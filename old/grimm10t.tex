\documentclass[12pt]{book}

\title{Grimms' Fairy Tales}
\author{The Brothers Grimm}
\date{}

\begin{document}

\maketitle

\begin{center}
{\Large ABOUT THIS E-TEXT}
\end{center}

\noindent \framebox[13cm]{
\parbox{12cm}{

The Project Gutenberg Etext Fairy Tales, by the Grimm Brothers


Copyright laws are changing all over the world, be sure to check
the copyright laws for your country before posting these files!!

Please take a look at the important information in this header.
We encourage you to keep this file on your own disk, keeping an
electronic path open for the next readers.  Do not remove this.

It must legally be the first thing seem when opening the book.
In fact, our legal advisors said we can't even change margins.

Welcome To The World of Free Plain Vanilla Electronic Texts

Etexts Readable By Both Humans and By Computers, Since 1971

These Etexts Prepared By Hundreds of Volunteers and Donations

Information on contacting Project Gutenberg to get Etexts, and
further information is included below.  We need your donations.
}
}

\vspace{1cm}

\begin{verse}
Title: Grimms' Fairy Tales \\
Author:  The Brothers Grimm
\end{verse}

\begin{verse}
Etext prepared by Emma Dudding, emma\_302@hotmail.com \\
John Bickers, jbickers@ihug.co.nz \\
and Dagny, dagnyj@hotmail.com. \\
Converted to \TeX{} by Carlos Campani, campani@ufpel.tche.br.
\end{verse}

\newpage

Project Gutenberg Etexts are usually created from multiple editions,
all of which are in the Public Domain in the United States, unless a
copyright notice is included.  Therefore, we usually do NOT keep any
of these books in compliance with any particular paper edition.


We are now trying to release all our books one month in advance
of the official release dates, leaving time for better editing.

Please note:  neither this list nor its contents are final till
midnight of the last day of the month of any such announcement.
The official release date of all Project Gutenberg Etexts is at
Midnight, Central Time, of the last day of the stated month.  A
preliminary version may often be posted for suggestion, comment
and editing by those who wish to do so.  To be sure you have an
up to date first edition [xxxxx10x.xxx] please check file sizes
in the first week of the next month.  Since our ftp program has
a bug in it that scrambles the date [tried to fix and failed] a
look at the file size will have to do, but we will try to see a
new copy has at least one byte more or less.

\newpage

\begin{center}
{\Large ABOUT PROJECT GUTENBERG}
\end{center}

We produce about two million dollars for each hour we work.  The
time it takes us, a rather conservative estimate, is fifty hours
to get any etext selected, entered, proofread, edited, copyright
searched and analyzed, the copyright letters written, etc.  This
projected audience is one hundred million readers.  If our value
per text is nominally estimated at one dollar then we produce \$2
million dollars per hour this year as we release thirty-six text
files per month, or 432 more Etexts in 1999 for a total of 2000+.
If these reach just 10% of the computerized population, then the
total should reach over 200 billion Etexts given away this year.

The Goal of Project Gutenberg is to Give Away One Trillion Etext
Files by December 31, 2001.  [10,000 x 100,000,000 = 1 Trillion]
This is ten thousand titles each to one hundred million readers,
which is only 5\% of the present number of computer users.

At our revised rates of production, we will reach only one-third
of that goal by the end of 2001, or about 3,333 Etexts unless we
manage to get some real funding; currently our funding is mostly
from Michael Hart's salary at Carnegie-Mellon University, and an
assortment of sporadic gifts; this salary is only good for a few
more years, so we are looking for something to replace it, as we
don't want Project Gutenberg to be so dependent on one person.

We need your donations more than ever!


All donations should be made to ``Project Gutenberg/CMU'': and are
tax deductible to the extent allowable by law.  (CMU = Carnegie-
Mellon University).

For these and other matters, please mail to:

\begin{center}
Project Gutenberg \\
P. O. Box  2782 \\
Champaign, IL 61825
\end{center}

When all other email fails\ldots try our Executive Director:
Michael S. Hart <hart@pobox.com>
hart@pobox.com forwards to hart@prairienet.org and \newline archive.org
if your mail bounces from archive.org, I will still see it, if
it bounces from prairienet.org, better resend later on\ldots

We would prefer to send you this information by email.

\newpage

To access Project Gutenberg etexts, use any Web browser
to view \newline http://promo.net/pg.  This site lists Etexts by
author and by title, and includes information about how
to get involved with Project Gutenberg.  You could also
download our past Newsletters, or subscribe here.  This
is one of our major sites, please email hart@pobox.com,
for a more complete list of our various sites.

To go directly to the etext collections, use FTP or any
Web browser to visit a Project Gutenberg mirror (mirror
sites are available on 7 continents; mirrors are listed
at http://promo.net/pg).

Mac users, do NOT point and click, typing works better.

Example FTP session:

\noindent ftp metalab.unc.edu\\
login: anonymous\\
password: your@login\\
cd pub/docs/books/gutenberg\\
cd etext90 through etext99\\
dir [to see files]\\
get or mget [to get files\ldots set bin for zip files]\\
GET GUTINDEX.??  [to get a year's listing of books, e.g., GUTINDEX.99]\\
GET GUTINDEX.ALL [to get a listing of ALL books]\\

\newpage

\begin{center}
{\Large INFORMATION PREPARED BY THE PROJECT GUTENBERG LEGAL ADVISOR}
\end{center}


-- START -- THE SMALL PRINT! -- FOR PUBLIC DOMAIN ETEXTS

Why is this ``Small Print!'' statement here?  You know: lawyers.
They tell us you might sue us if there is something wrong with
your copy of this etext, even if you got it for free from
someone other than us, and even if what's wrong is not our
fault.  So, among other things, this ``Small Print!'' statement
disclaims most of our liability to you.  It also tells you how
you can distribute copies of this etext if you want to.

\begin{center}
*BEFORE!* YOU USE OR READ THIS ETEXT
\end{center}

By using or reading any part of this PROJECT GUTENBERG-tm
etext, you indicate that you understand, agree to and accept
this ``Small Print!'' statement.  If you do not, you can receive
a refund of the money (if any) you paid for this etext by
sending a request within 30 days of receiving it to the person
you got it from.  If you received this etext on a physical
medium (such as a disk), you must return it with your request.

\begin{center}
ABOUT PROJECT GUTENBERG-TM ETEXTS
\end{center}

This PROJECT GUTENBERG-tm etext, like most PROJECT GUTEN\-BERG-
tm etexts, is a ``public domain'' work distributed by Professor
Michael S. Hart through the Project Gutenberg Association at
Carnegie-Mellon University (the ``Project'').  Among other
things, this means that no one owns a United States copyright
on or for this work, so the Project (and you!) can copy and
distribute it in the United States without permission and
without paying copyright royalties.  Special rules, set forth
below, apply if you wish to copy and distribute this etext
under the Project's ``PROJECT GUTENBERG'' trademark.

To create these etexts, the Project expends considerable
efforts to identify, transcribe and proofread public domain
works.  Despite these efforts, the Project's etexts and any
medium they may be on may contain ``Defects''.  Among other
things, Defects may take the form of incomplete, inaccurate or
corrupt data, transcription errors, a copyright or other
intellectual property infringement, a defective or damaged
disk or other etext medium, a computer virus, or computer
codes that damage or cannot be read by your equipment.

\begin{center}
LIMITED WARRANTY; DISCLAIMER OF DAMAGES
\end{center}

But for the ``Right of Replacement or Refund'' described below,
\begin{enumerate}
\item the Project (and any other party you may receive this
etext from as a PROJECT GUTENBERG-tm etext) disclaims all
liability to you for damages, costs and expenses, including
legal fees, and
\item YOU HAVE NO REMEDIES FOR NEGLIGENCE OR
UNDER STRICT LIABILITY, OR FOR BREACH OF WARRANTY OR CONTRACT,
INCLUDING BUT NOT LIMITED TO INDIRECT, CONSEQUENTIAL, PUNITIVE
OR INCIDENTAL DAMAGES, EVEN IF YOU GIVE NOTICE OF THE
POSSIBILITY OF SUCH DAMAGES.
\end{enumerate}

If you discover a Defect in this etext within 90 days of
receiving it, you can receive a refund of the money (if any)
you paid for it by sending an explanatory note within that
time to the person you received it from.  If you received it
on a physical medium, you must return it with your note, and
such person may choose to alternatively give you a replacement
copy.  If you received it electronically, such person may
choose to alternatively give you a second opportunity to
receive it electronically.

THIS ETEXT IS OTHERWISE PROVIDED TO YOU ``AS-IS''.  NO OTHER
WARRANTIES OF ANY KIND, EXPRESS OR IMPLIED, ARE MADE TO YOU AS
TO THE ETEXT OR ANY MEDIUM IT MAY BE ON, INCLUDING BUT NOT
LIMITED TO WARRANTIES OF MERCHANTABILITY OR FITNESS FOR A
PARTICULAR PURPOSE.

Some states do not allow disclaimers of implied warranties or
the exclusion or limitation of consequential damages, so the
above disclaimers and exclusions may not apply to you, and you
may have other legal rights.

\begin{center}
INDEMNITY
\end{center}

You will indemnify and hold the Project, its directors,
officers, members and agents harmless from all liability, cost
and expense, including legal fees, that arise directly or
indirectly from any of the following that you do or cause:

\begin{enumerate}
\item distribution of this etext,
\item  alteration, modification,
or addition to the etext, or
\item  any Defect.
\end{enumerate}

\begin{center}
DISTRIBUTION UNDER ``PROJECT GUTENBERG-tm''
\end{center}

You may distribute copies of this etext electronically, or by
disk, book or any other medium if you either delete this
``Small Print!'' and all other references to Project Gutenberg,
or:

\begin{enumerate}
\item  Only give exact copies of it.  Among other things, this
     requires that you do not remove, alter or modify the
     etext or this ``small print!'' statement.  You may however,
     if you wish, distribute this etext in machine readable
     binary, compressed, mark-up, or proprietary form,
     including any form resulting from conversion by word pro-
     cessing or hypertext software, but only so long as
     *EITHER*:
\begin{enumerate}
\item     The etext, when displayed, is clearly readable, and
          does *not* contain characters other than those
          intended by the author of the work, although tilde,
          asterisk (*) and underline (\_) characters may
          be used to convey punctuation intended by the
          author, and additional characters may be used to
          indicate hypertext links; OR

\item     The etext may be readily converted by the reader at
          no expense into plain ASCII, EBCDIC or equivalent
          form by the program that displays the etext (as is
          the case, for instance, with most word processors);
          OR

\item     You provide, or agree to also provide on request at
          no additional cost, fee or expense, a copy of the
          etext in its original plain ASCII form (or in EBCDIC
          or other equivalent proprietary form).
\end{enumerate}
\item  Honor the etext refund and replacement provisions of this
    ``Small Print!'' statement.

\item  Pay a trademark license fee to the Project of 20% of the
     net profits you derive calculated using the method you
     already use to calculate your applicable taxes.  If you
     don't derive profits, no royalty is due.  Royalties are
     payable to ``Project Gutenberg Association/Carnegie-Mellon
     University'' within the 60 days following each
     date you prepare (or were legally required to prepare)
     your annual (or equivalent periodic) tax return.
\end{enumerate}

\begin{center}
WHAT IF YOU *WANT* TO SEND MONEY EVEN IF YOU DON'T HAVE TO?
\end{center}

The Project gratefully accepts contributions in money, time,
scanning machines, OCR software, public domain etexts, royalty
free copyright licenses, and every other sort of contribution
you can think of.  Money should be paid to ``Project Gutenberg
Association / Carnegie-Mellon University''.

We are planning on making some changes in our donation structure
in 2000, so you might want to email me, hart@pobox.com beforehand.




*END THE SMALL PRINT! FOR PUBLIC DOMAIN ETEXTS* \newline Ver.04.29.93*END*


\newpage

\begin{verse}
Etext prepared by Emma Dudding, emma\_302@hotmail.com \\
John Bickers, jbickers@ihug.co.nz \\
and Dagny, dagnyj@hotmail.com \\
Converted to \TeX{} by Carlos Campani, campani@ufpel.tche.br.
\end{verse}

\newpage


The text is based on translations from
the Grimms' Kinder und Hausmarchen by
Edgar Taylor and
Marian Edwardes.



\newpage

\vspace*{8cm}

\begin{center}
{\Huge
THE BROTHERS GRIMM

\vspace*{3cm}

FAIRY TALES
}
\end{center}

\tableofcontents

\chapter{THE GOLDEN BIRD}

A certain king had a beautiful garden, and in the garden stood a tree
which bore golden apples. These apples were always counted, and about
the time when they began to grow ripe it was found that every night
one of them was gone. The king became very angry at this, and ordered
the gardener to keep watch all night under the tree. The gardener set
his eldest son to watch; but about twelve o'clock he fell asleep, and
in the morning another of the apples was missing. Then the second son
was ordered to watch; and at midnight he too fell asleep, and in the
morning another apple was gone. Then the third son offered to keep
watch; but the gardener at first would not let him, for fear some harm
should come to him: however, at last he consented, and the young man
laid himself under the tree to watch. As the clock struck twelve he
heard a rustling noise in the air, and a bird came flying that was of
pure gold; and as it was snapping at one of the apples with its beak,
the gardener's son jumped up and shot an arrow at it. But the arrow
did the bird no harm; only it dropped a golden feather from its tail,
and then flew away. The golden feather was brought to the king in the
morning, and all the council was called together. Everyone agreed that
it was worth more than all the wealth of the kingdom: but the king
said, 'One feather is of no use to me, I must have the whole bird.'

Then the gardener's eldest son set out and thought to find the golden
bird very easily; and when he had gone but a little way, he came to a
wood, and by the side of the wood he saw a fox sitting; so he took his
bow and made ready to shoot at it. Then the fox said, 'Do not shoot
me, for I will give you good counsel; I know what your business is,
and that you want to find the golden bird. You will reach a village in
the evening; and when you get there, you will see two inns opposite to
each other, one of which is very pleasant and beautiful to look at: go
not in there, but rest for the night in the other, though it may
appear to you to be very poor and mean.' But the son thought to
himself, 'What can such a beast as this know about the matter?' So he
shot his arrow at the fox; but he missed it, and it set up its tail
above its back and ran into the wood. Then he went his way, and in the
evening came to the village where the two inns were; and in one of
these were people singing, and dancing, and feasting; but the other
looked very dirty, and poor. 'I should be very silly,' said he, 'if I
went to that shabby house, and left this charming place'; so he went
into the smart house, and ate and drank at his ease, and forgot the
bird, and his country too.

Time passed on; and as the eldest son did not come back, and no
tidings were heard of him, the second son set out, and the same thing
happened to him. He met the fox, who gave him the good advice: but
when he came to the two inns, his eldest brother was standing at the
window where the merrymaking was, and called to him to come in; and he
could not withstand the temptation, but went in, and forgot the golden
bird and his country in the same manner.

Time passed on again, and the youngest son too wished to set out into
the wide world to seek for the golden bird; but his father would not
listen to it for a long while, for he was very fond of his son, and
was afraid that some ill luck might happen to him also, and prevent
his coming back. However, at last it was agreed he should go, for he
would not rest at home; and as he came to the wood, he met the fox,
and heard the same good counsel. But he was thankful to the fox, and
did not attempt his life as his brothers had done; so the fox said,
'Sit upon my tail, and you will travel faster.' So he sat down, and
the fox began to run, and away they went over stock and stone so quick
that their hair whistled in the wind.

When they came to the village, the son followed the fox's counsel, and
without looking about him went to the shabby inn and rested there all
night at his ease. In the morning came the fox again and met him as he
was beginning his journey, and said, 'Go straight forward, till you
come to a castle, before which lie a whole troop of soldiers fast
asleep and snoring: take no notice of them, but go into the castle and
pass on and on till you come to a room, where the golden bird sits in
a wooden cage; close by it stands a beautiful golden cage; but do not
try to take the bird out of the shabby cage and put it into the
handsome one, otherwise you will repent it.' Then the fox stretched
out his tail again, and the young man sat himself down, and away they
went over stock and stone till their hair whistled in the wind.

Before the castle gate all was as the fox had said: so the son went in
and found the chamber where the golden bird hung in a wooden cage, and
below stood the golden cage, and the three golden apples that had been
lost were lying close by it. Then thought he to himself, 'It will be a
very droll thing to bring away such a fine bird in this shabby cage';
so he opened the door and took hold of it and put it into the golden
cage. But the bird set up such a loud scream that all the soldiers
awoke, and they took him prisoner and carried him before the king. The
next morning the court sat to judge him; and when all was heard, it
sentenced him to die, unless he should bring the king the golden horse
which could run as swiftly as the wind; and if he did this, he was to
have the golden bird given him for his own.

So he set out once more on his journey, sighing, and in great despair,
when on a sudden his friend the fox met him, and said, 'You see now
what has happened on account of your not listening to my counsel. I
will still, however, tell you how to find the golden horse, if you
will do as I bid you. You must go straight on till you come to the
castle where the horse stands in his stall: by his side will lie the
groom fast asleep and snoring: take away the horse quietly, but be
sure to put the old leathern saddle upon him, and not the golden one
that is close by it.' Then the son sat down on the fox's tail, and
away they went over stock and stone till their hair whistled in the
wind.

All went right, and the groom lay snoring with his hand upon the
golden saddle. But when the son looked at the horse, he thought it a
great pity to put the leathern saddle upon it. 'I will give him the
good one,' said he; 'I am sure he deserves it.' As he took up the
golden saddle the groom awoke and cried out so loud, that all the
guards ran in and took him prisoner, and in the morning he was again
brought before the court to be judged, and was sentenced to die. But
it was agreed, that, if he could bring thither the beautiful princess,
he should live, and have the bird and the horse given him for his own.

Then he went his way very sorrowful; but the old fox came and said,
'Why did not you listen to me? If you had, you would have carried away
both the bird and the horse; yet will I once more give you counsel. Go
straight on, and in the evening you will arrive at a castle. At twelve
o'clock at night the princess goes to the bathing-house: go up to her
and give her a kiss, and she will let you lead her away; but take care
you do not suffer her to go and take leave of her father and mother.'
Then the fox stretched out his tail, and so away they went over stock
and stone till their hair whistled again.

As they came to the castle, all was as the fox had said, and at twelve
o'clock the young man met the princes going to the bath and gave her
the kiss, and she agreed to run away with him, but begged with many
tears that he would let her take leave of her father. At first he
refused, but she wept still more and more, and fell at his feet, till
at last he consented; but the moment she came to her father's house
the guards awoke and he was taken prisoner again.

Then he was brought before the king, and the king said, 'You shall
never have my daughter unless in eight days you dig away the hill that
stops the view from my window.' Now this hill was so big that the
whole world could not take it away: and when he had worked for seven
days, and had done very little, the fox came and said. 'Lie down and
go to sleep; I will work for you.' And in the morning he awoke and the
hill was gone; so he went merrily to the king, and told him that now
that it was removed he must give him the princess.

Then the king was obliged to keep his word, and away went the young
man and the princess; and the fox came and said to him, 'We will have
all three, the princess, the horse, and the bird.' 'Ah!' said the
young man, 'that would be a great thing, but how can you contrive it?'

'If you will only listen,' said the fox, 'it can be done. When you
come to the king, and he asks for the beautiful princess, you must
say, ``Here she is!'' Then he will be very joyful; and you will mount
the golden horse that they are to give you, and put out your hand to
take leave of them; but shake hands with the princess last. Then lift
her quickly on to the horse behind you; clap your spurs to his side,
and gallop away as fast as you can.'

All went right: then the fox said, 'When you come to the castle where
the bird is, I will stay with the princess at the door, and you will
ride in and speak to the king; and when he sees that it is the right
horse, he will bring out the bird; but you must sit still, and say
that you want to look at it, to see whether it is the true golden
bird; and when you get it into your hand, ride away.'

This, too, happened as the fox said; they carried off the bird, the
princess mounted again, and they rode on to a great wood. Then the fox
came, and said, 'Pray kill me, and cut off my head and my feet.' But
the young man refused to do it: so the fox said, 'I will at any rate
give you good counsel: beware of two things; ransom no one from the
gallows, and sit down by the side of no river.' Then away he went.
'Well,' thought the young man, 'it is no hard matter to keep that
advice.'

He rode on with the princess, till at last he came to the village
where he had left his two brothers. And there he heard a great noise
and uproar; and when he asked what was the matter, the people said,
'Two men are going to be hanged.' As he came nearer, he saw that the
two men were his brothers, who had turned robbers; so he said, 'Cannot
they in any way be saved?' But the people said 'No,' unless he would
bestow all his money upon the rascals and buy their liberty. Then he
did not stay to think about the matter, but paid what was asked, and
his brothers were given up, and went on with him towards their home.

And as they came to the wood where the fox first met them, it was so
cool and pleasant that the two brothers said, 'Let us sit down by the
side of the river, and rest a while, to eat and drink.' So he said,
'Yes,' and forgot the fox's counsel, and sat down on the side of the
river; and while he suspected nothing, they came behind, and threw him
down the bank, and took the princess, the horse, and the bird, and
went home to the king their master, and said. 'All this have we won by
our labour.' Then there was great rejoicing made; but the horse would
not eat, the bird would not sing, and the princess wept.

The youngest son fell to the bottom of the river's bed: luckily it was
nearly dry, but his bones were almost broken, and the bank was so
steep that he could find no way to get out. Then the old fox came once
more, and scolded him for not following his advice; otherwise no evil
would have befallen him: 'Yet,' said he, 'I cannot leave you here, so
lay hold of my tail and hold fast.' Then he pulled him out of the
river, and said to him, as he got upon the bank, 'Your brothers have
set watch to kill you, if they find you in the kingdom.' So he dressed
himself as a poor man, and came secretly to the king's court, and was
scarcely within the doors when the horse began to eat, and the bird to
sing, and princess left off weeping. Then he went to the king, and
told him all his brothers' roguery; and they were seized and punished,
and he had the princess given to him again; and after the king's death
he was heir to his kingdom.

A long while after, he went to walk one day in the wood, and the old
fox met him, and besought him with tears in his eyes to kill him, and
cut off his head and feet. And at last he did so, and in a moment the
fox was changed into a man, and turned out to be the brother of the
princess, who had been lost a great many many years.



\chapter{HANS IN LUCK}

Some men are born to good luck: all they do or try to do comes right--
all that falls to them is so much gain--all their geese are swans--all
their cards are trumps--toss them which way you will, they will
always, like poor puss, alight upon their legs, and only move on so
much the faster. The world may very likely not always think of them as
they think of themselves, but what care they for the world? what can
it know about the matter?

One of these lucky beings was neighbour Hans. Seven long years he had
worked hard for his master. At last he said, 'Master, my time is up; I
must go home and see my poor mother once more: so pray pay me my wages
and let me go.' And the master said, 'You have been a faithful and
good servant, Hans, so your pay shall be handsome.' Then he gave him a
lump of silver as big as his head.

Hans took out his pocket-handkerchief, put the piece of silver into
it, threw it over his shoulder, and jogged off on his road homewards.
As he went lazily on, dragging one foot after another, a man came in
sight, trotting gaily along on a capital horse. 'Ah!' said Hans aloud,
'what a fine thing it is to ride on horseback! There he sits as easy
and happy as if he was at home, in the chair by his fireside; he trips
against no stones, saves shoe-leather, and gets on he hardly knows
how.' Hans did not speak so softly but the horseman heard it all, and
said, 'Well, friend, why do you go on foot then?' 'Ah!' said he, 'I
have this load to carry: to be sure it is silver, but it is so heavy
that I can't hold up my head, and you must know it hurts my shoulder
sadly.' 'What do you say of making an exchange?' said the horseman. 'I
will give you my horse, and you shall give me the silver; which will
save you a great deal of trouble in carrying such a heavy load about
with you.' 'With all my heart,' said Hans: 'but as you are so kind to
me, I must tell you one thing--you will have a weary task to draw that
silver about with you.' However, the horseman got off, took the
silver, helped Hans up, gave him the bridle into one hand and the whip
into the other, and said, 'When you want to go very fast, smack your
lips loudly together, and cry ``Jip!'''

Hans was delighted as he sat on the horse, drew himself up, squared
his elbows, turned out his toes, cracked his whip, and rode merrily
off, one minute whistling a merry tune, and another singing,

\begin{verse}
 'No care and no sorrow,\\
  A fig for the morrow!\\
  We'll laugh and be merry,\\
  Sing neigh down derry!'
\end{verse}

After a time he thought he should like to go a little faster, so he
smacked his lips and cried 'Jip!' Away went the horse full gallop; and
before Hans knew what he was about, he was thrown off, and lay on his
back by the road-side. His horse would have ran off, if a shepherd who
was coming by, driving a cow, had not stopped it. Hans soon came to
himself, and got upon his legs again, sadly vexed, and said to the
shepherd, 'This riding is no joke, when a man has the luck to get upon
a beast like this that stumbles and flings him off as if it would
break his neck. However, I'm off now once for all: I like your cow now
a great deal better than this smart beast that played me this trick,
and has spoiled my best coat, you see, in this puddle; which, by the
by, smells not very like a nosegay. One can walk along at one's
leisure behind that cow--keep good company, and have milk, butter, and
cheese, every day, into the bargain. What would I give to have such a
prize!' 'Well,' said the shepherd, 'if you are so fond of her, I will
change my cow for your horse; I like to do good to my neighbours, even
though I lose by it myself.' 'Done!' said Hans, merrily. 'What a noble
heart that good man has!' thought he. Then the shepherd jumped upon
the horse, wished Hans and the cow good morning, and away he rode.

Hans brushed his coat, wiped his face and hands, rested a while, and
then drove off his cow quietly, and thought his bargain a very lucky
one. 'If I have only a piece of bread (and I certainly shall always be
able to get that), I can, whenever I like, eat my butter and cheese
with it; and when I am thirsty I can milk my cow and drink the milk:
and what can I wish for more?' When he came to an inn, he halted, ate
up all his bread, and gave away his last penny for a glass of beer.
When he had rested himself he set off again, driving his cow towards
his mother's village. But the heat grew greater as soon as noon came
on, till at last, as he found himself on a wide heath that would take
him more than an hour to cross, he began to be so hot and parched that
his tongue clave to the roof of his mouth. 'I can find a cure for
this,' thought he; 'now I will milk my cow and quench my thirst': so
he tied her to the stump of a tree, and held his leathern cap to milk
into; but not a drop was to be had. Who would have thought that this
cow, which was to bring him milk and butter and cheese, was all that
time utterly dry? Hans had not thought of looking to that.

While he was trying his luck in milking, and managing the matter very
clumsily, the uneasy beast began to think him very troublesome; and at
last gave him such a kick on the head as knocked him down; and there
he lay a long while senseless. Luckily a butcher soon came by, driving
a pig in a wheelbarrow. 'What is the matter with you, my man?' said
the butcher, as he helped him up. Hans told him what had happened, how
he was dry, and wanted to milk his cow, but found the cow was dry too.
Then the butcher gave him a flask of ale, saying, 'There, drink and
refresh yourself; your cow will give you no milk: don't you see she is
an old beast, good for nothing but the slaughter-house?' 'Alas, alas!'
said Hans, 'who would have thought it? What a shame to take my horse,
and give me only a dry cow! If I kill her, what will she be good for?
I hate cow-beef; it is not tender enough for me. If it were a pig now
--like that fat gentleman you are driving along at his ease--one could
do something with it; it would at any rate make sausages.' 'Well,'
said the butcher, 'I don't like to say no, when one is asked to do a
kind, neighbourly thing. To please you I will change, and give you my
fine fat pig for the cow.' 'Heaven reward you for your kindness and
self-denial!' said Hans, as he gave the butcher the cow; and taking
the pig off the wheel-barrow, drove it away, holding it by the string
that was tied to its leg.

So on he jogged, and all seemed now to go right with him: he had met
with some misfortunes, to be sure; but he was now well repaid for all.
How could it be otherwise with such a travelling companion as he had
at last got?

The next man he met was a countryman carrying a fine white goose. The
countryman stopped to ask what was o'clock; this led to further chat;
and Hans told him all his luck, how he had so many good bargains, and
how all the world went gay and smiling with him. The countryman than
began to tell his tale, and said he was going to take the goose to a
christening. 'Feel,' said he, 'how heavy it is, and yet it is only
eight weeks old. Whoever roasts and eats it will find plenty of fat
upon it, it has lived so well!' 'You're right,' said Hans, as he
weighed it in his hand; 'but if you talk of fat, my pig is no trifle.'
Meantime the countryman began to look grave, and shook his head. 'Hark
ye!' said he, 'my worthy friend, you seem a good sort of fellow, so I
can't help doing you a kind turn. Your pig may get you into a scrape.
In the village I just came from, the squire has had a pig stolen out
of his sty. I was dreadfully afraid when I saw you that you had got
the squire's pig. If you have, and they catch you, it will be a bad
job for you. The least they will do will be to throw you into the
horse-pond. Can you swim?'

Poor Hans was sadly frightened. 'Good man,' cried he, 'pray get me out
of this scrape. I know nothing of where the pig was either bred or
born; but he may have been the squire's for aught I can tell: you know
this country better than I do, take my pig and give me the goose.' 'I
ought to have something into the bargain,' said the countryman; 'give
a fat goose for a pig, indeed! 'Tis not everyone would do so much for
you as that. However, I will not be hard upon you, as you are in
trouble.' Then he took the string in his hand, and drove off the pig
by a side path; while Hans went on the way homewards free from care.
'After all,' thought he, 'that chap is pretty well taken in. I don't
care whose pig it is, but wherever it came from it has been a very
good friend to me. I have much the best of the bargain. First there
will be a capital roast; then the fat will find me in goose-grease for
six months; and then there are all the beautiful white feathers. I
will put them into my pillow, and then I am sure I shall sleep soundly
without rocking. How happy my mother will be! Talk of a pig, indeed!
Give me a fine fat goose.'

As he came to the next village, he saw a scissor-grinder with his
wheel, working and singing,

\begin{verse}
 'O'er hill and o'er dale\\
  So happy I roam,\\
  Work light and live well,\\
  All the world is my home;\\
  Then who so blythe, so merry as I?'
\end{verse}

Hans stood looking on for a while, and at last said, 'You must be well
off, master grinder! you seem so happy at your work.' 'Yes,' said the
other, 'mine is a golden trade; a good grinder never puts his hand
into his pocket without finding money in it--but where did you get
that beautiful goose?' 'I did not buy it, I gave a pig for it.' 'And
where did you get the pig?' 'I gave a cow for it.' 'And the cow?' 'I
gave a horse for it.' 'And the horse?' 'I gave a lump of silver as big
as my head for it.' 'And the silver?' 'Oh! I worked hard for that
seven long years.' 'You have thriven well in the world hitherto,' said
the grinder, 'now if you could find money in your pocket whenever you
put your hand in it, your fortune would be made.' 'Very true: but how
is that to be managed?' 'How? Why, you must turn grinder like myself,'
said the other; 'you only want a grindstone; the rest will come of
itself. Here is one that is but little the worse for wear: I would not
ask more than the value of your goose for it--will you buy?' 'How can
you ask?' said Hans; 'I should be the happiest man in the world, if I
could have money whenever I put my hand in my pocket: what could I
want more? there's the goose.' 'Now,' said the grinder, as he gave him
a common rough stone that lay by his side, 'this is a most capital
stone; do but work it well enough, and you can make an old nail cut
with it.'

Hans took the stone, and went his way with a light heart: his eyes
sparkled for joy, and he said to himself, 'Surely I must have been
born in a lucky hour; everything I could want or wish for comes of
itself. People are so kind; they seem really to think I do them a
favour in letting them make me rich, and giving me good bargains.'

Meantime he began to be tired, and hungry too, for he had given away
his last penny in his joy at getting the cow.

At last he could go no farther, for the stone tired him sadly: and he
dragged himself to the side of a river, that he might take a drink of
water, and rest a while. So he laid the stone carefully by his side on
the bank: but, as he stooped down to drink, he forgot it, pushed it a
little, and down it rolled, plump into the stream.

For a while he watched it sinking in the deep clear water; then sprang
up and danced for joy, and again fell upon his knees and thanked
Heaven, with tears in his eyes, for its kindness in taking away his
only plague, the ugly heavy stone.

'How happy am I!' cried he; 'nobody was ever so lucky as I.' Then up
he got with a light heart, free from all his troubles, and walked on
till he reached his mother's house, and told her how very easy the
road to good luck was.



\chapter{JORINDA AND JORINDEL}

There was once an old castle, that stood in the middle of a deep
gloomy wood, and in the castle lived an old fairy. Now this fairy
could take any shape she pleased. All the day long she flew about in
the form of an owl, or crept about the country like a cat; but at
night she always became an old woman again. When any young man came
within a hundred paces of her castle, he became quite fixed, and could
not move a step till she came and set him free; which she would not do
till he had given her his word never to come there again: but when any
pretty maiden came within that space she was changed into a bird, and
the fairy put her into a cage, and hung her up in a chamber in the
castle. There were seven hundred of these cages hanging in the castle,
and all with beautiful birds in them.

Now there was once a maiden whose name was Jorinda. She was prettier
than all the pretty girls that ever were seen before, and a shepherd
lad, whose name was Jorindel, was very fond of her, and they were soon
to be married. One day they went to walk in the wood, that they might
be alone; and Jorindel said, 'We must take care that we don't go too
near to the fairy's castle.' It was a beautiful evening; the last rays
of the setting sun shone bright through the long stems of the trees
upon the green underwood beneath, and the turtle-doves sang from the
tall birches.

Jorinda sat down to gaze upon the sun; Jorindel sat by her side; and
both felt sad, they knew not why; but it seemed as if they were to be
parted from one another for ever. They had wandered a long way; and
when they looked to see which way they should go home, they found
themselves at a loss to know what path to take.

The sun was setting fast, and already half of its circle had sunk
behind the hill: Jorindel on a sudden looked behind him, and saw
through the bushes that they had, without knowing it, sat down close
under the old walls of the castle. Then he shrank for fear, turned
pale, and trembled. Jorinda was just singing,

\begin{verse}
 'The ring-dove sang from the willow spray,\\
  Well-a-day! Well-a-day!\\
  He mourn'd for the fate of his darling mate,\\
  Well-a-day!'
\end{verse}

when her song stopped suddenly. Jorindel turned to see the reason, and
beheld his Jorinda changed into a nightingale, so that her song ended
with a mournful /jug, jug/. An owl with fiery eyes flew three times
round them, and three times screamed:

\begin{verse}
 'Tu whu! Tu whu! Tu whu!'
\end{verse}

Jorindel could not move; he stood fixed as a stone, and could neither
weep, nor speak, nor stir hand or foot. And now the sun went quite
down; the gloomy night came; the owl flew into a bush; and a moment
after the old fairy came forth pale and meagre, with staring eyes, and
a nose and chin that almost met one another.

She mumbled something to herself, seized the nightingale, and went
away with it in her hand. Poor Jorindel saw the nightingale was gone--
but what could he do? He could not speak, he could not move from the
spot where he stood. At last the fairy came back and sang with a
hoarse voice:

\begin{verse}
 'Till the prisoner is fast,\\
  And her doom is cast,\\
  There stay! Oh, stay!\\
  When the charm is around her,\\
  And the spell has bound her,\\
  Hie away! away!'
\end{verse}

On a sudden Jorindel found himself free. Then he fell on his knees
before the fairy, and prayed her to give him back his dear Jorinda:
but she laughed at him, and said he should never see her again; then
she went her way.

He prayed, he wept, he sorrowed, but all in vain. 'Alas!' he said,
'what will become of me?' He could not go back to his own home, so he
went to a strange village, and employed himself in keeping sheep. Many
a time did he walk round and round as near to the hated castle as he
dared go, but all in vain; he heard or saw nothing of Jorinda.

At last he dreamt one night that he found a beautiful purple flower,
and that in the middle of it lay a costly pearl; and he dreamt that he
plucked the flower, and went with it in his hand into the castle, and
that everything he touched with it was disenchanted, and that there he
found his Jorinda again.

In the morning when he awoke, he began to search over hill and dale
for this pretty flower; and eight long days he sought for it in vain:
but on the ninth day, early in the morning, he found the beautiful
purple flower; and in the middle of it was a large dewdrop, as big as
a costly pearl. Then he plucked the flower, and set out and travelled
day and night, till he came again to the castle.

He walked nearer than a hundred paces to it, and yet he did not become
fixed as before, but found that he could go quite close up to the
door. Jorindel was very glad indeed to see this. Then he touched the
door with the flower, and it sprang open; so that he went in through
the court, and listened when he heard so many birds singing. At last
he came to the chamber where the fairy sat, with the seven hundred
birds singing in the seven hundred cages. When she saw Jorindel she
was very angry, and screamed with rage; but she could not come within
two yards of him, for the flower he held in his hand was his
safeguard. He looked around at the birds, but alas! there were many,
many nightingales, and how then should he find out which was his
Jorinda? While he was thinking what to do, he saw the fairy had taken
down one of the cages, and was making the best of her way off through
the door. He ran or flew after her, touched the cage with the flower,
and Jorinda stood before him, and threw her arms round his neck
looking as beautiful as ever, as beautiful as when they walked
together in the wood.

Then he touched all the other birds with the flower, so that they all
took their old forms again; and he took Jorinda home, where they were
married, and lived happily together many years: and so did a good many
other lads, whose maidens had been forced to sing in the old fairy's
cages by themselves, much longer than they liked.



\chapter{THE TRAVELLING MUSICIANS}

An honest farmer had once an ass that had been a faithful servant to
him a great many years, but was now growing old and every day more and
more unfit for work. His master therefore was tired of keeping him and
began to think of putting an end to him; but the ass, who saw that
some mischief was in the wind, took himself slyly off, and began his
journey towards the great city, 'For there,' thought he, 'I may turn
musician.'

After he had travelled a little way, he spied a dog lying by the
roadside and panting as if he were tired. 'What makes you pant so, my
friend?' said the ass. 'Alas!' said the dog, 'my master was going to
knock me on the head, because I am old and weak, and can no longer
make myself useful to him in hunting; so I ran away; but what can I do
to earn my livelihood?' 'Hark ye!' said the ass, 'I am going to the
great city to turn musician: suppose you go with me, and try what you
can do in the same way?' The dog said he was willing, and they jogged
on together.

They had not gone far before they saw a cat sitting in the middle of
the road and making a most rueful face. 'Pray, my good lady,' said the
ass, 'what's the matter with you? You look quite out of spirits!' 'Ah,
me!' said the cat, 'how can one be in good spirits when one's life is
in danger? Because I am beginning to grow old, and had rather lie at
my ease by the fire than run about the house after the mice, my
mistress laid hold of me, and was going to drown me; and though I have
been lucky enough to get away from her, I do not know what I am to
live upon.' 'Oh,' said the ass, 'by all means go with us to the great
city; you are a good night singer, and may make your fortune as a
musician.' The cat was pleased with the thought, and joined the party.

Soon afterwards, as they were passing by a farmyard, they saw a cock
perched upon a gate, and screaming out with all his might and main.
'Bravo!' said the ass; 'upon my word, you make a famous noise; pray
what is all this about?' 'Why,' said the cock, 'I was just now saying
that we should have fine weather for our washing-day, and yet my
mistress and the cook don't thank me for my pains, but threaten to cut
off my head tomorrow, and make broth of me for the guests that are
coming on Sunday!' 'Heaven forbid!' said the ass, 'come with us Master
Chanticleer; it will be better, at any rate, than staying here to have
your head cut off! Besides, who knows? If we care to sing in tune, we
may get up some kind of a concert; so come along with us.' 'With all
my heart,' said the cock: so they all four went on jollily together.

They could not, however, reach the great city the first day; so when
night came on, they went into a wood to sleep. The ass and the dog
laid themselves down under a great tree, and the cat climbed up into
the branches; while the cock, thinking that the higher he sat the
safer he should be, flew up to the very top of the tree, and then,
according to his custom, before he went to sleep, looked out on all
sides of him to see that everything was well. In doing this, he saw
afar off something bright and shining and calling to his companions
said, 'There must be a house no great way off, for I see a light.' 'If
that be the case,' said the ass, 'we had better change our quarters,
for our lodging is not the best in the world!' 'Besides,' added the
dog, 'I should not be the worse for a bone or two, or a bit of meat.'
So they walked off together towards the spot where Chanticleer had
seen the light, and as they drew near it became larger and brighter,
till they at last came close to a house in which a gang of robbers
lived.

The ass, being the tallest of the company, marched up to the window
and peeped in. 'Well, Donkey,' said Chanticleer, 'what do you see?'
'What do I see?' replied the ass. 'Why, I see a table spread with all
kinds of good things, and robbers sitting round it making merry.'
'That would be a noble lodging for us,' said the cock. 'Yes,' said the
ass, 'if we could only get in'; so they consulted together how they
should contrive to get the robbers out; and at last they hit upon a
plan. The ass placed himself upright on his hind legs, with his
forefeet resting against the window; the dog got upon his back; the
cat scrambled up to the dog's shoulders, and the cock flew up and sat
upon the cat's head. When all was ready a signal was given, and they
began their music. The ass brayed, the dog barked, the cat mewed, and
the cock screamed; and then they all broke through the window at once,
and came tumbling into the room, amongst the broken glass, with a most
hideous clatter! The robbers, who had been not a little frightened by
the opening concert, had now no doubt that some frightful hobgoblin
had broken in upon them, and scampered away as fast as they could.

The coast once clear, our travellers soon sat down and dispatched what
the robbers had left, with as much eagerness as if they had not
expected to eat again for a month. As soon as they had satisfied
themselves, they put out the lights, and each once more sought out a
resting-place to his own liking. The donkey laid himself down upon a
heap of straw in the yard, the dog stretched himself upon a mat behind
the door, the cat rolled herself up on the hearth before the warm
ashes, and the cock perched upon a beam on the top of the house; and,
as they were all rather tired with their journey, they soon fell
asleep.

But about midnight, when the robbers saw from afar that the lights
were out and that all seemed quiet, they began to think that they had
been in too great a hurry to run away; and one of them, who was bolder
than the rest, went to see what was going on. Finding everything
still, he marched into the kitchen, and groped about till he found a
match in order to light a candle; and then, espying the glittering
fiery eyes of the cat, he mistook them for live coals, and held the
match to them to light it. But the cat, not understanding this joke,
sprang at his face, and spat, and scratched at him. This frightened
him dreadfully, and away he ran to the back door; but there the dog
jumped up and bit him in the leg; and as he was crossing over the yard
the ass kicked him; and the cock, who had been awakened by the noise,
crowed with all his might. At this the robber ran back as fast as he
could to his comrades, and told the captain how a horrid witch had got
into the house, and had spat at him and scratched his face with her
long bony fingers; how a man with a knife in his hand had hidden
himself behind the door, and stabbed him in the leg; how a black
monster stood in the yard and struck him with a club, and how the
devil had sat upon the top of the house and cried out, 'Throw the
rascal up here!' After this the robbers never dared to go back to the
house; but the musicians were so pleased with their quarters that they
took up their abode there; and there they are, I dare say, at this
very day.



\chapter{OLD SULTAN}

A shepherd had a faithful dog, called Sultan, who was grown very old,
and had lost all his teeth. And one day when the shepherd and his wife
were standing together before the house the shepherd said, 'I will
shoot old Sultan tomorrow morning, for he is of no use now.' But his
wife said, 'Pray let the poor faithful creature live; he has served us
well a great many years, and we ought to give him a livelihood for the
rest of his days.' 'But what can we do with him?' said the shepherd,
'he has not a tooth in his head, and the thieves don't care for him at
all; to be sure he has served us, but then he did it to earn his
livelihood; tomorrow shall be his last day, depend upon it.'

Poor Sultan, who was lying close by them, heard all that the shepherd
and his wife said to one another, and was very much frightened to
think tomorrow would be his last day; so in the evening he went to his
good friend the wolf, who lived in the wood, and told him all his
sorrows, and how his master meant to kill him in the morning. 'Make
yourself easy,' said the wolf, 'I will give you some good advice. Your
master, you know, goes out every morning very early with his wife into
the field; and they take their little child with them, and lay it down
behind the hedge in the shade while they are at work. Now do you lie
down close by the child, and pretend to be watching it, and I will
come out of the wood and run away with it; you must run after me as
fast as you can, and I will let it drop; then you may carry it back,
and they will think you have saved their child, and will be so
thankful to you that they will take care of you as long as you live.'
The dog liked this plan very well; and accordingly so it was managed.
The wolf ran with the child a little way; the shepherd and his wife
screamed out; but Sultan soon overtook him, and carried the poor
little thing back to his master and mistress. Then the shepherd patted
him on the head, and said, 'Old Sultan has saved our child from the
wolf, and therefore he shall live and be well taken care of, and have
plenty to eat. Wife, go home, and give him a good dinner, and let him
have my old cushion to sleep on as long as he lives.' So from this
time forward Sultan had all that he could wish for.

Soon afterwards the wolf came and wished him joy, and said, 'Now, my
good fellow, you must tell no tales, but turn your head the other way
when I want to taste one of the old shepherd's fine fat sheep.' 'No,'
said the Sultan; 'I will be true to my master.' However, the wolf
thought he was in joke, and came one night to get a dainty morsel. But
Sultan had told his master what the wolf meant to do; so he laid wait
for him behind the barn door, and when the wolf was busy looking out
for a good fat sheep, he had a stout cudgel laid about his back, that
combed his locks for him finely.

Then the wolf was very angry, and called Sultan 'an old rogue,' and
swore he would have his revenge. So the next morning the wolf sent the
boar to challenge Sultan to come into the wood to fight the matter.
Now Sultan had nobody he could ask to be his second but the shepherd's
old three-legged cat; so he took her with him, and as the poor thing
limped along with some trouble, she stuck up her tail straight in the
air.

The wolf and the wild boar were first on the ground; and when they
espied their enemies coming, and saw the cat's long tail standing
straight in the air, they thought she was carrying a sword for Sultan
to fight with; and every time she limped, they thought she was picking
up a stone to throw at them; so they said they should not like this
way of fighting, and the boar lay down behind a bush, and the wolf
jumped up into a tree. Sultan and the cat soon came up, and looked
about and wondered that no one was there. The boar, however, had not
quite hidden himself, for his ears stuck out of the bush; and when he
shook one of them a little, the cat, seeing something move, and
thinking it was a mouse, sprang upon it, and bit and scratched it, so
that the boar jumped up and grunted, and ran away, roaring out, 'Look
up in the tree, there sits the one who is to blame.' So they looked
up, and espied the wolf sitting amongst the branches; and they called
him a cowardly rascal, and would not suffer him to come down till he
was heartily ashamed of himself, and had promised to be good friends
again with old Sultan.



\chapter{THE STRAW, THE COAL, AND THE BEAN}

In a village dwelt a poor old woman, who had gathered together a dish
of beans and wanted to cook them. So she made a fire on her hearth,
and that it might burn the quicker, she lighted it with a handful of
straw. When she was emptying the beans into the pan, one dropped
without her observing it, and lay on the ground beside a straw, and
soon afterwards a burning coal from the fire leapt down to the two.
Then the straw began and said: 'Dear friends, from whence do you come
here?' The coal replied: 'I fortunately sprang out of the fire, and if
I had not escaped by sheer force, my death would have been certain,--I
should have been burnt to ashes.' The bean said: 'I too have escaped
with a whole skin, but if the old woman had got me into the pan, I
should have been made into broth without any mercy, like my comrades.'
'And would a better fate have fallen to my lot?' said the straw. 'The
old woman has destroyed all my brethren in fire and smoke; she seized
sixty of them at once, and took their lives. I luckily slipped through
her fingers.'

'But what are we to do now?' said the coal.

'I think,' answered the bean, 'that as we have so fortunately escaped
death, we should keep together like good companions, and lest a new
mischance should overtake us here, we should go away together, and
repair to a foreign country.'

The proposition pleased the two others, and they set out on their way
together. Soon, however, they came to a little brook, and as there was
no bridge or foot-plank, they did not know how they were to get over
it. The straw hit on a good idea, and said: 'I will lay myself
straight across, and then you can walk over on me as on a bridge.' The
straw therefore stretched itself from one bank to the other, and the
coal, who was of an impetuous disposition, tripped quite boldly on to
the newly-built bridge. But when she had reached the middle, and heard
the water rushing beneath her, she was after all, afraid, and stood
still, and ventured no farther. The straw, however, began to burn,
broke in two pieces, and fell into the stream. The coal slipped after
her, hissed when she got into the water, and breathed her last. The
bean, who had prudently stayed behind on the shore, could not but
laugh at the event, was unable to stop, and laughed so heartily that
she burst. It would have been all over with her, likewise, if, by good
fortune, a tailor who was travelling in search of work, had not sat
down to rest by the brook. As he had a compassionate heart he pulled
out his needle and thread, and sewed her together. The bean thanked
him most prettily, but as the tailor used black thread, all beans
since then have a black seam.



\chapter{BRIAR ROSE}

A king and queen once upon a time reigned in a country a great way
off, where there were in those days fairies. Now this king and queen
had plenty of money, and plenty of fine clothes to wear, and plenty of
good things to eat and drink, and a coach to ride out in every day:
but though they had been married many years they had no children, and
this grieved them very much indeed. But one day as the queen was
walking by the side of the river, at the bottom of the garden, she saw
a poor little fish, that had thrown itself out of the water, and lay
gasping and nearly dead on the bank. Then the queen took pity on the
little fish, and threw it back again into the river; and before it
swam away it lifted its head out of the water and said, 'I know what
your wish is, and it shall be fulfilled, in return for your kindness
to me--you will soon have a daughter.' What the little fish had
foretold soon came to pass; and the queen had a little girl, so very
beautiful that the king could not cease looking on it for joy, and
said he would hold a great feast and make merry, and show the child to
all the land. So he asked his kinsmen, and nobles, and friends, and
neighbours. But the queen said, 'I will have the fairies also, that
they might be kind and good to our little daughter.' Now there were
thirteen fairies in the kingdom; but as the king and queen had only
twelve golden dishes for them to eat out of, they were forced to leave
one of the fairies without asking her. So twelve fairies came, each
with a high red cap on her head, and red shoes with high heels on her
feet, and a long white wand in her hand: and after the feast was over
they gathered round in a ring and gave all their best gifts to the
little princess. One gave her goodness, another beauty, another
riches, and so on till she had all that was good in the world.

Just as eleven of them had done blessing her, a great noise was heard
in the courtyard, and word was brought that the thirteenth fairy was
come, with a black cap on her head, and black shoes on her feet, and a
broomstick in her hand: and presently up she came into the dining-
hall. Now, as she had not been asked to the feast she was very angry,
and scolded the king and queen very much, and set to work to take her
revenge. So she cried out, 'The king's daughter shall, in her
fifteenth year, be wounded by a spindle, and fall down dead.' Then the
twelfth of the friendly fairies, who had not yet given her gift, came
forward, and said that the evil wish must be fulfilled, but that she
could soften its mischief; so her gift was, that the king's daughter,
when the spindle wounded her, should not really die, but should only
fall asleep for a hundred years.

However, the king hoped still to save his dear child altogether from
the threatened evil; so he ordered that all the spindles in the
kingdom should be bought up and burnt. But all the gifts of the first
eleven fairies were in the meantime fulfilled; for the princess was so
beautiful, and well behaved, and good, and wise, that everyone who
knew her loved her.

It happened that, on the very day she was fifteen years old, the king
and queen were not at home, and she was left alone in the palace. So
she roved about by herself, and looked at all the rooms and chambers,
till at last she came to an old tower, to which there was a narrow
staircase ending with a little door. In the door there was a golden
key, and when she turned it the door sprang open, and there sat an old
lady spinning away very busily. 'Why, how now, good mother,' said the
princess; 'what are you doing there?' 'Spinning,' said the old lady,
and nodded her head, humming a tune, while buzz! went the wheel. 'How
prettily that little thing turns round!' said the princess, and took
the spindle and began to try and spin. But scarcely had she touched
it, before the fairy's prophecy was fulfilled; the spindle wounded
her, and she fell down lifeless on the ground.

However, she was not dead, but had only fallen into a deep sleep; and
the king and the queen, who had just come home, and all their court,
fell asleep too; and the horses slept in the stables, and the dogs in
the court, the pigeons on the house-top, and the very flies slept upon
the walls. Even the fire on the hearth left off blazing, and went to
sleep; the jack stopped, and the spit that was turning about with a
goose upon it for the king's dinner stood still; and the cook, who was
at that moment pulling the kitchen-boy by the hair to give him a box
on the ear for something he had done amiss, let him go, and both fell
asleep; the butler, who was slyly tasting the ale, fell asleep with
the jug at his lips: and thus everything stood still, and slept
soundly.

A large hedge of thorns soon grew round the palace, and every year it
became higher and thicker; till at last the old palace was surrounded
and hidden, so that not even the roof or the chimneys could be seen.
But there went a report through all the land of the beautiful sleeping
Briar Rose (for so the king's daughter was called): so that, from time
to time, several kings' sons came, and tried to break through the
thicket into the palace. This, however, none of them could ever do;
for the thorns and bushes laid hold of them, as it were with hands;
and there they stuck fast, and died wretchedly.

After many, many years there came a king's son into that land: and an
old man told him the story of the thicket of thorns; and how a
beautiful palace stood behind it, and how a wonderful princess, called
Briar Rose, lay in it asleep, with all her court. He told, too, how he
had heard from his grandfather that many, many princes had come, and
had tried to break through the thicket, but that they had all stuck
fast in it, and died. Then the young prince said, 'All this shall not
frighten me; I will go and see this Briar Rose.' The old man tried to
hinder him, but he was bent upon going.

Now that very day the hundred years were ended; and as the prince came
to the thicket he saw nothing but beautiful flowering shrubs, through
which he went with ease, and they shut in after him as thick as ever.
Then he came at last to the palace, and there in the court lay the
dogs asleep; and the horses were standing in the stables; and on the
roof sat the pigeons fast asleep, with their heads under their wings.
And when he came into the palace, the flies were sleeping on the
walls; the spit was standing still; the butler had the jug of ale at
his lips, going to drink a draught; the maid sat with a fowl in her
lap ready to be plucked; and the cook in the kitchen was still holding
up her hand, as if she was going to beat the boy.

Then he went on still farther, and all was so still that he could hear
every breath he drew; till at last he came to the old tower, and
opened the door of the little room in which Briar Rose was; and there
she lay, fast asleep on a couch by the window. She looked so beautiful
that he could not take his eyes off her, so he stooped down and gave
her a kiss. But the moment he kissed her she opened her eyes and
awoke, and smiled upon him; and they went out together; and soon the
king and queen also awoke, and all the court, and gazed on each other
with great wonder. And the horses shook themselves, and the dogs
jumped up and barked; the pigeons took their heads from under their
wings, and looked about and flew into the fields; the flies on the
walls buzzed again; the fire in the kitchen blazed up; round went the
jack, and round went the spit, with the goose for the king's dinner
upon it; the butler finished his draught of ale; the maid went on
plucking the fowl; and the cook gave the boy the box on his ear.

And then the prince and Briar Rose were married, and the wedding feast
was given; and they lived happily together all their lives long.



\chapter{THE DOG AND THE SPARROW}

A shepherd's dog had a master who took no care of him, but often let
him suffer the greatest hunger. At last he could bear it no longer; so
he took to his heels, and off he ran in a very sad and sorrowful mood.
On the road he met a sparrow that said to him, 'Why are you so sad, my
friend?' 'Because,' said the dog, 'I am very very hungry, and have
nothing to eat.' 'If that be all,' answered the sparrow, 'come with me
into the next town, and I will soon find you plenty of food.' So on
they went together into the town: and as they passed by a butcher's
shop, the sparrow said to the dog, 'Stand there a little while till I
peck you down a piece of meat.' So the sparrow perched upon the shelf:
and having first looked carefully about her to see if anyone was
watching her, she pecked and scratched at a steak that lay upon the
edge of the shelf, till at last down it fell. Then the dog snapped it
up, and scrambled away with it into a corner, where he soon ate it all
up. 'Well,' said the sparrow, 'you shall have some more if you will;
so come with me to the next shop, and I will peck you down another
steak.' When the dog had eaten this too, the sparrow said to him,
'Well, my good friend, have you had enough now?' 'I have had plenty of
meat,' answered he, 'but I should like to have a piece of bread to eat
after it.' 'Come with me then,' said the sparrow, 'and you shall soon
have that too.' So she took him to a baker's shop, and pecked at two
rolls that lay in the window, till they fell down: and as the dog
still wished for more, she took him to another shop and pecked down
some more for him. When that was eaten, the sparrow asked him whether
he had had enough now. 'Yes,' said he; 'and now let us take a walk a
little way out of the town.' So they both went out upon the high road;
but as the weather was warm, they had not gone far before the dog
said, 'I am very much tired--I should like to take a nap.' 'Very
well,' answered the sparrow, 'do so, and in the meantime I will perch
upon that bush.' So the dog stretched himself out on the road, and
fell fast asleep. Whilst he slept, there came by a carter with a cart
drawn by three horses, and loaded with two casks of wine. The sparrow,
seeing that the carter did not turn out of the way, but would go on in
the track in which the dog lay, so as to drive over him, called out,
'Stop! stop! Mr Carter, or it shall be the worse for you.' But the
carter, grumbling to himself, 'You make it the worse for me, indeed!
what can you do?' cracked his whip, and drove his cart over the poor
dog, so that the wheels crushed him to death. 'There,' cried the
sparrow, 'thou cruel villain, thou hast killed my friend the dog. Now
mind what I say. This deed of thine shall cost thee all thou art
worth.' 'Do your worst, and welcome,' said the brute, 'what harm can
you do me?' and passed on. But the sparrow crept under the tilt of the
cart, and pecked at the bung of one of the casks till she loosened it;
and than all the wine ran out, without the carter seeing it. At last
he looked round, and saw that the cart was dripping, and the cask
quite empty. 'What an unlucky wretch I am!' cried he. 'Not wretch
enough yet!' said the sparrow, as she alighted upon the head of one of
the horses, and pecked at him till he reared up and kicked. When the
carter saw this, he drew out his hatchet and aimed a blow at the
sparrow, meaning to kill her; but she flew away, and the blow fell
upon the poor horse's head with such force, that he fell down dead.
'Unlucky wretch that I am!' cried he. 'Not wretch enough yet!' said
the sparrow. And as the carter went on with the other two horses, she
again crept under the tilt of the cart, and pecked out the bung of the
second cask, so that all the wine ran out. When the carter saw this,
he again cried out, 'Miserable wretch that I am!' But the sparrow
answered, 'Not wretch enough yet!' and perched on the head of the
second horse, and pecked at him too. The carter ran up and struck at
her again with his hatchet; but away she flew, and the blow fell upon
the second horse and killed him on the spot. 'Unlucky wretch that I
am!' said he. 'Not wretch enough yet!' said the sparrow; and perching
upon the third horse, she began to peck him too. The carter was mad
with fury; and without looking about him, or caring what he was about,
struck again at the sparrow; but killed his third horse as he done the
other two. 'Alas! miserable wretch that I am!' cried he. 'Not wretch
enough yet!' answered the sparrow as she flew away; 'now will I plague
and punish thee at thy own house.' The carter was forced at last to
leave his cart behind him, and to go home overflowing with rage and
vexation. 'Alas!' said he to his wife, 'what ill luck has befallen me!
--my wine is all spilt, and my horses all three dead.' 'Alas!
husband,' replied she, 'and a wicked bird has come into the house, and
has brought with her all the birds in the world, I am sure, and they
have fallen upon our corn in the loft, and are eating it up at such a
rate!' Away ran the husband upstairs, and saw thousands of birds
sitting upon the floor eating up his corn, with the sparrow in the
midst of them. 'Unlucky wretch that I am!' cried the carter; for he
saw that the corn was almost all gone. 'Not wretch enough yet!' said
the sparrow; 'thy cruelty shall cost thee they life yet!' and away she
flew.

The carter seeing that he had thus lost all that he had, went down
into his kitchen; and was still not sorry for what he had done, but
sat himself angrily and sulkily in the chimney corner. But the sparrow
sat on the outside of the window, and cried 'Carter! thy cruelty shall
cost thee thy life!' With that he jumped up in a rage, seized his
hatchet, and threw it at the sparrow; but it missed her, and only
broke the window. The sparrow now hopped in, perched upon the window-
seat, and cried, 'Carter! it shall cost thee thy life!' Then he became
mad and blind with rage, and struck the window-seat with such force
that he cleft it in two: and as the sparrow flew from place to place,
the carter and his wife were so furious, that they broke all their
furniture, glasses, chairs, benches, the table, and at last the walls,
without touching the bird at all. In the end, however, they caught
her: and the wife said, 'Shall I kill her at once?' 'No,' cried he,
'that is letting her off too easily: she shall die a much more cruel
death; I will eat her.' But the sparrow began to flutter about, and
stretch out her neck and cried, 'Carter! it shall cost thee thy life
yet!' With that he could wait no longer: so he gave his wife the
hatchet, and cried, 'Wife, strike at the bird and kill her in my
hand.' And the wife struck; but she missed her aim, and hit her
husband on the head so that he fell down dead, and the sparrow flew
quietly home to her nest.



\chapter{THE TWELVE DANCING PRINCESSES}

There was a king who had twelve beautiful daughters. They slept in
twelve beds all in one room; and when they went to bed, the doors were
shut and locked up; but every morning their shoes were found to be
quite worn through as if they had been danced in all night; and yet
nobody could find out how it happened, or where they had been.

Then the king made it known to all the land, that if any person could
discover the secret, and find out where it was that the princesses
danced in the night, he should have the one he liked best for his
wife, and should be king after his death; but whoever tried and did
not succeed, after three days and nights, should be put to death.

A king's son soon came. He was well entertained, and in the evening
was taken to the chamber next to the one where the princesses lay in
their twelve beds. There he was to sit and watch where they went to
dance; and, in order that nothing might pass without his hearing it,
the door of his chamber was left open. But the king's son soon fell
asleep; and when he awoke in the morning he found that the princesses
had all been dancing, for the soles of their shoes were full of holes.
The same thing happened the second and third night: so the king
ordered his head to be cut off. After him came several others; but
they had all the same luck, and all lost their lives in the same
manner.

Now it chanced that an old soldier, who had been wounded in battle and
could fight no longer, passed through the country where this king
reigned: and as he was travelling through a wood, he met an old woman,
who asked him where he was going. 'I hardly know where I am going, or
what I had better do,' said the soldier; 'but I think I should like
very well to find out where it is that the princesses dance, and then
in time I might be a king.' 'Well,' said the old dame, 'that is no
very hard task: only take care not to drink any of the wine which one
of the princesses will bring to you in the evening; and as soon as she
leaves you pretend to be fast asleep.'

Then she gave him a cloak, and said, 'As soon as you put that on you
will become invisible, and you will then be able to follow the
princesses wherever they go.' When the soldier heard all this good
counsel, he determined to try his luck: so he went to the king, and
said he was willing to undertake the task.

He was as well received as the others had been, and the king ordered
fine royal robes to be given him; and when the evening came he was led
to the outer chamber. Just as he was going to lie down, the eldest of
the princesses brought him a cup of wine; but the soldier threw it all
away secretly, taking care not to drink a drop. Then he laid himself
down on his bed, and in a little while began to snore very loud as if
he was fast asleep. When the twelve princesses heard this they laughed
heartily; and the eldest said, 'This fellow too might have done a
wiser thing than lose his life in this way!' Then they rose up and
opened their drawers and boxes, and took out all their fine clothes,
and dressed themselves at the glass, and skipped about as if they were
eager to begin dancing. But the youngest said, 'I don't know how it
is, while you are so happy I feel very uneasy; I am sure some
mischance will befall us.' 'You simpleton,' said the eldest, 'you are
always afraid; have you forgotten how many kings' sons have already
watched in vain? And as for this soldier, even if I had not given him
his sleeping draught, he would have slept soundly enough.'

When they were all ready, they went and looked at the soldier; but he
snored on, and did not stir hand or foot: so they thought they were
quite safe; and the eldest went up to her own bed and clapped her
hands, and the bed sank into the floor and a trap-door flew open. The
soldier saw them going down through the trap-door one after another,
the eldest leading the way; and thinking he had no time to lose, he
jumped up, put on the cloak which the old woman had given him, and
followed them; but in the middle of the stairs he trod on the gown of
the youngest princess, and she cried out to her sisters, 'All is not
right; someone took hold of my gown.' 'You silly creature!' said the
eldest, 'it is nothing but a nail in the wall.' Then down they all
went, and at the bottom they found themselves in a most delightful
grove of trees; and the leaves were all of silver, and glittered and
sparkled beautifully. The soldier wished to take away some token of
the place; so he broke off a little branch, and there came a loud
noise from the tree. Then the youngest daughter said again, 'I am sure
all is not right--did not you hear that noise? That never happened
before.' But the eldest said, 'It is only our princes, who are
shouting for joy at our approach.'

Then they came to another grove of trees, where all the leaves were of
gold; and afterwards to a third, where the leaves were all glittering
diamonds. And the soldier broke a branch from each; and every time
there was a loud noise, which made the youngest sister tremble with
fear; but the eldest still said, it was only the princes, who were
crying for joy. So they went on till they came to a great lake; and at
the side of the lake there lay twelve little boats with twelve
handsome princes in them, who seemed to be waiting there for the
princesses.

One of the princesses went into each boat, and the soldier stepped
into the same boat with the youngest. As they were rowing over the
lake, the prince who was in the boat with the youngest princess and
the soldier said, 'I do not know why it is, but though I am rowing
with all my might we do not get on so fast as usual, and I am quite
tired: the boat seems very heavy today.' 'It is only the heat of the
weather,' said the princess: 'I feel it very warm too.'

On the other side of the lake stood a fine illuminated castle, from
which came the merry music of horns and trumpets. There they all
landed, and went into the castle, and each prince danced with his
princess; and the soldier, who was all the time invisible, danced with
them too; and when any of the princesses had a cup of wine set by her,
he drank it all up, so that when she put the cup to her mouth it was
empty. At this, too, the youngest sister was terribly frightened, but
the eldest always silenced her. They danced on till three o'clock in
the morning, and then all their shoes were worn out, so that they were
obliged to leave off. The princes rowed them back again over the lake
(but this time the soldier placed himself in the boat with the eldest
princess); and on the opposite shore they took leave of each other,
the princesses promising to come again the next night.

When they came to the stairs, the soldier ran on before the
princesses, and laid himself down; and as the twelve sisters slowly
came up very much tired, they heard him snoring in his bed; so they
said, 'Now all is quite safe'; then they undressed themselves, put
away their fine clothes, pulled off their shoes, and went to bed. In
the morning the soldier said nothing about what had happened, but
determined to see more of this strange adventure, and went again the
second and third night; and every thing happened just as before; the
princesses danced each time till their shoes were worn to pieces, and
then returned home. However, on the third night the soldier carried
away one of the golden cups as a token of where he had been.

As soon as the time came when he was to declare the secret, he was
taken before the king with the three branches and the golden cup; and
the twelve princesses stood listening behind the door to hear what he
would say. And when the king asked him. 'Where do my twelve daughters
dance at night?' he answered, 'With twelve princes in a castle under
ground.' And then he told the king all that had happened, and showed
him the three branches and the golden cup which he had brought with
him. Then the king called for the princesses, and asked them whether
what the soldier said was true: and when they saw that they were
discovered, and that it was of no use to deny what had happened, they
confessed it all. And the king asked the soldier which of them he
would choose for his wife; and he answered, 'I am not very young, so I
will have the eldest.'--And they were married that very day, and the
soldier was chosen to be the king's heir.



\chapter{THE FISHERMAN AND HIS WIFE}

There was once a fisherman who lived with his wife in a pigsty, close
by the seaside. The fisherman used to go out all day long a-fishing;
and one day, as he sat on the shore with his rod, looking at the
sparkling waves and watching his line, all on a sudden his float was
dragged away deep into the water: and in drawing it up he pulled out a
great fish. But the fish said, 'Pray let me live! I am not a real
fish; I am an enchanted prince: put me in the water again, and let me
go!' 'Oh, ho!' said the man, 'you need not make so many words about
the matter; I will have nothing to do with a fish that can talk: so
swim away, sir, as soon as you please!' Then he put him back into the
water, and the fish darted straight down to the bottom, and left a
long streak of blood behind him on the wave.

When the fisherman went home to his wife in the pigsty, he told her
how he had caught a great fish, and how it had told him it was an
enchanted prince, and how, on hearing it speak, he had let it go
again. 'Did not you ask it for anything?' said the wife, 'we live very
wretchedly here, in this nasty dirty pigsty; do go back and tell the
fish we want a snug little cottage.'

The fisherman did not much like the business: however, he went to the
seashore; and when he came back there the water looked all yellow and
green. And he stood at the water's edge, and said:

\begin{verse}
 'O man of the sea!\\
  Hearken to me!\\
  My wife Ilsabill\\
  Will have her own will,\\
  And hath sent me to beg a boon of thee!'
\end{verse}

Then the fish came swimming to him, and said, 'Well, what is her will?
What does your wife want?' 'Ah!' said the fisherman, 'she says that
when I had caught you, I ought to have asked you for something before
I let you go; she does not like living any longer in the pigsty, and
wants a snug little cottage.' 'Go home, then,' said the fish; 'she is
in the cottage already!' So the man went home, and saw his wife
standing at the door of a nice trim little cottage. 'Come in, come
in!' said she; 'is not this much better than the filthy pigsty we
had?' And there was a parlour, and a bedchamber, and a kitchen; and
behind the cottage there was a little garden, planted with all sorts
of flowers and fruits; and there was a courtyard behind, full of ducks
and chickens. 'Ah!' said the fisherman, 'how happily we shall live
now!' 'We will try to do so, at least,' said his wife.

Everything went right for a week or two, and then Dame Ilsabill said,
'Husband, there is not near room enough for us in this cottage; the
courtyard and the garden are a great deal too small; I should like to
have a large stone castle to live in: go to the fish again and tell
him to give us a castle.' 'Wife,' said the fisherman, 'I don't like to
go to him again, for perhaps he will be angry; we ought to be easy
with this pretty cottage to live in.' 'Nonsense!' said the wife; 'he
will do it very willingly, I know; go along and try!'

The fisherman went, but his heart was very heavy: and when he came to
the sea, it looked blue and gloomy, though it was very calm; and he
went close to the edge of the waves, and said:

\begin{verse}
 'O man of the sea!\\
  Hearken to me!\\
  My wife Ilsabill\\
  Will have her own will,\\
  And hath sent me to beg a boon of thee!'
\end{verse}

'Well, what does she want now?' said the fish. 'Ah!' said the man,
dolefully, 'my wife wants to live in a stone castle.' 'Go home, then,'
said the fish; 'she is standing at the gate of it already.' So away
went the fisherman, and found his wife standing before the gate of a
great castle. 'See,' said she, 'is not this grand?' With that they
went into the castle together, and found a great many servants there,
and the rooms all richly furnished, and full of golden chairs and
tables; and behind the castle was a garden, and around it was a park
half a mile long, full of sheep, and goats, and hares, and deer; and
in the courtyard were stables and cow-houses. 'Well,' said the man,
'now we will live cheerful and happy in this beautiful castle for the
rest of our lives.' 'Perhaps we may,' said the wife; 'but let us sleep
upon it, before we make up our minds to that.' So they went to bed.

The next morning when Dame Ilsabill awoke it was broad daylight, and
she jogged the fisherman with her elbow, and said, 'Get up, husband,
and bestir yourself, for we must be king of all the land.' 'Wife,
wife,' said the man, 'why should we wish to be the king? I will not be
king.' 'Then I will,' said she. 'But, wife,' said the fisherman, 'how
can you be king--the fish cannot make you a king?' 'Husband,' said
she, 'say no more about it, but go and try! I will be king.' So the
man went away quite sorrowful to think that his wife should want to be
king. This time the sea looked a dark grey colour, and was overspread
with curling waves and the ridges of foam as he cried out:

\begin{verse}
 'O man of the sea!\\
  Hearken to me!\\
  My wife Ilsabill\\
  Will have her own will,\\
  And hath sent me to beg a boon of thee!'
\end{verse}

'Well, what would she have now?' said the fish. 'Alas!' said the poor
man, 'my wife wants to be king.' 'Go home,' said the fish; 'she is
king already.'

Then the fisherman went home; and as he came close to the palace he
saw a troop of soldiers, and heard the sound of drums and trumpets.
And when he went in he saw his wife sitting on a throne of gold and
diamonds, with a golden crown upon her head; and on each side of her
stood six fair maidens, each a head taller than the other. 'Well,
wife,' said the fisherman, 'are you king?' 'Yes,' said she, 'I am
king.' And when he had looked at her for a long time, he said, 'Ah,
wife! what a fine thing it is to be king! Now we shall never have
anything more to wish for as long as we live.' 'I don't know how that
may be,' said she; 'never is a long time. I am king, it is true; but I
begin to be tired of that, and I think I should like to be emperor.'
'Alas, wife! why should you wish to be emperor?' said the fisherman.
'Husband,' said she, 'go to the fish! I say I will be emperor.' 'Ah,
wife!' replied the fisherman, 'the fish cannot make an emperor, I am
sure, and I should not like to ask him for such a thing.' 'I am king,'
said Ilsabill, 'and you are my slave; so go at once!'

So the fisherman was forced to go; and he muttered as he went along,
'This will come to no good, it is too much to ask; the fish will be
tired at last, and then we shall be sorry for what we have done.' He
soon came to the seashore; and the water was quite black and muddy,
and a mighty whirlwind blew over the waves and rolled them about, but
he went as near as he could to the water's brink, and said:

\begin{verse}
 'O man of the sea!\\
  Hearken to me!\\
  My wife Ilsabill\\
  Will have her own will,\\
  And hath sent me to beg a boon of thee!'
\end{verse}

'What would she have now?' said the fish. 'Ah!' said the fisherman,
'she wants to be emperor.' 'Go home,' said the fish; 'she is emperor
already.'

So he went home again; and as he came near he saw his wife Ilsabill
sitting on a very lofty throne made of solid gold, with a great crown
on her head full two yards high; and on each side of her stood her
guards and attendants in a row, each one smaller than the other, from
the tallest giant down to a little dwarf no bigger than my finger. And
before her stood princes, and dukes, and earls: and the fisherman went
up to her and said, 'Wife, are you emperor?' 'Yes,' said she, 'I am
emperor.' 'Ah!' said the man, as he gazed upon her, 'what a fine thing
it is to be emperor!' 'Husband,' said she, 'why should we stop at
being emperor? I will be pope next.' 'O wife, wife!' said he, 'how can
you be pope? there is but one pope at a time in Christendom.'
'Husband,' said she, 'I will be pope this very day.' 'But,' replied
the husband, 'the fish cannot make you pope.' 'What nonsense!' said
she; 'if he can make an emperor, he can make a pope: go and try him.'

So the fisherman went. But when he came to the shore the wind was
raging and the sea was tossed up and down in boiling waves, and the
ships were in trouble, and rolled fearfully upon the tops of the
billows. In the middle of the heavens there was a little piece of blue
sky, but towards the south all was red, as if a dreadful storm was
rising. At this sight the fisherman was dreadfully frightened, and he
trembled so that his knees knocked together: but still he went down
near to the shore, and said:

\begin{verse}
 'O man of the sea!\\
  Hearken to me!\\
  My wife Ilsabill\\
  Will have her own will,\\
  And hath sent me to beg a boon of thee!'
\end{verse}

'What does she want now?' said the fish. 'Ah!' said the fisherman, 'my
wife wants to be pope.' 'Go home,' said the fish; 'she is pope
already.'

Then the fisherman went home, and found Ilsabill sitting on a throne
that was two miles high. And she had three great crowns on her head,
and around her stood all the pomp and power of the Church. And on each
side of her were two rows of burning lights, of all sizes, the
greatest as large as the highest and biggest tower in the world, and
the least no larger than a small rushlight. 'Wife,' said the
fisherman, as he looked at all this greatness, 'are you pope?' 'Yes,'
said she, 'I am pope.' 'Well, wife,' replied he, 'it is a grand thing
to be pope; and now you must be easy, for you can be nothing greater.'
'I will think about that,' said the wife. Then they went to bed: but
Dame Ilsabill could not sleep all night for thinking what she should
be next. At last, as she was dropping asleep, morning broke, and the
sun rose. 'Ha!' thought she, as she woke up and looked at it through
the window, 'after all I cannot prevent the sun rising.' At this
thought she was very angry, and wakened her husband, and said,
'Husband, go to the fish and tell him I must be lord of the sun and
moon.' The fisherman was half asleep, but the thought frightened him
so much that he started and fell out of bed. 'Alas, wife!' said he,
'cannot you be easy with being pope?' 'No,' said she, 'I am very
uneasy as long as the sun and moon rise without my leave. Go to the
fish at once!'

Then the man went shivering with fear; and as he was going down to the
shore a dreadful storm arose, so that the trees and the very rocks
shook. And all the heavens became black with stormy clouds, and the
lightnings played, and the thunders rolled; and you might have seen in
the sea great black waves, swelling up like mountains with crowns of
white foam upon their heads. And the fisherman crept towards the sea,
and cried out, as well as he could:

\begin{verse}
 'O man of the sea!\\
  Hearken to me!\\
  My wife Ilsabill\\
  Will have her own will,\\
  And hath sent me to beg a boon of thee!'
\end{verse}

'What does she want now?' said the fish. 'Ah!' said he, 'she wants to
be lord of the sun and moon.' 'Go home,' said the fish, 'to your
pigsty again.'

And there they live to this very day.



\chapter{THE WILLOW-WREN AND THE BEAR}

Once in summer-time the bear and the wolf were walking in the forest,
and the bear heard a bird singing so beautifully that he said:
'Brother wolf, what bird is it that sings so well?' 'That is the King
of birds,' said the wolf, 'before whom we must bow down.' In reality
the bird was the willow-wren. 'IF that's the case,' said the bear, 'I
should very much like to see his royal palace; come, take me thither.'
'That is not done quite as you seem to think,' said the wolf; 'you
must wait until the Queen comes,' Soon afterwards, the Queen arrived
with some food in her beak, and the lord King came too, and they began
to feed their young ones. The bear would have liked to go at once, but
the wolf held him back by the sleeve, and said: 'No, you must wait
until the lord and lady Queen have gone away again.' So they took
stock of the hole where the nest lay, and trotted away. The bear,
however, could not rest until he had seen the royal palace, and when a
short time had passed, went to it again. The King and Queen had just
flown out, so he peeped in and saw five or six young ones lying there.
'Is that the royal palace?' cried the bear; 'it is a wretched palace,
and you are not King's children, you are disreputable children!' When
the young wrens heard that, they were frightfully angry, and screamed:
'No, that we are not! Our parents are honest people! Bear, you will
have to pay for that!'

The bear and the wolf grew uneasy, and turned back and went into their
holes. The young willow-wrens, however, continued to cry and scream,
and when their parents again brought food they said: 'We will not so
much as touch one fly's leg, no, not if we were dying of hunger, until
you have settled whether we are respectable children or not; the bear
has been here and has insulted us!' Then the old King said: 'Be easy,
he shall be punished,' and he at once flew with the Queen to the
bear's cave, and called in: 'Old Growler, why have you insulted my
children? You shall suffer for it--we will punish you by a bloody
war.' Thus war was announced to the Bear, and all four-footed animals
were summoned to take part in it, oxen, asses, cows, deer, and every
other animal the earth contained. And the willow-wren summoned
everything which flew in the air, not only birds, large and small, but
midges, and hornets, bees and flies had to come.

When the time came for the war to begin, the willow-wren sent out
spies to discover who was the enemy's commander-in-chief. The gnat,
who was the most crafty, flew into the forest where the enemy was
assembled, and hid herself beneath a leaf of the tree where the
password was to be announced. There stood the bear, and he called the
fox before him and said: 'Fox, you are the most cunning of all
animals, you shall be general and lead us.' 'Good,' said the fox, 'but
what signal shall we agree upon?' No one knew that, so the fox said:
'I have a fine long bushy tail, which almost looks like a plume of red
feathers. When I lift my tail up quite high, all is going well, and
you must charge; but if I let it hang down, run away as fast as you
can.' When the gnat had heard that, she flew away again, and revealed
everything, down to the minutest detail, to the willow-wren. When day
broke, and the battle was to begin, all the four-footed animals came
running up with such a noise that the earth trembled. The willow-wren
with his army also came flying through the air with such a humming,
and whirring, and swarming that every one was uneasy and afraid, and
on both sides they advanced against each other. But the willow-wren
sent down the hornet, with orders to settle beneath the fox's tail,
and sting with all his might. When the fox felt the first string, he
started so that he one leg, from pain, but he bore it, and
still kept his tail high in the air; at the second sting, he was
forced to put it down for a moment; at the third, he could hold out no
longer, screamed, and put his tail between his legs. When the animals
saw that, they thought all was lost, and began to flee, each into his
hole, and the birds had won the battle.

Then the King and Queen flew home to their children and cried:
'Children, rejoice, eat and drink to your heart's content, we have won
the battle!' But the young wrens said: 'We will not eat yet, the bear
must come to the nest, and beg for pardon and say that we are
honourable children, before we will do that.' Then the willow-wren
flew to the bear's hole and cried: 'Growler, you are to come to the
nest to my children, and beg their pardon, or else every rib of your
body shall be broken.' So the bear crept thither in the greatest fear,
and begged their pardon. And now at last the young wrens were
satisfied, and sat down together and ate and drank, and made merry
till quite late into the night.



\chapter{THE FROG-PRINCE}

One fine evening a young princess put on her bonnet and clogs, and
went out to take a walk by herself in a wood; and when she came to a
cool spring of water, that rose in the midst of it, she sat herself
down to rest a while. Now she had a golden ball in her hand, which was
her favourite plaything; and she was always tossing it up into the
air, and catching it again as it fell. After a time she threw it up so
high that she missed catching it as it fell; and the ball bounded
away, and rolled along upon the ground, till at last it fell down into
the spring. The princess looked into the spring after her ball, but it
was very deep, so deep that she could not see the bottom of it. Then
she began to bewail her loss, and said, 'Alas! if I could only get my
ball again, I would give all my fine clothes and jewels, and
everything that I have in the world.'

Whilst she was speaking, a frog put its head out of the water, and
said, 'Princess, why do you weep so bitterly?' 'Alas!' said she, 'what
can you do for me, you nasty frog? My golden ball has fallen into the
spring.' The frog said, 'I want not your pearls, and jewels, and fine
clothes; but if you will love me, and let me live with you and eat
from off your golden plate, and sleep upon your bed, I will bring you
your ball again.' 'What nonsense,' thought the princess, 'this silly
frog is talking! He can never even get out of the spring to visit me,
though he may be able to get my ball for me, and therefore I will tell
him he shall have what he asks.' So she said to the frog, 'Well, if
you will bring me my ball, I will do all you ask.' Then the frog put
his head down, and dived deep under the water; and after a little
while he came up again, with the ball in his mouth, and threw it on
the edge of the spring. As soon as the young princess saw her ball,
she ran to pick it up; and she was so overjoyed to have it in her hand
again, that she never thought of the frog, but ran home with it as
fast as she could. The frog called after her, 'Stay, princess, and
take me with you as you said,' But she did not stop to hear a word.

The next day, just as the princess had sat down to dinner, she heard a
strange noise--tap, tap--plash, plash--as if something was coming up
the marble staircase: and soon afterwards there was a gentle knock at
the door, and a little voice cried out and said:

\begin{verse}
 'Open the door, my princess dear,\\
  Open the door to thy true love here!\\
  And mind the words that thou and I said\\
  By the fountain cool, in the greenwood shade.'
\end{verse}

Then the princess ran to the door and opened it, and there she saw the
frog, whom she had quite forgotten. At this sight she was sadly
frightened, and shutting the door as fast as she could came back to
her seat. The king, her father, seeing that something had frightened
her, asked her what was the matter. 'There is a nasty frog,' said she,
'at the door, that lifted my ball for me out of the spring this
morning: I told him that he should live with me here, thinking that he
could never get out of the spring; but there he is at the door, and he
wants to come in.'

While she was speaking the frog knocked again at the door, and said:

\begin{verse}
 'Open the door, my princess dear,\\
  Open the door to thy true love here!\\
  And mind the words that thou and I said\\
  By the fountain cool, in the greenwood shade.'
\end{verse}

Then the king said to the young princess, 'As you have given your word
you must keep it; so go and let him in.' She did so, and the frog
hopped into the room, and then straight on--tap, tap--plash, plash--
from the bottom of the room to the top, till he came up close to the
table where the princess sat. 'Pray lift me upon chair,' said he to
the princess, 'and let me sit next to you.' As soon as she had done
this, the frog said, 'Put your plate nearer to me, that I may eat out
of it.' This she did, and when he had eaten as much as he could, he
said, 'Now I am tired; carry me upstairs, and put me into your bed.'
And the princess, though very unwilling, took him up in her hand, and
put him upon the pillow of her own bed, where he slept all night long.
As soon as it was light he jumped up, hopped downstairs, and went out
of the house. 'Now, then,' thought the princess, 'at last he is gone,
and I shall be troubled with him no more.'

But she was mistaken; for when night came again she heard the same
tapping at the door; and the frog came once more, and said:

\begin{verse}
 'Open the door, my princess dear,\\
  Open the door to thy true love here!\\
  And mind the words that thou and I said\\
  By the fountain cool, in the greenwood shade.'
\end{verse}

And when the princess opened the door the frog came in, and slept upon
her pillow as before, till the morning broke. And the third night he
did the same. But when the princess awoke on the following morning she
was astonished to see, instead of the frog, a handsome prince, gazing
on her with the most beautiful eyes she had ever seen, and standing at
the head of her bed.

He told her that he had been enchanted by a spiteful fairy, who had
changed him into a frog; and that he had been fated so to abide till
some princess should take him out of the spring, and let him eat from
her plate, and sleep upon her bed for three nights. 'You,' said the
prince, 'have broken his cruel charm, and now I have nothing to wish
for but that you should go with me into my father's kingdom, where I
will marry you, and love you as long as you live.'

The young princess, you may be sure, was not long in saying 'Yes' to
all this; and as they spoke a gay coach drove up, with eight beautiful
horses, decked with plumes of feathers and a golden harness; and
behind the coach rode the prince's servant, faithful Heinrich, who had
bewailed the misfortunes of his dear master during his enchantment so
long and so bitterly, that his heart had well-nigh burst.

They then took leave of the king, and got into the coach with eight
horses, and all set out, full of joy and merriment, for the prince's
kingdom, which they reached safely; and there they lived happily a
great many years.



\chapter{CAT AND MOUSE IN PARTNERSHIP}

A certain cat had made the acquaintance of a mouse, and had said so
much to her about the great love and friendship she felt for her, that
at length the mouse agreed that they should live and keep house
together. 'But we must make a provision for winter, or else we shall
suffer from hunger,' said the cat; 'and you, little mouse, cannot
venture everywhere, or you will be caught in a trap some day.' The
good advice was followed, and a pot of fat was bought, but they did
not know where to put it. At length, after much consideration, the cat
said: 'I know no place where it will be better stored up than in the
church, for no one dares take anything away from there. We will set it
beneath the altar, and not touch it until we are really in need of
it.' So the pot was placed in safety, but it was not long before the
cat had a great yearning for it, and said to the mouse: 'I want to
tell you something, little mouse; my cousin has brought a little son
into the world, and has asked me to be godmother; he is white with
brown spots, and I am to hold him over the font at the christening.
Let me go out today, and you look after the house by yourself.' 'Yes,
yes,' answered the mouse, 'by all means go, and if you get anything
very good to eat, think of me. I should like a drop of sweet red
christening wine myself.' All this, however, was untrue; the cat had
no cousin, and had not been asked to be godmother. She went straight
to the church, stole to the pot of fat, began to lick at it, and
licked the top of the fat off. Then she took a walk upon the roofs of
the town, looked out for opportunities, and then stretched herself in
the sun, and licked her lips whenever she thought of the pot of fat,
and not until it was evening did she return home. 'Well, here you are
again,' said the mouse, 'no doubt you have had a merry day.' 'All went
off well,' answered the cat. 'What name did they give the child?' 'Top
off!' said the cat quite coolly. 'Top off!' cried the mouse, 'that is
a very odd and uncommon name, is it a usual one in your family?' 'What
does that matter,' said the cat, 'it is no worse than Crumb-stealer,
as your godchildren are called.'

Before long the cat was seized by another fit of yearning. She said to
the mouse: 'You must do me a favour, and once more manage the house
for a day alone. I am again asked to be godmother, and, as the child
has a white ring round its neck, I cannot refuse.' The good mouse
consented, but the cat crept behind the town walls to the church, and
devoured half the pot of fat. 'Nothing ever seems so good as what one
keeps to oneself,' said she, and was quite satisfied with her day's
work. When she went home the mouse inquired: 'And what was the child
christened?' 'Half-done,' answered the cat. 'Half-done! What are you
saying? I never heard the name in my life, I'll wager anything it is
not in the calendar!'

The cat's mouth soon began to water for some more licking. 'All good
things go in threes,' said she, 'I am asked to stand godmother again.
The child is quite black, only it has white paws, but with that
exception, it has not a single white hair on its whole body; this only
happens once every few years, you will let me go, won't you?' 'Top-
off! Half-done!' answered the mouse, 'they are such odd names, they
make me very thoughtful.' 'You sit at home,' said the cat, 'in your
dark-grey fur coat and long tail, and are filled with fancies, that's
because you do not go out in the daytime.' During the cat's absence
the mouse cleaned the house, and put it in order, but the greedy cat
entirely emptied the pot of fat. 'When everything is eaten up one has
some peace,' said she to herself, and well filled and fat she did not
return home till night. The mouse at once asked what name had been
given to the third child. 'It will not please you more than the
others,' said the cat. 'He is called All-gone.' 'All-gone,' cried the
mouse 'that is the most suspicious name of all! I have never seen it
in print. All-gone; what can that mean?' and she shook her head,
curled herself up, and lay down to sleep.

>From this time forth no one invited the cat to be godmother, but when
the winter had come and there was no longer anything to be found
outside, the mouse thought of their provision, and said: 'Come, cat,
we will go to our pot of fat which we have stored up for ourselves--we
shall enjoy that.' 'Yes,' answered the cat, 'you will enjoy it as much
as you would enjoy sticking that dainty tongue of yours out of the
window.' They set out on their way, but when they arrived, the pot of
fat certainly was still in its place, but it was empty. 'Alas!' said
the mouse, 'now I see what has happened, now it comes to light! You a
true friend! You have devoured all when you were standing godmother.
First top off, then half-done, then--' 'Will you hold your tongue,'
cried the cat, 'one word more, and I will eat you too.' 'All-gone' was
already on the poor mouse's lips; scarcely had she spoken it before
the cat sprang on her, seized her, and swallowed her down. Verily,
that is the way of the world.



\chapter{THE GOOSE-GIRL}

The king of a great land died, and left his queen to take care of
their only child. This child was a daughter, who was very beautiful;
and her mother loved her dearly, and was very kind to her. And there
was a good fairy too, who was fond of the princess, and helped her
mother to watch over her. When she grew up, she was betrothed to a
prince who lived a great way off; and as the time drew near for her to
be married, she got ready to set off on her journey to his country.
Then the queen her mother, packed up a great many costly things;
jewels, and gold, and silver; trinkets, fine dresses, and in short
everything that became a royal bride. And she gave her a waiting-maid
to ride with her, and give her into the bridegroom's hands; and each
had a horse for the journey. Now the princess's horse was the fairy's
gift, and it was called Falada, and could speak.

When the time came for them to set out, the fairy went into her bed-
chamber, and took a little knife, and cut off a lock of her hair, and
gave it to the princess, and said, 'Take care of it, dear child; for
it is a charm that may be of use to you on the road.' Then they all
took a sorrowful leave of the princess; and she put the lock of hair
into her bosom, got upon her horse, and set off on her journey to her
bridegroom's kingdom.

One day, as they were riding along by a brook, the princess began to
feel very thirsty: and she said to her maid, 'Pray get down, and fetch
me some water in my golden cup out of yonder brook, for I want to
drink.' 'Nay,' said the maid, 'if you are thirsty, get off yourself,
and stoop down by the water and drink; I shall not be your waiting-
maid any longer.' Then she was so thirsty that she got down, and knelt
over the little brook, and drank; for she was frightened, and dared
not bring out her golden cup; and she wept and said, 'Alas! what will
become of me?' And the lock answered her, and said:

\begin{verse}
 'Alas! alas! if thy mother knew it,\\
  Sadly, sadly, would she rue it.'
\end{verse}

But the princess was very gentle and meek, so she said nothing to her
maid's ill behaviour, but got upon her horse again.

Then all rode farther on their journey, till the day grew so warm, and
the sun so scorching, that the bride began to feel very thirsty again;
and at last, when they came to a river, she forgot her maid's rude
speech, and said, 'Pray get down, and fetch me some water to drink in
my golden cup.' But the maid answered her, and even spoke more
haughtily than before: 'Drink if you will, but I shall not be your
waiting-maid.' Then the princess was so thirsty that she got off her
horse, and lay down, and held her head over the running stream, and
cried and said, 'What will become of me?' And the lock of hair
answered her again:

\begin{verse}
 'Alas! alas! if thy mother knew it,\\
  Sadly, sadly, would she rue it.'
\end{verse}

And as she leaned down to drink, the lock of hair fell from her bosom,
and floated away with the water. Now she was so frightened that she
did not see it; but her maid saw it, and was very glad, for she knew
the charm; and she saw that the poor bride would be in her power, now
that she had lost the hair. So when the bride had done drinking, and
would have got upon Falada again, the maid said, 'I shall ride upon
Falada, and you may have my horse instead'; so she was forced to give
up her horse, and soon afterwards to take off her royal clothes and
put on her maid's shabby ones.

At last, as they drew near the end of their journey, this treacherous
servant threatened to kill her mistress if she ever told anyone what
had happened. But Falada saw it all, and marked it well.

Then the waiting-maid got upon Falada, and the real bride rode upon
the other horse, and they went on in this way till at last they came
to the royal court. There was great joy at their coming, and the
prince flew to meet them, and lifted the maid from her horse, thinking
she was the one who was to be his wife; and she was led upstairs to
the royal chamber; but the true princess was told to stay in the court
below.

Now the old king happened just then to have nothing else to do; so he
amused himself by sitting at his kitchen window, looking at what was
going on; and he saw her in the courtyard. As she looked very pretty,
and too delicate for a waiting-maid, he went up into the royal chamber
to ask the bride who it was she had brought with her, that was thus
left standing in the court below. 'I brought her with me for the sake
of her company on the road,' said she; 'pray give the girl some work
to do, that she may not be idle.' The old king could not for some time
think of any work for her to do; but at last he said, 'I have a lad
who takes care of my geese; she may go and help him.' Now the name of
this lad, that the real bride was to help in watching the king's
geese, was Curdken.

But the false bride said to the prince, 'Dear husband, pray do me one
piece of kindness.' 'That I will,' said the prince. 'Then tell one of
your slaughterers to cut off the head of the horse I rode upon, for it
was very unruly, and plagued me sadly on the road'; but the truth was,
she was very much afraid lest Falada should some day or other speak,
and tell all she had done to the princess. She carried her point, and
the faithful Falada was killed; but when the true princess heard of
it, she wept, and begged the man to nail up Falada's head against a
large dark gate of the city, through which she had to pass every
morning and evening, that there she might still see him sometimes.
Then the slaughterer said he would do as she wished; and cut off the
head, and nailed it up under the dark gate.

Early the next morning, as she and Curdken went out through the gate,
she said sorrowfully:

\begin{verse}
 'Falada, Falada, there thou hangest!'
\end{verse}

and the head answered:

\begin{verse}
 'Bride, bride, there thou gangest!\\
  Alas! alas! if thy mother knew it,\\
  Sadly, sadly, would she rue it.'
\end{verse}

Then they went out of the city, and drove the geese on. And when she
came to the meadow, she sat down upon a bank there, and let down her
waving locks of hair, which were all of pure silver; and when Curdken
saw it glitter in the sun, he ran up, and would have pulled some of
the locks out, but she cried:

\begin{verse}
 'Blow, breezes, blow!\\
  Let Curdken's hat go!\\
  Blow, breezes, blow!\\
  Let him after it go!\\
  O'er hills, dales, and rocks,\\
  Away be it whirl'd\\
  Till the silvery locks\\
  Are all comb'd and curl'd!
\end{verse}

Then there came a wind, so strong that it blew off Curdken's hat; and
away it flew over the hills: and he was forced to turn and run after
it; till, by the time he came back, she had done combing and curling
her hair, and had put it up again safe. Then he was very angry and
sulky, and would not speak to her at all; but they watched the geese
until it grew dark in the evening, and then drove them homewards.

The next morning, as they were going through the dark gate, the poor
girl looked up at Falada's head, and cried:

\begin{verse}
 'Falada, Falada, there thou hangest!'
\end{verse}

and the head answered:

\begin{verse}
 'Bride, bride, there thou gangest!\\
  Alas! alas! if they mother knew it,\\
  Sadly, sadly, would she rue it.'
\end{verse}

Then she drove on the geese, and sat down again in the meadow, and
began to comb out her hair as before; and Curdken ran up to her, and
wanted to take hold of it; but she cried out quickly:

\begin{verse}
 'Blow, breezes, blow!\\
  Let Curdken's hat go!\\
  Blow, breezes, blow!\\
  Let him after it go!\\
  O'er hills, dales, and rocks,\\
  Away be it whirl'd\\
  Till the silvery locks\\
  Are all comb'd and curl'd!
\end{verse}

Then the wind came and blew away his hat; and off it flew a great way,
over the hills and far away, so that he had to run after it; and when
he came back she had bound up her hair again, and all was safe. So
they watched the geese till it grew dark.

In the evening, after they came home, Curdken went to the old king,
and said, 'I cannot have that strange girl to help me to keep the
geese any longer.' 'Why?' said the king. 'Because, instead of doing
any good, she does nothing but tease me all day long.' Then the king
made him tell him what had happened. And Curdken said, 'When we go in
the morning through the dark gate with our flock of geese, she cries
and talks with the head of a horse that hangs upon the wall, and says:

\begin{verse}
 'Falada, Falada, there thou hangest!'
\end{verse}

and the head answers:

\begin{verse}
 'Bride, bride, there thou gangest!\\
  Alas! alas! if they mother knew it,\\
  Sadly, sadly, would she rue it.'
\end{verse}

And Curdken went on telling the king what had happened upon the meadow
where the geese fed; how his hat was blown away; and how he was forced
to run after it, and to leave his flock of geese to themselves. But
the old king told the boy to go out again the next day: and when
morning came, he placed himself behind the dark gate, and heard how
she spoke to Falada, and how Falada answered. Then he went into the
field, and hid himself in a bush by the meadow's side; and he soon saw
with his own eyes how they drove the flock of geese; and how, after a
little time, she let down her hair that glittered in the sun. And then
he heard her say:

\begin{verse}
 'Blow, breezes, blow!\\
  Let Curdken's hat go!\\
  Blow, breezes, blow!\\
  Let him after it go!\\
  O'er hills, dales, and rocks,\\
  Away be it whirl'd\\
  Till the silvery locks\\
  Are all comb'd and curl'd!
\end{verse}

And soon came a gale of wind, and carried away Curdken's hat, and away
went Curdken after it, while the girl went on combing and curling her
hair. All this the old king saw: so he went home without being seen;
and when the little goose-girl came back in the evening he called her
aside, and asked her why she did so: but she burst into tears, and
said, 'That I must not tell you or any man, or I shall lose my life.'

But the old king begged so hard, that she had no peace till she had
told him all the tale, from beginning to end, word for word. And it
was very lucky for her that she did so, for when she had done the king
ordered royal clothes to be put upon her, and gazed on her with
wonder, she was so beautiful. Then he called his son and told him that
he had only a false bride; for that she was merely a waiting-maid,
while the true bride stood by. And the young king rejoiced when he saw
her beauty, and heard how meek and patient she had been; and without
saying anything to the false bride, the king ordered a great feast to
be got ready for all his court. The bridegroom sat at the top, with
the false princess on one side, and the true one on the other; but
nobody knew her again, for her beauty was quite dazzling to their
eyes; and she did not seem at all like the little goose-girl, now that
she had her brilliant dress on.

When they had eaten and drank, and were very merry, the old king said
he would tell them a tale. So he began, and told all the story of the
princess, as if it was one that he had once heard; and he asked the
true waiting-maid what she thought ought to be done to anyone who
would behave thus. 'Nothing better,' said this false bride, 'than that
she should be thrown into a cask stuck round with sharp nails, and
that two white horses should be put to it, and should drag it from
street to street till she was dead.' 'Thou art she!' said the old
king; 'and as thou has judged thyself, so shall it be done to thee.'
And the young king was then married to his true wife, and they reigned
over the kingdom in peace and happiness all their lives; and the good
fairy came to see them, and restored the faithful Falada to life
again.



\chapter[THE ADVENTURES OF CHANTICLEER AND\ldots]{THE ADVENTURES OF CHANTICLEER AND PARTLET}


\section[HOW THEY WENT TO THE MOUNTAINS TO\ldots]{HOW THEY WENT TO THE MOUNTAINS TO EAT NUTS}

'The nuts are quite ripe now,' said Chanticleer to his wife Partlet,
'suppose we go together to the mountains, and eat as many as we can,
before the squirrel takes them all away.' 'With all my heart,' said
Partlet, 'let us go and make a holiday of it together.'

So they went to the mountains; and as it was a lovely day, they stayed
there till the evening. Now, whether it was that they had eaten so
many nuts that they could not walk, or whether they were lazy and
would not, I do not know: however, they took it into their heads that
it did not become them to go home on foot. So Chanticleer began to
build a little carriage of nutshells: and when it was finished,
Partlet jumped into it and sat down, and bid Chanticleer harness
himself to it and draw her home. 'That's a good joke!' said
Chanticleer; 'no, that will never do; I had rather by half walk home;
I'll sit on the box and be coachman, if you like, but I'll not draw.'
While this was passing, a duck came quacking up and cried out, 'You
thieving vagabonds, what business have you in my grounds? I'll give it
you well for your insolence!' and upon that she fell upon Chanticleer
most lustily. But Chanticleer was no coward, and returned the duck's
blows with his sharp spurs so fiercely that she soon began to cry out
for mercy; which was only granted her upon condition that she would
draw the carriage home for them. This she agreed to do; and
Chanticleer got upon the box, and drove, crying, 'Now, duck, get on as
fast as you can.' And away they went at a pretty good pace.

After they had travelled along a little way, they met a needle and a
pin walking together along the road: and the needle cried out, 'Stop,
stop!' and said it was so dark that they could hardly find their way,
and such dirty walking they could not get on at all: he told them that
he and his friend, the pin, had been at a public-house a few miles
off, and had sat drinking till they had forgotten how late it was; he
begged therefore that the travellers would be so kind as to give them
a lift in their carriage. Chanticleer observing that they were but
thin fellows, and not likely to take up much room, told them they
might ride, but made them promise not to dirty the wheels of the
carriage in getting in, nor to tread on Partlet's toes.

Late at night they arrived at an inn; and as it was bad travelling in
the dark, and the duck seemed much tired, and waddled about a good
deal from one side to the other, they made up their minds to fix their
quarters there: but the landlord at first was unwilling, and said his
house was full, thinking they might not be very respectable company:
however, they spoke civilly to him, and gave him the egg which Partlet
had laid by the way, and said they would give him the duck, who was in
the habit of laying one every day: so at last he let them come in, and
they bespoke a handsome supper, and spent the evening very jollily.

Early in the morning, before it was quite light, and when nobody was
stirring in the inn, Chanticleer awakened his wife, and, fetching the
egg, they pecked a hole in it, ate it up, and threw the shells into
the fireplace: they then went to the pin and needle, who were fast
asleep, and seizing them by the heads, stuck one into the landlord's
easy chair and the other into his handkerchief; and, having done this,
they crept away as softly as possible. However, the duck, who slept in
the open air in the yard, heard them coming, and jumping into the
brook which ran close by the inn, soon swam out of their reach.

An hour or two afterwards the landlord got up, and took his
handkerchief to wipe his face, but the pin ran into him and pricked
him: then he walked into the kitchen to light his pipe at the fire,
but when he stirred it up the eggshells flew into his eyes, and almost
blinded him. 'Bless me!' said he, 'all the world seems to have a
design against my head this morning': and so saying, he threw himself
sulkily into his easy chair; but, oh dear! the needle ran into him;
and this time the pain was not in his head. He now flew into a very
great passion, and, suspecting the company who had come in the night
before, he went to look after them, but they were all off; so he swore
that he never again would take in such a troop of vagabonds, who ate a
great deal, paid no reckoning, and gave him nothing for his trouble
but their apish tricks.


\section[HOW CHANTICLEER AND PARTLET WENT TO\ldots]{HOW CHANTICLEER AND PARTLET WENT TO VIST MR KORBES}

Another day, Chanticleer and Partlet wished to ride out together; so
Chanticleer built a handsome carriage with four red wheels, and
harnessed six mice to it; and then he and Partlet got into the
carriage, and away they drove. Soon afterwards a cat met them, and
said, 'Where are you going?' And Chanticleer replied,

\begin{verse}
 'All on our way\\
  A visit to pay\\
  To Mr Korbes, the fox, today.'
\end{verse}

Then the cat said, 'Take me with you,' Chanticleer said, 'With all my
heart: get up behind, and be sure you do not fall off.'

\begin{verse}
 'Take care of this handsome coach of mine,\\
  Nor dirty my pretty red wheels so fine!\\
  Now, mice, be ready,\\
  And, wheels, run steady!\\
  For we are going a visit to pay\\
  To Mr Korbes, the fox, today.'
\end{verse}

Soon after came up a millstone, an egg, a duck, and a pin; and
Chanticleer gave them all leave to get into the carriage and go with
them.

When they arrived at Mr Korbes's house, he was not at home; so the
mice drew the carriage into the coach-house, Chanticleer and Partlet
flew upon a beam, the cat sat down in the fireplace, the duck got into
the washing cistern, the pin stuck himself into the bed pillow, the
millstone laid himself over the house door, and the egg rolled himself
up in the towel.

When Mr Korbes came home, he went to the fireplace to make a fire; but
the cat threw all the ashes in his eyes: so he ran to the kitchen to
wash himself; but there the duck splashed all the water in his face;
and when he tried to wipe himself, the egg broke to pieces in the
towel all over his face and eyes. Then he was very angry, and went
without his supper to bed; but when he laid his head on the pillow,
the pin ran into his cheek: at this he became quite furious, and,
jumping up, would have run out of the house; but when he came to the
door, the millstone fell down on his head, and killed him on the spot.


\section[HOW PARTLET DIED AND WAS BURIED, AND\ldots]{HOW PARTLET DIED AND WAS BUR\-IED, AND HOW CHANTICLEER DIED OF GRIEF}

Another day Chanticleer and Partlet agreed to go again to the
mountains to eat nuts; and it was settled that all the nuts which they
found should be shared equally between them. Now Partlet found a very
large nut; but she said nothing about it to Chanticleer, and kept it
all to herself: however, it was so big that she could not swallow it,
and it stuck in her throat. Then she was in a great fright, and cried
out to Chanticleer, 'Pray run as fast as you can, and fetch me some
water, or I shall be choked.' Chanticleer ran as fast as he could to
the river, and said, 'River, give me some water, for Partlet lies in
the mountain, and will be choked by a great nut.' The river said, 'Run
first to the bride, and ask her for a silken cord to draw up the
water.' Chanticleer ran to the bride, and said, 'Bride, you must give
me a silken cord, for then the river will give me water, and the water
I will carry to Partlet, who lies on the mountain, and will be choked
by a great nut.' But the bride said, 'Run first, and bring me my
garland that is hanging on a willow in the garden.' Then Chanticleer
ran to the garden, and took the garland from the bough where it hung,
and brought it to the bride; and then the bride gave him the silken
cord, and he took the silken cord to the river, and the river gave him
water, and he carried the water to Partlet; but in the meantime she
was choked by the great nut, and lay quite dead, and never moved any
more.

Then Chanticleer was very sorry, and cried bitterly; and all the
beasts came and wept with him over poor Partlet. And six mice built a
little hearse to carry her to her grave; and when it was ready they
harnessed themselves before it, and Chanticleer drove them. On the way
they met the fox. 'Where are you going, Chanticleer?' said he. 'To
bury my Partlet,' said the other. 'May I go with you?' said the fox.
'Yes; but you must get up behind, or my horses will not be able to
draw you.' Then the fox got up behind; and presently the wolf, the
bear, the goat, and all the beasts of the wood, came and climbed upon
the hearse.

So on they went till they came to a rapid stream. 'How shall we get
over?' said Chanticleer. Then said a straw, 'I will lay myself across,
and you may pass over upon me.' But as the mice were going over, the
straw slipped away and fell into the water, and the six mice all fell
in and were drowned. What was to be done? Then a large log of wood
came and said, 'I am big enough; I will lay myself across the stream,
and you shall pass over upon me.' So he laid himself down; but they
managed so clumsily, that the log of wood fell in and was carried away
by the stream. Then a stone, who saw what had happened, came up and
kindly offered to help poor Chanticleer by laying himself across the
stream; and this time he got safely to the other side with the hearse,
and managed to get Partlet out of it; but the fox and the other
mourners, who were sitting behind, were too heavy, and fell back into
the water and were all carried away by the stream and drowned.

Thus Chanticleer was left alone with his dead Partlet; and having dug
a grave for her, he laid her in it, and made a little hillock over
her. Then he sat down by the grave, and wept and mourned, till at last
he died too; and so all were dead.



\chapter{RAPUNZEL}

There were once a man and a woman who had long in vain wished for a
child. At length the woman hoped that God was about to grant her
desire. These people had a little window at the back of their house
from which a splendid garden could be seen, which was full of the most
beautiful flowers and herbs. It was, however, surrounded by a high
wall, and no one dared to go into it because it belonged to an
enchantress, who had great power and was dreaded by all the world. One
day the woman was standing by this window and looking down into the
garden, when she saw a bed which was planted with the most beautiful
rampion (rapunzel), and it looked so fresh and green that she longed
for it, she quite pined away, and began to look pale and miserable.
Then her husband was alarmed, and asked: 'What ails you, dear wife?'
'Ah,' she replied, 'if I can't eat some of the rampion, which is in
the garden behind our house, I shall die.' The man, who loved her,
thought: 'Sooner than let your wife die, bring her some of the rampion
yourself, let it cost what it will.' At twilight, he clambered down
over the wall into the garden of the enchantress, hastily clutched a
handful of rampion, and took it to his wife. She at once made herself
a salad of it, and ate it greedily. It tasted so good to her--so very
good, that the next day she longed for it three times as much as
before. If he was to have any rest, her husband must once more descend
into the garden. In the gloom of evening therefore, he let himself
down again; but when he had clambered down the wall he was terribly
afraid, for he saw the enchantress standing before him. 'How can you
dare,' said she with angry look, 'descend into my garden and steal my
rampion like a thief? You shall suffer for it!' 'Ah,' answered he,
'let mercy take the place of justice, I only made up my mind to do it
out of necessity. My wife saw your rampion from the window, and felt
such a longing for it that she would have died if she had not got some
to eat.' Then the enchantress allowed her anger to be softened, and
said to him: 'If the case be as you say, I will allow you to take away
with you as much rampion as you will, only I make one condition, you
must give me the child which your wife will bring into the world; it
shall be well treated, and I will care for it like a mother.' The man
in his terror consented to everything, and when the woman was brought
to bed, the enchantress appeared at once, gave the child the name of
Rapunzel, and took it away with her.

Rapunzel grew into the most beautiful child under the sun. When she
was twelve years old, the enchantress shut her into a tower, which lay
in a forest, and had neither stairs nor door, but quite at the top was
a little window. When the enchantress wanted to go in, she placed
herself beneath it and cried:

\begin{verse}
 'Rapunzel, Rapunzel,\\
  Let down your hair to me.'
\end{verse}

Rapunzel had magnificent long hair, fine as spun gold, and when she
heard the voice of the enchantress she unfastened her braided tresses,
wound them round one of the hooks of the window above, and then the
hair fell twenty ells down, and the enchantress climbed up by it.

After a year or two, it came to pass that the king's son rode through
the forest and passed by the tower. Then he heard a song, which was so
charming that he stood still and listened. This was Rapunzel, who in
her solitude passed her time in letting her sweet voice resound. The
king's son wanted to climb up to her, and looked for the door of the
tower, but none was to be found. He rode home, but the singing had so
deeply touched his heart, that every day he went out into the forest
and listened to it. Once when he was thus standing behind a tree, he
saw that an enchantress came there, and he heard how she cried:

\begin{verse}
 'Rapunzel, Rapunzel,\\
  Let down your hair to me.'
\end{verse}

Then Rapunzel let down the braids of her hair, and the enchantress
climbed up to her. 'If that is the ladder by which one mounts, I too
will try my fortune,' said he, and the next day when it began to grow
dark, he went to the tower and cried:

\begin{verse}
 'Rapunzel, Rapunzel,\\
  Let down your hair to me.'
\end{verse}

Immediately the hair fell down and the king's son climbed up.

At first Rapunzel was terribly frightened when a man, such as her eyes
had never yet beheld, came to her; but the king's son began to talk to
her quite like a friend, and told her that his heart had been so
stirred that it had let him have no rest, and he had been forced to
see her. Then Rapunzel lost her fear, and when he asked her if she
would take him for her husband, and she saw that he was young and
handsome, she thought: 'He will love me more than old Dame Gothel
does'; and she said yes, and laid her hand in his. She said: 'I will
willingly go away with you, but I do not know how to get down. Bring
with you a skein of silk every time that you come, and I will weave a
ladder with it, and when that is ready I will descend, and you will
take me on your horse.' They agreed that until that time he should
come to her every evening, for the old woman came by day. The
enchantress remarked nothing of this, until once Rapunzel said to her:
'Tell me, Dame Gothel, how it happens that you are so much heavier for
me to draw up than the young king's son--he is with me in a moment.'
'Ah! you wicked child,' cried the enchantress. 'What do I hear you
say! I thought I had separated you from all the world, and yet you
have deceived me!' In her anger she clutched Rapunzel's beautiful
tresses, wrapped them twice round her left hand, seized a pair of
scissors with the right, and snip, snap, they were cut off, and the
lovely braids lay on the ground. And she was so pitiless that she took
poor Rapunzel into a desert where she had to live in great grief and
misery.

On the same day that she cast out Rapunzel, however, the enchantress
fastened the braids of hair, which she had cut off, to the hook of the
window, and when the king's son came and cried:

\begin{verse}
 'Rapunzel, Rapunzel,\\
  Let down your hair to me.'
\end{verse}

she let the hair down. The king's son ascended, but instead of finding
his dearest Rapunzel, he found the enchantress, who gazed at him with
wicked and venomous looks. 'Aha!' she cried mockingly, 'you would
fetch your dearest, but the beautiful bird sits no longer singing in
the nest; the cat has got it, and will scratch out your eyes as well.
Rapunzel is lost to you; you will never see her again.' The king's son
was beside himself with pain, and in his despair he leapt down from
the tower. He escaped with his life, but the thorns into which he fell
pierced his eyes. Then he wandered quite blind about the forest, ate
nothing but roots and berries, and did naught but lament and weep over
the loss of his dearest wife. Thus he roamed about in misery for some
years, and at length came to the desert where Rapunzel, with the twins
to which she had given birth, a boy and a girl, lived in wretchedness.
He heard a voice, and it seemed so familiar to him that he went
towards it, and when he approached, Rapunzel knew him and fell on his
neck and wept. Two of her tears wetted his eyes and they grew clear
again, and he could see with them as before. He led her to his kingdom
where he was joyfully received, and they lived for a long time
afterwards, happy and contented.



\chapter{FUNDEVOGEL}

There was once a forester who went into the forest to hunt, and as he
entered it he heard a sound of screaming as if a little child were
there. He followed the sound, and at last came to a high tree, and at
the top of this a little child was sitting, for the mother had fallen
asleep under the tree with the child, and a bird of prey had seen it
in her arms, had flown down, snatched it away, and set it on the high
tree.

The forester climbed up, brought the child down, and thought to
himself: 'You will take him home with you, and bring him up with your
Lina.' He took it home, therefore, and the two children grew up
together. And the one, which he had found on a tree was called
Fundevogel, because a bird had carried it away. Fundevogel and Lina
loved each other so dearly that when they did not see each other they
were sad.

Now the forester had an old cook, who one evening took two pails and
began to fetch water, and did not go once only, but many times, out to
the spring. Lina saw this and said, 'Listen, old Sanna, why are you
fetching so much water?' 'If you will never repeat it to anyone, I
will tell you why.' So Lina said, no, she would never repeat it to
anyone, and then the cook said: 'Early tomorrow morning, when the
forester is out hunting, I will heat the water, and when it is boiling
in the kettle, I will throw in Fundevogel, and will boil him in it.'

Early next morning the forester got up and went out hunting, and when
he was gone the children were still in bed. Then Lina said to
Fundevogel: 'If you will never leave me, I too will never leave you.'
Fundevogel said: 'Neither now, nor ever will I leave you.' Then said
Lina: 'Then will I tell you. Last night, old Sanna carried so many
buckets of water into the house that I asked her why she was doing
that, and she said that if I would promise not to tell anyone, and she
said that early tomorrow morning when father was out hunting, she
would set the kettle full of water, throw you into it and boil you;
but we will get up quickly, dress ourselves, and go away together.'

The two children therefore got up, dressed themselves quickly, and
went away. When the water in the kettle was boiling, the cook went
into the bedroom to fetch Fundevogel and throw him into it. But when
she came in, and went to the beds, both the children were gone. Then
she was terribly alarmed, and she said to herself: 'What shall I say
now when the forester comes home and sees that the children are gone?
They must be followed instantly to get them back again.'

Then the cook sent three servants after them, who were to run and
overtake the children. The children, however, were sitting outside the
forest, and when they saw from afar the three servants running, Lina
said to Fundevogel: 'Never leave me, and I will never leave you.'
Fundevogel said: 'Neither now, nor ever.' Then said Lina: 'Do you
become a rose-tree, and I the rose upon it.' When the three servants
came to the forest, nothing was there but a rose-tree and one rose on
it, but the children were nowhere. Then said they: 'There is nothing
to be done here,' and they went home and told the cook that they had
seen nothing in the forest but a little rose-bush with one rose on it.
Then the old cook scolded and said: 'You simpletons, you should have
cut the rose-bush in two, and have broken off the rose and brought it
home with you; go, and do it at once.' They had therefore to go out
and look for the second time. The children, however, saw them coming
from a distance. Then Lina said: 'Fundevogel, never leave me, and I
will never leave you.' Fundevogel said: 'Neither now; nor ever.' Said
Lina: 'Then do you become a church, and I'll be the chandelier in it.'
So when the three servants came, nothing was there but a church, with
a chandelier in it. They said therefore to each other: 'What can we do
here, let us go home.' When they got home, the cook asked if they had
not found them; so they said no, they had found nothing but a church,
and there was a chandelier in it. And the cook scolded them and said:
'You fools! why did you not pull the church to pieces, and bring the
chandelier home with you?' And now the old cook herself got on her
legs, and went with the three servants in pursuit of the children. The
children, however, saw from afar that the three servants were coming,
and the cook waddling after them. Then said Lina: 'Fundevogel, never
leave me, and I will never leave you.' Then said Fundevogel: 'Neither
now, nor ever.' Said Lina: 'Be a fishpond, and I will be the duck upon
it.' The cook, however, came up to them, and when she saw the pond she
lay down by it, and was about to drink it up. But the duck swam
quickly to her, seized her head in its beak and drew her into the
water, and there the old witch had to drown. Then the children went
home together, and were heartily delighted, and if they have not died,
they are living still.



\chapter{THE VALIANT LITTLE TAILOR}

One summer's morning a little tailor was sitting on his table by the
window; he was in good spirits, and sewed with all his might. Then
came a peasant woman down the street crying: 'Good jams, cheap! Good
jams, cheap!' This rang pleasantly in the tailor's ears; he stretched
his delicate head out of the window, and called: 'Come up here, dear
woman; here you will get rid of your goods.' The woman came up the
three steps to the tailor with her heavy basket, and he made her
unpack all the pots for him. He inspected each one, lifted it up, put
his nose to it, and at length said: 'The jam seems to me to be good,
so weigh me out four ounces, dear woman, and if it is a quarter of a
pound that is of no consequence.' The woman who had hoped to find a
good sale, gave him what he desired, but went away quite angry and
grumbling. 'Now, this jam shall be blessed by God,' cried the little
tailor, 'and give me health and strength'; so he brought the bread out
of the cupboard, cut himself a piece right across the loaf and spread
the jam over it. 'This won't taste bitter,' said he, 'but I will just
finish the jacket before I take a bite.' He laid the bread near him,
sewed on, and in his joy, made bigger and bigger stitches. In the
meantime the smell of the sweet jam rose to where the flies were
sitting in great numbers, and they were attracted and descended on it
in hosts. 'Hi! who invited you?' said the little tailor, and drove the
unbidden guests away. The flies, however, who understood no German,
would not be turned away, but came back again in ever-increasing
companies. The little tailor at last lost all patience, and drew a
piece of cloth from the hole under his work-table, and saying: 'Wait,
and I will give it to you,' struck it mercilessly on them. When he
drew it away and counted, there lay before him no fewer than seven,
dead and with legs stretched out. 'Are you a fellow of that sort?'
said he, and could not help admiring his own bravery. 'The whole town
shall know of this!' And the little tailor hastened to cut himself a
girdle, stitched it, and embroidered on it in large letters: 'Seven at
one stroke!' 'What, the town!' he continued, 'the whole world shall
hear of it!' and his heart wagged with joy like a lamb's tail. The
tailor put on the girdle, and resolved to go forth into the world,
because he thought his workshop was too small for his valour. Before
he went away, he sought about in the house to see if there was
anything which he could take with him; however, he found nothing but
an old cheese, and that he put in his pocket. In front of the door he
observed a bird which had caught itself in the thicket. It had to go
into his pocket with the cheese. Now he took to the road boldly, and
as he was light and nimble, he felt no fatigue. The road led him up a
mountain, and when he had reached the highest point of it, there sat a
powerful giant looking peacefully about him. The little tailor went
bravely up, spoke to him, and said: 'Good day, comrade, so you are
sitting there overlooking the wide-spread world! I am just on my way
thither, and want to try my luck. Have you any inclination to go with
me?' The giant looked contemptuously at the tailor, and said: 'You
ragamuffin! You miserable creature!'

'Oh, indeed?' answered the little tailor, and unbuttoned his coat, and
showed the giant the girdle, 'there may you read what kind of a man I
am!' The giant read: 'Seven at one stroke,' and thought that they had
been men whom the tailor had killed, and began to feel a little
respect for the tiny fellow. Nevertheless, he wished to try him first,
and took a stone in his hand and squeezed it together so that water
dropped out of it. 'Do that likewise,' said the giant, 'if you have
strength.' 'Is that all?' said the tailor, 'that is child's play with
us!' and put his hand into his pocket, brought out the soft cheese,
and pressed it until the liquid ran out of it. 'Faith,' said he, 'that
was a little better, wasn't it?' The giant did not know what to say,
and could not believe it of the little man. Then the giant picked up a
stone and threw it so high that the eye could scarcely follow it.
'Now, little mite of a man, do that likewise,' 'Well thrown,' said the
tailor, 'but after all the stone came down to earth again; I will
throw you one which shall never come back at all,' and he put his hand
into his pocket, took out the bird, and threw it into the air. The
bird, delighted with its liberty, rose, flew away and did not come
back. 'How does that shot please you, comrade?' asked the tailor. 'You
can certainly throw,' said the giant, 'but now we will see if you are
able to carry anything properly.' He took the little tailor to a
mighty oak tree which lay there felled on the ground, and said: 'If
you are strong enough, help me to carry the tree out of the forest.'
'Readily,' answered the little man; 'take you the trunk on your
shoulders, and I will raise up the branches and twigs; after all, they
are the heaviest.' The giant took the trunk on his shoulder, but the
tailor seated himself on a branch, and the giant, who could not look
round, had to carry away the whole tree, and the little tailor into
the bargain: he behind, was quite merry and happy, and whistled the
song: 'Three tailors rode forth from the gate,' as if carrying the
tree were child's play. The giant, after he had dragged the heavy
burden part of the way, could go no further, and cried: 'Hark you, I
shall have to let the tree fall!' The tailor sprang nimbly down,
seized the tree with both arms as if he had been carrying it, and said
to the giant: 'You are such a great fellow, and yet cannot even carry
the tree!'

They went on together, and as they passed a cherry-tree, the giant
laid hold of the top of the tree where the ripest fruit was hanging,
bent it down, gave it into the tailor's hand, and bade him eat. But
the little tailor was much too weak to hold the tree, and when the
giant let it go, it sprang back again, and the tailor was tossed into
the air with it. When he had fallen down again without injury, the
giant said: 'What is this? Have you not strength enough to hold the
weak twig?' 'There is no lack of strength,' answered the little
tailor. 'Do you think that could be anything to a man who has struck
down seven at one blow? I leapt over the tree because the huntsmen are
shooting down there in the thicket. Jump as I did, if you can do it.'
The giant made the attempt but he could not get over the tree, and
remained hanging in the branches, so that in this also the tailor kept
the upper hand.

The giant said: 'If you are such a valiant fellow, come with me into
our cavern and spend the night with us.' The little tailor was
willing, and followed him. When they went into the cave, other giants
were sitting there by the fire, and each of them had a roasted sheep
in his hand and was eating it. The little tailor looked round and
thought: 'It is much more spacious here than in my workshop.' The
giant showed him a bed, and said he was to lie down in it and sleep.
The bed, however, was too big for the little tailor; he did not lie
down in it, but crept into a corner. When it was midnight, and the
giant thought that the little tailor was lying in a sound sleep, he
got up, took a great iron bar, cut through the bed with one blow, and
thought he had finished off the grasshopper for good. With the
earliest dawn the giants went into the forest, and had quite forgotten
the little tailor, when all at once he walked up to them quite merrily
and boldly. The giants were terrified, they were afraid that he would
strike them all dead, and ran away in a great hurry.

The little tailor went onwards, always following his own pointed nose.
After he had walked for a long time, he came to the courtyard of a
royal palace, and as he felt weary, he lay down on the grass and fell
asleep. Whilst he lay there, the people came and inspected him on all
sides, and read on his girdle: 'Seven at one stroke.' 'Ah!' said they,
'what does the great warrior want here in the midst of peace? He must
be a mighty lord.' They went and announced him to the king, and gave
it as their opinion that if war should break out, this would be a
weighty and useful man who ought on no account to be allowed to
depart. The counsel pleased the king, and he sent one of his courtiers
to the little tailor to offer him military service when he awoke. The
ambassador remained standing by the sleeper, waited until he stretched
his limbs and opened his eyes, and then conveyed to him this proposal.
'For this very reason have I come here,' the tailor replied, 'I am
ready to enter the king's service.' He was therefore honourably
received, and a special dwelling was assigned him.

The soldiers, however, were set against the little tailor, and wished
him a thousand miles away. 'What is to be the end of this?' they said
among themselves. 'If we quarrel with him, and he strikes about him,
seven of us will fall at every blow; not one of us can stand against
him.' They came therefore to a decision, betook themselves in a body
to the king, and begged for their dismissal. 'We are not prepared,'
said they, 'to stay with a man who kills seven at one stroke.' The
king was sorry that for the sake of one he should lose all his
faithful servants, wished that he had never set eyes on the tailor,
and would willingly have been rid of him again. But he did not venture
to give him his dismissal, for he dreaded lest he should strike him
and all his people dead, and place himself on the royal throne. He
thought about it for a long time, and at last found good counsel. He
sent to the little tailor and caused him to be informed that as he was
a great warrior, he had one request to make to him. In a forest of his
country lived two giants, who caused great mischief with their
robbing, murdering, ravaging, and burning, and no one could approach
them without putting himself in danger of death. If the tailor
conquered and killed these two giants, he would give him his only
daughter to wife, and half of his kingdom as a dowry, likewise one
hundred horsemen should go with him to assist him. 'That would indeed
be a fine thing for a man like me!' thought the little tailor. 'One is
not offered a beautiful princess and half a kingdom every day of one's
life!' 'Oh, yes,' he replied, 'I will soon subdue the giants, and do
not require the help of the hundred horsemen to do it; he who can hit
seven with one blow has no need to be afraid of two.'

The little tailor went forth, and the hundred horsemen followed him.
When he came to the outskirts of the forest, he said to his followers:
'Just stay waiting here, I alone will soon finish off the giants.'
Then he bounded into the forest and looked about right and left. After
a while he perceived both giants. They lay sleeping under a tree, and
snored so that the branches waved up and down. The little tailor, not
idle, gathered two pocketsful of stones, and with these climbed up the
tree. When he was halfway up, he slipped down by a branch, until he
sat just above the sleepers, and then let one stone after another fall
on the breast of one of the giants. For a long time the giant felt
nothing, but at last he awoke, pushed his comrade, and said: 'Why are
you knocking me?' 'You must be dreaming,' said the other, 'I am not
knocking you.' They laid themselves down to sleep again, and then the
tailor threw a stone down on the second. 'What is the meaning of
this?' cried the other 'Why are you pelting me?' 'I am not pelting
you,' answered the first, growling. They disputed about it for a time,
but as they were weary they let the matter rest, and their eyes closed
once more. The little tailor began his game again, picked out the
biggest stone, and threw it with all his might on the breast of the
first giant. 'That is too bad!' cried he, and sprang up like a madman,
and pushed his companion against the tree until it shook. The other
paid him back in the same coin, and they got into such a rage that
they tore up trees and belaboured each other so long, that at last
they both fell down dead on the ground at the same time. Then the
little tailor leapt down. 'It is a lucky thing,' said he, 'that they
did not tear up the tree on which I was sitting, or I should have had
to sprint on to another like a squirrel; but we tailors are nimble.'
He drew out his sword and gave each of them a couple of thrusts in the
breast, and then went out to the horsemen and said: 'The work is done;
I have finished both of them off, but it was hard work! They tore up
trees in their sore need, and defended themselves with them, but all
that is to no purpose when a man like myself comes, who can kill seven
at one blow.' 'But are you not wounded?' asked the horsemen. 'You need
not concern yourself about that,' answered the tailor, 'they have not
bent one hair of mine.' The horsemen would not believe him, and rode
into the forest; there they found the giants swimming in their blood,
and all round about lay the torn-up trees.

The little tailor demanded of the king the promised reward; he,
however, repented of his promise, and again bethought himself how he
could get rid of the hero. 'Before you receive my daughter, and the
half of my kingdom,' said he to him, 'you must perform one more heroic
deed. In the forest roams a unicorn which does great harm, and you
must catch it first.' 'I fear one unicorn still less than two giants.
Seven at one blow, is my kind of affair.' He took a rope and an axe
with him, went forth into the forest, and again bade those who were
sent with him to wait outside. He had not long to seek. The unicorn
soon came towards him, and rushed directly on the tailor, as if it
would gore him with its horn without more ado. 'Softly, softly; it
can't be done as quickly as that,' said he, and stood still and waited
until the animal was quite close, and then sprang nimbly behind the
tree. The unicorn ran against the tree with all its strength, and
stuck its horn so fast in the trunk that it had not the strength
enough to draw it out again, and thus it was caught. 'Now, I have got
the bird,' said the tailor, and came out from behind the tree and put
the rope round its neck, and then with his axe he hewed the horn out
of the tree, and when all was ready he led the beast away and took it
to the king.

The king still would not give him the promised reward, and made a
third demand. Before the wedding the tailor was to catch him a wild
boar that made great havoc in the forest, and the huntsmen should give
him their help. 'Willingly,' said the tailor, 'that is child's play!'
He did not take the huntsmen with him into the forest, and they were
well pleased that he did not, for the wild boar had several times
received them in such a manner that they had no inclination to lie in
wait for him. When the boar perceived the tailor, it ran on him with
foaming mouth and whetted tusks, and was about to throw him to the
ground, but the hero fled and sprang into a chapel which was near and
up to the window at once, and in one bound out again. The boar ran
after him, but the tailor ran round outside and shut the door behind
it, and then the raging beast, which was much too heavy and awkward to
leap out of the window, was caught. The little tailor called the
huntsmen thither that they might see the prisoner with their own eyes.
The hero, however, went to the king, who was now, whether he liked it
or not, obliged to keep his promise, and gave his daughter and the
half of his kingdom. Had he known that it was no warlike hero, but a
little tailor who was standing before him, it would have gone to his
heart still more than it did. The wedding was held with great
magnificence and small joy, and out of a tailor a king was made.

After some time the young queen heard her husband say in his dreams at
night: 'Boy, make me the doublet, and patch the pantaloons, or else I
will rap the yard-measure over your ears.' Then she discovered in what
state of life the young lord had been born, and next morning
complained of her wrongs to her father, and begged him to help her to
get rid of her husband, who was nothing else but a tailor. The king
comforted her and said: 'Leave your bedroom door open this night, and
my servants shall stand outside, and when he has fallen asleep shall
go in, bind him, and take him on board a ship which shall carry him
into the wide world.' The woman was satisfied with this; but the
king's armour-bearer, who had heard all, was friendly with the young
lord, and informed him of the whole plot. 'I'll put a screw into that
business,' said the little tailor. At night he went to bed with his
wife at the usual time, and when she thought that he had fallen
asleep, she got up, opened the door, and then lay down again. The
little tailor, who was only pretending to be asleep, began to cry out
in a clear voice: 'Boy, make me the doublet and patch me the
pantaloons, or I will rap the yard-measure over your ears. I smote
seven at one blow. I killed two giants, I brought away one unicorn,
and caught a wild boar, and am I to fear those who are standing
outside the room.' When these men heard the tailor speaking thus, they
were overcome by a great dread, and ran as if the wild huntsman were
behind them, and none of them would venture anything further against
him. So the little tailor was and remained a king to the end of his
life.



\chapter{HANSEL AND GRETEL}

Hard by a great forest dwelt a poor wood-cutter with his wife and his
two children. The boy was called Hansel and the girl Gretel. He had
little to bite and to break, and once when great dearth fell on the
land, he could no longer procure even daily bread. Now when he thought
over this by night in his bed, and tossed about in his anxiety, he
groaned and said to his wife: 'What is to become of us? How are we to
feed our poor children, when we no longer have anything even for
ourselves?' 'I'll tell you what, husband,' answered the woman, 'early
tomorrow morning we will take the children out into the forest to
where it is the thickest; there we will light a fire for them, and
give each of them one more piece of bread, and then we will go to our
work and leave them alone. They will not find the way home again, and
we shall be rid of them.' 'No, wife,' said the man, 'I will not do
that; how can I bear to leave my children alone in the forest?--the
wild animals would soon come and tear them to pieces.' 'O, you fool!'
said she, 'then we must all four die of hunger, you may as well plane
the planks for our coffins,' and she left him no peace until he
consented. 'But I feel very sorry for the poor children, all the
same,' said the man.

The two children had also not been able to sleep for hunger, and had
heard what their stepmother had said to their father. Gretel wept
bitter tears, and said to Hansel: 'Now all is over with us.' 'Be
quiet, Gretel,' said Hansel, 'do not distress yourself, I will soon
find a way to help us.' And when the old folks had fallen asleep, he
got up, put on his little coat, opened the door below, and crept
outside. The moon shone brightly, and the white pebbles which lay in
front of the house glittered like real silver pennies. Hansel stooped
and stuffed the little pocket of his coat with as many as he could get
in. Then he went back and said to Gretel: 'Be comforted, dear little
sister, and sleep in peace, God will not forsake us,' and he lay down
again in his bed. When day dawned, but before the sun had risen, the
woman came and awoke the two children, saying: 'Get up, you sluggards!
we are going into the forest to fetch wood.' She gave each a little
piece of bread, and said: 'There is something for your dinner, but do
not eat it up before then, for you will get nothing else.' Gretel took
the bread under her apron, as Hansel had the pebbles in his pocket.
Then they all set out together on the way to the forest. When they had
walked a short time, Hansel stood still and peeped back at the house,
and did so again and again. His father said: 'Hansel, what are you
looking at there and staying behind for? Pay attention, and do not
forget how to use your legs.' 'Ah, father,' said Hansel, 'I am looking
at my little white cat, which is sitting up on the roof, and wants to
say goodbye to me.' The wife said: 'Fool, that is not your little cat,
that is the morning sun which is shining on the chimneys.' Hansel,
however, had not been looking back at the cat, but had been constantly
throwing one of the white pebble-stones out of his pocket on the road.

When they had reached the middle of the forest, the father said: 'Now,
children, pile up some wood, and I will light a fire that you may not
be cold.' Hansel and Gretel gathered brushwood together, as high as a
little hill. The brushwood was lighted, and when the flames were
burning very high, the woman said: 'Now, children, lay yourselves down
by the fire and rest, we will go into the forest and cut some wood.
When we have done, we will come back and fetch you away.'

Hansel and Gretel sat by the fire, and when noon came, each ate a
little piece of bread, and as they heard the strokes of the wood-axe
they believed that their father was near. It was not the axe, however,
but a branch which he had fastened to a withered tree which the wind
was blowing backwards and forwards. And as they had been sitting such
a long time, their eyes closed with fatigue, and they fell fast
asleep. When at last they awoke, it was already dark night. Gretel
began to cry and said: 'How are we to get out of the forest now?' But
Hansel comforted her and said: 'Just wait a little, until the moon has
risen, and then we will soon find the way.' And when the full moon had
risen, Hansel took his little sister by the hand, and followed the
pebbles which shone like newly-coined silver pieces, and showed them
the way.

They walked the whole night long, and by break of day came once more
to their father's house. They knocked at the door, and when the woman
opened it and saw that it was Hansel and Gretel, she said: 'You
naughty children, why have you slept so long in the forest?--we
thought you were never coming back at all!' The father, however,
rejoiced, for it had cut him to the heart to leave them behind alone.

Not long afterwards, there was once more great dearth throughout the
land, and the children heard their mother saying at night to their
father: 'Everything is eaten again, we have one half loaf left, and
that is the end. The children must go, we will take them farther into
the wood, so that they will not find their way out again; there is no
other means of saving ourselves!' The man's heart was heavy, and he
thought: 'It would be better for you to share the last mouthful with
your children.' The woman, however, would listen to nothing that he
had to say, but scolded and reproached him. He who says A must say B,
likewise, and as he had yielded the first time, he had to do so a
second time also.

The children, however, were still awake and had heard the
conversation. When the old folks were asleep, Hansel again got up, and
wanted to go out and pick up pebbles as he had done before, but the
woman had locked the door, and Hansel could not get out. Nevertheless
he comforted his little sister, and said: 'Do not cry, Gretel, go to
sleep quietly, the good God will help us.'

Early in the morning came the woman, and took the children out of
their beds. Their piece of bread was given to them, but it was still
smaller than the time before. On the way into the forest Hansel
crumbled his in his pocket, and often stood still and threw a morsel
on the ground. 'Hansel, why do you stop and look round?' said the
father, 'go on.' 'I am looking back at my little pigeon which is
sitting on the roof, and wants to say goodbye to me,' answered Hansel.
'Fool!' said the woman, 'that is not your little pigeon, that is the
morning sun that is shining on the chimney.' Hansel, however little by
little, threw all the crumbs on the path.

The woman led the children still deeper into the forest, where they
had never in their lives been before. Then a great fire was again
made, and the mother said: 'Just sit there, you children, and when you
are tired you may sleep a little; we are going into the forest to cut
wood, and in the evening when we are done, we will come and fetch you
away.' When it was noon, Gretel shared her piece of bread with Hansel,
who had scattered his by the way. Then they fell asleep and evening
passed, but no one came to the poor children. They did not awake until
it was dark night, and Hansel comforted his little sister and said:
'Just wait, Gretel, until the moon rises, and then we shall see the
crumbs of bread which I have strewn about, they will show us our way
home again.' When the moon came they set out, but they found no
crumbs, for the many thousands of birds which fly about in the woods
and fields had picked them all up. Hansel said to Gretel: 'We shall
soon find the way,' but they did not find it. They walked the whole
night and all the next day too from morning till evening, but they did
not get out of the forest, and were very hungry, for they had nothing
to eat but two or three berries, which grew on the ground. And as they
were so weary that their legs would carry them no longer, they lay
down beneath a tree and fell asleep.

It was now three mornings since they had left their father's house.
They began to walk again, but they always came deeper into the forest,
and if help did not come soon, they must die of hunger and weariness.
When it was mid-day, they saw a beautiful snow-white bird sitting on a
bough, which sang so delightfully that they stood still and listened
to it. And when its song was over, it spread its wings and flew away
before them, and they followed it until they reached a little house,
on the roof of which it alighted; and when they approached the little
house they saw that it was built of bread and covered with cakes, but
that the windows were of clear sugar. 'We will set to work on that,'
said Hansel, 'and have a good meal. I will eat a bit of the roof, and
you Gretel, can eat some of the window, it will taste sweet.' Hansel
reached up above, and broke off a little of the roof to try how it
tasted, and Gretel leant against the window and nibbled at the panes.
Then a soft voice cried from the parlour:

\begin{verse}
 'Nibble, nibble, gnaw,\\
  Who is nibbling at my little house?'
\end{verse}

The children answered:

\begin{verse}
 'The wind, the wind,\\
  The heaven-born wind,'
\end{verse}

and went on eating without disturbing themselves. Hansel, who liked
the taste of the roof, tore down a great piece of it, and Gretel
pushed out the whole of one round window-pane, sat down, and enjoyed
herself with it. Suddenly the door opened, and a woman as old as the
hills, who supported herself on crutches, came creeping out. Hansel
and Gretel were so terribly frightened that they let fall what they
had in their hands. The old woman, however, nodded her head, and said:
'Oh, you dear children, who has brought you here? do come in, and stay
with me. No harm shall happen to you.' She took them both by the hand,
and led them into her little house. Then good food was set before
them, milk and pancakes, with sugar, apples, and nuts. Afterwards two
pretty little beds were covered with clean white linen, and Hansel and
Gretel lay down in them, and thought they were in heaven.

The old woman had only pretended to be so kind; she was in reality a
wicked witch, who lay in wait for children, and had only built the
little house of bread in order to entice them there. When a child fell
into her power, she killed it, cooked and ate it, and that was a feast
day with her. Witches have red eyes, and cannot see far, but they have
a keen scent like the beasts, and are aware when human beings draw
near. When Hansel and Gretel came into her neighbourhood, she laughed
with malice, and said mockingly: 'I have them, they shall not escape
me again!' Early in the morning before the children were awake, she
was already up, and when she saw both of them sleeping and looking so
pretty, with their plump and rosy cheeks she muttered to herself:
'That will be a dainty mouthful!' Then she seized Hansel with her
shrivelled hand, carried him into a little stable, and locked him in
behind a grated door. Scream as he might, it would not help him. Then
she went to Gretel, shook her till she awoke, and cried: 'Get up, lazy
thing, fetch some water, and cook something good for your brother, he
is in the stable outside, and is to be made fat. When he is fat, I
will eat him.' Gretel began to weep bitterly, but it was all in vain,
for she was forced to do what the wicked witch commanded.

And now the best food was cooked for poor Hansel, but Gretel got
nothing but crab-shells. Every morning the woman crept to the little
stable, and cried: 'Hansel, stretch out your finger that I may feel if
you will soon be fat.' Hansel, however, stretched out a little bone to
her, and the old woman, who had dim eyes, could not see it, and
thought it was Hansel's finger, and was astonished that there was no
way of fattening him. When four weeks had gone by, and Hansel still
remained thin, she was seized with impatience and would not wait any
longer. 'Now, then, Gretel,' she cried to the girl, 'stir yourself,
and bring some water. Let Hansel be fat or lean, tomorrow I will kill
him, and cook him.' Ah, how the poor little sister did lament when she
had to fetch the water, and how her tears did flow down her cheeks!
'Dear God, do help us,' she cried. 'If the wild beasts in the forest
had but devoured us, we should at any rate have died together.' 'Just
keep your noise to yourself,' said the old woman, 'it won't help you
at all.'

Early in the morning, Gretel had to go out and hang up the cauldron
with the water, and light the fire. 'We will bake first,' said the old
woman, 'I have already heated the oven, and kneaded the dough.' She
pushed poor Gretel out to the oven, from which flames of fire were
already darting. 'Creep in,' said the witch, 'and see if it is
properly heated, so that we can put the bread in.' And once Gretel was
inside, she intended to shut the oven and let her bake in it, and then
she would eat her, too. But Gretel saw what she had in mind, and said:
'I do not know how I am to do it; how do I get in?' 'Silly goose,'
said the old woman. 'The door is big enough; just look, I can get in
myself!' and she crept up and thrust her head into the oven. Then
Gretel gave her a push that drove her far into it, and shut the iron
door, and fastened the bolt. Oh! then she began to howl quite
horribly, but Gretel ran away and the godless witch was miserably
burnt to death.

Gretel, however, ran like lightning to Hansel, opened his little
stable, and cried: 'Hansel, we are saved! The old witch is dead!' Then
Hansel sprang like a bird from its cage when the door is opened. How
they did rejoice and embrace each other, and dance about and kiss each
other! And as they had no longer any need to fear her, they went into
the witch's house, and in every corner there stood chests full of
pearls and jewels. 'These are far better than pebbles!' said Hansel,
and thrust into his pockets whatever could be got in, and Gretel said:
'I, too, will take something home with me,' and filled her pinafore
full. 'But now we must be off,' said Hansel, 'that we may get out of
the witch's forest.'

When they had walked for two hours, they came to a great stretch of
water. 'We cannot cross,' said Hansel, 'I see no foot-plank, and no
bridge.' 'And there is also no ferry,' answered Gretel, 'but a white
duck is swimming there: if I ask her, she will help us over.' Then she
cried:

\begin{verse}
 'Little duck, little duck, dost thou see,\\
  Hansel and Gretel are waiting for thee?\\
  There's never a plank, or bridge in sight,\\
  Take us across on thy back so white.'
\end{verse}

The duck came to them, and Hansel seated himself on its back, and told
his sister to sit by him. 'No,' replied Gretel, 'that will be too
heavy for the little duck; she shall take us across, one after the
other.' The good little duck did so, and when they were once safely
across and had walked for a short time, the forest seemed to be more
and more familiar to them, and at length they saw from afar their
father's house. Then they began to run, rushed into the parlour, and
threw themselves round their father's neck. The man had not known one
happy hour since he had left the children in the forest; the woman,
however, was dead. Gretel emptied her pinafore until pearls and
precious stones ran about the room, and Hansel threw one handful after
another out of his pocket to add to them. Then all anxiety was at an
end, and they lived together in perfect happiness. My tale is done,
there runs a mouse; whosoever catches it, may make himself a big fur
cap out of it.



\chapter{THE MOUSE, THE BIRD, AND THE SAUSAGE}

Once upon a time, a mouse, a bird, and a sausage, entered into
partnership and set up house together. For a long time all went well;
they lived in great comfort, and prospered so far as to be able to add
considerably to their stores. The bird's duty was to fly daily into
the wood and bring in fuel; the mouse fetched the water, and the
sausage saw to the cooking.

When people are too well off they always begin to long for something
new. And so it came to pass, that the bird, while out one day, met a
fellow bird, to whom he boastfully expatiated on the excellence of his
household arrangements. But the other bird sneered at him for being a
poor simpleton, who did all the hard work, while the other two stayed
at home and had a good time of it. For, when the mouse had made the
fire and fetched in the water, she could retire into her little room
and rest until it was time to set the table. The sausage had only to
watch the pot to see that the food was properly cooked, and when it
was near dinner-time, he just threw himself into the broth, or rolled
in and out among the vegetables three or four times, and there they
were, buttered, and salted, and ready to be served. Then, when the
bird came home and had laid aside his burden, they sat down to table,
and when they had finished their meal, they could sleep their fill
till the following morning: and that was really a very delightful
life.

Influenced by those remarks, the bird next morning refused to bring in
the wood, telling the others that he had been their servant long
enough, and had been a fool into the bargain, and that it was now time
to make a change, and to try some other way of arranging the work. Beg
and pray as the mouse and the sausage might, it was of no use; the
bird remained master of the situation, and the venture had to be made.
They therefore drew lots, and it fell to the sausage to bring in the
wood, to the mouse to cook, and to the bird to fetch the water.

And now what happened? The sausage started in search of wood, the bird
made the fire, and the mouse put on the pot, and then these two waited
till the sausage returned with the fuel for the following day. But the
sausage remained so long away, that they became uneasy, and the bird
flew out to meet him. He had not flown far, however, when he came
across a dog who, having met the sausage, had regarded him as his
legitimate booty, and so seized and swallowed him. The bird complained
to the dog of this bare-faced robbery, but nothing he said was of any
avail, for the dog answered that he found false credentials on the
sausage, and that was the reason his life had been forfeited.

He picked up the wood, and flew sadly home, and told the mouse all he
had seen and heard. They were both very unhappy, but agreed to make
the best of things and to remain with one another.

So now the bird set the table, and the mouse looked after the food
and, wishing to prepare it in the same way as the sausage, by rolling
in and out among the vegetables to salt and butter them, she jumped
into the pot; but she stopped short long before she reached the
bottom, having already parted not only with her skin and hair, but
also with life.

Presently the bird came in and wanted to serve up the dinner, but he
could nowhere see the cook. In his alarm and flurry, he threw the wood
here and there about the floor, called and searched, but no cook was
to be found. Then some of the wood that had been carelessly thrown
down, caught fire and began to blaze. The bird hastened to fetch some
water, but his pail fell into the well, and he after it, and as he was
unable to recover himself, he was drowned.



\chapter{MOTHER HOLLE}

Once upon a time there was a widow who had two daughters; one of them
was beautiful and industrious, the other ugly and lazy. The mother,
however, loved the ugly and lazy one best, because she was her own
daughter, and so the other, who was only her stepdaughter, was made to
do all the work of the house, and was quite the Cinderella of the
family. Her stepmother sent her out every day to sit by the well in
the high road, there to spin until she made her fingers bleed. Now it
chanced one day that some blood fell on to the spindle, and as the
girl stopped over the well to wash it off, the spindle suddenly sprang
out of her hand and fell into the well. She ran home crying to tell of
her misfortune, but her stepmother spoke harshly to her, and after
giving her a violent scolding, said unkindly, 'As you have let the
spindle fall into the well you may go yourself and fetch it out.'

The girl went back to the well not knowing what to do, and at last in
her distress she jumped into the water after the spindle.

She remembered nothing more until she awoke and found herself in a
beautiful meadow, full of sunshine, and with countless flowers
blooming in every direction.

She walked over the meadow, and presently she came upon a baker's oven
full of bread, and the loaves cried out to her, 'Take us out, take us
out, or alas! we shall be burnt to a cinder; we were baked through
long ago.' So she took the bread-shovel and drew them all out.

She went on a little farther, till she came to a free full of apples.
'Shake me, shake me, I pray,' cried the tree; 'my apples, one and all,
are ripe.' So she shook the tree, and the apples came falling down
upon her like rain; but she continued shaking until there was not a
single apple left upon it. Then she carefully gathered the apples
together in a heap and walked on again.

The next thing she came to was a little house, and there she saw an
old woman looking out, with such large teeth, that she was terrified,
and turned to run away. But the old woman called after her, 'What are
you afraid of, dear child? Stay with me; if you will do the work of my
house properly for me, I will make you very happy. You must be very
careful, however, to make my bed in the right way, for I wish you
always to shake it thoroughly, so that the feathers fly about; then
they say, down there in the world, that it is snowing; for I am Mother
Holle.' The old woman spoke so kindly, that the girl summoned up
courage and agreed to enter into her service.

She took care to do everything according to the old woman's bidding
and every time she made the bed she shook it with all her might, so
that the feathers flew about like so many snowflakes. The old woman
was as good as her word: she never spoke angrily to her, and gave her
roast and boiled meats every day.

So she stayed on with Mother Holle for some time, and then she began
to grow unhappy. She could not at first tell why she felt sad, but she
became conscious at last of great longing to go home; then she knew
she was homesick, although she was a thousand times better off with
Mother Holle than with her mother and sister. After waiting awhile,
she went to Mother Holle and said, 'I am so homesick, that I cannot
stay with you any longer, for although I am so happy here, I must
return to my own people.'

Then Mother Holle said, 'I am pleased that you should want to go back
to your own people, and as you have served me so well and faithfully,
I will take you home myself.'

Thereupon she led the girl by the hand up to a broad gateway. The gate
was opened, and as the girl passed through, a shower of gold fell upon
her, and the gold clung to her, so that she was covered with it from
head to foot.

'That is a reward for your industry,' said Mother Holle, and as she
spoke she handed her the spindle which she had dropped into the well.

The gate was then closed, and the girl found herself back in the old
world close to her mother's house. As she entered the courtyard, the
cock who was perched on the well, called out:

\begin{verse}
 'Cock-a-doodle-doo!\\
  Your golden daughter's come back to you.'
\end{verse}

Then she went in to her mother and sister, and as she was so richly
covered with gold, they gave her a warm welcome. She related to them
all that had happened, and when the mother heard how she had come by
her great riches, she thought she should like her ugly, lazy daughter
to go and try her fortune. So she made the sister go and sit by the
well and spin, and the girl pricked her finger and thrust her hand
into a thorn-bush, so that she might drop some blood on to the
spindle; then she threw it into the well, and jumped in herself.

Like her sister she awoke in the beautiful meadow, and walked over it
till she came to the oven. 'Take us out, take us out, or alas! we
shall be burnt to a cinder; we were baked through long ago,' cried the
loaves as before. But the lazy girl answered, 'Do you think I am going
to dirty my hands for you?' and walked on.

Presently she came to the apple-tree. 'Shake me, shake me, I pray; my
apples, one and all, are ripe,' it cried. But she only answered, 'A
nice thing to ask me to do, one of the apples might fall on my head,'
and passed on.

At last she came to Mother Holle's house, and as she had heard all
about the large teeth from her sister, she was not afraid of them, and
engaged herself without delay to the old woman.

The first day she was very obedient and industrious, and exerted
herself to please Mother Holle, for she thought of the gold she should
get in return. The next day, however, she began to dawdle over her
work, and the third day she was more idle still; then she began to lie
in bed in the mornings and refused to get up. Worse still, she
neglected to make the old woman's bed properly, and forgot to shake it
so that the feathers might fly about. So Mother Holle very soon got
tired of her, and told her she might go. The lazy girl was delighted
at this, and thought to herself, 'The gold will soon be mine.' Mother
Holle led her, as she had led her sister, to the broad gateway; but as
she was passing through, instead of the shower of gold, a great
bucketful of pitch came pouring over her.

'That is in return for your services,' said the old woman, and she
shut the gate.

So the lazy girl had to go home covered with pitch, and the cock on
the well called out as she saw her:

\begin{verse}
 'Cock-a-doodle-doo!\\
  Your dirty daughter's come back to you.'
\end{verse}

But, try what she would, she could not get the pitch off and it stuck
to her as long as she lived.



\chapter[LITTLE RED-CAP (or LITTLE RED RIDING\ldots]{LITTLE RED-CAP (or LITTLE RED RIDING HOOD)}

Once upon a time there was a dear little girl who was loved by
everyone who looked at her, but most of all by her grandmother, and
there was nothing that she would not have given to the child. Once she
gave her a little cap of red velvet, which suited her so well that she
would never wear anything else; so she was always called 'Little Red-
Cap.'

One day her mother said to her: 'Come, Little Red-Cap, here is a piece
of cake and a bottle of wine; take them to your grandmother, she is
ill and weak, and they will do her good. Set out before it gets hot,
and when you are going, walk nicely and quietly and do not run off the
path, or you may fall and break the bottle, and then your grandmother
will get nothing; and when you go into her room, don't forget to say,
``Good morning'', and don't peep into every corner before you do it.'

'I will take great care,' said Little Red-Cap to her mother, and gave
her hand on it.

The grandmother lived out in the wood, half a league from the village,
and just as Little Red-Cap entered the wood, a wolf met her. Red-Cap
did not know what a wicked creature he was, and was not at all afraid
of him.

'Good day, Little Red-Cap,' said he.

'Thank you kindly, wolf.'

'Whither away so early, Little Red-Cap?'

'To my grandmother's.'

'What have you got in your apron?'

'Cake and wine; yesterday was baking-day, so poor sick grandmother is
to have something good, to make her stronger.'

'Where does your grandmother live, Little Red-Cap?'

'A good quarter of a league farther on in the wood; her house stands
under the three large oak-trees, the nut-trees are just below; you
surely must know it,' replied Little Red-Cap.

The wolf thought to himself: 'What a tender young creature! what a
nice plump mouthful--she will be better to eat than the old woman. I
must act craftily, so as to catch both.' So he walked for a short time
by the side of Little Red-Cap, and then he said: 'See, Little Red-Cap,
how pretty the flowers are about here--why do you not look round? I
believe, too, that you do not hear how sweetly the little birds are
singing; you walk gravely along as if you were going to school, while
everything else out here in the wood is merry.'

Little Red-Cap raised her eyes, and when she saw the sunbeams dancing
here and there through the trees, and pretty flowers growing
everywhere, she thought: 'Suppose I take grandmother a fresh nosegay;
that would please her too. It is so early in the day that I shall
still get there in good time'; and so she ran from the path into the
wood to look for flowers. And whenever she had picked one, she fancied
that she saw a still prettier one farther on, and ran after it, and so
got deeper and deeper into the wood.

Meanwhile the wolf ran straight to the grandmother's house and knocked
at the door.

'Who is there?'

'Little Red-Cap,' replied the wolf. 'She is bringing cake and wine;
open the door.'

'Lift the latch,' called out the grandmother, 'I am too weak, and
cannot get up.'

The wolf lifted the latch, the door sprang open, and without saying a
word he went straight to the grandmother's bed, and devoured her. Then
he put on her clothes, dressed himself in her cap laid himself in bed
and drew the curtains.

Little Red-Cap, however, had been running about picking flowers, and
when she had gathered so many that she could carry no more, she
remembered her grandmother, and set out on the way to her.

She was surprised to find the cottage-door standing open, and when she
went into the room, she had such a strange feeling that she said to
herself: 'Oh dear! how uneasy I feel today, and at other times I like
being with grandmother so much.' She called out: 'Good morning,' but
received no answer; so she went to the bed and drew back the curtains.
There lay her grandmother with her cap pulled far over her face, and
looking very strange.

'Oh! grandmother,' she said, 'what big ears you have!'

'The better to hear you with, my child,' was the reply.

'But, grandmother, what big eyes you have!' she said.

'The better to see you with, my dear.'

'But, grandmother, what large hands you have!'

'The better to hug you with.'

'Oh! but, grandmother, what a terrible big mouth you have!'

'The better to eat you with!'

And scarcely had the wolf said this, than with one bound he was out of
bed and swallowed up Red-Cap.

When the wolf had appeased his appetite, he lay down again in the bed,
fell asleep and began to snore very loud. The huntsman was just
passing the house, and thought to himself: 'How the old woman is
snoring! I must just see if she wants anything.' So he went into the
room, and when he came to the bed, he saw that the wolf was lying in
it. 'Do I find you here, you old sinner!' said he. 'I have long sought
you!' Then just as he was going to fire at him, it occurred to him
that the wolf might have devoured the grandmother, and that she might
still be saved, so he did not fire, but took a pair of scissors, and
began to cut open the stomach of the sleeping wolf. When he had made
two snips, he saw the little Red-Cap shining, and then he made two
snips more, and the little girl sprang out, crying: 'Ah, how
frightened I have been! How dark it was inside the wolf'; and after
that the aged grandmother came out alive also, but scarcely able to
breathe. Red-Cap, however, quickly fetched great stones with which
they filled the wolf's belly, and when he awoke, he wanted to run
away, but the stones were so heavy that he collapsed at once, and fell
dead.

Then all three were delighted. The huntsman drew off the wolf's skin
and went home with it; the grandmother ate the cake and drank the wine
which Red-Cap had brought, and revived, but Red-Cap thought to
herself: 'As long as I live, I will never by myself leave the path, to
run into the wood, when my mother has forbidden me to do so.'



It also related that once when Red-Cap was again taking cakes to the
old grandmother, another wolf spoke to her, and tried to entice her
from the path. Red-Cap, however, was on her guard, and went straight
forward on her way, and told her grandmother that she had met the
wolf, and that he had said 'good morning' to her, but with such a
wicked look in his eyes, that if they had not been on the public road
she was certain he would have eaten her up. 'Well,' said the
grandmother, 'we will shut the door, that he may not come in.' Soon
afterwards the wolf knocked, and cried: 'Open the door, grandmother, I
am Little Red-Cap, and am bringing you some cakes.' But they did not
speak, or open the door, so the grey-beard stole twice or thrice round
the house, and at last jumped on the roof, intending to wait until
Red-Cap went home in the evening, and then to steal after her and
devour her in the darkness. But the grandmother saw what was in his
thoughts. In front of the house was a great stone trough, so she said
to the child: 'Take the pail, Red-Cap; I made some sausages yesterday,
so carry the water in which I boiled them to the trough.' Red-Cap
carried until the great trough was quite full. Then the smell of the
sausages reached the wolf, and he sniffed and peeped down, and at last
stretched out his neck so far that he could no longer keep his footing
and began to slip, and slipped down from the roof straight into the
great trough, and was drowned. But Red-Cap went joyously home, and no
one ever did anything to harm her again.



\chapter{THE ROBBER BRIDEGROOM}

There was once a miller who had one beautiful daughter, and as she was
grown up, he was anxious that she should be well married and provided
for. He said to himself, 'I will give her to the first suitable man
who comes and asks for her hand.' Not long after a suitor appeared,
and as he appeared to be very rich and the miller could see nothing in
him with which to find fault, he betrothed his daughter to him. But
the girl did not care for the man as a girl ought to care for her
betrothed husband. She did not feel that she could trust him, and she
could not look at him nor think of him without an inward shudder. One
day he said to her, 'You have not yet paid me a visit, although we
have been betrothed for some time.' 'I do not know where your house
is,' she answered. 'My house is out there in the dark forest,' he
said. She tried to excuse herself by saying that she would not be able
to find the way thither. Her betrothed only replied, 'You must come
and see me next Sunday; I have already invited guests for that day,
and that you may not mistake the way, I will strew ashes along the
path.'

When Sunday came, and it was time for the girl to start, a feeling of
dread came over her which she could not explain, and that she might be
able to find her path again, she filled her pockets with peas and
lentils to sprinkle on the ground as she went along. On reaching the
entrance to the forest she found the path strewed with ashes, and
these she followed, throwing down some peas on either side of her at
every step she took. She walked the whole day until she came to the
deepest, darkest part of the forest. There she saw a lonely house,
looking so grim and mysterious, that it did not please her at all. She
stepped inside, but not a soul was to be seen, and a great silence
reigned throughout. Suddenly a voice cried:

\begin{verse}
 'Turn back, turn back, young maiden fair,\\
  Linger not in this murderers' lair.'
\end{verse}

The girl looked up and saw that the voice came from a bird hanging in
a cage on the wall. Again it cried:

\begin{verse}
 'Turn back, turn back, young maiden fair,\\
  Linger not in this murderers' lair.'
\end{verse}

The girl passed on, going from room to room of the house, but they
were all empty, and still she saw no one. At last she came to the
cellar, and there sat a very, very old woman, who could not keep her
head from shaking. 'Can you tell me,' asked the girl, 'if my betrothed
husband lives here?'

'Ah, you poor child,' answered the old woman, 'what a place for you to
come to! This is a murderers' den. You think yourself a promised
bride, and that your marriage will soon take place, but it is with
death that you will keep your marriage feast. Look, do you see that
large cauldron of water which I am obliged to keep on the fire! As
soon as they have you in their power they will kill you without mercy,
and cook and eat you, for they are eaters of men. If I did not take
pity on you and save you, you would be lost.'

Thereupon the old woman led her behind a large cask, which quite hid
her from view. 'Keep as still as a mouse,' she said; 'do not move or
speak, or it will be all over with you. Tonight, when the robbers are
all asleep, we will flee together. I have long been waiting for an
opportunity to escape.'

The words were hardly out of her mouth when the godless crew returned,
dragging another young girl along with them. They were all drunk, and
paid no heed to her cries and lamentations. They gave her wine to
drink, three glasses full, one of white wine, one of red, and one of
yellow, and with that her heart gave way and she died. Then they tore
of her dainty clothing, laid her on a table, and cut her beautiful
body into pieces, and sprinkled salt upon it.

The poor betrothed girl crouched trembling and shuddering behind the
cask, for she saw what a terrible fate had been intended for her by
the robbers. One of them now noticed a gold ring still remaining on
the little finger of the murdered girl, and as he could not draw it
off easily, he took a hatchet and cut off the finger; but the finger
sprang into the air, and fell behind the cask into the lap of the girl
who was hiding there. The robber took a light and began looking for
it, but he could not find it. 'Have you looked behind the large cask?'
said one of the others. But the old woman called out, 'Come and eat
your suppers, and let the thing be till tomorrow; the finger won't run
away.'

'The old woman is right,' said the robbers, and they ceased looking
for the finger and sat down.

The old woman then mixed a sleeping draught with their wine, and
before long they were all lying on the floor of the cellar, fast
asleep and snoring. As soon as the girl was assured of this, she came
from behind the cask. She was obliged to step over the bodies of the
sleepers, who were lying close together, and every moment she was
filled with renewed dread lest she should awaken them. But God helped
her, so that she passed safely over them, and then she and the old
woman went upstairs, opened the door, and hastened as fast as they
could from the murderers' den. They found the ashes scattered by the
wind, but the peas and lentils had sprouted, and grown sufficiently
above the ground, to guide them in the moonlight along the path. All
night long they walked, and it was morning before they reached the
mill. Then the girl told her father all that had happened.

The day came that had been fixed for the marriage. The bridegroom
arrived and also a large company of guests, for the miller had taken
care to invite all his friends and relations. As they sat at the
feast, each guest in turn was asked to tell a tale; the bride sat
still and did not say a word.

'And you, my love,' said the bridegroom, turning to her, 'is there no
tale you know? Tell us something.'

'I will tell you a dream, then,' said the bride. 'I went alone through
a forest and came at last to a house; not a soul could I find within,
but a bird that was hanging in a cage on the wall cried:

\begin{verse}
 'Turn back, turn back, young maiden fair,\\
  Linger not in this murderers' lair.'
\end{verse}

and again a second time it said these words.'

'My darling, this is only a dream.'

'I went on through the house from room to room, but they were all
empty, and everything was so grim and mysterious. At last I went down
to the cellar, and there sat a very, very old woman, who could not
keep her head still. I asked her if my betrothed lived here, and she
answered, ``Ah, you poor child, you are come to a murderers' den; your
betrothed does indeed live here, but he will kill you without mercy
and afterwards cook and eat you.'''

'My darling, this is only a dream.'

'The old woman hid me behind a large cask, and scarcely had she done
this when the robbers returned home, dragging a young girl along with
them. They gave her three kinds of wine to drink, white, red, and
yellow, and with that she died.'

'My darling, this is only a dream.'

'Then they tore off her dainty clothing, and cut her beautiful body
into pieces and sprinkled salt upon it.'

'My darling, this is only a dream.'

'And one of the robbers saw that there was a gold ring still left on
her finger, and as it was difficult to draw off, he took a hatchet and
cut off her finger; but the finger sprang into the air and fell behind
the great cask into my lap. And here is the finger with the ring.' and
with these words the bride drew forth the finger and shewed it to the
assembled guests.

The bridegroom, who during this recital had grown deadly pale, up and
tried to escape, but the guests seized him and held him fast. They
delivered him up to justice, and he and all his murderous band were
condemned to death for their wicked deeds.



\chapter{TOM THUMB}

A poor woodman sat in his cottage one night, smoking his pipe by the
fireside, while his wife sat by his side spinning. 'How lonely it is,
wife,' said he, as he puffed out a long curl of smoke, 'for you and me
to sit here by ourselves, without any children to play about and amuse
us while other people seem so happy and merry with their children!'
'What you say is very true,' said the wife, sighing, and turning round
her wheel; 'how happy should I be if I had but one child! If it were
ever so small--nay, if it were no bigger than my thumb--I should be
very happy, and love it dearly.' Now--odd as you may think it--it came
to pass that this good woman's wish was fulfilled, just in the very
way she had wished it; for, not long afterwards, she had a little boy,
who was quite healthy and strong, but was not much bigger than my
thumb. So they said, 'Well, we cannot say we have not got what we
wished for, and, little as he is, we will love him dearly.' And they
called him Thomas Thumb.

They gave him plenty of food, yet for all they could do he never grew
bigger, but kept just the same size as he had been when he was born.
Still, his eyes were sharp and sparkling, and he soon showed himself
to be a clever little fellow, who always knew well what he was about.

One day, as the woodman was getting ready to go into the wood to cut
fuel, he said, 'I wish I had someone to bring the cart after me, for I
want to make haste.' 'Oh, father,' cried Tom, 'I will take care of
that; the cart shall be in the wood by the time you want it.' Then the
woodman laughed, and said, 'How can that be? you cannot reach up to
the horse's bridle.' 'Never mind that, father,' said Tom; 'if my
mother will only harness the horse, I will get into his ear and tell
him which way to go.' 'Well,' said the father, 'we will try for once.'

When the time came the mother harnessed the horse to the cart, and put
Tom into his ear; and as he sat there the little man told the beast
how to go, crying out, 'Go on!' and 'Stop!' as he wanted: and thus the
horse went on just as well as if the woodman had driven it himself
into the wood. It happened that as the horse was going a little too
fast, and Tom was calling out, 'Gently! gently!' two strangers came
up. 'What an odd thing that is!' said one: 'there is a cart going
along, and I hear a carter talking to the horse, but yet I can see no
one.' 'That is queer, indeed,' said the other; 'let us follow the
cart, and see where it goes.' So they went on into the wood, till at
last they came to the place where the woodman was. Then Tom Thumb,
seeing his father, cried out, 'See, father, here I am with the cart,
all right and safe! now take me down!' So his father took hold of the
horse with one hand, and with the other took his son out of the
horse's ear, and put him down upon a straw, where he sat as merry as
you please.

The two strangers were all this time looking on, and did not know what
to say for wonder. At last one took the other aside, and said, 'That
little urchin will make our fortune, if we can get him, and carry him
about from town to town as a show; we must buy him.' So they went up
to the woodman, and asked him what he would take for the little man.
'He will be better off,' said they, 'with us than with you.' 'I won't
sell him at all,' said the father; 'my own flesh and blood is dearer
to me than all the silver and gold in the world.' But Tom, hearing of
the bargain they wanted to make, crept up his father's coat to his
shoulder and whispered in his ear, 'Take the money, father, and let
them have me; I'll soon come back to you.'

So the woodman at last said he would sell Tom to the strangers for a
large piece of gold, and they paid the price. 'Where would you like to
sit?' said one of them. 'Oh, put me on the rim of your hat; that will
be a nice gallery for me; I can walk about there and see the country
as we go along.' So they did as he wished; and when Tom had taken
leave of his father they took him away with them.

They journeyed on till it began to be dusky, and then the little man
said, 'Let me get down, I'm tired.' So the man took off his hat, and
put him down on a clod of earth, in a ploughed field by the side of
the road. But Tom ran about amongst the furrows, and at last slipped
into an old mouse-hole. 'Good night, my masters!' said he, 'I'm off!
mind and look sharp after me the next time.' Then they ran at once to
the place, and poked the ends of their sticks into the mouse-hole, but
all in vain; Tom only crawled farther and farther in; and at last it
became quite dark, so that they were forced to go their way without
their prize, as sulky as could be.

When Tom found they were gone, he came out of his hiding-place. 'What
dangerous walking it is,' said he, 'in this ploughed field! If I were
to fall from one of these great clods, I should undoubtedly break my
neck.' At last, by good luck, he found a large empty snail-shell.
'This is lucky,' said he, 'I can sleep here very well'; and in he
crept.

Just as he was falling asleep, he heard two men passing by, chatting
together; and one said to the other, 'How can we rob that rich
parson's house of his silver and gold?' 'I'll tell you!' cried Tom.
'What noise was that?' said the thief, frightened; 'I'm sure I heard
someone speak.' They stood still listening, and Tom said, 'Take me
with you, and I'll soon show you how to get the parson's money.' 'But
where are you?' said they. 'Look about on the ground,' answered he,
'and listen where the sound comes from.' At last the thieves found him
out, and lifted him up in their hands. 'You little urchin!' they said,
'what can you do for us?' 'Why, I can get between the iron window-bars
of the parson's house, and throw you out whatever you want.' 'That's a
good thought,' said the thieves; 'come along, we shall see what you
can do.'

When they came to the parson's house, Tom slipped through the window-
bars into the room, and then called out as loud as he could bawl,
'Will you have all that is here?' At this the thieves were frightened,
and said, 'Softly, softly! Speak low, that you may not awaken
anybody.' But Tom seemed as if he did not understand them, and bawled
out again, 'How much will you have? Shall I throw it all out?' Now the
cook lay in the next room; and hearing a noise she raised herself up
in her bed and listened. Meantime the thieves were frightened, and ran
off a little way; but at last they plucked up their hearts, and said,
'The little urchin is only trying to make fools of us.' So they came
back and whispered softly to him, saying, 'Now let us have no more of
your roguish jokes; but throw us out some of the money.' Then Tom
called out as loud as he could, 'Very well! hold your hands! here it
comes.'

The cook heard this quite plain, so she sprang out of bed, and ran to
open the door. The thieves ran off as if a wolf was at their tails:
and the maid, having groped about and found nothing, went away for a
light. By the time she came back, Tom had slipped off into the barn;
and when she had looked about and searched every hole and corner, and
found nobody, she went to bed, thinking she must have been dreaming
with her eyes open.

The little man crawled about in the hay-loft, and at last found a snug
place to finish his night's rest in; so he laid himself down, meaning
to sleep till daylight, and then find his way home to his father and
mother. But alas! how woefully he was undone! what crosses and sorrows
happen to us all in this world! The cook got up early, before
daybreak, to feed the cows; and going straight to the hay-loft,
carried away a large bundle of hay, with the little man in the middle
of it, fast asleep. He still, however, slept on, and did not awake
till he found himself in the mouth of the cow; for the cook had put
the hay into the cow's rick, and the cow had taken Tom up in a
mouthful of it. 'Good lack-a-day!' said he, 'how came I to tumble into
the mill?' But he soon found out where he really was; and was forced
to have all his wits about him, that he might not get between the
cow's teeth, and so be crushed to death. At last down he went into her
stomach. 'It is rather dark,' said he; 'they forgot to build windows
in this room to let the sun in; a candle would be no bad thing.'

Though he made the best of his bad luck, he did not like his quarters
at all; and the worst of it was, that more and more hay was always
coming down, and the space left for him became smaller and smaller. At
last he cried out as loud as he could, 'Don't bring me any more hay!
Don't bring me any more hay!'

The maid happened to be just then milking the cow; and hearing someone
speak, but seeing nobody, and yet being quite sure it was the same
voice that she had heard in the night, she was so frightened that she
fell off her stool, and overset the milk-pail. As soon as she could
pick herself up out of the dirt, she ran off as fast as she could to
her master the parson, and said, 'Sir, sir, the cow is talking!' But
the parson said, 'Woman, thou art surely mad!' However, he went with
her into the cow-house, to try and see what was the matter.

Scarcely had they set foot on the threshold, when Tom called out,
'Don't bring me any more hay!' Then the parson himself was frightened;
and thinking the cow was surely bewitched, told his man to kill her on
the spot. So the cow was killed, and cut up; and the stomach, in which
Tom lay, was thrown out upon a dunghill.

Tom soon set himself to work to get out, which was not a very easy
task; but at last, just as he had made room to get his head out, fresh
ill-luck befell him. A hungry wolf sprang out, and swallowed up the
whole stomach, with Tom in it, at one gulp, and ran away.

Tom, however, was still not disheartened; and thinking the wolf would
not dislike having some chat with him as he was going along, he called
out, 'My good friend, I can show you a famous treat.' 'Where's that?'
said the wolf. 'In such and such a house,' said Tom, describing his
own father's house. 'You can crawl through the drain into the kitchen
and then into the pantry, and there you will find cakes, ham, beef,
cold chicken, roast pig, apple-dumplings, and everything that your
heart can wish.'

The wolf did not want to be asked twice; so that very night he went to
the house and crawled through the drain into the kitchen, and then
into the pantry, and ate and drank there to his heart's content. As
soon as he had had enough he wanted to get away; but he had eaten so
much that he could not go out by the same way he came in.

This was just what Tom had reckoned upon; and now he began to set up a
great shout, making all the noise he could. 'Will you be easy?' said
the wolf; 'you'll awaken everybody in the house if you make such a
clatter.' 'What's that to me?' said the little man; 'you have had your
frolic, now I've a mind to be merry myself'; and he began, singing and
shouting as loud as he could.

The woodman and his wife, being awakened by the noise, peeped through
a crack in the door; but when they saw a wolf was there, you may well
suppose that they were sadly frightened; and the woodman ran for his
axe, and gave his wife a scythe. 'Do you stay behind,' said the
woodman, 'and when I have knocked him on the head you must rip him up
with the scythe.' Tom heard all this, and cried out, 'Father, father!
I am here, the wolf has swallowed me.' And his father said, 'Heaven be
praised! we have found our dear child again'; and he told his wife not
to use the scythe for fear she should hurt him. Then he aimed a great
blow, and struck the wolf on the head, and killed him on the spot! and
when he was dead they cut open his body, and set Tommy free. 'Ah!'
said the father, 'what fears we have had for you!' 'Yes, father,'
answered he; 'I have travelled all over the world, I think, in one way
or other, since we parted; and now I am very glad to come home and get
fresh air again.' 'Why, where have you been?' said his father. 'I have
been in a mouse-hole--and in a snail-shell--and down a cow's throat--
and in the wolf's belly; and yet here I am again, safe and sound.'

'Well,' said they, 'you are come back, and we will not sell you again
for all the riches in the world.'

Then they hugged and kissed their dear little son, and gave him plenty
to eat and drink, for he was very hungry; and then they fetched new
clothes for him, for his old ones had been quite spoiled on his
journey. So Master Thumb stayed at home with his father and mother, in
peace; for though he had been so great a traveller, and had done and
seen so many fine things, and was fond enough of telling the whole
story, he always agreed that, after all, there's no place like HOME!



\chapter{RUMPELSTILTSKIN}

By the side of a wood, in a country a long way off, ran a fine stream
of water; and upon the stream there stood a mill. The miller's house
was close by, and the miller, you must know, had a very beautiful
daughter. She was, moreover, very shrewd and clever; and the miller
was so proud of her, that he one day told the king of the land, who
used to come and hunt in the wood, that his daughter could spin gold
out of straw. Now this king was very fond of money; and when he heard
the miller's boast his greediness was raised, and he sent for the girl
to be brought before him. Then he led her to a chamber in his palace
where there was a great heap of straw, and gave her a spinning-wheel,
and said, 'All this must be spun into gold before morning, as you love
your life.' It was in vain that the poor maiden said that it was only
a silly boast of her father, for that she could do no such thing as
spin straw into gold: the chamber door was locked, and she was left
alone.

She sat down in one corner of the room, and began to bewail her hard
fate; when on a sudden the door opened, and a droll-looking little man
hobbled in, and said, 'Good morrow to you, my good lass; what are you
weeping for?' 'Alas!' said she, 'I must spin this straw into gold, and
I know not how.' 'What will you give me,' said the hobgoblin, 'to do
it for you?' 'My necklace,' replied the maiden. He took her at her
word, and sat himself down to the wheel, and whistled and sang:

\begin{verse}
 'Round about, round about,\\
    Lo and behold!\\
  Reel away, reel away,\\
    Straw into gold!'
\end{verse}

And round about the wheel went merrily; the work was quickly done, and
the straw was all spun into gold.

When the king came and saw this, he was greatly astonished and
pleased; but his heart grew still more greedy of gain, and he shut up
the poor miller's daughter again with a fresh task. Then she knew not
what to do, and sat down once more to weep; but the dwarf soon opened
the door, and said, 'What will you give me to do your task?' 'The ring
on my finger,' said she. So her little friend took the ring, and began
to work at the wheel again, and whistled and sang:

\begin{verse}
 'Round about, round about,\\
    Lo and behold!\\
  Reel away, reel away,\\
    Straw into gold!'
\end{verse}

till, long before morning, all was done again.

The king was greatly delighted to see all this glittering treasure;
but still he had not enough: so he took the miller's daughter to a yet
larger heap, and said, 'All this must be spun tonight; and if it is,
you shall be my queen.' As soon as she was alone that dwarf came in,
and said, 'What will you give me to spin gold for you this third
time?' 'I have nothing left,' said she. 'Then say you will give me,'
said the little man, 'the first little child that you may have when
you are queen.' 'That may never be,' thought the miller's daughter:
and as she knew no other way to get her task done, she said she would
do what he asked. Round went the wheel again to the old song, and the
manikin once more spun the heap into gold. The king came in the
morning, and, finding all he wanted, was forced to keep his word; so
he married the miller's daughter, and she really became queen.

At the birth of her first little child she was very glad, and forgot
the dwarf, and what she had said. But one day he came into her room,
where she was sitting playing with her baby, and put her in mind of
it. Then she grieved sorely at her misfortune, and said she would give
him all the wealth of the kingdom if he would let her off, but in
vain; till at last her tears softened him, and he said, 'I will give
you three days' grace, and if during that time you tell me my name,
you shall keep your child.'

Now the queen lay awake all night, thinking of all the odd names that
she had ever heard; and she sent messengers all over the land to find
out new ones. The next day the little man came, and she began with
TIMOTHY, ICHABOD, BENJAMIN, JEREMIAH, and all the names she could
remember; but to all and each of them he said, 'Madam, that is not my
name.'

The second day she began with all the comical names she could hear of,
BANDY-LEGS, HUNCHBACK, CROOK-SHANKS, and so on; but the little
gentleman still said to every one of them, 'Madam, that is not my
name.'

The third day one of the messengers came back, and said, 'I have
travelled two days without hearing of any other names; but yesterday,
as I was climbing a high hill, among the trees of the forest where the
fox and the hare bid each other good night, I saw a little hut; and
before the hut burnt a fire; and round about the fire a funny little
dwarf was dancing upon one leg, and singing:

\begin{verse}
 '''Merrily the feast I'll make.\\
  Today I'll brew, tomorrow bake;\\
  Merrily I'll dance and sing,\\
  For next day will a stranger bring.\\
  Little does my lady dream\\
  Rumpelstiltskin is my name!'''
\end{verse}

When the queen heard this she jumped for joy, and as soon as her
little friend came she sat down upon her throne, and called all her
court round to enjoy the fun; and the nurse stood by her side with the
baby in her arms, as if it was quite ready to be given up. Then the
little man began to chuckle at the thought of having the poor child,
to take home with him to his hut in the woods; and he cried out, 'Now,
lady, what is my name?' 'Is it JOHN?' asked she. 'No, madam!' 'Is it
TOM?' 'No, madam!' 'Is it JEMMY?' 'It is not.' 'Can your name be
RUMPELSTILTSKIN?' said the lady slyly. 'Some witch told you that!--
some witch told you that!' cried the little man, and dashed his right
foot in a rage so deep into the floor, that he was forced to lay hold
of it with both hands to pull it out.

Then he made the best of his way off, while the nurse laughed and the
baby crowed; and all the court jeered at him for having had so much
trouble for nothing, and said, 'We wish you a very good morning, and a
merry feast, Mr RUMPLESTILTSKIN!'



\chapter{CLEVER GRETEL}

There was once a cook named Gretel, who wore shoes with red heels, and
when she walked out with them on, she turned herself this way and
that, was quite happy and thought: 'You certainly are a pretty girl!'
And when she came home she drank, in her gladness of heart, a draught
of wine, and as wine excites a desire to eat, she tasted the best of
whatever she was cooking until she was satisfied, and said: 'The cook
must know what the food is like.'

It came to pass that the master one day said to her: 'Gretel, there is
a guest coming this evening; prepare me two fowls very daintily.' 'I
will see to it, master,' answered Gretel. She killed two fowls,
scalded them, plucked them, put them on the spit, and towards evening
set them before the fire, that they might roast. The fowls began to
turn brown, and were nearly ready, but the guest had not yet arrived.
Then Gretel called out to her master: 'If the guest does not come, I
must take the fowls away from the fire, but it will be a sin and a
shame if they are not eaten the moment they are at their juiciest.'
The master said: 'I will run myself, and fetch the guest.' When the
master had turned his back, Gretel laid the spit with the fowls on one
side, and thought: 'Standing so long by the fire there, makes one
sweat and thirsty; who knows when they will come? Meanwhile, I will
run into the cellar, and take a drink.' She ran down, set a jug, said:
'God bless it for you, Gretel,' and took a good drink, and thought
that wine should flow on, and should not be interrupted, and took yet
another hearty draught.

Then she went and put the fowls down again to the fire, basted them,
and drove the spit merrily round. But as the roast meat smelt so good,
Gretel thought: 'Something might be wrong, it ought to be tasted!' She
touched it with her finger, and said: 'Ah! how good fowls are! It
certainly is a sin and a shame that they are not eaten at the right
time!' She ran to the window, to see if the master was not coming with
his guest, but she saw no one, and went back to the fowls and thought:
'One of the wings is burning! I had better take it off and eat it.' So
she cut it off, ate it, and enjoyed it, and when she had done, she
thought: 'The other must go down too, or else master will observe that
something is missing.' When the two wings were eaten, she went and
looked for her master, and did not see him. It suddenly occurred to
her: 'Who knows? They are perhaps not coming at all, and have turned
in somewhere.' Then she said: 'Well, Gretel, enjoy yourself, one fowl
has been cut into, take another drink, and eat it up entirely; when it
is eaten you will have some peace, why should God's good gifts be
spoilt?' So she ran into the cellar again, took an enormous drink and
ate up the one chicken in great glee. When one of the chickens was
swallowed down, and still her master did not come, Gretel looked at
the other and said: 'What one is, the other should be likewise, the
two go together; what's right for the one is right for the other; I
think if I were to take another draught it would do me no harm.' So
she took another hearty drink, and let the second chicken follow the
first.

While she was making the most of it, her master came and cried: 'Hurry
up, Gretel, the guest is coming directly after me!' 'Yes, sir, I will
soon serve up,' answered Gretel. Meantime the master looked to see
what the table was properly laid, and took the great knife, wherewith
he was going to carve the chickens, and sharpened it on the steps.
Presently the guest came, and knocked politely and courteously at the
house-door. Gretel ran, and looked to see who was there, and when she
saw the guest, she put her finger to her lips and said: 'Hush! hush!
go away as quickly as you can, if my master catches you it will be the
worse for you; he certainly did ask you to supper, but his intention
is to cut off your two ears. Just listen how he is sharpening the
knife for it!' The guest heard the sharpening, and hurried down the
steps again as fast as he could. Gretel was not idle; she ran
screaming to her master, and cried: 'You have invited a fine guest!'
'Why, Gretel? What do you mean by that?' 'Yes,' said she, 'he has
taken the chickens which I was just going to serve up, off the dish,
and has run away with them!' 'That's a nice trick!' said her master,
and lamented the fine chickens. 'If he had but left me one, so that
something remained for me to eat.' He called to him to stop, but the
guest pretended not to hear. Then he ran after him with the knife
still in his hand, crying: 'Just one, just one,' meaning that the
guest should leave him just one chicken, and not take both. The guest,
however, thought no otherwise than that he was to give up one of his
ears, and ran as if fire were burning under him, in order to take them
both with him.



\chapter{THE OLD MAN AND HIS GRANDSON}

There was once a very old man, whose eyes had become dim, his ears
dull of hearing, his knees trembled, and when he sat at table he could
hardly hold the spoon, and spilt the broth upon the table-cloth or let
it run out of his mouth. His son and his son's wife were disgusted at
this, so the old grandfather at last had to sit in the corner behind
the stove, and they gave him his food in an earthenware bowl, and not
even enough of it. And he used to look towards the table with his eyes
full of tears. Once, too, his trembling hands could not hold the bowl,
and it fell to the ground and broke. The young wife scolded him, but
he said nothing and only sighed. Then they brought him a wooden bowl
for a few half-pence, out of which he had to eat.

They were once sitting thus when the little grandson of four years old
began to gather together some bits of wood upon the ground. 'What are
you doing there?' asked the father. 'I am making a little trough,'
answered the child, 'for father and mother to eat out of when I am
big.'

The man and his wife looked at each other for a while, and presently
began to cry. Then they took the old grandfather to the table, and
henceforth always let him eat with them, and likewise said nothing if
he did spill a little of anything.



\chapter{THE LITTLE PEASANT}

There was a certain village wherein no one lived but really rich
peasants, and just one poor one, whom they called the little peasant.
He had not even so much as a cow, and still less money to buy one, and
yet he and his wife did so wish to have one. One day he said to her:
'Listen, I have a good idea, there is our gossip the carpenter, he
shall make us a wooden calf, and paint it brown, so that it looks like
any other, and in time it will certainly get big and be a cow.' the
woman also liked the idea, and their gossip the carpenter cut and
planed the calf, and painted it as it ought to be, and made it with
its head hanging down as if it were eating.

Next morning when the cows were being driven out, the little peasant
called the cow-herd in and said: 'Look, I have a little calf there,
but it is still small and has to be carried.' The cow-herd said: 'All
right,' and took it in his arms and carried it to the pasture, and set
it among the grass. The little calf always remained standing like one
which was eating, and the cow-herd said: 'It will soon run by itself,
just look how it eats already!' At night when he was going to drive
the herd home again, he said to the calf: 'If you can stand there and
eat your fill, you can also go on your four legs; I don't care to drag
you home again in my arms.' But the little peasant stood at his door,
and waited for his little calf, and when the cow-herd drove the cows
through the village, and the calf was missing, he inquired where it
was. The cow-herd answered: 'It is still standing out there eating. It
would not stop and come with us.' But the little peasant said: 'Oh,
but I must have my beast back again.' Then they went back to the
meadow together, but someone had stolen the calf, and it was gone. The
cow-herd said: 'It must have run away.' The peasant, however, said:
'Don't tell me that,' and led the cow-herd before the mayor, who for
his carelessness condemned him to give the peasant a cow for the calf
which had run away.

And now the little peasant and his wife had the cow for which they had
so long wished, and they were heartily glad, but they had no food for
it, and could give it nothing to eat, so it soon had to be killed.
They salted the flesh, and the peasant went into the town and wanted
to sell the skin there, so that he might buy a new calf with the
proceeds. On the way he passed by a mill, and there sat a raven with
broken wings, and out of pity he took him and wrapped him in the skin.
But as the weather grew so bad and there was a storm of rain and wind,
he could go no farther, and turned back to the mill and begged for
shelter. The miller's wife was alone in the house, and said to the
peasant: 'Lay yourself on the straw there,' and gave him a slice of
bread and cheese. The peasant ate it, and lay down with his skin
beside him, and the woman thought: 'He is tired and has gone to
sleep.' In the meantime came the parson; the miller's wife received
him well, and said: 'My husband is out, so we will have a feast.' The
peasant listened, and when he heard them talk about feasting he was
vexed that he had been forced to make shift with a slice of bread and
cheese. Then the woman served up four different things, roast meat,
salad, cakes, and wine.

Just as they were about to sit down and eat, there was a knocking
outside. The woman said: 'Oh, heavens! It is my husband!' she quickly
hid the roast meat inside the tiled stove, the wine under the pillow,
the salad on the bed, the cakes under it, and the parson in the closet
on the porch. Then she opened the door for her husband, and said:
'Thank heaven, you are back again! There is such a storm, it looks as
if the world were coming to an end.' The miller saw the peasant lying
on the straw, and asked, 'What is that fellow doing there?' 'Ah,' said
the wife, 'the poor knave came in the storm and rain, and begged for
shelter, so I gave him a bit of bread and cheese, and showed him where
the straw was.' The man said: 'I have no objection, but be quick and
get me something to eat.' The woman said: 'But I have nothing but
bread and cheese.' 'I am contented with anything,' replied the
husband, 'so far as I am concerned, bread and cheese will do,' and
looked at the peasant and said: 'Come and eat some more with me.' The
peasant did not require to be invited twice, but got up and ate. After
this the miller saw the skin in which the raven was, lying on the
ground, and asked: 'What have you there?' The peasant answered: 'I
have a soothsayer inside it.' 'Can he foretell anything to me?' said
the miller. 'Why not?' answered the peasant: 'but he only says four
things, and the fifth he keeps to himself.' The miller was curious,
and said: 'Let him foretell something for once.' Then the peasant
pinched the raven's head, so that he croaked and made a noise like
krr, krr. The miller said: 'What did he say?' The peasant answered:
'In the first place, he says that there is some wine hidden under the
pillow.' 'Bless me!' cried the miller, and went there and found the
wine. 'Now go on,' said he. The peasant made the raven croak again,
and said: 'In the second place, he says that there is some roast meat
in the tiled stove.' 'Upon my word!' cried the miller, and went
thither, and found the roast meat. The peasant made the raven prophesy
still more, and said: 'Thirdly, he says that there is some salad on
the bed.' 'That would be a fine thing!' cried the miller, and went
there and found the salad. At last the peasant pinched the raven once
more till he croaked, and said: 'Fourthly, he says that there are some
cakes under the bed.' 'That would be a fine thing!' cried the miller,
and looked there, and found the cakes.

And now the two sat down to the table together, but the miller's wife
was frightened to death, and went to bed and took all the keys with
her. The miller would have liked much to know the fifth, but the
little peasant said: 'First, we will quickly eat the four things, for
the fifth is something bad.' So they ate, and after that they
bargained how much the miller was to give for the fifth prophecy,
until they agreed on three hundred talers. Then the peasant once more
pinched the raven's head till he croaked loudly. The miller asked:
'What did he say?' The peasant replied: 'He says that the Devil is
hiding outside there in the closet on the porch.' The miller said:
'The Devil must go out,' and opened the house-door; then the woman was
forced to give up the keys, and the peasant unlocked the closet. The
parson ran out as fast as he could, and the miller said: 'It was true;
I saw the black rascal with my own eyes.' The peasant, however, made
off next morning by daybreak with the three hundred talers.

At home the small peasant gradually launched out; he built a beautiful
house, and the peasants said: 'The small peasant has certainly been to
the place where golden snow falls, and people carry the gold home in
shovels.' Then the small peasant was brought before the mayor, and
bidden to say from whence his wealth came. He answered: 'I sold my
cow's skin in the town, for three hundred talers.' When the peasants
heard that, they too wished to enjoy this great profit, and ran home,
killed all their cows, and stripped off their skins in order to sell
them in the town to the greatest advantage. The mayor, however, said:
'But my servant must go first.' When she came to the merchant in the
town, he did not give her more than two talers for a skin, and when
the others came, he did not give them so much, and said: 'What can I
do with all these skins?'

Then the peasants were vexed that the small peasant should have thus
outwitted them, wanted to take vengeance on him, and accused him of
this treachery before the major. The innocent little peasant was
unanimously sentenced to death, and was to be rolled into the water,
in a barrel pierced full of holes. He was led forth, and a priest was
brought who was to say a mass for his soul. The others were all
obliged to retire to a distance, and when the peasant looked at the
priest, he recognized the man who had been with the miller's wife. He
said to him: 'I set you free from the closet, set me free from the
barrel.' At this same moment up came, with a flock of sheep, the very
shepherd whom the peasant knew had long been wishing to be mayor, so
he cried with all his might: 'No, I will not do it; if the whole world
insists on it, I will not do it!' The shepherd hearing that, came up
to him, and asked: 'What are you about? What is it that you will not
do?' The peasant said: 'They want to make me mayor, if I will but put
myself in the barrel, but I will not do it.' The shepherd said: 'If
nothing more than that is needful in order to be mayor, I would get
into the barrel at once.' The peasant said: 'If you will get in, you
will be mayor.' The shepherd was willing, and got in, and the peasant
shut the top down on him; then he took the shepherd's flock for
himself, and drove it away. The parson went to the crowd, and declared
that the mass had been said. Then they came and rolled the barrel
towards the water. When the barrel began to roll, the shepherd cried:
'I am quite willing to be mayor.' They believed no otherwise than that
it was the peasant who was saying this, and answered: 'That is what we
intend, but first you shall look about you a little down below there,'
and they rolled the barrel down into the water.

After that the peasants went home, and as they were entering the
village, the small peasant also came quietly in, driving a flock of
sheep and looking quite contented. Then the peasants were astonished,
and said: 'Peasant, from whence do you come? Have you come out of the
water?' 'Yes, truly,' replied the peasant, 'I sank deep, deep down,
until at last I got to the bottom; I pushed the bottom out of the
barrel, and crept out, and there were pretty meadows on which a number
of lambs were feeding, and from thence I brought this flock away with
me.' Said the peasants: 'Are there any more there?' 'Oh, yes,' said
he, 'more than I could want.' Then the peasants made up their minds
that they too would fetch some sheep for themselves, a flock apiece,
but the mayor said: 'I come first.' So they went to the water
together, and just then there were some of the small fleecy clouds in
the blue sky, which are called little lambs, and they were reflected
in the water, whereupon the peasants cried: 'We already see the sheep
down below!' The mayor pressed forward and said: 'I will go down
first, and look about me, and if things promise well I'll call you.'
So he jumped in; splash! went the water; it sounded as if he were
calling them, and the whole crowd plunged in after him as one man.
Then the entire village was dead, and the small peasant, as sole heir,
became a rich man.



\chapter{FREDERICK AND CATHERINE}

There was once a man called Frederick: he had a wife whose name was
Catherine, and they had not long been married. One day Frederick said.
'Kate! I am going to work in the fields; when I come back I shall be
hungry so let me have something nice cooked, and a good draught of
ale.' 'Very well,' said she, 'it shall all be ready.' When dinner-time
drew nigh, Catherine took a nice steak, which was all the meat she
had, and put it on the fire to fry. The steak soon began to look
brown, and to crackle in the pan; and Catherine stood by with a fork
and turned it: then she said to herself, 'The steak is almost ready, I
may as well go to the cellar for the ale.' So she left the pan on the
fire and took a large jug and went into the cellar and tapped the ale
cask. The beer ran into the jug and Catherine stood looking on. At
last it popped into her head, 'The dog is not shut up--he may be
running away with the steak; that's well thought of.' So up she ran
from the cellar; and sure enough the rascally cur had got the steak in
his mouth, and was making off with it.

Away ran Catherine, and away ran the dog across the field: but he ran
faster than she, and stuck close to the steak. 'It's all gone, and
``what can't be cured must be endured'',' said Catherine. So she turned
round; and as she had run a good way and was tired, she walked home
leisurely to cool herself.

Now all this time the ale was running too, for Catherine had not
turned the cock; and when the jug was full the liquor ran upon the
floor till the cask was empty. When she got to the cellar stairs she
saw what had happened. 'My stars!' said she, 'what shall I do to keep
Frederick from seeing all this slopping about?' So she thought a
while; and at last remembered that there was a sack of fine meal
bought at the last fair, and that if she sprinkled this over the floor
it would suck up the ale nicely. 'What a lucky thing,' said she, 'that
we kept that meal! we have now a good use for it.' So away she went
for it: but she managed to set it down just upon the great jug full of
beer, and upset it; and thus all the ale that had been saved was set
swimming on the floor also. 'Ah! well,' said she, 'when one goes
another may as well follow.' Then she strewed the meal all about the
cellar, and was quite pleased with her cleverness, and said, 'How very
neat and clean it looks!'

At noon Frederick came home. 'Now, wife,' cried he, 'what have you for
dinner?' 'O Frederick!' answered she, 'I was cooking you a steak; but
while I went down to draw the ale, the dog ran away with it; and while
I ran after him, the ale ran out; and when I went to dry up the ale
with the sack of meal that we got at the fair, I upset the jug: but
the cellar is now quite dry, and looks so clean!' 'Kate, Kate,' said
he, 'how could you do all this?' Why did you leave the steak to fry,
and the ale to run, and then spoil all the meal?' 'Why, Frederick,'
said she, 'I did not know I was doing wrong; you should have told me
before.'

The husband thought to himself, 'If my wife manages matters thus, I
must look sharp myself.' Now he had a good deal of gold in the house:
so he said to Catherine, 'What pretty yellow buttons these are! I
shall put them into a box and bury them in the garden; but take care
that you never go near or meddle with them.' 'No, Frederick,' said
she, 'that I never will.' As soon as he was gone, there came by some
pedlars with earthenware plates and dishes, and they asked her whether
she would buy. 'Oh dear me, I should like to buy very much, but I have
no money: if you had any use for yellow buttons, I might deal with
you.' 'Yellow buttons!' said they: 'let us have a look at them.' 'Go
into the garden and dig where I tell you, and you will find the yellow
buttons: I dare not go myself.' So the rogues went: and when they
found what these yellow buttons were, they took them all away, and
left her plenty of plates and dishes. Then she set them all about the
house for a show: and when Frederick came back, he cried out, 'Kate,
what have you been doing?' 'See,' said she, 'I have bought all these
with your yellow buttons: but I did not touch them myself; the pedlars
went themselves and dug them up.' 'Wife, wife,' said Frederick, 'what
a pretty piece of work you have made! those yellow buttons were all my
money: how came you to do such a thing?' 'Why,' answered she, 'I did
not know there was any harm in it; you should have told me.'

Catherine stood musing for a while, and at last said to her husband,
'Hark ye, Frederick, we will soon get the gold back: let us run after
the thieves.' 'Well, we will try,' answered he; 'but take some butter
and cheese with you, that we may have something to eat by the way.'
'Very well,' said she; and they set out: and as Frederick walked the
fastest, he left his wife some way behind. 'It does not matter,'
thought she: 'when we turn back, I shall be so much nearer home than
he.'

Presently she came to the top of a hill, down the side of which there
was a road so narrow that the cart wheels always chafed the trees on
each side as they passed. 'Ah, see now,' said she, 'how they have
bruised and wounded those poor trees; they will never get well.' So
she took pity on them, and made use of the butter to grease them all,
so that the wheels might not hurt them so much. While she was doing
this kind office one of her cheeses fell out of the basket, and rolled
down the hill. Catherine looked, but could not see where it had gone;
so she said, 'Well, I suppose the other will go the same way and find
you; he has younger legs than I have.' Then she rolled the other
cheese after it; and away it went, nobody knows where, down the hill.
But she said she supposed that they knew the road, and would follow
her, and she could not stay there all day waiting for them.

At last she overtook Frederick, who desired her to give him something
to eat. Then she gave him the dry bread. 'Where are the butter and
cheese?' said he. 'Oh!' answered she, 'I used the butter to grease
those poor trees that the wheels chafed so: and one of the cheeses ran
away so I sent the other after it to find it, and I suppose they are
both on the road together somewhere.' 'What a goose you are to do such
silly things!' said the husband. 'How can you say so?' said she; 'I am
sure you never told me not.'

They ate the dry bread together; and Frederick said, 'Kate, I hope you
locked the door safe when you came away.' 'No,' answered she, 'you did
not tell me.' 'Then go home, and do it now before we go any farther,'
said Frederick, 'and bring with you something to eat.'

Catherine did as he told her, and thought to herself by the way,
'Frederick wants something to eat; but I don't think he is very fond
of butter and cheese: I'll bring him a bag of fine nuts, and the
vinegar, for I have often seen him take some.'

When she reached home, she bolted the back door, but the front door
she took off the hinges, and said, 'Frederick told me to lock the
door, but surely it can nowhere be so safe if I take it with me.' So
she took her time by the way; and when she overtook her husband she
cried out, 'There, Frederick, there is the door itself, you may watch
it as carefully as you please.' 'Alas! alas!' said he, 'what a clever
wife I have! I sent you to make the house fast, and you take the door
away, so that everybody may go in and out as they please--however, as
you have brought the door, you shall carry it about with you for your
pains.' 'Very well,' answered she, 'I'll carry the door; but I'll not
carry the nuts and vinegar bottle also--that would be too much of a
load; so if you please, I'll fasten them to the door.'

Frederick of course made no objection to that plan, and they set off
into the wood to look for the thieves; but they could not find them:
and when it grew dark, they climbed up into a tree to spend the night
there. Scarcely were they up, than who should come by but the very
rogues they were looking for. They were in truth great rascals, and
belonged to that class of people who find things before they are lost;
they were tired; so they sat down and made a fire under the very tree
where Frederick and Catherine were. Frederick slipped down on the
other side, and picked up some stones. Then he climbed up again, and
tried to hit the thieves on the head with them: but they only said,
'It must be near morning, for the wind shakes the fir-apples down.'

Catherine, who had the door on her shoulder, began to be very tired;
but she thought it was the nuts upon it that were so heavy: so she
said softly, 'Frederick, I must let the nuts go.' 'No,' answered he,
'not now, they will discover us.' 'I can't help that: they must go.'
'Well, then, make haste and throw them down, if you will.' Then away
rattled the nuts down among the boughs and one of the thieves cried,
'Bless me, it is hailing.'

A little while after, Catherine thought the door was still very heavy:
so she whispered to Frederick, 'I must throw the vinegar down.' 'Pray
don't,' answered he, 'it will discover us.' 'I can't help that,' said
she, 'go it must.' So she poured all the vinegar down; and the thieves
said, 'What a heavy dew there is!'

At last it popped into Catherine's head that it was the door itself
that was so heavy all the time: so she whispered, 'Frederick, I must
throw the door down soon.' But he begged and prayed her not to do so,
for he was sure it would betray them. 'Here goes, however,' said she:
and down went the door with such a clatter upon the thieves, that they
cried out 'Murder!' and not knowing what was coming, ran away as fast
as they could, and left all the gold. So when Frederick and Catherine
came down, there they found all their money safe and sound.



\chapter{SWEETHEART ROLAND}

There was once upon a time a woman who was a real witch and had two
daughters, one ugly and wicked, and this one she loved because she was
her own daughter, and one beautiful and good, and this one she hated,
because she was her stepdaughter. The stepdaughter once had a pretty
apron, which the other fancied so much that she became envious, and
told her mother that she must and would have that apron. 'Be quiet, my
child,' said the old woman, 'and you shall have it. Your stepsister
has long deserved death; tonight when she is asleep I will come and
cut her head off. Only be careful that you are at the far side of the
bed, and push her well to the front.' It would have been all over with
the poor girl if she had not just then been standing in a corner, and
heard everything. All day long she dared not go out of doors, and when
bedtime had come, the witch's daughter got into bed first, so as to
lie at the far side, but when she was asleep, the other pushed her
gently to the front, and took for herself the place at the back, close
by the wall. In the night, the old woman came creeping in, she held an
axe in her right hand, and felt with her left to see if anyone were
lying at the outside, and then she grasped the axe with both hands,
and cut her own child's head off.

When she had gone away, the girl got up and went to her sweetheart,
who was called Roland, and knocked at his door. When he came out, she
said to him: 'Listen, dearest Roland, we must fly in all haste; my
stepmother wanted to kill me, but has struck her own child. When
daylight comes, and she sees what she has done, we shall be lost.'
'But,' said Roland, 'I counsel you first to take away her magic wand,
or we cannot escape if she pursues us.' The maiden fetched the magic
wand, and she took the dead girl's head and dropped three drops of
blood on the ground, one in front of the bed, one in the kitchen, and
one on the stairs. Then she hurried away with her lover.

When the old witch got up next morning, she called her daughter, and
wanted to give her the apron, but she did not come. Then the witch
cried: 'Where are you?' 'Here, on the stairs, I am sweeping,' answered
the first drop of blood. The old woman went out, but saw no one on the
stairs, and cried again: 'Where are you?' 'Here in the kitchen, I am
warming myself,' cried the second drop of blood. She went into the
kitchen, but found no one. Then she cried again: 'Where are you?' 'Ah,
here in the bed, I am sleeping,' cried the third drop of blood. She
went into the room to the bed. What did she see there? Her own child,
whose head she had cut off, bathed in her blood. The witch fell into a
passion, sprang to the window, and as she could look forth quite far
into the world, she perceived her stepdaughter hurrying away with her
sweetheart Roland. 'That shall not help you,' cried she, 'even if you
have got a long way off, you shall still not escape me.' She put on
her many-league boots, in which she covered an hour's walk at every
step, and it was not long before she overtook them. The girl, however,
when she saw the old woman striding towards her, changed, with her
magic wand, her sweetheart Roland into a lake, and herself into a duck
swimming in the middle of it. The witch placed herself on the shore,
threw breadcrumbs in, and went to endless trouble to entice the duck;
but the duck did not let herself be enticed, and the old woman had to
go home at night as she had come. At this the girl and her sweetheart
Roland resumed their natural shapes again, and they walked on the
whole night until daybreak. Then the maiden changed herself into a
beautiful flower which stood in the midst of a briar hedge, and her
sweetheart Roland into a fiddler. It was not long before the witch
came striding up towards them, and said to the musician: 'Dear
musician, may I pluck that beautiful flower for myself?' 'Oh, yes,' he
replied, 'I will play to you while you do it.' As she was hastily
creeping into the hedge and was just going to pluck the flower,
knowing perfectly well who the flower was, he began to play, and
whether she would or not, she was forced to dance, for it was a
magical dance. The faster he played, the more violent springs was she
forced to make, and the thorns tore her clothes from her body, and
pricked her and wounded her till she bled, and as he did not stop, she
had to dance till she lay dead on the ground.

As they were now set free, Roland said: 'Now I will go to my father
and arrange for the wedding.' 'Then in the meantime I will stay here
and wait for you,' said the girl, 'and that no one may recognize me, I
will change myself into a red stone landmark.' Then Roland went away,
and the girl stood like a red landmark in the field and waited for her
beloved. But when Roland got home, he fell into the snares of another,
who so fascinated him that he forgot the maiden. The poor girl
remained there a long time, but at length, as he did not return at
all, she was sad, and changed herself into a flower, and thought:
'Someone will surely come this way, and trample me down.'

It befell, however, that a shepherd kept his sheep in the field and
saw the flower, and as it was so pretty, plucked it, took it with him,
and laid it away in his chest. From that time forth, strange things
happened in the shepherd's house. When he arose in the morning, all
the work was already done, the room was swept, the table and benches
cleaned, the fire in the hearth was lighted, and the water was
fetched, and at noon, when he came home, the table was laid, and a
good dinner served. He could not conceive how this came to pass, for
he never saw a human being in his house, and no one could have
concealed himself in it. He was certainly pleased with this good
attendance, but still at last he was so afraid that he went to a wise
woman and asked for her advice. The wise woman said: 'There is some
enchantment behind it, listen very early some morning if anything is
moving in the room, and if you see anything, no matter what it is,
throw a white cloth over it, and then the magic will be stopped.'

The shepherd did as she bade him, and next morning just as day dawned,
he saw the chest open, and the flower come out. Swiftly he sprang
towards it, and threw a white cloth over it. Instantly the
transformation came to an end, and a beautiful girl stood before him,
who admitted to him that she had been the flower, and that up to this
time she had attended to his house-keeping. She told him her story,
and as she pleased him he asked her if she would marry him, but she
answered: 'No,' for she wanted to remain faithful to her sweetheart
Roland, although he had deserted her. Nevertheless, she promised not
to go away, but to continue keeping house for the shepherd.

And now the time drew near when Roland's wedding was to be celebrated,
and then, according to an old custom in the country, it was announced
that all the girls were to be present at it, and sing in honour of the
bridal pair. When the faithful maiden heard of this, she grew so sad
that she thought her heart would break, and she would not go thither,
but the other girls came and took her. When it came to her turn to
sing, she stepped back, until at last she was the only one left, and
then she could not refuse. But when she began her song, and it reached
Roland's ears, he sprang up and cried: 'I know the voice, that is the
true bride, I will have no other!' Everything he had forgotten, and
which had vanished from his mind, had suddenly come home again to his
heart. Then the faithful maiden held her wedding with her sweetheart
Roland, and grief came to an end and joy began.



\chapter{SNOWDROP}

It was the middle of winter, when the broad flakes of snow were
falling around, that the queen of a country many thousand miles off
sat working at her window. The frame of the window was made of fine
black ebony, and as she sat looking out upon the snow, she pricked her
finger, and three drops of blood fell upon it. Then she gazed
thoughtfully upon the red drops that sprinkled the white snow, and
said, 'Would that my little daughter may be as white as that snow, as
red as that blood, and as black as this ebony windowframe!' And so the
little girl really did grow up; her skin was as white as snow, her
cheeks as rosy as the blood, and her hair as black as ebony; and she
was called Snowdrop.

But this queen died; and the king soon married another wife, who
became queen, and was very beautiful, but so vain that she could not
bear to think that anyone could be handsomer than she was. She had a
fairy looking-glass, to which she used to go, and then she would gaze
upon herself in it, and say:

\begin{verse}
 'Tell me, glass, tell me true!\\
  Of all the ladies in the land,\\
  Who is fairest, tell me, who?'
\end{verse}

And the glass had always answered:

\begin{verse}
 'Thou, queen, art the fairest in all the land.'
\end{verse}

But Snowdrop grew more and more beautiful; and when she was seven
years old she was as bright as the day, and fairer than the queen
herself. Then the glass one day answered the queen, when she went to
look in it as usual:

\begin{verse}
 'Thou, queen, art fair, and beauteous to see,\\
  But Snowdrop is lovelier far than thee!'
\end{verse}

When she heard this she turned pale with rage and envy, and called to
one of her servants, and said, 'Take Snowdrop away into the wide wood,
that I may never see her any more.' Then the servant led her away; but
his heart melted when Snowdrop begged him to spare her life, and he
said, 'I will not hurt you, thou pretty child.' So he left her by
herself; and though he thought it most likely that the wild beasts
would tear her in pieces, he felt as if a great weight were taken off
his heart when he had made up his mind not to kill her but to leave
her to her fate, with the chance of someone finding and saving her.

Then poor Snowdrop wandered along through the wood in great fear; and
the wild beasts roared about her, but none did her any harm. In the
evening she came to a cottage among the hills, and went in to rest,
for her little feet would carry her no further. Everything was spruce
and neat in the cottage: on the table was spread a white cloth, and
there were seven little plates, seven little loaves, and seven little
glasses with wine in them; and seven knives and forks laid in order;
and by the wall stood seven little beds. As she was very hungry, she
picked a little piece of each loaf and drank a very little wine out of
each glass; and after that she thought she would lie down and rest. So
she tried all the little beds; but one was too long, and another was
too short, till at last the seventh suited her: and there she laid
herself down and went to sleep.

By and by in came the masters of the cottage. Now they were seven
little dwarfs, that lived among the mountains, and dug and searched
for gold. They lighted up their seven lamps, and saw at once that all
was not right. The first said, 'Who has been sitting on my stool?' The
second, 'Who has been eating off my plate?' The third, 'Who has been
picking my bread?' The fourth, 'Who has been meddling with my spoon?'
The fifth, 'Who has been handling my fork?' The sixth, 'Who has been
cutting with my knife?' The seventh, 'Who has been drinking my wine?'
Then the first looked round and said, 'Who has been lying on my bed?'
And the rest came running to him, and everyone cried out that somebody
had been upon his bed. But the seventh saw Snowdrop, and called all
his brethren to come and see her; and they cried out with wonder and
astonishment and brought their lamps to look at her, and said, 'Good
heavens! what a lovely child she is!' And they were very glad to see
her, and took care not to wake her; and the seventh dwarf slept an
hour with each of the other dwarfs in turn, till the night was gone.

In the morning Snowdrop told them all her story; and they pitied her,
and said if she would keep all things in order, and cook and wash and
knit and spin for them, she might stay where she was, and they would
take good care of her. Then they went out all day long to their work,
seeking for gold and silver in the mountains: but Snowdrop was left at
home; and they warned her, and said, 'The queen will soon find out
where you are, so take care and let no one in.'

But the queen, now that she thought Snowdrop was dead, believed that
she must be the handsomest lady in the land; and she went to her glass
and said:

\begin{verse}
 'Tell me, glass, tell me true!\\
  Of all the ladies in the land,\\
  Who is fairest, tell me, who?'
\end{verse}

And the glass answered:

\begin{verse}
 'Thou, queen, art the fairest in all this land:\\
  But over the hills, in the greenwood shade,\\
  Where the seven dwarfs their dwelling have made,\\
  There Snowdrop is hiding her head; and she\\
  Is lovelier far, O queen! than thee.'
\end{verse}

Then the queen was very much frightened; for she knew that the glass
always spoke the truth, and was sure that the servant had betrayed
her. And she could not bear to think that anyone lived who was more
beautiful than she was; so she dressed herself up as an old pedlar,
and went her way over the hills, to the place where the dwarfs dwelt.
Then she knocked at the door, and cried, 'Fine wares to sell!'
Snowdrop looked out at the window, and said, 'Good day, good woman!
what have you to sell?' 'Good wares, fine wares,' said she; 'laces and
bobbins of all colours.' 'I will let the old lady in; she seems to be
a very good sort of body,' thought Snowdrop, as she ran down and
unbolted the door. 'Bless me!' said the old woman, 'how badly your
stays are laced! Let me lace them up with one of my nice new laces.'
Snowdrop did not dream of any mischief; so she stood before the old
woman; but she set to work so nimbly, and pulled the lace so tight,
that Snowdrop's breath was stopped, and she fell down as if she were
dead. 'There's an end to all thy beauty,' said the spiteful queen,
and went away home.

In the evening the seven dwarfs came home; and I need not say how
grieved they were to see their faithful Snowdrop stretched out upon
the ground, as if she was quite dead. However, they lifted her up, and
when they found what ailed her, they cut the lace; and in a little
time she began to breathe, and very soon came to life again. Then they
said, 'The old woman was the queen herself; take care another time,
and let no one in when we are away.'

When the queen got home, she went straight to her glass, and spoke to
it as before; but to her great grief it still said:

\begin{verse}
 'Thou, queen, art the fairest in all this land:\\
  But over the hills, in the greenwood shade,\\
  Where the seven dwarfs their dwelling have made,\\
  There Snowdrop is hiding her head; and she\\
  Is lovelier far, O queen! than thee.'
\end{verse}

Then the blood ran cold in her heart with spite and malice, to see
that Snowdrop still lived; and she dressed herself up again, but in
quite another dress from the one she wore before, and took with her a
poisoned comb. When she reached the dwarfs' cottage, she knocked at
the door, and cried, 'Fine wares to sell!' But Snowdrop said, 'I dare
not let anyone in.' Then the queen said, 'Only look at my beautiful
combs!' and gave her the poisoned one. And it looked so pretty, that
she took it up and put it into her hair to try it; but the moment it
touched her head, the poison was so powerful that she fell down
senseless. 'There you may lie,' said the queen, and went her way. But
by good luck the dwarfs came in very early that evening; and when they
saw Snowdrop lying on the ground, they thought what had happened, and
soon found the poisoned comb. And when they took it away she got well,
and told them all that had passed; and they warned her once more not
to open the door to anyone.

Meantime the queen went home to her glass, and shook with rage when
she read the very same answer as before; and she said, 'Snowdrop shall
die, if it cost me my life.' So she went by herself into her chamber,
and got ready a poisoned apple: the outside looked very rosy and
tempting, but whoever tasted it was sure to die. Then she dressed
herself up as a peasant's wife, and travelled over the hills to the
dwarfs' cottage, and knocked at the door; but Snowdrop put her head
out of the window and said, 'I dare not let anyone in, for the dwarfs
have told me not.' 'Do as you please,' said the old woman, 'but at any
rate take this pretty apple; I will give it you.' 'No,' said Snowdrop,
'I dare not take it.' 'You silly girl!' answered the other, 'what are
you afraid of? Do you think it is poisoned? Come! do you eat one part,
and I will eat the other.' Now the apple was so made up that one side
was good, though the other side was poisoned. Then Snowdrop was much
tempted to taste, for the apple looked so very nice; and when she saw
the old woman eat, she could wait no longer. But she had scarcely put
the piece into her mouth, when she fell down dead upon the ground.
'This time nothing will save thee,' said the queen; and she went home
to her glass, and at last it said:

\begin{verse}
 'Thou, queen, art the fairest of all the fair.'
\end{verse}

And then her wicked heart was glad, and as happy as such a heart could
be.

When evening came, and the dwarfs had gone home, they found Snowdrop
lying on the ground: no breath came from her lips, and they were
afraid that she was quite dead. They lifted her up, and combed her
hair, and washed her face with wine and water; but all was in vain,
for the little girl seemed quite dead. So they laid her down upon a
bier, and all seven watched and bewailed her three whole days; and
then they thought they would bury her: but her cheeks were still rosy;
and her face looked just as it did while she was alive; so they said,
'We will never bury her in the cold ground.' And they made a coffin of
glass, so that they might still look at her, and wrote upon it in
golden letters what her name was, and that she was a king's daughter.
And the coffin was set among the hills, and one of the dwarfs always
sat by it and watched. And the birds of the air came too, and bemoaned
Snowdrop; and first of all came an owl, and then a raven, and at last
a dove, and sat by her side.

And thus Snowdrop lay for a long, long time, and still only looked as
though she was asleep; for she was even now as white as snow, and as
red as blood, and as black as ebony. At last a prince came and called
at the dwarfs' house; and he saw Snowdrop, and read what was written
in golden letters. Then he offered the dwarfs money, and prayed and
besought them to let him take her away; but they said, 'We will not
part with her for all the gold in the world.' At last, however, they
had pity on him, and gave him the coffin; but the moment he lifted it
up to carry it home with him, the piece of apple fell from between her
lips, and Snowdrop awoke, and said, 'Where am I?' And the prince said,
'Thou art quite safe with me.'

Then he told her all that had happened, and said, 'I love you far
better than all the world; so come with me to my father's palace, and
you shall be my wife.' And Snowdrop consented, and went home with the
prince; and everything was got ready with great pomp and splendour for
their wedding.

To the feast was asked, among the rest, Snowdrop's old enemy the
queen; and as she was dressing herself in fine rich clothes, she
looked in the glass and said:

\begin{verse}
 'Tell me, glass, tell me true!\\
  Of all the ladies in the land,\\
  Who is fairest, tell me, who?'
\end{verse}

And the glass answered:

\begin{verse}
 'Thou, lady, art loveliest here, I ween;\\
  But lovelier far is the new-made queen.'
\end{verse}

When she heard this she started with rage; but her envy and curiosity
were so great, that she could not help setting out to see the bride.
And when she got there, and saw that it was no other than Snowdrop,
who, as she thought, had been dead a long while, she choked with rage,
and fell down and died: but Snowdrop and the prince lived and reigned
happily over that land many, many years; and sometimes they went up
into the mountains, and paid a visit to the little dwarfs, who had
been so kind to Snowdrop in her time of need.



\chapter{THE PINK}

There was once upon a time a queen to whom God had given no children.
Every morning she went into the garden and prayed to God in heaven to
bestow on her a son or a daughter. Then an angel from heaven came to
her and said: 'Be at rest, you shall have a son with the power of
wishing, so that whatsoever in the world he wishes for, that shall he
have.' Then she went to the king, and told him the joyful tidings, and
when the time was come she gave birth to a son, and the king was
filled with gladness.

Every morning she went with the child to the garden where the wild
beasts were kept, and washed herself there in a clear stream. It
happened once when the child was a little older, that it was lying in
her arms and she fell asleep. Then came the old cook, who knew that
the child had the power of wishing, and stole it away, and he took a
hen, and cut it in pieces, and dropped some of its blood on the
queen's apron and on her dress. Then he carried the child away to a
secret place, where a nurse was obliged to suckle it, and he ran to
the king and accused the queen of having allowed her child to be taken
from her by the wild beasts. When the king saw the blood on her apron,
he believed this, fell into such a passion that he ordered a high
tower to be built, in which neither sun nor moon could be seen and had
his wife put into it, and walled up. Here she was to stay for seven
years without meat or drink, and die of hunger. But God sent two
angels from heaven in the shape of white doves, which flew to her
twice a day, and carried her food until the seven years were over.

The cook, however, thought to himself: 'If the child has the power of
wishing, and I am here, he might very easily get me into trouble.' So
he left the palace and went to the boy, who was already big enough to
speak, and said to him: 'Wish for a beautiful palace for yourself with
a garden, and all else that pertains to it.' Scarcely were the words
out of the boy's mouth, when everything was there that he had wished
for. After a while the cook said to him: 'It is not well for you to be
so alone, wish for a pretty girl as a companion.' Then the king's son
wished for one, and she immediately stood before him, and was more
beautiful than any painter could have painted her. The two played
together, and loved each other with all their hearts, and the old cook
went out hunting like a nobleman. The thought occurred to him,
however, that the king's son might some day wish to be with his
father, and thus bring him into great peril. So he went out and took
the maiden aside, and said: 'Tonight when the boy is asleep, go to his
bed and plunge this knife into his heart, and bring me his heart and
tongue, and if you do not do it, you shall lose your life.' Thereupon
he went away, and when he returned next day she had not done it, and
said: 'Why should I shed the blood of an innocent boy who has never
harmed anyone?' The cook once more said: 'If you do not do it, it
shall cost you your own life.' When he had gone away, she had a little
hind brought to her, and ordered her to be killed, and took her heart
and tongue, and laid them on a plate, and when she saw the old man
coming, she said to the boy: 'Lie down in your bed, and draw the
clothes over you.' Then the wicked wretch came in and said: 'Where are
the boy's heart and tongue?' The girl reached the plate to him, but
the king's son threw off the quilt, and said: 'You old sinner, why did
you want to kill me? Now will I pronounce thy sentence. You shall
become a black poodle and have a gold collar round your neck, and
shall eat burning coals, till the flames burst forth from your
throat.' And when he had spoken these words, the old man was changed
into a poodle dog, and had a gold collar round his neck, and the cooks
were ordered to bring up some live coals, and these he ate, until the
flames broke forth from his throat. The king's son remained there a
short while longer, and he thought of his mother, and wondered if she
were still alive. At length he said to the maiden: 'I will go home to
my own country; if you will go with me, I will provide for you.' 'Ah,'
she replied, 'the way is so long, and what shall I do in a strange
land where I am unknown?' As she did not seem quite willing, and as
they could not be parted from each other, he wished that she might be
changed into a beautiful pink, and took her with him. Then he went
away to his own country, and the poodle had to run after him. He went
to the tower in which his mother was confined, and as it was so high,
he wished for a ladder which would reach up to the very top. Then he
mounted up and looked inside, and cried: 'Beloved mother, Lady Queen,
are you still alive, or are you dead?' She answered: 'I have just
eaten, and am still satisfied,' for she thought the angels were there.
Said he: 'I am your dear son, whom the wild beasts were said to have
torn from your arms; but I am alive still, and will soon set you
free.' Then he descended again, and went to his father, and caused
himself to be announced as a strange huntsman, and asked if he could
offer him service. The king said yes, if he was skilful and could get
game for him, he should come to him, but that deer had never taken up
their quarters in any part of the district or country. Then the
huntsman promised to procure as much game for him as he could possibly
use at the royal table. So he summoned all the huntsmen together, and
bade them go out into the forest with him. And he went with them and
made them form a great circle, open at one end where he stationed
himself, and began to wish. Two hundred deer and more came running
inside the circle at once, and the huntsmen shot them. Then they were
all placed on sixty country carts, and driven home to the king, and
for once he was able to deck his table with game, after having had
none at all for years.

Now the king felt great joy at this, and commanded that his entire
household should eat with him next day, and made a great feast. When
they were all assembled together, he said to the huntsman: 'As you are
so clever, you shall sit by me.' He replied: 'Lord King, your majesty
must excuse me, I am a poor huntsman.' But the king insisted on it,
and said: 'You shall sit by me,' until he did it. Whilst he was
sitting there, he thought of his dearest mother, and wished that one
of the king's principal servants would begin to speak of her, and
would ask how it was faring with the queen in the tower, and if she
were alive still, or had perished. Hardly had he formed the wish than
the marshal began, and said: 'Your majesty, we live joyously here, but
how is the queen living in the tower? Is she still alive, or has she
died?' But the king replied: 'She let my dear son be torn to pieces by
wild beasts; I will not have her named.' Then the huntsman arose and
said: 'Gracious lord father she is alive still, and I am her son, and
I was not carried away by wild beasts, but by that wretch the old
cook, who tore me from her arms when she was asleep, and sprinkled her
apron with the blood of a chicken.' Thereupon he took the dog with the
golden collar, and said: 'That is the wretch!' and caused live coals
to be brought, and these the dog was compelled to devour before the
sight of all, until flames burst forth from its throat. On this the
huntsman asked the king if he would like to see the dog in his true
shape, and wished him back into the form of the cook, in the which he
stood immediately, with his white apron, and his knife by his side.
When the king saw him he fell into a passion, and ordered him to be
cast into the deepest dungeon. Then the huntsman spoke further and
said: 'Father, will you see the maiden who brought me up so tenderly
and who was afterwards to murder me, but did not do it, though her own
life depended on it?' The king replied: 'Yes, I would like to see
her.' The son said: 'Most gracious father, I will show her to you in
the form of a beautiful flower,' and he thrust his hand into his
pocket and brought forth the pink, and placed it on the royal table,
and it was so beautiful that the king had never seen one to equal it.
Then the son said: 'Now will I show her to you in her own form,' and
wished that she might become a maiden, and she stood there looking so
beautiful that no painter could have made her look more so.

And the king sent two waiting-maids and two attendants into the tower,
to fetch the queen and bring her to the royal table. But when she was
led in she ate nothing, and said: 'The gracious and merciful God who
has supported me in the tower, will soon set me free.' She lived three
days more, and then died happily, and when she was buried, the two
white doves which had brought her food to the tower, and were angels
of heaven, followed her body and seated themselves on her grave. The
aged king ordered the cook to be torn in four pieces, but grief
consumed the king's own heart, and he soon died. His son married the
beautiful maiden whom he had brought with him as a flower in his
pocket, and whether they are still alive or not, is known to God.



\chapter{CLEVER ELSIE}

There was once a man who had a daughter who was called Clever Elsie.
And when she had grown up her father said: 'We will get her married.'
'Yes,' said the mother, 'if only someone would come who would have
her.' At length a man came from a distance and wooed her, who was
called Hans; but he stipulated that Clever Elsie should be really
smart. 'Oh,' said the father, 'she has plenty of good sense'; and the
mother said: 'Oh, she can see the wind coming up the street, and hear
the flies coughing.' 'Well,' said Hans, 'if she is not really smart, I
won't have her.' When they were sitting at dinner and had eaten, the
mother said: 'Elsie, go into the cellar and fetch some beer.' Then
Clever Elsie took the pitcher from the wall, went into the cellar, and
tapped the lid briskly as she went, so that the time might not appear
long. When she was below she fetched herself a chair, and set it
before the barrel so that she had no need to stoop, and did not hurt
her back or do herself any unexpected injury. Then she placed the can
before her, and turned the tap, and while the beer was running she
would not let her eyes be idle, but looked up at the wall, and after
much peering here and there, saw a pick-axe exactly above her, which
the masons had accidentally left there.

Then Clever Elsie began to weep and said: 'If I get Hans, and we have
a child, and he grows big, and we send him into the cellar here to
draw beer, then the pick-axe will fall on his head and kill him.' Then
she sat and wept and screamed with all the strength of her body, over
the misfortune which lay before her. Those upstairs waited for the
drink, but Clever Elsie still did not come. Then the woman said to the
servant: 'Just go down into the cellar and see where Elsie is.' The
maid went and found her sitting in front of the barrel, screaming
loudly. 'Elsie why do you weep?' asked the maid. 'Ah,' she answered,
'have I not reason to weep? If I get Hans, and we have a child, and he
grows big, and has to draw beer here, the pick-axe will perhaps fall
on his head, and kill him.' Then said the maid: 'What a clever Elsie
we have!' and sat down beside her and began loudly to weep over the
misfortune. After a while, as the maid did not come back, and those
upstairs were thirsty for the beer, the man said to the boy: 'Just go
down into the cellar and see where Elsie and the girl are.' The boy
went down, and there sat Clever Elsie and the girl both weeping
together. Then he asked: 'Why are you weeping?' 'Ah,' said Elsie,
'have I not reason to weep? If I get Hans, and we have a child, and he
grows big, and has to draw beer here, the pick-axe will fall on his
head and kill him.' Then said the boy: 'What a clever Elsie we have!'
and sat down by her, and likewise began to howl loudly. Upstairs they
waited for the boy, but as he still did not return, the man said to
the woman: 'Just go down into the cellar and see where Elsie is!' The
woman went down, and found all three in the midst of their
lamentations, and inquired what was the cause; then Elsie told her
also that her future child was to be killed by the pick-axe, when it
grew big and had to draw beer, and the pick-axe fell down. Then said
the mother likewise: 'What a clever Elsie we have!' and sat down and
wept with them. The man upstairs waited a short time, but as his wife
did not come back and his thirst grew ever greater, he said: 'I must
go into the cellar myself and see where Elsie is.' But when he got
into the cellar, and they were all sitting together crying, and he
heard the reason, and that Elsie's child was the cause, and the Elsie
might perhaps bring one into the world some day, and that he might be
killed by the pick-axe, if he should happen to be sitting beneath it,
drawing beer just at the very time when it fell down, he cried: 'Oh,
what a clever Elsie!' and sat down, and likewise wept with them. The
bridegroom stayed upstairs alone for along time; then as no one would
come back he thought: 'They must be waiting for me below: I too must
go there and see what they are about.' When he got down, the five of
them were sitting screaming and lamenting quite piteously, each out-
doing the other. 'What misfortune has happened then?' asked he. 'Ah,
dear Hans,' said Elsie, 'if we marry each other and have a child, and
he is big, and we perhaps send him here to draw something to drink,
then the pick-axe which has been left up there might dash his brains
out if it were to fall down, so have we not reason to weep?' 'Come,'
said Hans, 'more understanding than that is not needed for my
household, as you are such a clever Elsie, I will have you,' and
seized her hand, took her upstairs with him, and married her.

After Hans had had her some time, he said: 'Wife, I am going out to
work and earn some money for us; go into the field and cut the corn
that we may have some bread.' 'Yes, dear Hans, I will do that.' After
Hans had gone away, she cooked herself some good broth and took it
into the field with her. When she came to the field she said to
herself: 'What shall I do; shall I cut first, or shall I eat first?
Oh, I will eat first.' Then she drank her cup of broth and when she
was fully satisfied, she once more said: 'What shall I do? Shall I cut
first, or shall I sleep first? I will sleep first.' Then she lay down
among the corn and fell asleep. Hans had been at home for a long time,
but Elsie did not come; then said he: 'What a clever Elsie I have; she
is so industrious that she does not even come home to eat.' But when
evening came and she still stayed away, Hans went out to see what she
had cut, but nothing was cut, and she was lying among the corn asleep.
Then Hans hastened home and brought a fowler's net with little bells
and hung it round about her, and she still went on sleeping. Then he
ran home, shut the house-door, and sat down in his chair and worked.
At length, when it was quite dark, Clever Elsie awoke and when she got
up there was a jingling all round about her, and the bells rang at
each step which she took. Then she was alarmed, and became uncertain
whether she really was Clever Elsie or not, and said: 'Is it I, or is
it not I?' But she knew not what answer to make to this, and stood for
a time in doubt; at length she thought: 'I will go home and ask if it
be I, or if it be not I, they will be sure to know.' She ran to the
door of her own house, but it was shut; then she knocked at the window
and cried: 'Hans, is Elsie within?' 'Yes,' answered Hans, 'she is
within.' Hereupon she was terrified, and said: 'Ah, heavens! Then it
is not I,' and went to another door; but when the people heard the
jingling of the bells they would not open it, and she could get in
nowhere. Then she ran out of the village, and no one has seen her
since.



\chapter{THE MISER IN THE BUSH}

A farmer had a faithful and diligent servant, who had worked hard for
him three years, without having been paid any wages. At last it came
into the man's head that he would not go on thus without pay any
longer; so he went to his master, and said, 'I have worked hard for
you a long time, I will trust to you to give me what I deserve to have
for my trouble.' The farmer was a sad miser, and knew that his man was
very simple-hearted; so he took out threepence, and gave him for every
year's service a penny. The poor fellow thought it was a great deal of
money to have, and said to himself, 'Why should I work hard, and live
here on bad fare any longer? I can now travel into the wide world, and
make myself merry.' With that he put his money into his purse, and set
out, roaming over hill and valley.

As he jogged along over the fields, singing and dancing, a little
dwarf met him, and asked him what made him so merry. 'Why, what should
make me down-hearted?' said he; 'I am sound in health and rich in
purse, what should I care for? I have saved up my three years'
earnings and have it all safe in my pocket.' 'How much may it come
to?' said the little man. 'Full threepence,' replied the countryman.
'I wish you would give them to me,' said the other; 'I am very poor.'
Then the man pitied him, and gave him all he had; and the little dwarf
said in return, 'As you have such a kind honest heart, I will grant
you three wishes--one for every penny; so choose whatever you like.'
Then the countryman rejoiced at his good luck, and said, 'I like many
things better than money: first, I will have a bow that will bring
down everything I shoot at; secondly, a fiddle that will set everyone
dancing that hears me play upon it; and thirdly, I should like that
everyone should grant what I ask.' The dwarf said he should have his
three wishes; so he gave him the bow and fiddle, and went his way.

Our honest friend journeyed on his way too; and if he was merry
before, he was now ten times more so. He had not gone far before he
met an old miser: close by them stood a tree, and on the topmost twig
sat a thrush singing away most joyfully. 'Oh, what a pretty bird!'
said the miser; 'I would give a great deal of money to have such a
one.' 'If that's all,' said the countryman, 'I will soon bring it
down.' Then he took up his bow, and down fell the thrush into the
bushes at the foot of the tree. The miser crept into the bush to find
it; but directly he had got into the middle, his companion took up his
fiddle and played away, and the miser began to dance and spring about,
capering higher and higher in the air. The thorns soon began to tear
his clothes till they all hung in rags about him, and he himself was
all scratched and wounded, so that the blood ran down. 'Oh, for
heaven's sake!' cried the miser, 'Master! master! pray let the fiddle
alone. What have I done to deserve this?' 'Thou hast shaved many a
poor soul close enough,' said the other; 'thou art only meeting thy
reward': so he played up another tune. Then the miser began to beg and
promise, and offered money for his liberty; but he did not come up to
the musician's price for some time, and he danced him along brisker
and brisker, and the miser bid higher and higher, till at last he
offered a round hundred of florins that he had in his purse, and had
just gained by cheating some poor fellow. When the countryman saw so
much money, he said, 'I will agree to your proposal.' So he took the
purse, put up his fiddle, and travelled on very pleased with his
bargain.

Meanwhile the miser crept out of the bush half-naked and in a piteous
plight, and began to ponder how he should take his revenge, and serve
his late companion some trick. At last he went to the judge, and
complained that a rascal had robbed him of his money, and beaten him
into the bargain; and that the fellow who did it carried a bow at his
back and a fiddle hung round his neck. Then the judge sent out his
officers to bring up the accused wherever they should find him; and he
was soon caught and brought up to be tried.

The miser began to tell his tale, and said he had been robbed of his
money. 'No, you gave it me for playing a tune to you.' said the
countryman; but the judge told him that was not likely, and cut the
matter short by ordering him off to the gallows.

So away he was taken; but as he stood on the steps he said, 'My Lord
Judge, grant me one last request.' 'Anything but thy life,' replied
the other. 'No,' said he, 'I do not ask my life; only to let me play
upon my fiddle for the last time.' The miser cried out, 'Oh, no! no!
for heaven's sake don't listen to him! don't listen to him!' But the
judge said, 'It is only this once, he will soon have done.' The fact
was, he could not refuse the request, on account of the dwarf's third
gift.

Then the miser said, 'Bind me fast, bind me fast, for pity's sake.'
But the countryman seized his fiddle, and struck up a tune, and at the
first note judge, clerks, and jailer were in motion; all began
capering, and no one could hold the miser. At the second note the
hangman let his prisoner go, and danced also, and by the time he had
played the first bar of the tune, all were dancing together--judge,
court, and miser, and all the people who had followed to look on. At
first the thing was merry and pleasant enough; but when it had gone on
a while, and there seemed to be no end of playing or dancing, they
began to cry out, and beg him to leave off; but he stopped not a whit
the more for their entreaties, till the judge not only gave him his
life, but promised to return him the hundred florins.

Then he called to the miser, and said, 'Tell us now, you vagabond,
where you got that gold, or I shall play on for your amusement only,'
'I stole it,' said the miser in the presence of all the people; 'I
acknowledge that I stole it, and that you earned it fairly.' Then the
countryman stopped his fiddle, and left the miser to take his place at
the gallows.



\chapter{ASHPUTTEL}

The wife of a rich man fell sick; and when she felt that her end drew
nigh, she called her only daughter to her bed-side, and said, 'Always
be a good girl, and I will look down from heaven and watch over you.'
Soon afterwards she shut her eyes and died, and was buried in the
garden; and the little girl went every day to her grave and wept, and
was always good and kind to all about her. And the snow fell and
spread a beautiful white covering over the grave; but by the time the
spring came, and the sun had melted it away again, her father had
married another wife. This new wife had two daughters of her own, that
she brought home with her; they were fair in face but foul at heart,
and it was now a sorry time for the poor little girl. 'What does the
good-for-nothing want in the parlour?' said they; 'they who would eat
bread should first earn it; away with the kitchen-maid!' Then they
took away her fine clothes, and gave her an old grey frock to put on,
and laughed at her, and turned her into the kitchen.

There she was forced to do hard work; to rise early before daylight,
to bring the water, to make the fire, to cook and to wash. Besides
that, the sisters plagued her in all sorts of ways, and laughed at
her. In the evening when she was tired, she had no bed to lie down on,
but was made to lie by the hearth among the ashes; and as this, of
course, made her always dusty and dirty, they called her Ashputtel.

It happened once that the father was going to the fair, and asked his
wife's daughters what he should bring them. 'Fine clothes,' said the
first; 'Pearls and diamonds,' cried the second. 'Now, child,' said he
to his own daughter, 'what will you have?' 'The first twig, dear
father, that brushes against your hat when you turn your face to come
homewards,' said she. Then he bought for the first two the fine
clothes and pearls and diamonds they had asked for: and on his way
home, as he rode through a green copse, a hazel twig brushed against
him, and almost pushed off his hat: so he broke it off and brought it
away; and when he got home he gave it to his daughter. Then she took
it, and went to her mother's grave and planted it there; and cried so
much that it was watered with her tears; and there it grew and became
a fine tree. Three times every day she went to it and cried; and soon
a little bird came and built its nest upon the tree, and talked with
her, and watched over her, and brought her whatever she wished for.

Now it happened that the king of that land held a feast, which was to
last three days; and out of those who came to it his son was to choose
a bride for himself. Ashputtel's two sisters were asked to come; so
they called her up, and said, 'Now, comb our hair, brush our shoes,
and tie our sashes for us, for we are going to dance at the king's
feast.' Then she did as she was told; but when all was done she could
not help crying, for she thought to herself, she should so have liked
to have gone with them to the ball; and at last she begged her mother
very hard to let her go. 'You, Ashputtel!' said she; 'you who have
nothing to wear, no clothes at all, and who cannot even dance--you
want to go to the ball? And when she kept on begging, she said at
last, to get rid of her, 'I will throw this dishful of peas into the
ash-heap, and if in two hours' time you have picked them all out, you
shall go to the feast too.'

Then she threw the peas down among the ashes, but the little maiden
ran out at the back door into the garden, and cried out:

\begin{verse}
 'Hither, hither, through the sky,\\
  Turtle-doves and linnets, fly!\\
  Blackbird, thrush, and chaffinch gay,\\
  Hither, hither, haste away!\\
  One and all come help me, quick!\\
  Haste ye, haste ye!--pick, pick, pick!'
\end{verse}

Then first came two white doves, flying in at the kitchen window; next
came two turtle-doves; and after them came all the little birds under
heaven, chirping and fluttering in: and they flew down into the ashes.
And the little doves stooped their heads down and set to work, pick,
pick, pick; and then the others began to pick, pick, pick: and among
them all they soon picked out all the good grain, and put it into a
dish but left the ashes. Long before the end of the hour the work was
quite done, and all flew out again at the windows.

Then Ashputtel brought the dish to her mother, overjoyed at the
thought that now she should go to the ball. But the mother said, 'No,
no! you slut, you have no clothes, and cannot dance; you shall not
go.' And when Ashputtel begged very hard to go, she said, 'If you can
in one hour's time pick two of those dishes of peas out of the ashes,
you shall go too.' And thus she thought she should at least get rid of
her. So she shook two dishes of peas into the ashes.

But the little maiden went out into the garden at the back of the
house, and cried out as before:

\begin{verse}
 'Hither, hither, through the sky,\\
  Turtle-doves and linnets, fly!\\
  Blackbird, thrush, and chaffinch gay,\\
  Hither, hither, haste away!\\
  One and all come help me, quick!\\
  Haste ye, haste ye!--pick, pick, pick!'
\end{verse}

Then first came two white doves in at the kitchen window; next came
two turtle-doves; and after them came all the little birds under
heaven, chirping and hopping about. And they flew down into the ashes;
and the little doves put their heads down and set to work, pick, pick,
pick; and then the others began pick, pick, pick; and they put all the
good grain into the dishes, and left all the ashes. Before half an
hour's time all was done, and out they flew again. And then Ashputtel
took the dishes to her mother, rejoicing to think that she should now
go to the ball. But her mother said, 'It is all of no use, you cannot
go; you have no clothes, and cannot dance, and you would only put us
to shame': and off she went with her two daughters to the ball.

Now when all were gone, and nobody left at home, Ashputtel went
sorrowfully and sat down under the hazel-tree, and cried out:

\begin{verse}
 'Shake, shake, hazel-tree,\\
  Gold and silver over me!'
\end{verse}

Then her friend the bird flew out of the tree, and brought a gold and
silver dress for her, and slippers of spangled silk; and she put them
on, and followed her sisters to the feast. But they did not know her,
and thought it must be some strange princess, she looked so fine and
beautiful in her rich clothes; and they never once thought of
Ashputtel, taking it for granted that she was safe at home in the
dirt.

The king's son soon came up to her, and took her by the hand and
danced with her, and no one else: and he never left her hand; but when
anyone else came to ask her to dance, he said, 'This lady is dancing
with me.'

Thus they danced till a late hour of the night; and then she wanted to
go home: and the king's son said, 'I shall go and take care of you to
your home'; for he wanted to see where the beautiful maiden lived. But
she slipped away from him, unawares, and ran off towards home; and as
the prince followed her, she jumped up into the pigeon-house and shut
the door. Then he waited till her father came home, and told him that
the unknown maiden, who had been at the feast, had hid herself in the
pigeon-house. But when they had broken open the door they found no one
within; and as they came back into the house, Ashputtel was lying, as
she always did, in her dirty frock by the ashes, and her dim little
lamp was burning in the chimney. For she had run as quickly as she
could through the pigeon-house and on to the hazel-tree, and had there
taken off her beautiful clothes, and put them beneath the tree, that
the bird might carry them away, and had lain down again amid the ashes
in her little grey frock.

The next day when the feast was again held, and her father, mother,
and sisters were gone, Ashputtel went to the hazel-tree, and said:

\begin{verse}
 'Shake, shake, hazel-tree,\\
  Gold and silver over me!'
\end{verse}

And the bird came and brought a still finer dress than the one she had
worn the day before. And when she came in it to the ball, everyone
wondered at her beauty: but the king's son, who was waiting for her,
took her by the hand, and danced with her; and when anyone asked her
to dance, he said as before, 'This lady is dancing with me.'

When night came she wanted to go home; and the king's son followed
here as before, that he might see into what house she went: but she
sprang away from him all at once into the garden behind her father's
house. In this garden stood a fine large pear-tree full of ripe fruit;
and Ashputtel, not knowing where to hide herself, jumped up into it
without being seen. Then the king's son lost sight of her, and could
not find out where she was gone, but waited till her father came home,
and said to him, 'The unknown lady who danced with me has slipped
away, and I think she must have sprung into the pear-tree.' The father
thought to himself, 'Can it be Ashputtel?' So he had an axe brought;
and they cut down the tree, but found no one upon it. And when they
came back into the kitchen, there lay Ashputtel among the ashes; for
she had slipped down on the other side of the tree, and carried her
beautiful clothes back to the bird at the hazel-tree, and then put on
her little grey frock.

The third day, when her father and mother and sisters were gone, she
went again into the garden, and said:

\begin{verse}
 'Shake, shake, hazel-tree,\\
  Gold and silver over me!'
\end{verse}

Then her kind friend the bird brought a dress still finer than the
former one, and slippers which were all of gold: so that when she came
to the feast no one knew what to say, for wonder at her beauty: and
the king's son danced with nobody but her; and when anyone else asked
her to dance, he said, 'This lady is /my/ partner, sir.'

When night came she wanted to go home; and the king's son would go
with her, and said to himself, 'I will not lose her this time'; but,
however, she again slipped away from him, though in such a hurry that
she dropped her left golden slipper upon the stairs.

The prince took the shoe, and went the next day to the king his
father, and said, 'I will take for my wife the lady that this golden
slipper fits.' Then both the sisters were overjoyed to hear it; for
they had beautiful feet, and had no doubt that they could wear the
golden slipper. The eldest went first into the room where the slipper
was, and wanted to try it on, and the mother stood by. But her great
toe could not go into it, and the shoe was altogether much too small
for her. Then the mother gave her a knife, and said, 'Never mind, cut
it off; when you are queen you will not care about toes; you will not
want to walk.' So the silly girl cut off her great toe, and thus
squeezed on the shoe, and went to the king's son. Then he took her for
his bride, and set her beside him on his horse, and rode away with her
homewards.

But on their way home they had to pass by the hazel-tree that
Ashputtel had planted; and on the branch sat a little dove singing:

\begin{verse}
 'Back again! back again! look to the shoe!\\
  The shoe is too small, and not made for you!\\
  Prince! prince! look again for thy bride,\\
  For she's not the true one that sits by thy side.'
\end{verse}

Then the prince got down and looked at her foot; and he saw, by the
blood that streamed from it, what a trick she had played him. So he
turned his horse round, and brought the false bride back to her home,
and said, 'This is not the right bride; let the other sister try and
put on the slipper.' Then she went into the room and got her foot into
the shoe, all but the heel, which was too large. But her mother
squeezed it in till the blood came, and took her to the king's son:
and he set her as his bride by his side on his horse, and rode away
with her.

But when they came to the hazel-tree the little dove sat there still,
and sang:

\begin{verse}
 'Back again! back again! look to the shoe!\\
  The shoe is too small, and not made for you!\\
  Prince! prince! look again for thy bride,\\
  For she's not the true one that sits by thy side.'
\end{verse}

Then he looked down, and saw that the blood streamed so much from the
shoe, that her white stockings were quite red. So he turned his horse
and brought her also back again. 'This is not the true bride,' said he
to the father; 'have you no other daughters?' 'No,' said he; 'there is
only a little dirty Ashputtel here, the child of my first wife; I am
sure she cannot be the bride.' The prince told him to send her. But
the mother said, 'No, no, she is much too dirty; she will not dare to
show herself.' However, the prince would have her come; and she first
washed her face and hands, and then went in and curtsied to him, and
he reached her the golden slipper. Then she took her clumsy shoe off
her left foot, and put on the golden slipper; and it fitted her as if
it had been made for her. And when he drew near and looked at her face
he knew her, and said, 'This is the right bride.' But the mother and
both the sisters were frightened, and turned pale with anger as he
took Ashputtel on his horse, and rode away with her. And when they
came to the hazel-tree, the white dove sang:

\begin{verse}
 'Home! home! look at the shoe!\\
  Princess! the shoe was made for you!\\
  Prince! prince! take home thy bride,\\
  For she is the true one that sits by thy side!'
\end{verse}

And when the dove had done its song, it came flying, and perched upon
her right shoulder, and so went home with her.



\chapter{THE WHITE SNAKE}

A long time ago there lived a king who was famed for his wisdom
through all the land. Nothing was hidden from him, and it seemed as if
news of the most secret things was brought to him through the air. But
he had a strange custom; every day after dinner, when the table was
cleared, and no one else was present, a trusty servant had to bring
him one more dish. It was covered, however, and even the servant did
not know what was in it, neither did anyone know, for the king never
took off the cover to eat of it until he was quite alone.

This had gone on for a long time, when one day the servant, who took
away the dish, was overcome with such curiosity that he could not help
carrying the dish into his room. When he had carefully locked the
door, he lifted up the cover, and saw a white snake lying on the dish.
But when he saw it he could not deny himself the pleasure of tasting
it, so he cut of a little bit and put it into his mouth. No sooner had
it touched his tongue than he heard a strange whispering of little
voices outside his window. He went and listened, and then noticed that
it was the sparrows who were chattering together, and telling one
another of all kinds of things which they had seen in the fields and
woods. Eating the snake had given him power of understanding the
language of animals.

Now it so happened that on this very day the queen lost her most
beautiful ring, and suspicion of having stolen it fell upon this
trusty servant, who was allowed to go everywhere. The king ordered the
man to be brought before him, and threatened with angry words that
unless he could before the morrow point out the thief, he himself
should be looked upon as guilty and executed. In vain he declared his
innocence; he was dismissed with no better answer.

In his trouble and fear he went down into the courtyard and took
thought how to help himself out of his trouble. Now some ducks were
sitting together quietly by a brook and taking their rest; and, whilst
they were making their feathers smooth with their bills, they were
having a confidential conversation together. The servant stood by and
listened. They were telling one another of all the places where they
had been waddling about all the morning, and what good food they had
found; and one said in a pitiful tone: 'Something lies heavy on my
stomach; as I was eating in haste I swallowed a ring which lay under
the queen's window.' The servant at once seized her by the neck,
carried her to the kitchen, and said to the cook: 'Here is a fine
duck; pray, kill her.' 'Yes,' said the cook, and weighed her in his
hand; 'she has spared no trouble to fatten herself, and has been
waiting to be roasted long enough.' So he cut off her head, and as she
was being dressed for the spit, the queen's ring was found inside her.

The servant could now easily prove his innocence; and the king, to
make amends for the wrong, allowed him to ask a favour, and promised
him the best place in the court that he could wish for. The servant
refused everything, and only asked for a horse and some money for
travelling, as he had a mind to see the world and go about a little.
When his request was granted he set out on his way, and one day came
to a pond, where he saw three fishes caught in the reeds and gasping
for water. Now, though it is said that fishes are dumb, he heard them
lamenting that they must perish so miserably, and, as he had a kind
heart, he got off his horse and put the three prisoners back into the
water. They leapt with delight, put out their heads, and cried to him:
'We will remember you and repay you for saving us!'

He rode on, and after a while it seemed to him that he heard a voice
in the sand at his feet. He listened, and heard an ant-king complain:
'Why cannot folks, with their clumsy beasts, keep off our bodies? That
stupid horse, with his heavy hoofs, has been treading down my people
without mercy!' So he turned on to a side path and the ant-king cried
out to him: 'We will remember you--one good turn deserves another!'

The path led him into a wood, and there he saw two old ravens standing
by their nest, and throwing out their young ones. 'Out with you, you
idle, good-for-nothing creatures!' cried they; 'we cannot find food
for you any longer; you are big enough, and can provide for
yourselves.' But the poor young ravens lay upon the ground, flapping
their wings, and crying: 'Oh, what helpless chicks we are! We must
shift for ourselves, and yet we cannot fly! What can we do, but lie
here and starve?' So the good young fellow alighted and killed his
horse with his sword, and gave it to them for food. Then they came
hopping up to it, satisfied their hunger, and cried: 'We will remember
you--one good turn deserves another!'

And now he had to use his own legs, and when he had walked a long way,
he came to a large city. There was a great noise and crowd in the
streets, and a man rode up on horseback, crying aloud: 'The king's
daughter wants a husband; but whoever seeks her hand must perform a
hard task, and if he does not succeed he will forfeit his life.' Many
had already made the attempt, but in vain; nevertheless when the youth
saw the king's daughter he was so overcome by her great beauty that he
forgot all danger, went before the king, and declared himself a
suitor.

So he was led out to the sea, and a gold ring was thrown into it,
before his eyes; then the king ordered him to fetch this ring up from
the bottom of the sea, and added: 'If you come up again without it you
will be thrown in again and again until you perish amid the waves.'
All the people grieved for the handsome youth; then they went away,
leaving him alone by the sea.

He stood on the shore and considered what he should do, when suddenly
he saw three fishes come swimming towards him, and they were the very
fishes whose lives he had saved. The one in the middle held a mussel
in its mouth, which it laid on the shore at the youth's feet, and when
he had taken it up and opened it, there lay the gold ring in the
shell. Full of joy he took it to the king and expected that he would
grant him the promised reward.

But when the proud princess perceived that he was not her equal in
birth, she scorned him, and required him first to perform another
task. She went down into the garden and strewed with her own hands ten
sacksful of millet-seed on the grass; then she said: 'Tomorrow morning
before sunrise these must be picked up, and not a single grain be
wanting.'

The youth sat down in the garden and considered how it might be
possible to perform this task, but he could think of nothing, and
there he sat sorrowfully awaiting the break of day, when he should be
led to death. But as soon as the first rays of the sun shone into the
garden he saw all the ten sacks standing side by side, quite full, and
not a single grain was missing. The ant-king had come in the night
with thousands and thousands of ants, and the grateful creatures had
by great industry picked up all the millet-seed and gathered them into
the sacks.

Presently the king's daughter herself came down into the garden, and
was amazed to see that the young man had done the task she had given
him. But she could not yet conquer her proud heart, and said:
'Although he has performed both the tasks, he shall not be my husband
until he had brought me an apple from the Tree of Life.' The youth did
not know where the Tree of Life stood, but he set out, and would have
gone on for ever, as long as his legs would carry him, though he had
no hope of finding it. After he had wandered through three kingdoms,
he came one evening to a wood, and lay down under a tree to sleep. But
he heard a rustling in the branches, and a golden apple fell into his
hand. At the same time three ravens flew down to him, perched
themselves upon his knee, and said: 'We are the three young ravens
whom you saved from starving; when we had grown big, and heard that
you were seeking the Golden Apple, we flew over the sea to the end of
the world, where the Tree of Life stands, and have brought you the
apple.' The youth, full of joy, set out homewards, and took the Golden
Apple to the king's beautiful daughter, who had now no more excuses
left to make. They cut the Apple of Life in two and ate it together;
and then her heart became full of love for him, and they lived in
undisturbed happiness to a great age.



\chapter{THE WOLF AND THE SEVEN LITTLE KIDS}

There was once upon a time an old goat who had seven little kids, and
loved them with all the love of a mother for her children. One day she
wanted to go into the forest and fetch some food. So she called all
seven to her and said: 'Dear children, I have to go into the forest,
be on your guard against the wolf; if he comes in, he will devour you
all--skin, hair, and everything. The wretch often disguises himself,
but you will know him at once by his rough voice and his black feet.'
The kids said: 'Dear mother, we will take good care of ourselves; you
may go away without any anxiety.' Then the old one bleated, and went
on her way with an easy mind.

It was not long before someone knocked at the house-door and called:
'Open the door, dear children; your mother is here, and has brought
something back with her for each of you.' But the little kids knew
that it was the wolf, by the rough voice. 'We will not open the door,'
cried they, 'you are not our mother. She has a soft, pleasant voice,
but your voice is rough; you are the wolf!' Then the wolf went away to
a shopkeeper and bought himself a great lump of chalk, ate this and
made his voice soft with it. Then he came back, knocked at the door of
the house, and called: 'Open the door, dear children, your mother is
here and has brought something back with her for each of you.' But the
wolf had laid his black paws against the window, and the children saw
them and cried: 'We will not open the door, our mother has not black
feet like you: you are the wolf!' Then the wolf ran to a baker and
said: 'I have hurt my feet, rub some dough over them for me.' And when
the baker had rubbed his feet over, he ran to the miller and said:
'Strew some white meal over my feet for me.' The miller thought to
himself: 'The wolf wants to deceive someone,' and refused; but the
wolf said: 'If you will not do it, I will devour you.' Then the miller
was afraid, and made his paws white for him. Truly, this is the way of
mankind.

So now the wretch went for the third time to the house-door, knocked
at it and said: 'Open the door for me, children, your dear little
mother has come home, and has brought every one of you something back
from the forest with her.' The little kids cried: 'First show us your
paws that we may know if you are our dear little mother.' Then he put
his paws in through the window and when the kids saw that they were
white, they believed that all he said was true, and opened the door.
But who should come in but the wolf! They were terrified and wanted to
hide themselves. One sprang under the table, the second into the bed,
the third into the stove, the fourth into the kitchen, the fifth into
the cupboard, the sixth under the washing-bowl, and the seventh into
the clock-case. But the wolf found them all, and used no great
ceremony; one after the other he swallowed them down his throat. The
youngest, who was in the clock-case, was the only one he did not find.
When the wolf had satisfied his appetite he took himself off, laid
himself down under a tree in the green meadow outside, and began to
sleep. Soon afterwards the old goat came home again from the forest.
Ah! what a sight she saw there! The house-door stood wide open. The
table, chairs, and benches were thrown down, the washing-bowl lay
broken to pieces, and the quilts and pillows were pulled off the bed.
She sought her children, but they were nowhere to be found. She called
them one after another by name, but no one answered. At last, when she
came to the youngest, a soft voice cried: 'Dear mother, I am in the
clock-case.' She took the kid out, and it told her that the wolf had
come and had eaten all the others. Then you may imagine how she wept
over her poor children.

At length in her grief she went out, and the youngest kid ran with
her. When they came to the meadow, there lay the wolf by the tree and
snored so loud that the branches shook. She looked at him on every
side and saw that something was moving and struggling in his gorged
belly. 'Ah, heavens,' she said, 'is it possible that my poor children
whom he has swallowed down for his supper, can be still alive?' Then
the kid had to run home and fetch scissors, and a needle and thread,
and the goat cut open the monster's stomach, and hardly had she made
one cut, than one little kid thrust its head out, and when she had cut
farther, all six sprang out one after another, and were all still
alive, and had suffered no injury whatever, for in his greediness the
monster had swallowed them down whole. What rejoicing there was! They
embraced their dear mother, and jumped like a tailor at his wedding.
The mother, however, said: 'Now go and look for some big stones, and
we will fill the wicked beast's stomach with them while he is still
asleep.' Then the seven kids dragged the stones thither with all
speed, and put as many of them into this stomach as they could get in;
and the mother sewed him up again in the greatest haste, so that he
was not aware of anything and never once stirred.

When the wolf at length had had his fill of sleep, he got on his legs,
and as the stones in his stomach made him very thirsty, he wanted to
go to a well to drink. But when he began to walk and to move about,
the stones in his stomach knocked against each other and rattled. Then
cried he:

\begin{verse}
 'What rumbles and tumbles\\
  Against my poor bones?\\
  I thought 'twas six kids,\\
  But it feels like big stones.'
\end{verse}

And when he got to the well and stooped over the water to drink, the
heavy stones made him fall in, and he drowned miserably. When the
seven kids saw that, they came running to the spot and cried aloud:
'The wolf is dead! The wolf is dead!' and danced for joy round about
the well with their mother.



\chapter{THE QUEEN BEE}

Two kings' sons once upon a time went into the world to seek their
fortunes; but they soon fell into a wasteful foolish way of living, so
that they could not return home again. Then their brother, who was a
little insignificant dwarf, went out to seek for his brothers: but
when he had found them they only laughed at him, to think that he, who
was so young and simple, should try to travel through the world, when
they, who were so much wiser, had been unable to get on. However, they
all set out on their journey together, and came at last to an ant-
hill. The two elder brothers would have pulled it down, in order to
see how the poor ants in their fright would run about and carry off
their eggs. But the little dwarf said, 'Let the poor things enjoy
themselves, I will not suffer you to trouble them.'

So on they went, and came to a lake where many many ducks were
swimming about. The two brothers wanted to catch two, and roast them.
But the dwarf said, 'Let the poor things enjoy themselves, you shall
not kill them.' Next they came to a bees'-nest in a hollow tree, and
there was so much honey that it ran down the trunk; and the two
brothers wanted to light a fire under the tree and kill the bees, so
as to get their honey. But the dwarf held them back, and said, 'Let
the pretty insects enjoy themselves, I cannot let you burn them.'

At length the three brothers came to a castle: and as they passed by
the stables they saw fine horses standing there, but all were of
marble, and no man was to be seen. Then they went through all the
rooms, till they came to a door on which were three locks: but in the
middle of the door was a wicket, so that they could look into the next
room. There they saw a little grey old man sitting at a table; and
they called to him once or twice, but he did not hear: however, they
called a third time, and then he rose and came out to them.

He said nothing, but took hold of them and led them to a beautiful
table covered with all sorts of good things: and when they had eaten
and drunk, he showed each of them to a bed-chamber.

The next morning he came to the eldest and took him to a marble table,
where there were three tablets, containing an account of the means by
which the castle might be disenchanted. The first tablet said: 'In the
wood, under the moss, lie the thousand pearls belonging to the king's
daughter; they must all be found: and if one be missing by set of sun,
he who seeks them will be turned into marble.'

The eldest brother set out, and sought for the pearls the whole day:
but the evening came, and he had not found the first hundred: so he
was turned into stone as the tablet had foretold.

The next day the second brother undertook the task; but he succeeded
no better than the first; for he could only find the second hundred of
the pearls; and therefore he too was turned into stone.

At last came the little dwarf's turn; and he looked in the moss; but
it was so hard to find the pearls, and the job was so tiresome!--so he
sat down upon a stone and cried. And as he sat there, the king of the
ants (whose life he had saved) came to help him, with five thousand
ants; and it was not long before they had found all the pearls and
laid them in a heap.

The second tablet said: 'The key of the princess's bed-chamber must be
fished up out of the lake.' And as the dwarf came to the brink of it,
he saw the two ducks whose lives he had saved swimming about; and they
dived down and soon brought in the key from the bottom.

The third task was the hardest. It was to choose out the youngest and
the best of the king's three daughters. Now they were all beautiful,
and all exactly alike: but he was told that the eldest had eaten a
piece of sugar, the next some sweet syrup, and the youngest a spoonful
of honey; so he was to guess which it was that had eaten the honey.

Then came the queen of the bees, who had been saved by the little
dwarf from the fire, and she tried the lips of all three; but at last
she sat upon the lips of the one that had eaten the honey: and so the
dwarf knew which was the youngest. Thus the spell was broken, and all
who had been turned into stones awoke, and took their proper forms.
And the dwarf married the youngest and the best of the princesses, and
was king after her father's death; but his two brothers married the
other two sisters.



\chapter{THE ELVES AND THE SHOEMAKER}

There was once a shoemaker, who worked very hard and was very honest:
but still he could not earn enough to live upon; and at last all he
had in the world was gone, save just leather enough to make one pair
of shoes.

Then he cut his leather out, all ready to make up the next day,
meaning to rise early in the morning to his work. His conscience was
clear and his heart light amidst all his troubles; so he went
peaceably to bed, left all his cares to Heaven, and soon fell asleep.
In the morning after he had said his prayers, he sat himself down to
his work; when, to his great wonder, there stood the shoes all ready
made, upon the table. The good man knew not what to say or think at
such an odd thing happening. He looked at the workmanship; there was
not one false stitch in the whole job; all was so neat and true, that
it was quite a masterpiece.

The same day a customer came in, and the shoes suited him so well that
he willingly paid a price higher than usual for them; and the poor
shoemaker, with the money, bought leather enough to make two pairs
more. In the evening he cut out the work, and went to bed early, that
he might get up and begin betimes next day; but he was saved all the
trouble, for when he got up in the morning the work was done ready to
his hand. Soon in came buyers, who paid him handsomely for his goods,
so that he bought leather enough for four pair more. He cut out the
work again overnight and found it done in the morning, as before; and
so it went on for some time: what was got ready in the evening was
always done by daybreak, and the good man soon became thriving and
well off again.

One evening, about Christmas-time, as he and his wife were sitting
over the fire chatting together, he said to her, 'I should like to sit
up and watch tonight, that we may see who it is that comes and does my
work for me.' The wife liked the thought; so they left a light
burning, and hid themselves in a corner of the room, behind a curtain
that was hung up there, and watched what would happen.

As soon as it was midnight, there came in two little naked dwarfs; and
they sat themselves upon the shoemaker's bench, took up all the work
that was cut out, and began to ply with their little fingers,
stitching and rapping and tapping away at such a rate, that the
shoemaker was all wonder, and could not take his eyes off them. And on
they went, till the job was quite done, and the shoes stood ready for
use upon the table. This was long before daybreak; and then they
bustled away as quick as lightning.

The next day the wife said to the shoemaker. 'These little wights have
made us rich, and we ought to be thankful to them, and do them a good
turn if we can. I am quite sorry to see them run about as they do; and
indeed it is not very decent, for they have nothing upon their backs
to keep off the cold. I'll tell you what, I will make each of them a
shirt, and a coat and waistcoat, and a pair of pantaloons into the
bargain; and do you make each of them a little pair of shoes.'

The thought pleased the good cobbler very much; and one evening, when
all the things were ready, they laid them on the table, instead of the
work that they used to cut out, and then went and hid themselves, to
watch what the little elves would do.

About midnight in they came, dancing and skipping, hopped round the
room, and then went to sit down to their work as usual; but when they
saw the clothes lying for them, they laughed and chuckled, and seemed
mightily delighted.

Then they dressed themselves in the twinkling of an eye, and danced
and capered and sprang about, as merry as could be; till at last they
danced out at the door, and away over the green.

The good couple saw them no more; but everything went well with them
from that time forward, as long as they lived.



\chapter{THE JUNIPER-TREE}

Long, long ago, some two thousand years or so, there lived a rich man
with a good and beautiful wife. They loved each other dearly, but
sorrowed much that they had no children. So greatly did they desire to
have one, that the wife prayed for it day and night, but still they
remained childless.

In front of the house there was a court, in which grew a juniper-tree.
One winter's day the wife stood under the tree to peel some apples,
and as she was peeling them, she cut her finger, and the blood fell on
the snow. 'Ah,' sighed the woman heavily, 'if I had but a child, as
red as blood and as white as snow,' and as she spoke the words, her
heart grew light within her, and it seemed to her that her wish was
granted, and she returned to the house feeling glad and comforted. A
month passed, and the snow had all disappeared; then another month
went by, and all the earth was green. So the months followed one
another, and first the trees budded in the woods, and soon the green
branches grew thickly intertwined, and then the blossoms began to
fall. Once again the wife stood under the juniper-tree, and it was so
full of sweet scent that her heart leaped for joy, and she was so
overcome with her happiness, that she fell on her knees. Presently the
fruit became round and firm, and she was glad and at peace; but when
they were fully ripe she picked the berries and ate eagerly of them,
and then she grew sad and ill. A little while later she called her
husband, and said to him, weeping. 'If I die, bury me under the
juniper-tree.' Then she felt comforted and happy again, and before
another month had passed she had a little child, and when she saw that
it was as white as snow and as red as blood, her joy was so great that
she died.

Her husband buried her under the juniper-tree, and wept bitterly for
her. By degrees, however, his sorrow grew less, and although at times
he still grieved over his loss, he was able to go about as usual, and
later on he married again.

He now had a little daughter born to him; the child of his first wife
was a boy, who was as red as blood and as white as snow. The mother
loved her daughter very much, and when she looked at her and then
looked at the boy, it pierced her heart to think that he would always
stand in the way of her own child, and she was continually thinking
how she could get the whole of the property for her. This evil thought
took possession of her more and more, and made her behave very
unkindly to the boy. She drove him from place to place with cuffings
and buffetings, so that the poor child went about in fear, and had no
peace from the time he left school to the time he went back.

One day the little daughter came running to her mother in the store-
room, and said, 'Mother, give me an apple.' 'Yes, my child,' said the
wife, and she gave her a beautiful apple out of the chest; the chest
had a very heavy lid and a large iron lock.

'Mother,' said the little daughter again, 'may not brother have one
too?' The mother was angry at this, but she answered, 'Yes, when he
comes out of school.'

Just then she looked out of the window and saw him coming, and it
seemed as if an evil spirit entered into her, for she snatched the
apple out of her little daughter's hand, and said, 'You shall not have
one before your brother.' She threw the apple into the chest and shut
it to. The little boy now came in, and the evil spirit in the wife
made her say kindly to him, 'My son, will you have an apple?' but she
gave him a wicked look. 'Mother,' said the boy, 'how dreadful you
look! Yes, give me an apple.' The thought came to her that she would
kill him. 'Come with me,' she said, and she lifted up the lid of the
chest; 'take one out for yourself.' And as he bent over to do so, the
evil spirit urged her, and crash! down went the lid, and off went the
little boy's head. Then she was overwhelmed with fear at the thought
of what she had done. 'If only I can prevent anyone knowing that I did
it,' she thought. So she went upstairs to her room, and took a white
handkerchief out of her top drawer; then she set the boy's head again
on his shoulders, and bound it with the handkerchief so that nothing
could be seen, and placed him on a chair by the door with an apple in
his hand.

Soon after this, little Marleen came up to her mother who was stirring
a pot of boiling water over the fire, and said, 'Mother, brother is
sitting by the door with an apple in his hand, and he looks so pale;
and when I asked him to give me the apple, he did not answer, and that
frightened me.'

'Go to him again,' said her mother, 'and if he does not answer, give
him a box on the ear.' So little Marleen went, and said, 'Brother,
give me that apple,' but he did not say a word; then she gave him a
box on the ear, and his head rolled off. She was so terrified at this,
that she ran crying and screaming to her mother. 'Oh!' she said, 'I
have knocked off brother's head,' and then she wept and wept, and
nothing would stop her.

'What have you done!' said her mother, 'but no one must know about it,
so you must keep silence; what is done can't be undone; we will make
him into puddings.' And she took the little boy and cut him up, made
him into puddings, and put him in the pot. But Marleen stood looking
on, and wept and wept, and her tears fell into the pot, so that there
was no need of salt.

Presently the father came home and sat down to his dinner; he asked,
'Where is my son?' The mother said nothing, but gave him a large dish
of black pudding, and Marleen still wept without ceasing.

The father again asked, 'Where is my son?'

'Oh,' answered the wife, 'he is gone into the country to his mother's
great uncle; he is going to stay there some time.'

'What has he gone there for, and he never even said goodbye to me!'

'Well, he likes being there, and he told me he should be away quite
six weeks; he is well looked after there.'

'I feel very unhappy about it,' said the husband, 'in case it should
not be all right, and he ought to have said goodbye to me.'

With this he went on with his dinner, and said, 'Little Marleen, why
do you weep? Brother will soon be back.' Then he asked his wife for
more pudding, and as he ate, he threw the bones under the table.

Little Marleen went upstairs and took her best silk handkerchief out
of her bottom drawer, and in it she wrapped all the bones from under
the table and carried them outside, and all the time she did nothing
but weep. Then she laid them in the green grass under the juniper-
tree, and she had no sooner done so, then all her sadness seemed to
leave her, and she wept no more. And now the juniper-tree began to
move, and the branches waved backwards and forwards, first away from
one another, and then together again, as it might be someone clapping
their hands for joy. After this a mist came round the tree, and in the
midst of it there was a burning as of fire, and out of the fire there
flew a beautiful bird, that rose high into the air, singing
magnificently, and when it could no more be seen, the juniper-tree
stood there as before, and the silk handkerchief and the bones were
gone.

Little Marleen now felt as lighthearted and happy as if her brother
were still alive, and she went back to the house and sat down
cheerfully to the table and ate.

The bird flew away and alighted on the house of a goldsmith and began
to sing:

\begin{verse}
 'My mother killed her little son;\\
  My father grieved when I was gone;\\
  My sister loved me best of all;\\
  She laid her kerchief over me,\\
  And took my bones that they might lie\\
  Underneath the juniper-tree\\
  Kywitt, Kywitt, what a beautiful bird am I!'
\end{verse}

The goldsmith was in his workshop making a gold chain, when he heard
the song of the bird on his roof. He thought it so beautiful that he
got up and ran out, and as he crossed the threshold he lost one of his
slippers. But he ran on into the middle of the street, with a slipper
on one foot and a sock on the other; he still had on his apron, and
still held the gold chain and the pincers in his hands, and so he
stood gazing up at the bird, while the sun came shining brightly down
on the street.

'Bird,' he said, 'how beautifully you sing! Sing me that song again.'

'Nay,' said the bird, 'I do not sing twice for nothing. Give that gold
chain, and I will sing it you again.'

'Here is the chain, take it,' said the goldsmith. 'Only sing me that
again.'

The bird flew down and took the gold chain in his right claw, and then
he alighted again in front of the goldsmith and sang:

\begin{verse}
 'My mother killed her little son;\\
  My father grieved when I was gone;\\
  My sister loved me best of all;\\
  She laid her kerchief over me,\\
  And took my bones that they might lie\\
  Underneath the juniper-tree\\
  Kywitt, Kywitt, what a beautiful bird am I!'
\end{verse}

Then he flew away, and settled on the roof of a shoemaker's house and
sang:

\begin{verse}
 'My mother killed her little son;\\
  My father grieved when I was gone;\\
  My sister loved me best of all;\\
  She laid her kerchief over me,\\
  And took my bones that they might lie\\
  Underneath the juniper-tree\\
  Kywitt, Kywitt, what a beautiful bird am I!'
\end{verse}

The shoemaker heard him, and he jumped up and ran out in his shirt-
sleeves, and stood looking up at the bird on the roof with his hand
over his eyes to keep himself from being blinded by the sun.

'Bird,' he said, 'how beautifully you sing!' Then he called through
the door to his wife: 'Wife, come out; here is a bird, come and look
at it and hear how beautifully it sings.' Then he called his daughter
and the children, then the apprentices, girls and boys, and they all
ran up the street to look at the bird, and saw how splendid it was
with its red and green feathers, and its neck like burnished gold, and
eyes like two bright stars in its head.

'Bird,' said the shoemaker, 'sing me that song again.'

'Nay,' answered the bird, 'I do not sing twice for nothing; you must
give me something.'

'Wife,' said the man, 'go into the garret; on the upper shelf you will
see a pair of red shoes; bring them to me.' The wife went in and
fetched the shoes.

'There, bird,' said the shoemaker, 'now sing me that song again.'

The bird flew down and took the red shoes in his left claw, and then
he went back to the roof and sang:

\begin{verse}
 'My mother killed her little son;\\
  My father grieved when I was gone;\\
  My sister loved me best of all;\\
  She laid her kerchief over me,\\
  And took my bones that they might lie\\
  Underneath the juniper-tree\\
  Kywitt, Kywitt, what a beautiful bird am I!'
\end{verse}

When he had finished, he flew away. He had the chain in his right claw
and the shoes in his left, and he flew right away to a mill, and the
mill went 'Click clack, click clack, click clack.' Inside the mill
were twenty of the miller's men hewing a stone, and as they went 'Hick
hack, hick hack, hick hack,' the mill went 'Click clack, click clack,
click clack.'

The bird settled on a lime-tree in front of the mill and sang:

\begin{verse}
 'My mother killed her little son;
\end{verse}

then one of the men left off,

\begin{verse}
  My father grieved when I was gone;
\end{verse}

two more men left off and listened,

\begin{verse}
  My sister loved me best of all;
\end{verse}

then four more left off,

\begin{verse}
  She laid her kerchief over me,\\
  And took my bones that they might lie
\end{verse}

now there were only eight at work,

\begin{verse}
  Underneath
\end{verse}

And now only five,

\begin{verse}
the juniper-tree.
\end{verse}

and now only one,

\begin{verse}
  Kywitt, Kywitt, what a beautiful bird am I!'
\end{verse}

then he looked up and the last one had left off work.

'Bird,' he said, 'what a beautiful song that is you sing! Let me hear
it too; sing it again.'

'Nay,' answered the bird, 'I do not sing twice for nothing; give me
that millstone, and I will sing it again.'

'If it belonged to me alone,' said the man, 'you should have it.'

'Yes, yes,' said the others: 'if he will sing again, he can have it.'

The bird came down, and all the twenty millers set to and lifted up
the stone with a beam; then the bird put his head through the hole and
took the stone round his neck like a collar, and flew back with it to
the tree and sang--

\begin{verse}
 'My mother killed her little son;\\
  My father grieved when I was gone;\\
  My sister loved me best of all;\\
  She laid her kerchief over me,\\
  And took my bones that they might lie\\
  Underneath the juniper-tree\\
  Kywitt, Kywitt, what a beautiful bird am I!'
\end{verse}

And when he had finished his song, he spread his wings, and with the
chain in his right claw, the shoes in his left, and the millstone
round his neck, he flew right away to his father's house.

The father, the mother, and little Marleen were having their dinner.

'How lighthearted I feel,' said the father, 'so pleased and cheerful.'

'And I,' said the mother, 'I feel so uneasy, as if a heavy
thunderstorm were coming.'

But little Marleen sat and wept and wept.

Then the bird came flying towards the house and settled on the roof.

'I do feel so happy,' said the father, 'and how beautifully the sun
shines; I feel just as if I were going to see an old friend again.'

'Ah!' said the wife, 'and I am so full of distress and uneasiness that
my teeth chatter, and I feel as if there were a fire in my veins,' and
she tore open her dress; and all the while little Marleen sat in the
corner and wept, and the plate on her knees was wet with her tears.

The bird now flew to the juniper-tree and began singing:

\begin{verse}
 'My mother killed her little son;
\end{verse}

the mother shut her eyes and her ears, that she might see and hear
nothing, but there was a roaring sound in her ears like that of a
violent storm, and in her eyes a burning and flashing like lightning:

\begin{verse}
  My father grieved when I was gone;
\end{verse}

'Look, mother,' said the man, 'at the beautiful bird that is singing
so magnificently; and how warm and bright the sun is, and what a
delicious scent of spice in the air!'

\begin{verse}
  My sister loved me best of all;
\end{verse}

then little Marleen laid her head down on her knees and sobbed.

'I must go outside and see the bird nearer,' said the man.

'Ah, do not go!' cried the wife. 'I feel as if the whole house were in
flames!'

But the man went out and looked at the bird.

\begin{verse}
 She laid her kerchief over me,\\
 And took my bones that they might lie\\
 Underneath the juniper-tree\\
 Kywitt, Kywitt, what a beautiful bird am I!'
\end{verse}

With that the bird let fall the gold chain, and it fell just round the
man's neck, so that it fitted him exactly.

He went inside, and said, 'See, what a splendid bird that is; he has
given me this beautiful gold chain, and looks so beautiful himself.'

But the wife was in such fear and trouble, that she fell on the floor,
and her cap fell from her head.

Then the bird began again:

\begin{verse}
 'My mother killed her little son;
\end{verse}

'Ah me!' cried the wife, 'if I were but a thousand feet beneath the
earth, that I might not hear that song.'

\begin{verse}
  My father grieved when I was gone;
\end{verse}

then the woman fell down again as if dead.

\begin{verse}
  My sister loved me best of all;
\end{verse}

'Well,' said little Marleen, 'I will go out too and see if the bird
will give me anything.'

So she went out.

\begin{verse}
  She laid her kerchief over me,\\
  And took my bones that they might lie
\end{verse}

and he threw down the shoes to her,

\begin{verse}
  Underneath the juniper-tree\\
  Kywitt, Kywitt, what a beautiful bird am I!'
\end{verse}

And she now felt quite happy and lighthearted; she put on the shoes
and danced and jumped about in them. 'I was so miserable,' she said,
'when I came out, but that has all passed away; that is indeed a
splendid bird, and he has given me a pair of red shoes.'

The wife sprang up, with her hair standing out from her head like
flames of fire. 'Then I will go out too,' she said, 'and see if it
will lighten my misery, for I feel as if the world were coming to an
end.'

But as she crossed the threshold, crash! the bird threw the millstone
down on her head, and she was crushed to death.

The father and little Marleen heard the sound and ran out, but they
only saw mist and flame and fire rising from the spot, and when these
had passed, there stood the little brother, and he took the father and
little Marleen by the hand; then they all three rejoiced, and went
inside together and sat down to their dinners and ate.



\chapter{THE TURNIP}

There were two brothers who were both soldiers; the one was rich and
the other poor. The poor man thought he would try to better himself;
so, pulling off his red coat, he became a gardener, and dug his ground
well, and sowed turnips.

When the seed came up, there was one plant bigger than all the rest;
and it kept getting larger and larger, and seemed as if it would never
cease growing; so that it might have been called the prince of turnips
for there never was such a one seen before, and never will again. At
last it was so big that it filled a cart, and two oxen could hardly
draw it; and the gardener knew not what in the world to do with it,
nor whether it would be a blessing or a curse to him. One day he said
to himself, 'What shall I do with it? if I sell it, it will bring no
more than another; and for eating, the little turnips are better than
this; the best thing perhaps is to carry it and give it to the king as
a mark of respect.'

Then he yoked his oxen, and drew the turnip to the court, and gave it
to the king. 'What a wonderful thing!' said the king; 'I have seen
many strange things, but such a monster as this I never saw. Where did
you get the seed? or is it only your good luck? If so, you are a true
child of fortune.' 'Ah, no!' answered the gardener, 'I am no child of
fortune; I am a poor soldier, who never could get enough to live upon;
so I laid aside my red coat, and set to work, tilling the ground. I
have a brother, who is rich, and your majesty knows him well, and all
the world knows him; but because I am poor, everybody forgets me.'

The king then took pity on him, and said, 'You shall be poor no
longer. I will give you so much that you shall be even richer than
your brother.' Then he gave him gold and lands and flocks, and made
him so rich that his brother's fortune could not at all be compared
with his.

When the brother heard of all this, and how a turnip had made the
gardener so rich, he envied him sorely, and bethought himself how he
could contrive to get the same good fortune for himself. However, he
determined to manage more cleverly than his brother, and got together
a rich present of gold and fine horses for the king; and thought he
must have a much larger gift in return; for if his brother had
received so much for only a turnip, what must his present be wroth?

The king took the gift very graciously, and said he knew not what to
give in return more valuable and wonderful than the great turnip; so
the soldier was forced to put it into a cart, and drag it home with
him. When he reached home, he knew not upon whom to vent his rage and
spite; and at length wicked thoughts came into his head, and he
resolved to kill his brother.

So he hired some villains to murder him; and having shown them where
to lie in ambush, he went to his brother, and said, 'Dear brother, I
have found a hidden treasure; let us go and dig it up, and share it
between us.' The other had no suspicions of his roguery: so they went
out together, and as they were travelling along, the murderers rushed
out upon him, bound him, and were going to hang him on a tree.

But whilst they were getting all ready, they heard the trampling of a
horse at a distance, which so frightened them that they pushed their
prisoner neck and shoulders together into a sack, and swung him up by
a cord to the tree, where they left him dangling, and ran away.
Meantime he worked and worked away, till he made a hole large enough
to put out his head.

When the horseman came up, he proved to be a student, a merry fellow,
who was journeying along on his nag, and singing as he went. As soon
as the man in the sack saw him passing under the tree, he cried out,
'Good morning! good morning to thee, my friend!' The student looked
about everywhere; and seeing no one, and not knowing where the voice
came from, cried out, 'Who calls me?'

Then the man in the tree answered, 'Lift up thine eyes, for behold
here I sit in the sack of wisdom; here have I, in a short time,
learned great and wondrous things. Compared to this seat, all the
learning of the schools is as empty air. A little longer, and I shall
know all that man can know, and shall come forth wiser than the wisest
of mankind. Here I discern the signs and motions of the heavens and
the stars; the laws that control the winds; the number of the sands on
the seashore; the healing of the sick; the virtues of all simples, of
birds, and of precious stones. Wert thou but once here, my friend,
though wouldst feel and own the power of knowledge.

The student listened to all this and wondered much; at last he said,
'Blessed be the day and hour when I found you; cannot you contrive to
let me into the sack for a little while?' Then the other answered, as
if very unwillingly, 'A little space I may allow thee to sit here, if
thou wilt reward me well and entreat me kindly; but thou must tarry
yet an hour below, till I have learnt some little matters that are yet
unknown to me.'

So the student sat himself down and waited a while; but the time hung
heavy upon him, and he begged earnestly that he might ascend
forthwith, for his thirst for knowledge was great. Then the other
pretended to give way, and said, 'Thou must let the sack of wisdom
descend, by untying yonder cord, and then thou shalt enter.' So the
student let him down, opened the sack, and set him free. 'Now then,'
cried he, 'let me ascend quickly.' As he began to put himself into the
sack heels first, 'Wait a while,' said the gardener, 'that is not the
way.' Then he pushed him in head first, tied up the sack, and soon
swung up the searcher after wisdom dangling in the air. 'How is it
with thee, friend?' said he, 'dost thou not feel that wisdom comes
unto thee? Rest there in peace, till thou art a wiser man than thou
wert.'

So saying, he trotted off on the student's nag, and left the poor
fellow to gather wisdom till somebody should come and let him down.



\chapter{CLEVER HANS}

The mother of Hans said: 'Whither away, Hans?' Hans answered: 'To
Gretel.' 'Behave well, Hans.' 'Oh, I'll behave well. Goodbye, mother.'
'Goodbye, Hans.' Hans comes to Gretel. 'Good day, Gretel.' 'Good day,
Hans. What do you bring that is good?' 'I bring nothing, I want to
have something given me.' Gretel presents Hans with a needle, Hans
says: 'Goodbye, Gretel.' 'Goodbye, Hans.'

Hans takes the needle, sticks it into a hay-cart, and follows the cart
home. 'Good evening, mother.' 'Good evening, Hans. Where have you
been?' 'With Gretel.' 'What did you take her?' 'Took nothing; had
something given me.' 'What did Gretel give you?' 'Gave me a needle.'
'Where is the needle, Hans?' 'Stuck in the hay-cart.' 'That was ill
done, Hans. You should have stuck the needle in your sleeve.' 'Never
mind, I'll do better next time.'

'Whither away, Hans?' 'To Gretel, mother.' 'Behave well, Hans.' 'Oh,
I'll behave well. Goodbye, mother.' 'Goodbye, Hans.' Hans comes to
Gretel. 'Good day, Gretel.' 'Good day, Hans. What do you bring that is
good?' 'I bring nothing. I want to have something given to me.' Gretel
presents Hans with a knife. 'Goodbye, Gretel.' 'Goodbye, Hans.' Hans
takes the knife, sticks it in his sleeve, and goes home. 'Good
evening, mother.' 'Good evening, Hans. Where have you been?' 'With
Gretel.' What did you take her?' 'Took her nothing, she gave me
something.' 'What did Gretel give you?' 'Gave me a knife.' 'Where is
the knife, Hans?' 'Stuck in my sleeve.' 'That's ill done, Hans, you
should have put the knife in your pocket.' 'Never mind, will do better
next time.'

'Whither away, Hans?' 'To Gretel, mother.' 'Behave well, Hans.' 'Oh,
I'll behave well. Goodbye, mother.' 'Goodbye, Hans.' Hans comes to
Gretel. 'Good day, Gretel.' 'Good day, Hans. What good thing do you
bring?' 'I bring nothing, I want something given me.' Gretel presents
Hans with a young goat. 'Goodbye, Gretel.' 'Goodbye, Hans.' Hans takes
the goat, ties its legs, and puts it in his pocket. When he gets home
it is suffocated. 'Good evening, mother.' 'Good evening, Hans. Where
have you been?' 'With Gretel.' 'What did you take her?' 'Took nothing,
she gave me something.' 'What did Gretel give you?' 'She gave me a
goat.' 'Where is the goat, Hans?' 'Put it in my pocket.' 'That was ill
done, Hans, you should have put a rope round the goat's neck.' 'Never
mind, will do better next time.'

'Whither away, Hans?' 'To Gretel, mother.' 'Behave well, Hans.' 'Oh,
I'll behave well. Goodbye, mother.' 'Goodbye, Hans.' Hans comes to
Gretel. 'Good day, Gretel.' 'Good day, Hans. What good thing do you
bring?' 'I bring nothing, I want something given me.' Gretel presents
Hans with a piece of bacon. 'Goodbye, Gretel.' 'Goodbye, Hans.'

Hans takes the bacon, ties it to a rope, and drags it away behind him.
The dogs come and devour the bacon. When he gets home, he has the rope
in his hand, and there is no longer anything hanging on to it. 'Good
evening, mother.' 'Good evening, Hans. Where have you been?' 'With
Gretel.' 'What did you take her?' 'I took her nothing, she gave me
something.' 'What did Gretel give you?' 'Gave me a bit of bacon.'
'Where is the bacon, Hans?' 'I tied it to a rope, brought it home,
dogs took it.' 'That was ill done, Hans, you should have carried the
bacon on your head.' 'Never mind, will do better next time.'

'Whither away, Hans?' 'To Gretel, mother.' 'Behave well, Hans.' 'I'll
behave well. Goodbye, mother.' 'Goodbye, Hans.' Hans comes to Gretel.
'Good day, Gretel.' 'Good day, Hans, What good thing do you bring?' 'I
bring nothing, but would have something given.' Gretel presents Hans
with a calf. 'Goodbye, Gretel.' 'Goodbye, Hans.'

Hans takes the calf, puts it on his head, and the calf kicks his face.
'Good evening, mother.' 'Good evening, Hans. Where have you been?'
'With Gretel.' 'What did you take her?' 'I took nothing, but had
something given me.' 'What did Gretel give you?' 'A calf.' 'Where have
you the calf, Hans?' 'I set it on my head and it kicked my face.'
'That was ill done, Hans, you should have led the calf, and put it in
the stall.' 'Never mind, will do better next time.'

'Whither away, Hans?' 'To Gretel, mother.' 'Behave well, Hans.' 'I'll
behave well. Goodbye, mother.' 'Goodbye, Hans.'

Hans comes to Gretel. 'Good day, Gretel.' 'Good day, Hans. What good
thing do you bring?' 'I bring nothing, but would have something
given.' Gretel says to Hans: 'I will go with you.'

Hans takes Gretel, ties her to a rope, leads her to the rack, and
binds her fast. Then Hans goes to his mother. 'Good evening, mother.'
'Good evening, Hans. Where have you been?' 'With Gretel.' 'What did
you take her?' 'I took her nothing.' 'What did Gretel give you?' 'She
gave me nothing, she came with me.' 'Where have you left Gretel?' 'I
led her by the rope, tied her to the rack, and scattered some grass
for her.' 'That was ill done, Hans, you should have cast friendly eyes
on her.' 'Never mind, will do better.'

Hans went into the stable, cut out all the calves' and sheep's eyes,
and threw them in Gretel's face. Then Gretel became angry, tore
herself loose and ran away, and was no longer the bride of Hans.



\chapter{THE THREE LANGUAGES}

An aged count once lived in Switzerland, who had an only son, but he
was stupid, and could learn nothing. Then said the father: 'Hark you,
my son, try as I will I can get nothing into your head. You must go
from hence, I will give you into the care of a celebrated master, who
shall see what he can do with you.' The youth was sent into a strange
town, and remained a whole year with the master. At the end of this
time, he came home again, and his father asked: 'Now, my son, what
have you learnt?' 'Father, I have learnt what the dogs say when they
bark.' 'Lord have mercy on us!' cried the father; 'is that all you
have learnt? I will send you into another town, to another master.'
The youth was taken thither, and stayed a year with this master
likewise. When he came back the father again asked: 'My son, what have
you learnt?' He answered: 'Father, I have learnt what the birds say.'
Then the father fell into a rage and said: 'Oh, you lost man, you have
spent the precious time and learnt nothing; are you not ashamed to
appear before my eyes? I will send you to a third master, but if you
learn nothing this time also, I will no longer be your father.' The
youth remained a whole year with the third master also, and when he
came home again, and his father inquired: 'My son, what have you
learnt?' he answered: 'Dear father, I have this year learnt what the
frogs croak.' Then the father fell into the most furious anger, sprang
up, called his people thither, and said: 'This man is no longer my
son, I drive him forth, and command you to take him out into the
forest, and kill him.' They took him forth, but when they should have
killed him, they could not do it for pity, and let him go, and they
cut the eyes and tongue out of a deer that they might carry them to
the old man as a token.

The youth wandered on, and after some time came to a fortress where he
begged for a night's lodging. 'Yes,' said the lord of the castle, 'if
you will pass the night down there in the old tower, go thither; but I
warn you, it is at the peril of your life, for it is full of wild
dogs, which bark and howl without stopping, and at certain hours a man
has to be given to them, whom they at once devour.' The whole district
was in sorrow and dismay because of them, and yet no one could do
anything to stop this. The youth, however, was without fear, and said:
'Just let me go down to the barking dogs, and give me something that I
can throw to them; they will do nothing to harm me.' As he himself
would have it so, they gave him some food for the wild animals, and
led him down to the tower. When he went inside, the dogs did not bark
at him, but wagged their tails quite amicably around him, ate what he
set before them, and did not hurt one hair of his head. Next morning,
to the astonishment of everyone, he came out again safe and unharmed,
and said to the lord of the castle: 'The dogs have revealed to me, in
their own language, why they dwell there, and bring evil on the land.
They are bewitched, and are obliged to watch over a great treasure
which is below in the tower, and they can have no rest until it is
taken away, and I have likewise learnt, from their discourse, how that
is to be done.' Then all who heard this rejoiced, and the lord of the
castle said he would adopt him as a son if he accomplished it
successfully. He went down again, and as he knew what he had to do, he
did it thoroughly, and brought a chest full of gold out with him. The
howling of the wild dogs was henceforth heard no more; they had
disappeared, and the country was freed from the trouble.

After some time he took it in his head that he would travel to Rome.
On the way he passed by a marsh, in which a number of frogs were
sitting croaking. He listened to them, and when he became aware of
what they were saying, he grew very thoughtful and sad. At last he
arrived in Rome, where the Pope had just died, and there was great
doubt among the cardinals as to whom they should appoint as his
successor. They at length agreed that the person should be chosen as
pope who should be distinguished by some divine and miraculous token.
And just as that was decided on, the young count entered into the
church, and suddenly two snow-white doves flew on his shoulders and
remained sitting there. The ecclesiastics recognized therein the token
from above, and asked him on the spot if he would be pope. He was
undecided, and knew not if he were worthy of this, but the doves
counselled him to do it, and at length he said yes. Then was he
anointed and consecrated, and thus was fulfilled what he had heard
from the frogs on his way, which had so affected him, that he was to
be his Holiness the Pope. Then he had to sing a mass, and did not know
one word of it, but the two doves sat continually on his shoulders,
and said it all in his ear.



\chapter{THE FOX AND THE CAT}

It happened that the cat met the fox in a forest, and as she thought
to herself: 'He is clever and full of experience, and much esteemed in
the world,' she spoke to him in a friendly way. 'Good day, dear Mr
Fox, how are you? How is all with you? How are you getting on in these
hard times?' The fox, full of all kinds of arrogance, looked at the
cat from head to foot, and for a long time did not know whether he
would give any answer or not. At last he said: 'Oh, you wretched
beard-cleaner, you piebald fool, you hungry mouse-hunter, what can you
be thinking of? Have you the cheek to ask how I am getting on? What
have you learnt? How many arts do you understand?' 'I understand but
one,' replied the cat, modestly. 'What art is that?' asked the fox.
'When the hounds are following me, I can spring into a tree and save
myself.' 'Is that all?' said the fox. 'I am master of a hundred arts,
and have into the bargain a sackful of cunning. You make me sorry for
you; come with me, I will teach you how people get away from the
hounds.' Just then came a hunter with four dogs. The cat sprang nimbly
up a tree, and sat down at the top of it, where the branches and
foliage quite concealed her. 'Open your sack, Mr Fox, open your sack,'
cried the cat to him, but the dogs had already seized him, and were
holding him fast. 'Ah, Mr Fox,' cried the cat. 'You with your hundred
arts are left in the lurch! Had you been able to climb like me, you
would not have lost your life.'



\chapter{THE FOUR CLEVER BROTHERS}

'Dear children,' said a poor man to his four sons, 'I have nothing to
give you; you must go out into the wide world and try your luck. Begin
by learning some craft or another, and see how you can get on.' So the
four brothers took their walking-sticks in their hands, and their
little bundles on their shoulders, and after bidding their father
goodbye, went all out at the gate together. When they had got on some
way they came to four crossways, each leading to a different country.
Then the eldest said, 'Here we must part; but this day four years we
will come back to this spot, and in the meantime each must try what he
can do for himself.'

So each brother went his way; and as the eldest was hastening on a man
met him, and asked him where he was going, and what he wanted. 'I am
going to try my luck in the world, and should like to begin by
learning some art or trade,' answered he. 'Then,' said the man, 'go
with me, and I will teach you to become the cunningest thief that ever
was.' 'No,' said the other, 'that is not an honest calling, and what
can one look to earn by it in the end but the gallows?' 'Oh!' said the
man, 'you need not fear the gallows; for I will only teach you to
steal what will be fair game: I meddle with nothing but what no one
else can get or care anything about, and where no one can find you
out.' So the young man agreed to follow his trade, and he soon showed
himself so clever, that nothing could escape him that he had once set
his mind upon.

The second brother also met a man, who, when he found out what he was
setting out upon, asked him what craft he meant to follow. 'I do not
know yet,' said he. 'Then come with me, and be a star-gazer. It is a
noble art, for nothing can be hidden from you, when once you
understand the stars.' The plan pleased him much, and he soon became
such a skilful star-gazer, that when he had served out his time, and
wanted to leave his master, he gave him a glass, and said, 'With this
you can see all that is passing in the sky and on earth, and nothing
can be hidden from you.'

The third brother met a huntsman, who took him with him, and taught
him so well all that belonged to hunting, that he became very clever
in the craft of the woods; and when he left his master he gave him a
bow, and said, 'Whatever you shoot at with this bow you will be sure
to hit.'

The youngest brother likewise met a man who asked him what he wished
to do. 'Would not you like,' said he, 'to be a tailor?' 'Oh, no!' said
the young man; 'sitting cross-legged from morning to night, working
backwards and forwards with a needle and goose, will never suit me.'
'Oh!' answered the man, 'that is not my sort of tailoring; come with
me, and you will learn quite another kind of craft from that.' Not
knowing what better to do, he came into the plan, and learnt tailoring
from the beginning; and when he left his master, he gave him a needle,
and said, 'You can sew anything with this, be it as soft as an egg or
as hard as steel; and the joint will be so fine that no seam will be
seen.'

After the space of four years, at the time agreed upon, the four
brothers met at the four cross-roads; and having welcomed each other,
set off towards their father's home, where they told him all that had
happened to them, and how each had learned some craft.

Then, one day, as they were sitting before the house under a very high
tree, the father said, 'I should like to try what each of you can do
in this way.' So he looked up, and said to the second son, 'At the top
of this tree there is a chaffinch's nest; tell me how many eggs there
are in it.' The star-gazer took his glass, looked up, and said,
'Five.' 'Now,' said the father to the eldest son, 'take away the eggs
without letting the bird that is sitting upon them and hatching them
know anything of what you are doing.' So the cunning thief climbed up
the tree, and brought away to his father the five eggs from under the
bird; and it never saw or felt what he was doing, but kept sitting on
at its ease. Then the father took the eggs, and put one on each corner
of the table, and the fifth in the middle, and said to the huntsman,
'Cut all the eggs in two pieces at one shot.' The huntsman took up his
bow, and at one shot struck all the five eggs as his father wished.

'Now comes your turn,' said he to the young tailor; 'sew the eggs and
the young birds in them together again, so neatly that the shot shall
have done them no harm.' Then the tailor took his needle, and sewed
the eggs as he was told; and when he had done, the thief was sent to
take them back to the nest, and put them under the bird without its
knowing it. Then she went on sitting, and hatched them: and in a few
days they crawled out, and had only a little red streak across their
necks, where the tailor had sewn them together.

'Well done, sons!' said the old man; 'you have made good use of your
time, and learnt something worth the knowing; but I am sure I do not
know which ought to have the prize. Oh, that a time might soon come
for you to turn your skill to some account!'

Not long after this there was a great bustle in the country; for the
king's daughter had been carried off by a mighty dragon, and the king
mourned over his loss day and night, and made it known that whoever
brought her back to him should have her for a wife. Then the four
brothers said to each other, 'Here is a chance for us; let us try what
we can do.' And they agreed to see whether they could not set the
princess free. 'I will soon find out where she is, however,' said the
star-gazer, as he looked through his glass; and he soon cried out, 'I
see her afar off, sitting upon a rock in the sea, and I can spy the
dragon close by, guarding her.' Then he went to the king, and asked
for a ship for himself and his brothers; and they sailed together over
the sea, till they came to the right place. There they found the
princess sitting, as the star-gazer had said, on the rock; and the
dragon was lying asleep, with his head upon her lap. 'I dare not shoot
at him,' said the huntsman, 'for I should kill the beautiful young
lady also.' 'Then I will try my skill,' said the thief, and went and
stole her away from under the dragon, so quietly and gently that the
beast did not know it, but went on snoring.

Then away they hastened with her full of joy in their boat towards the
ship; but soon came the dragon roaring behind them through the air;
for he awoke and missed the princess. But when he got over the boat,
and wanted to pounce upon them and carry off the princess, the
huntsman took up his bow and shot him straight through the heart so
that he fell down dead. They were still not safe; for he was such a
great beast that in his fall he overset the boat, and they had to swim
in the open sea upon a few planks. So the tailor took his needle, and
with a few large stitches put some of the planks together; and he sat
down upon these, and sailed about and gathered up all pieces of the
boat; and then tacked them together so quickly that the boat was soon
ready, and they then reached the ship and got home safe.

When they had brought home the princess to her father, there was great
rejoicing; and he said to the four brothers, 'One of you shall marry
her, but you must settle amongst yourselves which it is to be.' Then
there arose a quarrel between them; and the star-gazer said, 'If I had
not found the princess out, all your skill would have been of no use;
therefore she ought to be mine.' 'Your seeing her would have been of
no use,' said the thief, 'if I had not taken her away from the dragon;
therefore she ought to be mine.' 'No, she is mine,' said the huntsman;
'for if I had not killed the dragon, he would, after all, have torn
you and the princess into pieces.' 'And if I had not sewn the boat
together again,' said the tailor, 'you would all have been drowned,
therefore she is mine.' Then the king put in a word, and said, 'Each
of you is right; and as all cannot have the young lady, the best way
is for neither of you to have her: for the truth is, there is somebody
she likes a great deal better. But to make up for your loss, I will
give each of you, as a reward for his skill, half a kingdom.' So the
brothers agreed that this plan would be much better than either
quarrelling or marrying a lady who had no mind to have them. And the
king then gave to each half a kingdom, as he had said; and they lived
very happily the rest of their days, and took good care of their
father; and somebody took better care of the young lady, than to let
either the dragon or one of the craftsmen have her again.



\chapter{LILY AND THE LION}

A merchant, who had three daughters, was once setting out upon a
journey; but before he went he asked each daughter what gift he should
bring back for her. The eldest wished for pearls; the second for
jewels; but the third, who was called Lily, said, 'Dear father, bring
me a rose.' Now it was no easy task to find a rose, for it was the
middle of winter; yet as she was his prettiest daughter, and was very
fond of flowers, her father said he would try what he could do. So he
kissed all three, and bid them goodbye.

And when the time came for him to go home, he had bought pearls and
jewels for the two eldest, but he had sought everywhere in vain for
the rose; and when he went into any garden and asked for such a thing,
the people laughed at him, and asked him whether he thought roses grew
in snow. This grieved him very much, for Lily was his dearest child;
and as he was journeying home, thinking what he should bring her, he
came to a fine castle; and around the castle was a garden, in one half
of which it seemed to be summer-time and in the other half winter. On
one side the finest flowers were in full bloom, and on the other
everything looked dreary and buried in the snow. 'A lucky hit!' said
he, as he called to his servant, and told him to go to a beautiful bed
of roses that was there, and bring him away one of the finest flowers.

This done, they were riding away well pleased, when up sprang a fierce
lion, and roared out, 'Whoever has stolen my roses shall be eaten up
alive!' Then the man said, 'I knew not that the garden belonged to
you; can nothing save my life?' 'No!' said the lion, 'nothing, unless
you undertake to give me whatever meets you on your return home; if
you agree to this, I will give you your life, and the rose too for
your daughter.' But the man was unwilling to do so and said, 'It may
be my youngest daughter, who loves me most, and always runs to meet me
when I go home.' Then the servant was greatly frightened, and said,
'It may perhaps be only a cat or a dog.' And at last the man yielded
with a heavy heart, and took the rose; and said he would give the lion
whatever should meet him first on his return.

And as he came near home, it was Lily, his youngest and dearest
daughter, that met him; she came running, and kissed him, and welcomed
him home; and when she saw that he had brought her the rose, she was
still more glad. But her father began to be very sorrowful, and to
weep, saying, 'Alas, my dearest child! I have bought this flower at a
high price, for I have said I would give you to a wild lion; and when
he has you, he will tear you in pieces, and eat you.' Then he told her
all that had happened, and said she should not go, let what would
happen.

But she comforted him, and said, 'Dear father, the word you have given
must be kept; I will go to the lion, and soothe him: perhaps he will
let me come safe home again.'

The next morning she asked the way she was to go, and took leave of
her father, and went forth with a bold heart into the wood. But the
lion was an enchanted prince. By day he and all his court were lions,
but in the evening they took their right forms again. And when Lily
came to the castle, he welcomed her so courteously that she agreed to
marry him. The wedding-feast was held, and they lived happily together
a long time. The prince was only to be seen as soon as evening came,
and then he held his court; but every morning he left his bride, and
went away by himself, she knew not whither, till the night came again.

After some time he said to her, 'Tomorrow there will be a great feast
in your father's house, for your eldest sister is to be married; and
if you wish to go and visit her my lions shall lead you thither.' Then
she rejoiced much at the thoughts of seeing her father once more, and
set out with the lions; and everyone was overjoyed to see her, for
they had thought her dead long since. But she told them how happy she
was, and stayed till the feast was over, and then went back to the
wood.

Her second sister was soon after married, and when Lily was asked to
go to the wedding, she said to the prince, 'I will not go alone this
time--you must go with me.' But he would not, and said that it would
be a very hazardous thing; for if the least ray of the torch-light
should fall upon him his enchantment would become still worse, for he
should be changed into a dove, and be forced to wander about the world
for seven long years. However, she gave him no rest, and said she
would take care no light should fall upon him. So at last they set out
together, and took with them their little child; and she chose a large
hall with thick walls for him to sit in while the wedding-torches were
lighted; but, unluckily, no one saw that there was a crack in the
door. Then the wedding was held with great pomp, but as the train came
from the church, and passed with the torches before the hall, a very
small ray of light fell upon the prince. In a moment he disappeared,
and when his wife came in and looked for him, she found only a white
dove; and it said to her, 'Seven years must I fly up and down over the
face of the earth, but every now and then I will let fall a white
feather, that will show you the way I am going; follow it, and at last
you may overtake and set me free.'

This said, he flew out at the door, and poor Lily followed; and every
now and then a white feather fell, and showed her the way she was to
journey. Thus she went roving on through the wide world, and looked
neither to the right hand nor to the left, nor took any rest, for
seven years. Then she began to be glad, and thought to herself that
the time was fast coming when all her troubles should end; yet repose
was still far off, for one day as she was travelling on she missed the
white feather, and when she lifted up her eyes she could nowhere see
the dove. 'Now,' thought she to herself, 'no aid of man can be of use
to me.' So she went to the sun and said, 'Thou shinest everywhere, on
the hill's top and the valley's depth--hast thou anywhere seen my
white dove?' 'No,' said the sun, 'I have not seen it; but I will give
thee a casket--open it when thy hour of need comes.'

So she thanked the sun, and went on her way till eventide; and when
the moon arose, she cried unto it, and said, 'Thou shinest through the
night, over field and grove--hast thou nowhere seen my white dove?'
'No,' said the moon, 'I cannot help thee but I will give thee an egg--
break it when need comes.'

Then she thanked the moon, and went on till the night-wind blew; and
she raised up her voice to it, and said, 'Thou blowest through every
tree and under every leaf--hast thou not seen my white dove?' 'No,'
said the night-wind, 'but I will ask three other winds; perhaps they
have seen it.' Then the east wind and the west wind came, and said
they too had not seen it, but the south wind said, 'I have seen the
white dove--he has fled to the Red Sea, and is changed once more into
a lion, for the seven years are passed away, and there he is fighting
with a dragon; and the dragon is an enchanted princess, who seeks to
separate him from you.' Then the night-wind said, 'I will give thee
counsel. Go to the Red Sea; on the right shore stand many rods--count
them, and when thou comest to the eleventh, break it off, and smite
the dragon with it; and so the lion will have the victory, and both of
them will appear to you in their own forms. Then look round and thou
wilt see a griffin, winged like bird, sitting by the Red Sea; jump on
to his back with thy beloved one as quickly as possible, and he will
carry you over the waters to your home. I will also give thee this
nut,' continued the night-wind. 'When you are half-way over, throw it
down, and out of the waters will immediately spring up a high nut-tree
on which the griffin will be able to rest, otherwise he would not have
the strength to bear you the whole way; if, therefore, thou dost
forget to throw down the nut, he will let you both fall into the sea.'

So our poor wanderer went forth, and found all as the night-wind had
said; and she plucked the eleventh rod, and smote the dragon, and the
lion forthwith became a prince, and the dragon a princess again. But
no sooner was the princess released from the spell, than she seized
the prince by the arm and sprang on to the griffin's back, and went
off carrying the prince away with her.

Thus the unhappy traveller was again forsaken and forlorn; but she
took heart and said, 'As far as the wind blows, and so long as the
cock crows, I will journey on, till I find him once again.' She went
on for a long, long way, till at length she came to the castle whither
the princess had carried the prince; and there was a feast got ready,
and she heard that the wedding was about to be held. 'Heaven aid me
now!' said she; and she took the casket that the sun had given her,
and found that within it lay a dress as dazzling as the sun itself. So
she put it on, and went into the palace, and all the people gazed upon
her; and the dress pleased the bride so much that she asked whether it
was to be sold. 'Not for gold and silver.' said she, 'but for flesh
and blood.' The princess asked what she meant, and she said, 'Let me
speak with the bridegroom this night in his chamber, and I will give
thee the dress.' At last the princess agreed, but she told her
chamberlain to give the prince a sleeping draught, that he might not
hear or see her. When evening came, and the prince had fallen asleep,
she was led into his chamber, and she sat herself down at his feet,
and said: 'I have followed thee seven years. I have been to the sun,
the moon, and the night-wind, to seek thee, and at last I have helped
thee to overcome the dragon. Wilt thou then forget me quite?' But the
prince all the time slept so soundly, that her voice only passed over
him, and seemed like the whistling of the wind among the fir-trees.

Then poor Lily was led away, and forced to give up the golden dress;
and when she saw that there was no help for her, she went out into a
meadow, and sat herself down and wept. But as she sat she bethought
herself of the egg that the moon had given her; and when she broke it,
there ran out a hen and twelve chickens of pure gold, that played
about, and then nestled under the old one's wings, so as to form the
most beautiful sight in the world. And she rose up and drove them
before her, till the bride saw them from her window, and was so
pleased that she came forth and asked her if she would sell the brood.
'Not for gold or silver, but for flesh and blood: let me again this
evening speak with the bridegroom in his chamber, and I will give thee
the whole brood.'

Then the princess thought to betray her as before, and agreed to what
she asked: but when the prince went to his chamber he asked the
chamberlain why the wind had whistled so in the night. And the
chamberlain told him all--how he had given him a sleeping draught, and
how a poor maiden had come and spoken to him in his chamber, and was
to come again that night. Then the prince took care to throw away the
sleeping draught; and when Lily came and began again to tell him what
woes had befallen her, and how faithful and true to him she had been,
he knew his beloved wife's voice, and sprang up, and said, 'You have
awakened me as from a dream, for the strange princess had thrown a
spell around me, so that I had altogether forgotten you; but Heaven
hath sent you to me in a lucky hour.'

And they stole away out of the palace by night unawares, and seated
themselves on the griffin, who flew back with them over the Red Sea.
When they were half-way across Lily let the nut fall into the water,
and immediately a large nut-tree arose from the sea, whereon the
griffin rested for a while, and then carried them safely home. There
they found their child, now grown up to be comely and fair; and after
all their troubles they lived happily together to the end of their
days.



\chapter{THE FOX AND THE HORSE}

A farmer had a horse that had been an excellent faithful servant to
him: but he was now grown too old to work; so the farmer would give
him nothing more to eat, and said, 'I want you no longer, so take
yourself off out of my stable; I shall not take you back again until
you are stronger than a lion.' Then he opened the door and turned him
adrift.

The poor horse was very melancholy, and wandered up and down in the
wood, seeking some little shelter from the cold wind and rain.
Presently a fox met him: 'What's the matter, my friend?' said he, 'why
do you hang down your head and look so lonely and woe-begone?' 'Ah!'
replied the horse, 'justice and avarice never dwell in one house; my
master has forgotten all that I have done for him so many years, and
because I can no longer work he has turned me adrift, and says unless
I become stronger than a lion he will not take me back again; what
chance can I have of that? he knows I have none, or he would not talk
so.'

However, the fox bid him be of good cheer, and said, 'I will help you;
lie down there, stretch yourself out quite stiff, and pretend to be
dead.' The horse did as he was told, and the fox went straight to the
lion who lived in a cave close by, and said to him, 'A little way off
lies a dead horse; come with me and you may make an excellent meal of
his carcase.' The lion was greatly pleased, and set off immediately;
and when they came to the horse, the fox said, 'You will not be able
to eat him comfortably here; I'll tell you what--I will tie you fast
to his tail, and then you can draw him to your den, and eat him at
your leisure.'

This advice pleased the lion, so he laid himself down quietly for the
fox to make him fast to the horse. But the fox managed to tie his legs
together and bound all so hard and fast that with all his strength he
could not set himself free. When the work was done, the fox clapped
the horse on the shoulder, and said, 'Jip! Dobbin! Jip!' Then up he
sprang, and moved off, dragging the lion behind him. The beast began
to roar and bellow, till all the birds of the wood flew away for
fright; but the horse let him sing on, and made his way quietly over
the fields to his master's house.

'Here he is, master,' said he, 'I have got the better of him': and
when the farmer saw his old servant, his heart relented, and he said.
'Thou shalt stay in thy stable and be well taken care of.' And so the
poor old horse had plenty to eat, and lived--till he died.



\chapter{THE BLUE LIGHT}

There was once upon a time a soldier who for many years had served the
king faithfully, but when the war came to an end could serve no longer
because of the many wounds which he had received. The king said to
him: 'You may return to your home, I need you no longer, and you will
not receive any more money, for he only receives wages who renders me
service for them.' Then the soldier did not know how to earn a living,
went away greatly troubled, and walked the whole day, until in the
evening he entered a forest. When darkness came on, he saw a light,
which he went up to, and came to a house wherein lived a witch. 'Do
give me one night's lodging, and a little to eat and drink,' said he
to her, 'or I shall starve.' 'Oho!' she answered, 'who gives anything
to a run-away soldier? Yet will I be compassionate, and take you in,
if you will do what I wish.' 'What do you wish?' said the soldier.
'That you should dig all round my garden for me, tomorrow.' The
soldier consented, and next day laboured with all his strength, but
could not finish it by the evening. 'I see well enough,' said the
witch, 'that you can do no more today, but I will keep you yet another
night, in payment for which you must tomorrow chop me a load of wood,
and chop it small.' The soldier spent the whole day in doing it, and
in the evening the witch proposed that he should stay one night more.
'Tomorrow, you shall only do me a very trifling piece of work. Behind
my house, there is an old dry well, into which my light has fallen, it
burns blue, and never goes out, and you shall bring it up again.' Next
day the old woman took him to the well, and let him down in a basket.
He found the blue light, and made her a signal to draw him up again.
She did draw him up, but when he came near the edge, she stretched
down her hand and wanted to take the blue light away from him. 'No,'
said he, perceiving her evil intention, 'I will not give you the light
until I am standing with both feet upon the ground.' The witch fell
into a passion, let him fall again into the well, and went away.

The poor soldier fell without injury on the moist ground, and the blue
light went on burning, but of what use was that to him? He saw very
well that he could not escape death. He sat for a while very
sorrowfully, then suddenly he felt in his pocket and found his tobacco
pipe, which was still half full. 'This shall be my last pleasure,'
thought he, pulled it out, lit it at the blue light and began to
smoke. When the smoke had circled about the cavern, suddenly a little
black dwarf stood before him, and said: 'Lord, what are your
commands?' 'What my commands are?' replied the soldier, quite
astonished. 'I must do everything you bid me,' said the little man.
'Good,' said the soldier; 'then in the first place help me out of this
well.' The little man took him by the hand, and led him through an
underground passage, but he did not forget to take the blue light with
him. On the way the dwarf showed him the treasures which the witch had
collected and hidden there, and the soldier took as much gold as he
could carry. When he was above, he said to the little man: 'Now go and
bind the old witch, and carry her before the judge.' In a short time
she came by like the wind, riding on a wild tom-cat and screaming
frightfully. Nor was it long before the little man reappeared. 'It is
all done,' said he, 'and the witch is already hanging on the gallows.
What further commands has my lord?' inquired the dwarf. 'At this
moment, none,' answered the soldier; 'you can return home, only be at
hand immediately, if I summon you.' 'Nothing more is needed than that
you should light your pipe at the blue light, and I will appear before
you at once.' Thereupon he vanished from his sight.

The soldier returned to the town from which he come. He went to the
best inn, ordered himself handsome clothes, and then bade the landlord
furnish him a room as handsome as possible. When it was ready and the
soldier had taken possession of it, he summoned the little black
manikin and said: 'I have served the king faithfully, but he has
dismissed me, and left me to hunger, and now I want to take my
revenge.' 'What am I to do?' asked the little man. 'Late at night,
when the king's daughter is in bed, bring her here in her sleep, she
shall do servant's work for me.' The manikin said: 'That is an easy
thing for me to do, but a very dangerous thing for you, for if it is
discovered, you will fare ill.' When twelve o'clock had struck, the
door sprang open, and the manikin carried in the princess. 'Aha! are
you there?' cried the soldier, 'get to your work at once! Fetch the
broom and sweep the chamber.' When she had done this, he ordered her
to come to his chair, and then he stretched out his feet and said:
'Pull off my boots,' and then he threw them in her face, and made her
pick them up again, and clean and brighten them. She, however, did
everything he bade her, without opposition, silently and with half-
shut eyes. When the first cock crowed, the manikin carried her back to
the royal palace, and laid her in her bed.

Next morning when the princess arose she went to her father, and told
him that she had had a very strange dream. 'I was carried through the
streets with the rapidity of lightning,' said she, 'and taken into a
soldier's room, and I had to wait upon him like a servant, sweep his
room, clean his boots, and do all kinds of menial work. It was only a
dream, and yet I am just as tired as if I really had done everything.'
'The dream may have been true,' said the king. 'I will give you a
piece of advice. Fill your pocket full of peas, and make a small hole
in the pocket, and then if you are carried away again, they will fall
out and leave a track in the streets.' But unseen by the king, the
manikin was standing beside him when he said that, and heard all. At
night when the sleeping princess was again carried through the
streets, some peas certainly did fall out of her pocket, but they made
no track, for the crafty manikin had just before scattered peas in
every street there was. And again the princess was compelled to do
servant's work until cock-crow.

Next morning the king sent his people out to seek the track, but it
was all in vain, for in every street poor children were sitting,
picking up peas, and saying: 'It must have rained peas, last night.'
'We must think of something else,' said the king; 'keep your shoes on
when you go to bed, and before you come back from the place where you
are taken, hide one of them there, I will soon contrive to find it.'
The black manikin heard this plot, and at night when the soldier again
ordered him to bring the princess, revealed it to him, and told him
that he knew of no expedient to counteract this stratagem, and that if
the shoe were found in the soldier's house it would go badly with him.
'Do what I bid you,' replied the soldier, and again this third night
the princess was obliged to work like a servant, but before she went
away, she hid her shoe under the bed.

Next morning the king had the entire town searched for his daughter's
shoe. It was found at the soldier's, and the soldier himself, who at
the entreaty of the dwarf had gone outside the gate, was soon brought
back, and thrown into prison. In his flight he had forgotten the most
valuable things he had, the blue light and the gold, and had only one
ducat in his pocket. And now loaded with chains, he was standing at
the window of his dungeon, when he chanced to see one of his comrades
passing by. The soldier tapped at the pane of glass, and when this man
came up, said to him: 'Be so kind as to fetch me the small bundle I
have left lying in the inn, and I will give you a ducat for doing it.'
His comrade ran thither and brought him what he wanted. As soon as the
soldier was alone again, he lighted his pipe and summoned the black
manikin. 'Have no fear,' said the latter to his master. 'Go
wheresoever they take you, and let them do what they will, only take
the blue light with you.' Next day the soldier was tried, and though
he had done nothing wicked, the judge condemned him to death. When he
was led forth to die, he begged a last favour of the king. 'What is
it?' asked the king. 'That I may smoke one more pipe on my way.' 'You
may smoke three,' answered the king, 'but do not imagine that I will
spare your life.' Then the soldier pulled out his pipe and lighted it
at the blue light, and as soon as a few wreaths of smoke had ascended,
the manikin was there with a small cudgel in his hand, and said: 'What
does my lord command?' 'Strike down to earth that false judge there,
and his constable, and spare not the king who has treated me so ill.'
Then the manikin fell on them like lightning, darting this way and
that way, and whosoever was so much as touched by his cudgel fell to
earth, and did not venture to stir again. The king was terrified; he
threw himself on the soldier's mercy, and merely to be allowed to live
at all, gave him his kingdom for his own, and his daughter to wife.



\chapter{THE RAVEN}

There was once a queen who had a little daughter, still too young to
run alone. One day the child was very troublesome, and the mother
could not quiet it, do what she would. She grew impatient, and seeing
the ravens flying round the castle, she opened the window, and said:
'I wish you were a raven and would fly away, then I should have a
little peace.' Scarcely were the words out of her mouth, when the
child in her arms was turned into a raven, and flew away from her
through the open window. The bird took its flight to a dark wood and
remained there for a long time, and meanwhile the parents could hear
nothing of their child.

Long after this, a man was making his way through the wood when he
heard a raven calling, and he followed the sound of the voice. As he
drew near, the raven said, 'I am by birth a king's daughter, but am
now under the spell of some enchantment; you can, however, set me
free.' 'What am I to do?' he asked. She replied, 'Go farther into the
wood until you come to a house, wherein lives an old woman; she will
offer you food and drink, but you must not take of either; if you do,
you will fall into a deep sleep, and will not be able to help me. In
the garden behind the house is a large tan-heap, and on that you must
stand and watch for me. I shall drive there in my carriage at two
o'clock in the afternoon for three successive days; the first day it
will be drawn by four white, the second by four chestnut, and the last
by four black horses; but if you fail to keep awake and I find you
sleeping, I shall not be set free.'

The man promised to do all that she wished, but the raven said, 'Alas!
I know even now that you will take something from the woman and be
unable to save me.' The man assured her again that he would on no
account touch a thing to eat or drink.

When he came to the house and went inside, the old woman met him, and
said, 'Poor man! how tired you are! Come in and rest and let me give
you something to eat and drink.'

'No,' answered the man, 'I will neither eat not drink.'

But she would not leave him alone, and urged him saying, 'If you will
not eat anything, at least you might take a draught of wine; one drink
counts for nothing,' and at last he allowed himself to be persuaded,
and drank.

As it drew towards the appointed hour, he went outside into the garden
and mounted the tan-heap to await the raven. Suddenly a feeling of
fatigue came over him, and unable to resist it, he lay down for a
little while, fully determined, however, to keep awake; but in another
minute his eyes closed of their own accord, and he fell into such a
deep sleep, that all the noises in the world would not have awakened
him. At two o'clock the raven came driving along, drawn by her four
white horses; but even before she reached the spot, she said to
herself, sighing, 'I know he has fallen asleep.' When she entered the
garden, there she found him as she had feared, lying on the tan-heap,
fast asleep. She got out of her carriage and went to him; she called
him and shook him, but it was all in vain, he still continued
sleeping.

The next day at noon, the old woman came to him again with food and
drink which he at first refused. At last, overcome by her persistent
entreaties that he would take something, he lifted the glass and drank
again.

Towards two o'clock he went into the garden and on to the tan-heap to
watch for the raven. He had not been there long before he began to
feel so tired that his limbs seemed hardly able to support him, and he
could not stand upright any longer; so again he lay down and fell fast
asleep. As the raven drove along her four chestnut horses, she said
sorrowfully to herself, 'I know he has fallen asleep.' She went as
before to look for him, but he slept, and it was impossible to awaken
him.

The following day the old woman said to him, 'What is this? You are
not eating or drinking anything, do you want to kill yourself?'

He answered, 'I may not and will not either eat or drink.'

But she put down the dish of food and the glass of wine in front of
him, and when he smelt the wine, he was unable to resist the
temptation, and took a deep draught.

When the hour came round again he went as usual on to the tan-heap in
the garden to await the king's daughter, but he felt even more
overcome with weariness than on the two previous days, and throwing
himself down, he slept like a log. At two o'clock the raven could be
seen approaching, and this time her coachman and everything about her,
as well as her horses, were black.

She was sadder than ever as she drove along, and said mournfully, 'I
know he has fallen asleep, and will not be able to set me free.' She
found him sleeping heavily, and all her efforts to awaken him were of
no avail. Then she placed beside him a loaf, and some meat, and a
flask of wine, of such a kind, that however much he took of them, they
would never grow less. After that she drew a gold ring, on which her
name was engraved, off her finger, and put it upon one of his.
Finally, she laid a letter near him, in which, after giving him
particulars of the food and drink she had left for him, she finished
with the following words: 'I see that as long as you remain here you
will never be able to set me free; if, however, you still wish to do
so, come to the golden castle of Stromberg; this is well within your
power to accomplish.' She then returned to her carriage and drove to
the golden castle of Stromberg.

When the man awoke and found that he had been sleeping, he was grieved
at heart, and said, 'She has no doubt been here and driven away again,
and it is now too late for me to save her.' Then his eyes fell on the
things which were lying beside him; he read the letter, and knew from
it all that had happened. He rose up without delay, eager to start on
his way and to reach the castle of Stromberg, but he had no idea in
which direction he ought to go. He travelled about a long time in
search of it and came at last to a dark forest, through which he went
on walking for fourteen days and still could not find a way out. Once
more the night came on, and worn out he lay down under a bush and fell
asleep. Again the next day he pursued his way through the forest, and
that evening, thinking to rest again, he lay down as before, but he
heard such a howling and wailing that he found it impossible to sleep.
He waited till it was darker and people had begun to light up their
houses, and then seeing a little glimmer ahead of him, he went towards
it.

He found that the light came from a house which looked smaller than it
really was, from the contrast of its height with that of an immense
giant who stood in front of it. He thought to himself, 'If the giant
sees me going in, my life will not be worth much.' However, after a
while he summoned up courage and went forward. When the giant saw him,
he called out, 'It is lucky for that you have come, for I have not had
anything to eat for a long time. I can have you now for my supper.' 'I
would rather you let that alone,' said the man, 'for I do not
willingly give myself up to be eaten; if you are wanting food I have
enough to satisfy your hunger.' 'If that is so,' replied the giant, 'I
will leave you in peace; I only thought of eating you because I had
nothing else.'

So they went indoors together and sat down, and the man brought out
the bread, meat, and wine, which although he had eaten and drunk of
them, were still unconsumed. The giant was pleased with the good
cheer, and ate and drank to his heart's content. When he had finished
his supper the man asked him if he could direct him to the castle of
Stromberg. The giant said, 'I will look on my map; on it are marked
all the towns, villages, and houses.' So he fetched his map, and
looked for the castle, but could not find it. 'Never mind,' he said,
'I have larger maps upstairs in the cupboard, we will look on those,'
but they searched in vain, for the castle was not marked even on
these. The man now thought he should like to continue his journey, but
the giant begged him to remain for a day or two longer until the
return of his brother, who was away in search of provisions. When the
brother came home, they asked him about the castle of Stromberg, and
he told them he would look on his own maps as soon as he had eaten and
appeased his hunger. Accordingly, when he had finished his supper,
they all went up together to his room and looked through his maps, but
the castle was not to be found. Then he fetched other older maps, and
they went on looking for the castle until at last they found it, but
it was many thousand miles away. 'How shall I be able to get there?'
asked the man. 'I have two hours to spare,' said the giant, 'and I
will carry you into the neighbourhood of the castle; I must then
return to look after the child who is in our care.'

The giant, thereupon, carried the man to within about a hundred
leagues of the castle, where he left him, saying, 'You will be able to
walk the remainder of the way yourself.' The man journeyed on day and
night till he reached the golden castle of Stromberg. He found it
situated, however, on a glass mountain, and looking up from the foot
he saw the enchanted maiden drive round her castle and then go inside.
He was overjoyed to see her, and longed to get to the top of the
mountain, but the sides were so slippery that every time he attempted
to climb he fell back again. When he saw that it was impossible to
reach her, he was greatly grieved, and said to himself, 'I will remain
here and wait for her,' so he built himself a little hut, and there he
sat and watched for a whole year, and every day he saw the king's
daughter driving round her castle, but still was unable to get nearer
to her.

Looking out from his hut one day he saw three robbers fighting and he
called out to them, 'God be with you.' They stopped when they heard
the call, but looking round and seeing nobody, they went on again with
their fighting, which now became more furious. 'God be with you,' he
cried again, and again they paused and looked about, but seeing no one
went back to their fighting. A third time he called out, 'God be with
you,' and then thinking he should like to know the cause of dispute
between the three men, he went out and asked them why they were
fighting so angrily with one another. One of them said that he had
found a stick, and that he had but to strike it against any door
through which he wished to pass, and it immediately flew open. Another
told him that he had found a cloak which rendered its wearer
invisible; and the third had caught a horse which would carry its
rider over any obstacle, and even up the glass mountain. They had been
unable to decide whether they would keep together and have the things
in common, or whether they would separate. On hearing this, the man
said, 'I will give you something in exchange for those three things;
not money, for that I have not got, but something that is of far more
value. I must first, however, prove whether all you have told me about
your three things is true.' The robbers, therefore, made him get on
the horse, and handed him the stick and the cloak, and when he had put
this round him he was no longer visible. Then he fell upon them with
the stick and beat them one after another, crying, 'There, you idle
vagabonds, you have got what you deserve; are you satisfied now!'

After this he rode up the glass mountain. When he reached the gate of
the castle, he found it closed, but he gave it a blow with his stick,
and it flew wide open at once and he passed through. He mounted the
steps and entered the room where the maiden was sitting, with a golden
goblet full of wine in front of her. She could not see him for he
still wore his cloak. He took the ring which she had given him off his
finger, and threw it into the goblet, so that it rang as it touched
the bottom. 'That is my own ring,' she exclaimed, 'and if that is so
the man must also be here who is coming to set me free.'

She sought for him about the castle, but could find him nowhere.
Meanwhile he had gone outside again and mounted his horse and thrown
off the cloak. When therefore she came to the castle gate she saw him,
and cried aloud for joy. Then he dismounted and took her in his arms;
and she kissed him, and said, 'Now you have indeed set me free, and
tomorrow we will celebrate our marriage.'



\chapter{THE GOLDEN GOOSE}

There was a man who had three sons, the youngest of whom was called
Dummling,\footnote{Simpleton} and was despised, mocked, and sneered at on every
occasion.

It happened that the eldest wanted to go into the forest to hew wood,
and before he went his mother gave him a beautiful sweet cake and a
bottle of wine in order that he might not suffer from hunger or
thirst.

When he entered the forest he met a little grey-haired old man who
bade him good day, and said: 'Do give me a piece of cake out of your
pocket, and let me have a draught of your wine; I am so hungry and
thirsty.' But the clever son answered: 'If I give you my cake and
wine, I shall have none for myself; be off with you,' and he left the
little man standing and went on.

But when he began to hew down a tree, it was not long before he made a
false stroke, and the axe cut him in the arm, so that he had to go
home and have it bound up. And this was the little grey man's doing.

After this the second son went into the forest, and his mother gave
him, like the eldest, a cake and a bottle of wine. The little old grey
man met him likewise, and asked him for a piece of cake and a drink of
wine. But the second son, too, said sensibly enough: 'What I give you
will be taken away from myself; be off!' and he left the little man
standing and went on. His punishment, however, was not delayed; when
he had made a few blows at the tree he struck himself in the leg, so
that he had to be carried home.

Then Dummling said: 'Father, do let me go and cut wood.' The father
answered: 'Your brothers have hurt themselves with it, leave it alone,
you do not understand anything about it.' But Dummling begged so long
that at last he said: 'Just go then, you will get wiser by hurting
yourself.' His mother gave him a cake made with water and baked in the
cinders, and with it a bottle of sour beer.

When he came to the forest the little old grey man met him likewise,
and greeting him, said: 'Give me a piece of your cake and a drink out
of your bottle; I am so hungry and thirsty.' Dummling answered: 'I
have only cinder-cake and sour beer; if that pleases you, we will sit
down and eat.' So they sat down, and when Dummling pulled out his
cinder-cake, it was a fine sweet cake, and the sour beer had become
good wine. So they ate and drank, and after that the little man said:
'Since you have a good heart, and are willing to divide what you have,
I will give you good luck. There stands an old tree, cut it down, and
you will find something at the roots.' Then the little man took leave
of him.

Dummling went and cut down the tree, and when it fell there was a
goose sitting in the roots with feathers of pure gold. He lifted her
up, and taking her with him, went to an inn where he thought he would
stay the night. Now the host had three daughters, who saw the goose
and were curious to know what such a wonderful bird might be, and
would have liked to have one of its golden feathers.

The eldest thought: 'I shall soon find an opportunity of pulling out a
feather,' and as soon as Dummling had gone out she seized the goose by
the wing, but her finger and hand remained sticking fast to it.

The second came soon afterwards, thinking only of how she might get a
feather for herself, but she had scarcely touched her sister than she
was held fast.

At last the third also came with the like intent, and the others
screamed out: 'Keep away; for goodness' sake keep away!' But she did
not understand why she was to keep away. 'The others are there,' she
thought, 'I may as well be there too,' and ran to them; but as soon as
she had touched her sister, she remained sticking fast to her. So they
had to spend the night with the goose.

The next morning Dummling took the goose under his arm and set out,
without troubling himself about the three girls who were hanging on to
it. They were obliged to run after him continually, now left, now
right, wherever his legs took him.

In the middle of the fields the parson met them, and when he saw the
procession he said: 'For shame, you good-for-nothing girls, why are
you running across the fields after this young man? Is that seemly?'
At the same time he seized the youngest by the hand in order to pull
her away, but as soon as he touched her he likewise stuck fast, and
was himself obliged to run behind.

Before long the sexton came by and saw his master, the parson, running
behind three girls. He was astonished at this and called out: 'Hi!
your reverence, whither away so quickly? Do not forget that we have a
christening today!' and running after him he took him by the sleeve,
but was also held fast to it.

Whilst the five were trotting thus one behind the other, two labourers
came with their hoes from the fields; the parson called out to them
and begged that they would set him and the sexton free. But they had
scarcely touched the sexton when they were held fast, and now there
were seven of them running behind Dummling and the goose.

Soon afterwards he came to a city, where a king ruled who had a
daughter who was so serious that no one could make her laugh. So he
had put forth a decree that whosoever should be able to make her laugh
should marry her. When Dummling heard this, he went with his goose and
all her train before the king's daughter, and as soon as she saw the
seven people running on and on, one behind the other, she began to
laugh quite loudly, and as if she would never stop. Thereupon Dummling
asked to have her for his wife; but the king did not like the son-in-
law, and made all manner of excuses and said he must first produce a
man who could drink a cellarful of wine. Dummling thought of the
little grey man, who could certainly help him; so he went into the
forest, and in the same place where he had felled the tree, he saw a
man sitting, who had a very sorrowful face. Dummling asked him what he
was taking to heart so sorely, and he answered: 'I have such a great
thirst and cannot quench it; cold water I cannot stand, a barrel of
wine I have just emptied, but that to me is like a drop on a hot
stone!'

'There, I can help you,' said Dummling, 'just come with me and you
shall be satisfied.'

He led him into the king's cellar, and the man bent over the huge
barrels, and drank and drank till his loins hurt, and before the day
was out he had emptied all the barrels. Then Dummling asked once more
for his bride, but the king was vexed that such an ugly fellow, whom
everyone called Dummling, should take away his daughter, and he made a
new condition; he must first find a man who could eat a whole mountain
of bread. Dummling did not think long, but went straight into the
forest, where in the same place there sat a man who was tying up his
body with a strap, and making an awful face, and saying: 'I have eaten
a whole ovenful of rolls, but what good is that when one has such a
hunger as I? My stomach remains empty, and I must tie myself up if I
am not to die of hunger.'

At this Dummling was glad, and said: 'Get up and come with me; you
shall eat yourself full.' He led him to the king's palace where all
the flour in the whole Kingdom was collected, and from it he caused a
huge mountain of bread to be baked. The man from the forest stood
before it, began to eat, and by the end of one day the whole mountain
had vanished. Then Dummling for the third time asked for his bride;
but the king again sought a way out, and ordered a ship which could
sail on land and on water. 'As soon as you come sailing back in it,'
said he, 'you shall have my daughter for wife.'

Dummling went straight into the forest, and there sat the little grey
man to whom he had given his cake. When he heard what Dummling wanted,
he said: 'Since you have given me to eat and to drink, I will give you
the ship; and I do all this because you once were kind to me.' Then he
gave him the ship which could sail on land and water, and when the
king saw that, he could no longer prevent him from having his
daughter. The wedding was celebrated, and after the king's death,
Dummling inherited his kingdom and lived for a long time contentedly
with his wife.




\chapter{THE WATER OF LIFE}

Long before you or I were born, there reigned, in a country a great
way off, a king who had three sons. This king once fell very ill--so
ill that nobody thought he could live. His sons were very much grieved
at their father's sickness; and as they were walking together very
mournfully in the garden of the palace, a little old man met them and
asked what was the matter. They told him that their father was very
ill, and that they were afraid nothing could save him. 'I know what
would,' said the little old man; 'it is the Water of Life. If he could
have a draught of it he would be well again; but it is very hard to
get.' Then the eldest son said, 'I will soon find it': and he went to
the sick king, and begged that he might go in search of the Water of
Life, as it was the only thing that could save him. 'No,' said the
king. 'I had rather die than place you in such great danger as you
must meet with in your journey.' But he begged so hard that the king
let him go; and the prince thought to himself, 'If I bring my father
this water, he will make me sole heir to his kingdom.'

Then he set out: and when he had gone on his way some time he came to
a deep valley, overhung with rocks and woods; and as he looked around,
he saw standing above him on one of the rocks a little ugly dwarf,
with a sugarloaf cap and a scarlet cloak; and the dwarf called to him
and said, 'Prince, whither so fast?' 'What is that to thee, you ugly
imp?' said the prince haughtily, and rode on.

But the dwarf was enraged at his behaviour, and laid a fairy spell of
ill-luck upon him; so that as he rode on the mountain pass became
narrower and narrower, and at last the way was so straitened that he
could not go to step forward: and when he thought to have turned his
horse round and go back the way he came, he heard a loud laugh ringing
round him, and found that the path was closed behind him, so that he
was shut in all round. He next tried to get off his horse and make his
way on foot, but again the laugh rang in his ears, and he found
himself unable to move a step, and thus he was forced to abide
spellbound.

Meantime the old king was lingering on in daily hope of his son's
return, till at last the second son said, 'Father, I will go in search
of the Water of Life.' For he thought to himself, 'My brother is
surely dead, and the kingdom will fall to me if I find the water.' The
king was at first very unwilling to let him go, but at last yielded to
his wish. So he set out and followed the same road which his brother
had done, and met with the same elf, who stopped him at the same spot
in the mountains, saying, as before, 'Prince, prince, whither so
fast?' 'Mind your own affairs, busybody!' said the prince scornfully,
and rode on.

But the dwarf put the same spell upon him as he put on his elder
brother, and he, too, was at last obliged to take up his abode in the
heart of the mountains. Thus it is with proud silly people, who think
themselves above everyone else, and are too proud to ask or take
advice.

When the second prince had thus been gone a long time, the youngest
son said he would go and search for the Water of Life, and trusted he
should soon be able to make his father well again. So he set out, and
the dwarf met him too at the same spot in the valley, among the
mountains, and said, 'Prince, whither so fast?' And the prince said,
'I am going in search of the Water of Life, because my father is ill,
and like to die: can you help me? Pray be kind, and aid me if you
can!' 'Do you know where it is to be found?' asked the dwarf. 'No,'
said the prince, 'I do not. Pray tell me if you know.' 'Then as you
have spoken to me kindly, and are wise enough to seek for advice, I
will tell you how and where to go. The water you seek springs from a
well in an enchanted castle; and, that you may be able to reach it in
safety, I will give you an iron wand and two little loaves of bread;
strike the iron door of the castle three times with the wand, and it
will open: two hungry lions will be lying down inside gaping for their
prey, but if you throw them the bread they will let you pass; then
hasten on to the well, and take some of the Water of Life before the
clock strikes twelve; for if you tarry longer the door will shut upon
you for ever.'

Then the prince thanked his little friend with the scarlet cloak for
his friendly aid, and took the wand and the bread, and went travelling
on and on, over sea and over land, till he came to his journey's end,
and found everything to be as the dwarf had told him. The door flew
open at the third stroke of the wand, and when the lions were quieted
he went on through the castle and came at length to a beautiful hall.
Around it he saw several knights sitting in a trance; then he pulled
off their rings and put them on his own fingers. In another room he
saw on a table a sword and a loaf of bread, which he also took.
Further on he came to a room where a beautiful young lady sat upon a
couch; and she welcomed him joyfully, and said, if he would set her
free from the spell that bound her, the kingdom should be his, if he
would come back in a year and marry her. Then she told him that the
well that held the Water of Life was in the palace gardens; and bade
him make haste, and draw what he wanted before the clock struck
twelve.

He walked on; and as he walked through beautiful gardens he came to a
delightful shady spot in which stood a couch; and he thought to
himself, as he felt tired, that he would rest himself for a while, and
gaze on the lovely scenes around him. So he laid himself down, and
sleep fell upon him unawares, so that he did not wake up till the
clock was striking a quarter to twelve. Then he sprang from the couch
dreadfully frightened, ran to the well, filled a cup that was standing
by him full of water, and hastened to get away in time. Just as he was
going out of the iron door it struck twelve, and the door fell so
quickly upon him that it snapped off a piece of his heel.

When he found himself safe, he was overjoyed to think that he had got
the Water of Life; and as he was going on his way homewards, he passed
by the little dwarf, who, when he saw the sword and the loaf, said,
'You have made a noble prize; with the sword you can at a blow slay
whole armies, and the bread will never fail you.' Then the prince
thought to himself, 'I cannot go home to my father without my
brothers'; so he said, 'My dear friend, cannot you tell me where my
two brothers are, who set out in search of the Water of Life before
me, and never came back?' 'I have shut them up by a charm between two
mountains,' said the dwarf, 'because they were proud and ill-behaved,
and scorned to ask advice.' The prince begged so hard for his
brothers, that the dwarf at last set them free, though unwillingly,
saying, 'Beware of them, for they have bad hearts.' Their brother,
however, was greatly rejoiced to see them, and told them all that had
happened to him; how he had found the Water of Life, and had taken a
cup full of it; and how he had set a beautiful princess free from a
spell that bound her; and how she had engaged to wait a whole year,
and then to marry him, and to give him the kingdom.

Then they all three rode on together, and on their way home came to a
country that was laid waste by war and a dreadful famine, so that it
was feared all must die for want. But the prince gave the king of the
land the bread, and all his kingdom ate of it. And he lent the king
the wonderful sword, and he slew the enemy's army with it; and thus
the kingdom was once more in peace and plenty. In the same manner he
befriended two other countries through which they passed on their way.

When they came to the sea, they got into a ship and during their
voyage the two eldest said to themselves, 'Our brother has got the
water which we could not find, therefore our father will forsake us
and give him the kingdom, which is our right'; so they were full of
envy and revenge, and agreed together how they could ruin him. Then
they waited till he was fast asleep, and poured the Water of Life out
of the cup, and took it for themselves, giving him bitter sea-water
instead.

When they came to their journey's end, the youngest son brought his
cup to the sick king, that he might drink and be healed. Scarcely,
however, had he tasted the bitter sea-water when he became worse even
than he was before; and then both the elder sons came in, and blamed
the youngest for what they had done; and said that he wanted to poison
their father, but that they had found the Water of Life, and had
brought it with them. He no sooner began to drink of what they brought
him, than he felt his sickness leave him, and was as strong and well
as in his younger days. Then they went to their brother, and laughed
at him, and said, 'Well, brother, you found the Water of Life, did
you? You have had the trouble and we shall have the reward. Pray, with
all your cleverness, why did not you manage to keep your eyes open?
Next year one of us will take away your beautiful princess, if you do
not take care. You had better say nothing about this to our father,
for he does not believe a word you say; and if you tell tales, you
shall lose your life into the bargain: but be quiet, and we will let
you off.'

The old king was still very angry with his youngest son, and thought
that he really meant to have taken away his life; so he called his
court together, and asked what should be done, and all agreed that he
ought to be put to death. The prince knew nothing of what was going
on, till one day, when the king's chief huntsmen went a-hunting with
him, and they were alone in the wood together, the huntsman looked so
sorrowful that the prince said, 'My friend, what is the matter with
you?' 'I cannot and dare not tell you,' said he. But the prince begged
very hard, and said, 'Only tell me what it is, and do not think I
shall be angry, for I will forgive you.' 'Alas!' said the huntsman;
'the king has ordered me to shoot you.' The prince started at this,
and said, 'Let me live, and I will change dresses with you; you shall
take my royal coat to show to my father, and do you give me your
shabby one.' 'With all my heart,' said the huntsman; 'I am sure I
shall be glad to save you, for I could not have shot you.' Then he
took the prince's coat, and gave him the shabby one, and went away
through the wood.

Some time after, three grand embassies came to the old king's court,
with rich gifts of gold and precious stones for his youngest son; now
all these were sent from the three kings to whom he had lent his sword
and loaf of bread, in order to rid them of their enemy and feed their
people. This touched the old king's heart, and he thought his son
might still be guiltless, and said to his court, 'O that my son were
still alive! how it grieves me that I had him killed!' 'He is still
alive,' said the huntsman; 'and I am glad that I had pity on him, but
let him go in peace, and brought home his royal coat.' At this the
king was overwhelmed with joy, and made it known thoughout all his
kingdom, that if his son would come back to his court he would forgive
him.

Meanwhile the princess was eagerly waiting till her deliverer should
come back; and had a road made leading up to her palace all of shining
gold; and told her courtiers that whoever came on horseback, and rode
straight up to the gate upon it, was her true lover; and that they
must let him in: but whoever rode on one side of it, they must be sure
was not the right one; and that they must send him away at once.

The time soon came, when the eldest brother thought that he would make
haste to go to the princess, and say that he was the one who had set
her free, and that he should have her for his wife, and the kingdom
with her. As he came before the palace and saw the golden road, he
stopped to look at it, and he thought to himself, 'It is a pity to
ride upon this beautiful road'; so he turned aside and rode on the
right-hand side of it. But when he came to the gate, the guards, who
had seen the road he took, said to him, he could not be what he said
he was, and must go about his business.

The second prince set out soon afterwards on the same errand; and when
he came to the golden road, and his horse had set one foot upon it, he
stopped to look at it, and thought it very beautiful, and said to
himself, 'What a pity it is that anything should tread here!' Then he
too turned aside and rode on the left side of it. But when he came to
the gate the guards said he was not the true prince, and that he too
must go away about his business; and away he went.

Now when the full year was come round, the third brother left the
forest in which he had lain hid for fear of his father's anger, and
set out in search of his betrothed bride. So he journeyed on, thinking
of her all the way, and rode so quickly that he did not even see what
the road was made of, but went with his horse straight over it; and as
he came to the gate it flew open, and the princess welcomed him with
joy, and said he was her deliverer, and should now be her husband and
lord of the kingdom. When the first joy at their meeting was over, the
princess told him she had heard of his father having forgiven him, and
of his wish to have him home again: so, before his wedding with the
princess, he went to visit his father, taking her with him. Then he
told him everything; how his brothers had cheated and robbed him, and
yet that he had borne all those wrongs for the love of his father. And
the old king was very angry, and wanted to punish his wicked sons; but
they made their escape, and got into a ship and sailed away over the
wide sea, and where they went to nobody knew and nobody cared.

And now the old king gathered together his court, and asked all his
kingdom to come and celebrate the wedding of his son and the princess.
And young and old, noble and squire, gentle and simple, came at once
on the summons; and among the rest came the friendly dwarf, with the
sugarloaf hat, and a new scarlet cloak.

\begin{verse}
  And the wedding was held, and the merry bells run.\\
  And all the good people they danced and they sung,\\
  And feasted and frolick'd I can't tell how long.
\end{verse}


\chapter{THE TWELVE HUNTSMEN}

There was once a king's son who had a bride whom he loved very much.
And when he was sitting beside her and very happy, news came that his
father lay sick unto death, and desired to see him once again before
his end. Then he said to his beloved: 'I must now go and leave you, I
give you a ring as a remembrance of me. When I am king, I will return
and fetch you.' So he rode away, and when he reached his father, the
latter was dangerously ill, and near his death. He said to him: 'Dear
son, I wished to see you once again before my end, promise me to marry
as I wish,' and he named a certain king's daughter who was to be his
wife. The son was in such trouble that he did not think what he was
doing, and said: 'Yes, dear father, your will shall be done,' and
thereupon the king shut his eyes, and died.

When therefore the son had been proclaimed king, and the time of
mourning was over, he was forced to keep the promise which he had
given his father, and caused the king's daughter to be asked in
marriage, and she was promised to him. His first betrothed heard of
this, and fretted so much about his faithfulness that she nearly died.
Then her father said to her: 'Dearest child, why are you so sad? You
shall have whatsoever you will.' She thought for a moment and said:
'Dear father, I wish for eleven girls exactly like myself in face,
figure, and size.' The father said: 'If it be possible, your desire
shall be fulfilled,' and he caused a search to be made in his whole
kingdom, until eleven young maidens were found who exactly resembled
his daughter in face, figure, and size.

When they came to the king's daughter, she had twelve suits of
huntsmen's clothes made, all alike, and the eleven maidens had to put
on the huntsmen's clothes, and she herself put on the twelfth suit.
Thereupon she took her leave of her father, and rode away with them,
and rode to the court of her former betrothed, whom she loved so
dearly. Then she asked if he required any huntsmen, and if he would
take all of them into his service. The king looked at her and did not
know her, but as they were such handsome fellows, he said: 'Yes,' and
that he would willingly take them, and now they were the king's twelve
huntsmen.

The king, however, had a lion which was a wondrous animal, for he knew
all concealed and secret things. It came to pass that one evening he
said to the king: 'You think you have twelve huntsmen?' 'Yes,' said
the king, 'they are twelve huntsmen.' The lion continued: 'You are
mistaken, they are twelve girls.' The king said: 'That cannot be true!
How will you prove that to me?' 'Oh, just let some peas be strewn in
the ante-chamber,' answered the lion, 'and then you will soon see.
Men have a firm step, and when they walk over peas none of them stir,
but girls trip and skip, and drag their feet, and the peas roll
about.' The king was well pleased with the counsel, and caused the
peas to be strewn.

There was, however, a servant of the king's who favoured the huntsmen,
and when he heard that they were going to be put to this test he went
to them and repeated everything, and said: 'The lion wants to make the
king believe that you are girls.' Then the king's daughter thanked
him, and said to her maidens: 'Show some strength, and step firmly on
the peas.' So next morning when the king had the twelve huntsmen
called before him, and they came into the ante-chamber where the peas
were lying, they stepped so firmly on them, and had such a strong,
sure walk, that not one of the peas either rolled or stirred. Then
they went away again, and the king said to the lion: 'You have lied to
me, they walk just like men.' The lion said: 'They have been informed
that they were going to be put to the test, and have assumed some
strength. Just let twelve spinning-wheels be brought into the ante-
chamber, and they will go to them and be pleased with them, and that
is what no man would do.' The king liked the advice, and had the
spinning-wheels placed in the ante-chamber.

But the servant, who was well disposed to the huntsmen, went to them,
and disclosed the project. So when they were alone the king's daughter
said to her eleven girls: 'Show some constraint, and do not look round
at the spinning-wheels.' And next morning when the king had his twelve
huntsmen summoned, they went through the ante-chamber, and never once
looked at the spinning-wheels. Then the king again said to the lion:
'You have deceived me, they are men, for they have not looked at the
spinning-wheels.' The lion replied: 'They have restrained themselves.'
The king, however, would no longer believe the lion.

The twelve huntsmen always followed the king to the chase, and his
liking for them continually increased. Now it came to pass that once
when they were out hunting, news came that the king's bride was
approaching. When the true bride heard that, it hurt her so much that
her heart was almost broken, and she fell fainting to the ground. The
king thought something had happened to his dear huntsman, ran up to
him, wanted to help him, and drew his glove off. Then he saw the ring
which he had given to his first bride, and when he looked in her face
he recognized her. Then his heart was so touched that he kissed her,
and when she opened her eyes he said: 'You are mine, and I am yours,
and no one in the world can alter that.' He sent a messenger to the
other bride, and entreated her to return to her own kingdom, for he
had a wife already, and someone who had just found an old key did not
require a new one. Thereupon the wedding was celebrated, and the lion
was again taken into favour, because, after all, he had told the
truth.



\chapter{THE KING OF THE GOLDEN MOUNTAIN}

There was once a merchant who had only one child, a son, that was very
young, and barely able to run alone. He had two richly laden ships
then making a voyage upon the seas, in which he had embarked all his
wealth, in the hope of making great gains, when the news came that
both were lost. Thus from being a rich man he became all at once so
very poor that nothing was left to him but one small plot of land; and
there he often went in an evening to take his walk, and ease his mind
of a little of his trouble.

One day, as he was roaming along in a brown study, thinking with no
great comfort on what he had been and what he now was, and was like to
be, all on a sudden there stood before him a little, rough-looking,
black dwarf. 'Prithee, friend, why so sorrowful?' said he to the
merchant; 'what is it you take so deeply to heart?' 'If you would do
me any good I would willingly tell you,' said the merchant. 'Who knows
but I may?' said the little man: 'tell me what ails you, and perhaps
you will find I may be of some use.' Then the merchant told him how
all his wealth was gone to the bottom of the sea, and how he had
nothing left but that little plot of land. 'Oh, trouble not yourself
about that,' said the dwarf; 'only undertake to bring me here, twelve
years hence, whatever meets you first on your going home, and I will
give you as much as you please.' The merchant thought this was no
great thing to ask; that it would most likely be his dog or his cat,
or something of that sort, but forgot his little boy Heinel; so he
agreed to the bargain, and signed and sealed the bond to do what was
asked of him.

But as he drew near home, his little boy was so glad to see him that
he crept behind him, and laid fast hold of his legs, and looked up in
his face and laughed. Then the father started, trembling with fear and
horror, and saw what it was that he had bound himself to do; but as no
gold was come, he made himself easy by thinking that it was only a
joke that the dwarf was playing him, and that, at any rate, when the
money came, he should see the bearer, and would not take it in.

About a month afterwards he went upstairs into a lumber-room to look
for some old iron, that he might sell it and raise a little money; and
there, instead of his iron, he saw a large pile of gold lying on the
floor. At the sight of this he was overjoyed, and forgetting all about
his son, went into trade again, and became a richer merchant than
before.

Meantime little Heinel grew up, and as the end of the twelve years
drew near the merchant began to call to mind his bond, and became very
sad and thoughtful; so that care and sorrow were written upon his
face. The boy one day asked what was the matter, but his father would
not tell for some time; at last, however, he said that he had, without
knowing it, sold him for gold to a little, ugly-looking, black dwarf,
and that the twelve years were coming round when he must keep his
word. Then Heinel said, 'Father, give yourself very little trouble
about that; I shall be too much for the little man.'

When the time came, the father and son went out together to the place
agreed upon: and the son drew a circle on the ground, and set himself
and his father in the middle of it. The little black dwarf soon came,
and walked round and round about the circle, but could not find any
way to get into it, and he either could not, or dared not, jump over
it. At last the boy said to him. 'Have you anything to say to us, my
friend, or what do you want?' Now Heinel had found a friend in a good
fairy, that was fond of him, and had told him what to do; for this
fairy knew what good luck was in store for him. 'Have you brought me
what you said you would?' said the dwarf to the merchant. The old man
held his tongue, but Heinel said again, 'What do you want here?' The
dwarf said, 'I come to talk with your father, not with you.' 'You have
cheated and taken in my father,' said the son; 'pray give him up his
bond at once.' 'Fair and softly,' said the little old man; 'right is
right; I have paid my money, and your father has had it, and spent it;
so be so good as to let me have what I paid it for.' 'You must have my
consent to that first,' said Heinel, 'so please to step in here, and
let us talk it over.' The old man grinned, and showed his teeth, as if
he should have been very glad to get into the circle if he could. Then
at last, after a long talk, they came to terms. Heinel agreed that his
father must give him up, and that so far the dwarf should have his
way: but, on the other hand, the fairy had told Heinel what fortune
was in store for him, if he followed his own course; and he did not
choose to be given up to his hump-backed friend, who seemed so anxious
for his company.

So, to make a sort of drawn battle of the matter, it was settled that
Heinel should be put into an open boat, that lay on the sea-shore hard
by; that the father should push him off with his own hand, and that he
should thus be set adrift, and left to the bad or good luck of wind
and weather. Then he took leave of his father, and set himself in the
boat, but before it got far off a wave struck it, and it fell with one
side low in the water, so the merchant thought that poor Heinel was
lost, and went home very sorrowful, while the dwarf went his way,
thinking that at any rate he had had his revenge.

The boat, however, did not sink, for the good fairy took care of her
friend, and soon raised the boat up again, and it went safely on. The
young man sat safe within, till at length it ran ashore upon an
unknown land. As he jumped upon the shore he saw before him a
beautiful castle but empty and dreary within, for it was enchanted.
'Here,' said he to himself, 'must I find the prize the good fairy told
me of.' So he once more searched the whole palace through, till at
last he found a white snake, lying coiled up on a cushion in one of
the chambers.

Now the white snake was an enchanted princess; and she was very glad
to see him, and said, 'Are you at last come to set me free? Twelve
long years have I waited here for the fairy to bring you hither as she
promised, for you alone can save me. This night twelve men will come:
their faces will be black, and they will be dressed in chain armour.
They will ask what you do here, but give no answer; and let them do
what they will--beat, whip, pinch, prick, or torment you--bear all;
only speak not a word, and at twelve o'clock they must go away. The
second night twelve others will come: and the third night twenty-four,
who will even cut off your head; but at the twelfth hour of that night
their power is gone, and I shall be free, and will come and bring you
the Water of Life, and will wash you with it, and bring you back to
life and health.' And all came to pass as she had said; Heinel bore
all, and spoke not a word; and the third night the princess came, and
fell on his neck and kissed him. Joy and gladness burst forth
throughout the castle, the wedding was celebrated, and he was crowned
king of the Golden Mountain.

They lived together very happily, and the queen had a son. And thus
eight years had passed over their heads, when the king thought of his
father; and he began to long to see him once again. But the queen was
against his going, and said, 'I know well that misfortunes will come
upon us if you go.' However, he gave her no rest till she agreed. At
his going away she gave him a wishing-ring, and said, 'Take this ring,
and put it on your finger; whatever you wish it will bring you; only
promise never to make use of it to bring me hence to your father's
house.' Then he said he would do what she asked, and put the ring on
his finger, and wished himself near the town where his father lived.

Heinel found himself at the gates in a moment; but the guards would
not let him go in, because he was so strangely clad. So he went up to
a neighbouring hill, where a shepherd dwelt, and borrowed his old
frock, and thus passed unknown into the town. When he came to his
father's house, he said he was his son; but the merchant would not
believe him, and said he had had but one son, his poor Heinel, who he
knew was long since dead: and as he was only dressed like a poor
shepherd, he would not even give him anything to eat. The king,
however, still vowed that he was his son, and said, 'Is there no mark
by which you would know me if I am really your son?' 'Yes,' said his
mother, 'our Heinel had a mark like a raspberry on his right arm.'
Then he showed them the mark, and they knew that what he had said was
true.

He next told them how he was king of the Golden Mountain, and was
married to a princess, and had a son seven years old. But the merchant
said, 'that can never be true; he must be a fine king truly who travels
about in a shepherd's frock!' At this the son was vexed; and forgetting
his word, turned his ring, and wished for his queen and son. In an
instant they stood before him; but the queen wept, and said he had
broken his word, and bad luck would follow. He did all he could to
soothe her, and she at last seemed to be appeased; but she was not so in
truth, and was only thinking how she should punish him.

One day he took her to walk with him out of the town, and showed her
the spot where the boat was set adrift upon the wide waters. Then he
sat himself down, and said, 'I am very much tired; sit by me, I will
rest my head in your lap, and sleep a while.' As soon as he had fallen
asleep, however, she drew the ring from his finger, and crept softly
away, and wished herself and her son at home in their kingdom. And
when he awoke he found himself alone, and saw that the ring was gone
from his finger. 'I can never go back to my father's house,' said he;
'they would say I am a sorcerer: I will journey forth into the world,
till I come again to my kingdom.'

So saying he set out and travelled till he came to a hill, where three
giants were sharing their father's goods; and as they saw him pass
they cried out and said, 'Little men have sharp wits; he shall part
the goods between us.' Now there was a sword that cut off an enemy's
head whenever the wearer gave the words, 'Heads off!'; a cloak that
made the owner invisible, or gave him any form he pleased; and a pair
of boots that carried the wearer wherever he wished. Heinel said they
must first let him try these wonderful things, then he might know how
to set a value upon them. Then they gave him the cloak, and he wished
himself a fly, and in a moment he was a fly. 'The cloak is very well,'
said he: 'now give me the sword.' 'No,' said they; 'not unless you
undertake not to say, ``Heads off!'' for if you do we are all dead men.'
So they gave it him, charging him to try it on a tree. He next asked
for the boots also; and the moment he had all three in his power, he
wished himself at the Golden Mountain; and there he was at once. So
the giants were left behind with no goods to share or quarrel about.

As Heinel came near his castle he heard the sound of merry music; and
the people around told him that his queen was about to marry another
husband. Then he threw his cloak around him, and passed through the
castle hall, and placed himself by the side of the queen, where no one
saw him. But when anything to eat was put upon her plate, he took it
away and ate it himself; and when a glass of wine was handed to her,
he took it and drank it; and thus, though they kept on giving her meat
and drink, her plate and cup were always empty.

Upon this, fear and remorse came over her, and she went into her
chamber alone, and sat there weeping; and he followed her there.
'Alas!' said she to herself, 'was I not once set free? Why then does
this enchantment still seem to bind me?'

'False and fickle one!' said he. 'One indeed came who set thee free,
and he is now near thee again; but how have you used him? Ought he to
have had such treatment from thee?' Then he went out and sent away the
company, and said the wedding was at an end, for that he was come back
to the kingdom. But the princes, peers, and great men mocked at him.
However, he would enter into no parley with them, but only asked them
if they would go in peace or not. Then they turned upon him and tried
to seize him; but he drew his sword. 'Heads Off!' cried he; and with
the word the traitors' heads fell before him, and Heinel was once more
king of the Golden Mountain.



\chapter{DOCTOR KNOWALL}

There was once upon a time a poor peasant called Crabb, who drove with
two oxen a load of wood to the town, and sold it to a doctor for two
talers. When the money was being counted out to him, it so happened
that the doctor was sitting at table, and when the peasant saw how
well he ate and drank, his heart desired what he saw, and would
willingly have been a doctor too. So he remained standing a while, and
at length inquired if he too could not be a doctor. 'Oh, yes,' said
the doctor, 'that is soon managed.' 'What must I do?' asked the
peasant. 'In the first place buy yourself an A B C book of the kind
which has a cock on the frontispiece; in the second, turn your cart
and your two oxen into money, and get yourself some clothes, and
whatsoever else pertains to medicine; thirdly, have a sign painted for
yourself with the words: ``I am Doctor Knowall,'' and have that nailed
up above your house-door.' The peasant did everything that he had been
told to do. When he had doctored people awhile, but not long, a rich
and great lord had some money stolen. Then he was told about Doctor
Knowall who lived in such and such a village, and must know what had
become of the money. So the lord had the horses harnessed to his
carriage, drove out to the village, and asked Crabb if he were Doctor
Knowall. Yes, he was, he said. Then he was to go with him and bring
back the stolen money. 'Oh, yes, but Grete, my wife, must go too.' The
lord was willing, and let both of them have a seat in the carriage,
and they all drove away together. When they came to the nobleman's
castle, the table was spread, and Crabb was told to sit down and eat.
'Yes, but my wife, Grete, too,' said he, and he seated himself with
her at the table. And when the first servant came with a dish of
delicate fare, the peasant nudged his wife, and said: 'Grete, that was
the first,' meaning that was the servant who brought the first dish.
The servant, however, thought he intended by that to say: 'That is the
first thief,' and as he actually was so, he was terrified, and said to
his comrade outside: 'The doctor knows all: we shall fare ill, he said
I was the first.' The second did not want to go in at all, but was
forced. So when he went in with his dish, the peasant nudged his wife,
and said: 'Grete, that is the second.' This servant was equally
alarmed, and he got out as fast as he could. The third fared no
better, for the peasant again said: 'Grete, that is the third.' The
fourth had to carry in a dish that was covered, and the lord told the
doctor that he was to show his skill, and guess what was beneath the
cover. Actually, there were crabs. The doctor looked at the dish, had
no idea what to say, and cried: 'Ah, poor Crabb.' When the lord heard
that, he cried: 'There! he knows it; he must also know who has the
money!'

On this the servants looked terribly uneasy, and made a sign to the
doctor that they wished him to step outside for a moment. When
therefore he went out, all four of them confessed to him that they had
stolen the money, and said that they would willingly restore it and
give him a heavy sum into the bargain, if he would not denounce them,
for if he did they would be hanged. They led him to the spot where the
money was concealed. With this the doctor was satisfied, and returned
to the hall, sat down to the table, and said: 'My lord, now will I
search in my book where the gold is hidden.' The fifth servant,
however, crept into the stove to hear if the doctor knew still more.
But the doctor sat still and opened his A B C book, turned the pages
backwards and forwards, and looked for the cock. As he could not find
it immediately he said: 'I know you are there, so you had better come
out!' Then the fellow in the stove thought that the doctor meant him,
and full of terror, sprang out, crying: 'That man knows everything!'
Then Doctor Knowall showed the lord where the money was, but did not
say who had stolen it, and received from both sides much money in
reward, and became a renowned man.



\chapter{THE SEVEN RAVENS}

There was once a man who had seven sons, and last of all one daughter.
Although the little girl was very pretty, she was so weak and small
that they thought she could not live; but they said she should at once
be christened.

So the father sent one of his sons in haste to the spring to get some
water, but the other six ran with him. Each wanted to be first at
drawing the water, and so they were in such a hurry that all let their
pitchers fall into the well, and they stood very foolishly looking at
one another, and did not know what to do, for none dared go home. In
the meantime the father was uneasy, and could not tell what made the
young men stay so long. 'Surely,' said he, 'the whole seven must have
forgotten themselves over some game of play'; and when he had waited
still longer and they yet did not come, he flew into a rage and wished
them all turned into ravens. Scarcely had he spoken these words when
he heard a croaking over his head, and looked up and saw seven ravens
as black as coal flying round and round. Sorry as he was to see his
wish so fulfilled, he did not know how what was done could be undone,
and comforted himself as well as he could for the loss of his seven
sons with his dear little daughter, who soon became stronger and every
day more beautiful.

For a long time she did not know that she had ever had any brothers;
for her father and mother took care not to speak of them before her:
but one day by chance she heard the people about her speak of them.
'Yes,' said they, 'she is beautiful indeed, but still 'tis a pity that
her brothers should have been lost for her sake.' Then she was much
grieved, and went to her father and mother, and asked if she had any
brothers, and what had become of them. So they dared no longer hide
the truth from her, but said it was the will of Heaven, and that her
birth was only the innocent cause of it; but the little girl mourned
sadly about it every day, and thought herself bound to do all she
could to bring her brothers back; and she had neither rest nor ease,
till at length one day she stole away, and set out into the wide world
to find her brothers, wherever they might be, and free them, whatever
it might cost her.

She took nothing with her but a little ring which her father and
mother had given her, a loaf of bread in case she should be hungry, a
little pitcher of water in case she should be thirsty, and a little
stool to rest upon when she should be weary. Thus she went on and on,
and journeyed till she came to the world's end; then she came to the
sun, but the sun looked much too hot and fiery; so she ran away
quickly to the moon, but the moon was cold and chilly, and said, 'I
smell flesh and blood this way!' so she took herself away in a hurry
and came to the stars, and the stars were friendly and kind to her,
and each star sat upon his own little stool; but the morning star rose
up and gave her a little piece of wood, and said, 'If you have not
this little piece of wood, you cannot unlock the castle that stands on
the glass-mountain, and there your brothers live.' The little girl
took the piece of wood, rolled it up in a little cloth, and went on
again until she came to the glass-mountain, and found the door shut.
Then she felt for the little piece of wood; but when she unwrapped the
cloth it was not there, and she saw she had lost the gift of the good
stars. What was to be done? She wanted to save her brothers, and had
no key of the castle of the glass-mountain; so this faithful little
sister took a knife out of her pocket and cut off her little finger,
that was just the size of the piece of wood she had lost, and put it
in the door and opened it.

As she went in, a little dwarf came up to her, and said, 'What are you
seeking for?' 'I seek for my brothers, the seven ravens,' answered
she. Then the dwarf said, 'My masters are not at home; but if you will
wait till they come, pray step in.' Now the little dwarf was getting
their dinner ready, and he brought their food upon seven little
plates, and their drink in seven little glasses, and set them upon the
table, and out of each little plate their sister ate a small piece,
and out of each little glass she drank a small drop; but she let the
ring that she had brought with her fall into the last glass.

On a sudden she heard a fluttering and croaking in the air, and the
dwarf said, 'Here come my masters.' When they came in, they wanted to
eat and drink, and looked for their little plates and glasses. Then
said one after the other,

'Who has eaten from my little plate? And who has been drinking out of
my little glass?'

\begin{verse}
 'Caw! Caw! well I ween\\
  Mortal lips have this way been.'
\end{verse}

When the seventh came to the bottom of his glass, and found there the
ring, he looked at it, and knew that it was his father's and mother's,
and said, 'O that our little sister would but come! then we should be
free.' When the little girl heard this (for she stood behind the door
all the time and listened), she ran forward, and in an instant all the
ravens took their right form again; and all hugged and kissed each
other, and went merrily home.



\chapter{THE WEDDING OF MRS FOX}


\section{FIRST STORY}

There was once upon a time an old fox with nine tails, who believed
that his wife was not faithful to him, and wished to put her to the
test. He stretched himself out under the bench, did not move a limb,
and behaved as if he were stone dead. Mrs Fox went up to her room,
shut herself in, and her maid, Miss Cat, sat by the fire, and did the
cooking. When it became known that the old fox was dead, suitors
presented themselves. The maid heard someone standing at the house-
door, knocking. She went and opened it, and it was a young fox, who
said:

\begin{verse}
 'What may you be about, Miss Cat?\\
  Do you sleep or do you wake?'
\end{verse}

She answered:

\begin{verse}
 'I am not sleeping, I am waking,\\
  Would you know what I am making?\\
  I am boiling warm beer with butter,\\
  Will you be my guest for supper?'
\end{verse}

'No, thank you, miss,' said the fox, 'what is Mrs Fox doing?' The maid
replied:

\begin{verse}
 'She is sitting in her room,\\
  Moaning in her gloom,\\
  Weeping her little eyes quite red,\\
  Because old Mr Fox is dead.'
\end{verse}

'Do just tell her, miss, that a young fox is here, who would like to
woo her.' 'Certainly, young sir.'

\begin{verse}
  The cat goes up the stairs trip, trap,\\
  The door she knocks at tap, tap, tap,\\
 'Mistress Fox, are you inside?'\\
 'Oh, yes, my little cat,' she cried.\\
 'A wooer he stands at the door out there.'\\
 'What does he look like, my dear?'
\end{verse}

'Has he nine as beautiful tails as the late Mr Fox?' 'Oh, no,'
answered the cat, 'he has only one.' 'Then I will not have him.'

Miss Cat went downstairs and sent the wooer away. Soon afterwards
there was another knock, and another fox was at the door who wished to
woo Mrs Fox. He had two tails, but he did not fare better than the
first. After this still more came, each with one tail more than the
other, but they were all turned away, until at last one came who had
nine tails, like old Mr Fox. When the widow heard that, she said
joyfully to the cat:

\begin{verse}
 'Now open the gates and doors all wide,\\
  And carry old Mr Fox outside.'
\end{verse}

But just as the wedding was going to be solemnized, old Mr Fox stirred
under the bench, and cudgelled all the rabble, and drove them and Mrs
Fox out of the house.


\section{SECOND STORY}

When old Mr Fox was dead, the wolf came as a suitor, and knocked at
the door, and the cat who was servant to Mrs Fox, opened it for him.
The wolf greeted her, and said:

\begin{verse}
 'Good day, Mrs Cat of Kehrewit,\\
  How comes it that alone you sit?\\
  What are you making good?'
\end{verse}

The cat replied:

\begin{verse}
 'In milk I'm breaking bread so sweet,\\
  Will you be my guest, and eat?'
\end{verse}

'No, thank you, Mrs Cat,' answered the wolf. 'Is Mrs Fox not at home?'

The cat said:

\begin{verse}
 'She sits upstairs in her room,\\
  Bewailing her sorrowful doom,\\
  Bewailing her trouble so sore,\\
  For old Mr Fox is no more.'
\end{verse}

The wolf answered:

\begin{verse}
 'If she's in want of a husband now,\\
  Then will it please her to step below?'\\
  The cat runs quickly up the stair,\\
  And lets her tail fly here and there,\\
  Until she comes to the parlour door.\\
  With her five gold rings at the door she knocks:\\
 'Are you within, good Mistress Fox?\\
  If you're in want of a husband now,\\
  Then will it please you to step below?
\end{verse}

Mrs Fox asked: 'Has the gentleman red stockings on, and has he a
pointed mouth?' 'No,' answered the cat. 'Then he won't do for me.'

When the wolf was gone, came a dog, a stag, a hare, a bear, a lion,
and all the beasts of the forest, one after the other. But one of the
good qualities which old Mr Fox had possessed, was always lacking, and
the cat had continually to send the suitors away. At length came a
young fox. Then Mrs Fox said: 'Has the gentleman red stockings on, and
has a little pointed mouth?' 'Yes,' said the cat, 'he has.' 'Then let
him come upstairs,' said Mrs Fox, and ordered the servant to prepare
the wedding feast.

\begin{verse}
 'Sweep me the room as clean as you can,\\
  Up with the window, fling out my old man!\\
  For many a fine fat mouse he brought,\\
  Yet of his wife he never thought,\\
  But ate up every one he caught.'
\end{verse}

Then the wedding was solemnized with young Mr Fox, and there was much
rejoicing and dancing; and if they have not left off, they are dancing
still.



\chapter{THE SALAD}

As a merry young huntsman was once going briskly along through a wood,
there came up a little old woman, and said to him, 'Good day, good
day; you seem merry enough, but I am hungry and thirsty; do pray give
me something to eat.' The huntsman took pity on her, and put his hand
in his pocket and gave her what he had. Then he wanted to go his way;
but she took hold of him, and said, 'Listen, my friend, to what I am
going to tell you; I will reward you for your kindness; go your way,
and after a little time you will come to a tree where you will see
nine birds sitting on a cloak. Shoot into the midst of them, and one
will fall down dead: the cloak will fall too; take it, it is a
wishing-cloak, and when you wear it you will find yourself at any
place where you may wish to be. Cut open the dead bird, take out its
heart and keep it, and you will find a piece of gold under your pillow
every morning when you rise. It is the bird's heart that will bring
you this good luck.'

The huntsman thanked her, and thought to himself, 'If all this does
happen, it will be a fine thing for me.' When he had gone a hundred
steps or so, he heard a screaming and chirping in the branches over
him, and looked up and saw a flock of birds pulling a cloak with their
bills and feet; screaming, fighting, and tugging at each other as if
each wished to have it himself. 'Well,' said the huntsman, 'this is
wonderful; this happens just as the old woman said'; then he shot into
the midst of them so that their feathers flew all about. Off went the
flock chattering away; but one fell down dead, and the cloak with it.
Then the huntsman did as the old woman told him, cut open the bird,
took out the heart, and carried the cloak home with him.

The next morning when he awoke he lifted up his pillow, and there lay
the piece of gold glittering underneath; the same happened next day,
and indeed every day when he arose. He heaped up a great deal of gold,
and at last thought to himself, 'Of what use is this gold to me whilst
I am at home? I will go out into the world and look about me.'

Then he took leave of his friends, and hung his bag and bow about his
neck, and went his way. It so happened that his road one day led
through a thick wood, at the end of which was a large castle in a
green meadow, and at one of the windows stood an old woman with a very
beautiful young lady by her side looking about them. Now the old woman
was a witch, and said to the young lady, 'There is a young man coming
out of the wood who carries a wonderful prize; we must get it away
from him, my dear child, for it is more fit for us than for him. He
has a bird's heart that brings a piece of gold under his pillow every
morning.' Meantime the huntsman came nearer and looked at the lady,
and said to himself, 'I have been travelling so long that I should
like to go into this castle and rest myself, for I have money enough
to pay for anything I want'; but the real reason was, that he wanted
to see more of the beautiful lady. Then he went into the house, and
was welcomed kindly; and it was not long before he was so much in love
that he thought of nothing else but looking at the lady's eyes, and
doing everything that she wished. Then the old woman said, 'Now is the
time for getting the bird's heart.' So the lady stole it away, and he
never found any more gold under his pillow, for it lay now under the
young lady's, and the old woman took it away every morning; but he was
so much in love that he never missed his prize.

'Well,' said the old witch, 'we have got the bird's heart, but not the
wishing-cloak yet, and that we must also get.' 'Let us leave him
that,' said the young lady; 'he has already lost his wealth.' Then the
witch was very angry, and said, 'Such a cloak is a very rare and
wonderful thing, and I must and will have it.' So she did as the old
woman told her, and set herself at the window, and looked about the
country and seemed very sorrowful; then the huntsman said, 'What makes
you so sad?' 'Alas! dear sir,' said she, 'yonder lies the granite rock
where all the costly diamonds grow, and I want so much to go there,
that whenever I think of it I cannot help being sorrowful, for who can
reach it? only the birds and the flies--man cannot.' 'If that's all
your grief,' said the huntsman, 'I'll take there with all my heart';
so he drew her under his cloak, and the moment he wished to be on the
granite mountain they were both there. The diamonds glittered so on
all sides that they were delighted with the sight and picked up the
finest. But the old witch made a deep sleep come upon him, and he said
to the young lady, 'Let us sit down and rest ourselves a little, I am
so tired that I cannot stand any longer.' So they sat down, and he
laid his head in her lap and fell asleep; and whilst he was sleeping
on she took the cloak from his shoulders, hung it on her own, picked
up the diamonds, and wished herself home again.

When he awoke and found that his lady had tricked him, and left him
alone on the wild rock, he said, 'Alas! what roguery there is in the
world!' and there he sat in great grief and fear, not knowing what to
do. Now this rock belonged to fierce giants who lived upon it; and as
he saw three of them striding about, he thought to himself, 'I can
only save myself by feigning to be asleep'; so he laid himself down as
if he were in a sound sleep. When the giants came up to him, the first
pushed him with his foot, and said, 'What worm is this that lies here
curled up?' 'Tread upon him and kill him,' said the second. 'It's not
worth the trouble,' said the third; 'let him live, he'll go climbing
higher up the mountain, and some cloud will come rolling and carry him
away.' And they passed on. But the huntsman had heard all they said;
and as soon as they were gone, he climbed to the top of the mountain,
and when he had sat there a short time a cloud came rolling around
him, and caught him in a whirlwind and bore him along for some time,
till it settled in a garden, and he fell quite gently to the ground
amongst the greens and cabbages.

Then he looked around him, and said, 'I wish I had something to eat,
if not I shall be worse off than before; for here I see neither apples
nor pears, nor any kind of fruits, nothing but vegetables.' At last he
thought to himself, 'I can eat salad, it will refresh and strengthen
me.' So he picked out a fine head and ate of it; but scarcely had he
swallowed two bites when he felt himself quite changed, and saw with
horror that he was turned into an ass. However, he still felt very
hungry, and the salad tasted very nice; so he ate on till he came to
another kind of salad, and scarcely had he tasted it when he felt
another change come over him, and soon saw that he was lucky enough to
have found his old shape again.

Then he laid himself down and slept off a little of his weariness; and
when he awoke the next morning he broke off a head both of the good
and the bad salad, and thought to himself, 'This will help me to my
fortune again, and enable me to pay off some folks for their
treachery.' So he went away to try and find the castle of his friends;
and after wandering about a few days he luckily found it. Then he
stained his face all over brown, so that even his mother would not
have known him, and went into the castle and asked for a lodging; 'I
am so tired,' said he, 'that I can go no farther.' 'Countryman,' said
the witch, 'who are you? and what is your business?' 'I am,' said he,
'a messenger sent by the king to find the finest salad that grows
under the sun. I have been lucky enough to find it, and have brought
it with me; but the heat of the sun scorches so that it begins to
wither, and I don't know that I can carry it farther.'

When the witch and the young lady heard of his beautiful salad, they
longed to taste it, and said, 'Dear countryman, let us just taste it.'
'To be sure,' answered he; 'I have two heads of it with me, and will
give you one'; so he opened his bag and gave them the bad. Then the
witch herself took it into the kitchen to be dressed; and when it was
ready she could not wait till it was carried up, but took a few leaves
immediately and put them in her mouth, and scarcely were they
swallowed when she lost her own form and ran braying down into the
court in the form of an ass. Now the servant-maid came into the
kitchen, and seeing the salad ready, was going to carry it up; but on
the way she too felt a wish to taste it as the old woman had done, and
ate some leaves; so she also was turned into an ass and ran after the
other, letting the dish with the salad fall on the ground. The
messenger sat all this time with the beautiful young lady, and as
nobody came with the salad and she longed to taste it, she said, 'I
don't know where the salad can be.' Then he thought something must
have happened, and said, 'I will go into the kitchen and see.' And as
he went he saw two asses in the court running about, and the salad
lying on the ground. 'All right!' said he; 'those two have had their
share.' Then he took up the rest of the leaves, laid them on the dish
and brought them to the young lady, saying, 'I bring you the dish
myself that you may not wait any longer.' So she ate of it, and like
the others ran off into the court braying away.

Then the huntsman washed his face and went into the court that they
might know him. 'Now you shall be paid for your roguery,' said he; and
tied them all three to a rope and took them along with him till he
came to a mill and knocked at the window. 'What's the matter?' said
the miller. 'I have three tiresome beasts here,' said the other; 'if
you will take them, give them food and room, and treat them as I tell
you, I will pay you whatever you ask.' 'With all my heart,' said the
miller; 'but how shall I treat them?' Then the huntsman said, 'Give
the old one stripes three times a day and hay once; give the next (who
was the servant-maid) stripes once a day and hay three times; and give
the youngest (who was the beautiful lady) hay three times a day and no
stripes': for he could not find it in his heart to have her beaten.
After this he went back to the castle, where he found everything he
wanted.

Some days after, the miller came to him and told him that the old ass
was dead; 'The other two,' said he, 'are alive and eat, but are so
sorrowful that they cannot last long.' Then the huntsman pitied them,
and told the miller to drive them back to him, and when they came, he
gave them some of the good salad to eat. And the beautiful young lady
fell upon her knees before him, and said, 'O dearest huntsman! forgive
me all the ill I have done you; my mother forced me to it, it was
against my will, for I always loved you very much. Your wishing-cloak
hangs up in the closet, and as for the bird's heart, I will give it
you too.' But he said, 'Keep it, it will be just the same thing, for I
mean to make you my wife.' So they were married, and lived together
very happily till they died.



\chapter[THE STORY OF THE YOUTH WHO WENT\ldots]{THE STORY OF THE YOUTH WHO WENT FORTH TO LEARN WHAT FEAR WAS}

A certain father had two sons, the elder of who was smart and
sensible, and could do everything, but the younger was stupid and
could neither learn nor understand anything, and when people saw him
they said: 'There's a fellow who will give his father some trouble!'
When anything had to be done, it was always the elder who was forced
to do it; but if his father bade him fetch anything when it was late,
or in the night-time, and the way led through the churchyard, or any
other dismal place, he answered: 'Oh, no father, I'll not go there, it
makes me shudder!' for he was afraid. Or when stories were told by the
fire at night which made the flesh creep, the listeners sometimes
said: 'Oh, it makes us shudder!' The younger sat in a corner and
listened with the rest of them, and could not imagine what they could
mean. 'They are always saying: ``It makes me shudder, it makes me
shudder!'' It does not make me shudder,' thought he. 'That, too, must
be an art of which I understand nothing!'

Now it came to pass that his father said to him one day: 'Hearken to
me, you fellow in the corner there, you are growing tall and strong,
and you too must learn something by which you can earn your bread.
Look how your brother works, but you do not even earn your salt.'
'Well, father,' he replied, 'I am quite willing to learn something--
indeed, if it could but be managed, I should like to learn how to
shudder. I don't understand that at all yet.' The elder brother smiled
when he heard that, and thought to himself: 'Goodness, what a
blockhead that brother of mine is! He will never be good for anything
as long as he lives! He who wants to be a sickle must bend himself
betimes.'

The father sighed, and answered him: 'You shall soon learn what it is
to shudder, but you will not earn your bread by that.'

Soon after this the sexton came to the house on a visit, and the
father bewailed his trouble, and told him how his younger son was so
backward in every respect that he knew nothing and learnt nothing.
'Just think,' said he, 'when I asked him how he was going to earn his
bread, he actually wanted to learn to shudder.' 'If that be all,'
replied the sexton, 'he can learn that with me. Send him to me, and I
will soon polish him.' The father was glad to do it, for he thought:
'It will train the boy a little.' The sexton therefore took him into
his house, and he had to ring the church bell. After a day or two, the
sexton awoke him at midnight, and bade him arise and go up into the
church tower and ring the bell. 'You shall soon learn what shuddering
is,' thought he, and secretly went there before him; and when the boy
was at the top of the tower and turned round, and was just going to
take hold of the bell rope, he saw a white figure standing on the
stairs opposite the sounding hole. 'Who is there?' cried he, but the
figure made no reply, and did not move or stir. 'Give an answer,'
cried the boy, 'or take yourself off, you have no business here at
night.'

The sexton, however, remained standing motionless that the boy might
think he was a ghost. The boy cried a second time: 'What do you want
here?--speak if you are an honest fellow, or I will throw you down the
steps!' The sexton thought: 'He can't mean to be as bad as his words,'
uttered no sound and stood as if he were made of stone. Then the boy
called to him for the third time, and as that was also to no purpose,
he ran against him and pushed the ghost down the stairs, so that it
fell down the ten steps and remained lying there in a corner.
Thereupon he rang the bell, went home, and without saying a word went
to bed, and fell asleep. The sexton's wife waited a long time for her
husband, but he did not come back. At length she became uneasy, and
wakened the boy, and asked: 'Do you know where my husband is? He
climbed up the tower before you did.' 'No, I don't know,' replied the
boy, 'but someone was standing by the sounding hole on the other side
of the steps, and as he would neither gave an answer nor go away, I
took him for a scoundrel, and threw him downstairs. Just go there and
you will see if it was he. I should be sorry if it were.' The woman
ran away and found her husband, who was lying moaning in the corner,
and had broken his leg.

She carried him down, and then with loud screams she hastened to the
boy's father, 'Your boy,' cried she, 'has been the cause of a great
misfortune! He has thrown my husband down the steps so that he broke
his leg. Take the good-for-nothing fellow out of our house.' The
father was terrified, and ran thither and scolded the boy. 'What
wicked tricks are these?' said he. 'The devil must have put them into
your head.' 'Father,' he replied, 'do listen to me. I am quite
innocent. He was standing there by night like one intent on doing
evil. I did not know who it was, and I entreated him three times
either to speak or to go away.' 'Ah,' said the father, 'I have nothing
but unhappiness with you. Go out of my sight. I will see you no more.'

'Yes, father, right willingly, wait only until it is day. Then will I
go forth and learn how to shudder, and then I shall, at any rate,
understand one art which will support me.' 'Learn what you will,'
spoke the father, 'it is all the same to me. Here are fifty talers for
you. Take these and go into the wide world, and tell no one from
whence you come, and who is your father, for I have reason to be
ashamed of you.' 'Yes, father, it shall be as you will. If you desire
nothing more than that, I can easily keep it in mind.'

When the day dawned, therefore, the boy put his fifty talers into his
pocket, and went forth on the great highway, and continually said to
himself: 'If I could but shudder! If I could but shudder!' Then a man
approached who heard this conversation which the youth was holding
with himself, and when they had walked a little farther to where they
could see the gallows, the man said to him: 'Look, there is the tree
where seven men have married the ropemaker's daughter, and are now
learning how to fly. Sit down beneath it, and wait till night comes,
and you will soon learn how to shudder.' 'If that is all that is
wanted,' answered the youth, 'it is easily done; but if I learn how to
shudder as fast as that, you shall have my fifty talers. Just come
back to me early in the morning.' Then the youth went to the gallows,
sat down beneath it, and waited till evening came. And as he was cold,
he lighted himself a fire, but at midnight the wind blew so sharply
that in spite of his fire, he could not get warm. And as the wind
knocked the hanged men against each other, and they moved backwards
and forwards, he thought to himself: 'If you shiver below by the fire,
how those up above must freeze and suffer!' And as he felt pity for
them, he raised the ladder, and climbed up, unbound one of them after
the other, and brought down all seven. Then he stoked the fire, blew
it, and set them all round it to warm themselves. But they sat there
and did not stir, and the fire caught their clothes. So he said: 'Take
care, or I will hang you up again.' The dead men, however, did not
hear, but were quite silent, and let their rags go on burning. At this
he grew angry, and said: 'If you will not take care, I cannot help
you, I will not be burnt with you,' and he hung them up again each in
his turn. Then he sat down by his fire and fell asleep, and the next
morning the man came to him and wanted to have the fifty talers, and
said: 'Well do you know how to shudder?' 'No,' answered he, 'how
should I know? Those fellows up there did not open their mouths, and
were so stupid that they let the few old rags which they had on their
bodies get burnt.' Then the man saw that he would not get the fifty
talers that day, and went away saying: 'Such a youth has never come my
way before.'

The youth likewise went his way, and once more began to mutter to
himself: 'Ah, if I could but shudder! Ah, if I could but shudder!' A
waggoner who was striding behind him heard this and asked: 'Who are
you?' 'I don't know,' answered the youth. Then the waggoner asked:
'From whence do you come?' 'I know not.' 'Who is your father?' 'That I
may not tell you.' 'What is it that you are always muttering between
your teeth?' 'Ah,' replied the youth, 'I do so wish I could shudder,
but no one can teach me how.' 'Enough of your foolish chatter,' said
the waggoner. 'Come, go with me, I will see about a place for you.'
The youth went with the waggoner, and in the evening they arrived at
an inn where they wished to pass the night. Then at the entrance of
the parlour the youth again said quite loudly: 'If I could but
shudder! If I could but shudder!' The host who heard this, laughed and
said: 'If that is your desire, there ought to be a good opportunity
for you here.' 'Ah, be silent,' said the hostess, 'so many prying
persons have already lost their lives, it would be a pity and a shame
if such beautiful eyes as these should never see the daylight again.'

But the youth said: 'However difficult it may be, I will learn it. For
this purpose indeed have I journeyed forth.' He let the host have no
rest, until the latter told him, that not far from thence stood a
haunted castle where anyone could very easily learn what shuddering
was, if he would but watch in it for three nights. The king had
promised that he who would venture should have his daughter to wife,
and she was the most beautiful maiden the sun shone on. Likewise in
the castle lay great treasures, which were guarded by evil spirits,
and these treasures would then be freed, and would make a poor man
rich enough. Already many men had gone into the castle, but as yet
none had come out again. Then the youth went next morning to the king,
and said: 'If it be allowed, I will willingly watch three nights in
the haunted castle.'

The king looked at him, and as the youth pleased him, he said: 'You
may ask for three things to take into the castle with you, but they
must be things without life.' Then he answered: 'Then I ask for a
fire, a turning lathe, and a cutting-board with the knife.'

The king had these things carried into the castle for him during the
day. When night was drawing near, the youth went up and made himself a
bright fire in one of the rooms, placed the cutting-board and knife
beside it, and seated himself by the turning-lathe. 'Ah, if I could
but shudder!' said he, 'but I shall not learn it here either.' Towards
midnight he was about to poke his fire, and as he was blowing it,
something cried suddenly from one corner: 'Au, miau! how cold we are!'
'You fools!' cried he, 'what are you crying about? If you are cold,
come and take a seat by the fire and warm yourselves.' And when he had
said that, two great black cats came with one tremendous leap and sat
down on each side of him, and looked savagely at him with their fiery
eyes. After a short time, when they had warmed themselves, they said:
'Comrade, shall we have a game of cards?' 'Why not?' he replied, 'but
just show me your paws.' Then they stretched out their claws. 'Oh,'
said he, 'what long nails you have! Wait, I must first cut them for
you.' Thereupon he seized them by the throats, put them on the
cutting-board and screwed their feet fast. 'I have looked at your
fingers,' said he, 'and my fancy for card-playing has gone,' and he
struck them dead and threw them out into the water. But when he had
made away with these two, and was about to sit down again by his fire,
out from every hole and corner came black cats and black dogs with
red-hot chains, and more and more of them came until he could no
longer move, and they yelled horribly, and got on his fire, pulled it
to pieces, and tried to put it out. He watched them for a while
quietly, but at last when they were going too far, he seized his
cutting-knife, and cried: 'Away with you, vermin,' and began to cut
them down. Some of them ran away, the others he killed, and threw out
into the fish-pond. When he came back he fanned the embers of his fire
again and warmed himself. And as he thus sat, his eyes would keep open
no longer, and he felt a desire to sleep. Then he looked round and saw
a great bed in the corner. 'That is the very thing for me,' said he,
and got into it. When he was just going to shut his eyes, however, the
bed began to move of its own accord, and went over the whole of the
castle. 'That's right,' said he, 'but go faster.' Then the bed rolled
on as if six horses were harnessed to it, up and down, over thresholds
and stairs, but suddenly hop, hop, it turned over upside down, and lay
on him like a mountain. But he threw quilts and pillows up in the air,
got out and said: 'Now anyone who likes, may drive,' and lay down by
his fire, and slept till it was day. In the morning the king came, and
when he saw him lying there on the ground, he thought the evil spirits
had killed him and he was dead. Then said he: 'After all it is a
pity,--for so handsome a man.' The youth heard it, got up, and said:
'It has not come to that yet.' Then the king was astonished, but very
glad, and asked how he had fared. 'Very well indeed,' answered he;
'one night is past, the two others will pass likewise.' Then he went
to the innkeeper, who opened his eyes very wide, and said: 'I never
expected to see you alive again! Have you learnt how to shudder yet?'
'No,' said he, 'it is all in vain. If someone would but tell me!'

The second night he again went up into the old castle, sat down by the
fire, and once more began his old song: 'If I could but shudder!' When
midnight came, an uproar and noise of tumbling about was heard; at
first it was low, but it grew louder and louder. Then it was quiet for
a while, and at length with a loud scream, half a man came down the
chimney and fell before him. 'Hullo!' cried he, 'another half belongs
to this. This is not enough!' Then the uproar began again, there was a
roaring and howling, and the other half fell down likewise. 'Wait,'
said he, 'I will just stoke up the fire a little for you.' When he had
done that and looked round again, the two pieces were joined together,
and a hideous man was sitting in his place. 'That is no part of our
bargain,' said the youth, 'the bench is mine.' The man wanted to push
him away; the youth, however, would not allow that, but thrust him off
with all his strength, and seated himself again in his own place. Then
still more men fell down, one after the other; they brought nine dead
men's legs and two skulls, and set them up and played at nine-pins
with them. The youth also wanted to play and said: 'Listen you, can I
join you?' 'Yes, if you have any money.' 'Money enough,' replied he,
'but your balls are not quite round.' Then he took the skulls and put
them in the lathe and turned them till they were round. 'There, now
they will roll better!' said he. 'Hurrah! now we'll have fun!' He
played with them and lost some of his money, but when it struck
twelve, everything vanished from his sight. He lay down and quietly
fell asleep. Next morning the king came to inquire after him. 'How has
it fared with you this time?' asked he. 'I have been playing at nine-
pins,' he answered, 'and have lost a couple of farthings.' 'Have you
not shuddered then?' 'What?' said he, 'I have had a wonderful time! If
I did but know what it was to shudder!'

The third night he sat down again on his bench and said quite sadly:
'If I could but shudder.' When it grew late, six tall men came in and
brought a coffin. Then he said: 'Ha, ha, that is certainly my little
cousin, who died only a few days ago,' and he beckoned with his
finger, and cried: 'Come, little cousin, come.' They placed the coffin
on the ground, but he went to it and took the lid off, and a dead man
lay therein. He felt his face, but it was cold as ice. 'Wait,' said
he, 'I will warm you a little,' and went to the fire and warmed his
hand and laid it on the dead man's face, but he remained cold. Then he
took him out, and sat down by the fire and laid him on his breast and
rubbed his arms that the blood might circulate again. As this also did
no good, he thought to himself: 'When two people lie in bed together,
they warm each other,' and carried him to the bed, covered him over
and lay down by him. After a short time the dead man became warm too,
and began to move. Then said the youth, 'See, little cousin, have I
not warmed you?' The dead man, however, got up and cried: 'Now will I
strangle you.'

'What!' said he, 'is that the way you thank me? You shall at once go
into your coffin again,' and he took him up, threw him into it, and
shut the lid. Then came the six men and carried him away again. 'I
cannot manage to shudder,' said he. 'I shall never learn it here as
long as I live.'

Then a man entered who was taller than all others, and looked
terrible. He was old, however, and had a long white beard. 'You
wretch,' cried he, 'you shall soon learn what it is to shudder, for
you shall die.' 'Not so fast,' replied the youth. 'If I am to die, I
shall have to have a say in it.' 'I will soon seize you,' said the
fiend. 'Softly, softly, do not talk so big. I am as strong as you are,
and perhaps even stronger.' 'We shall see,' said the old man. 'If you
are stronger, I will let you go--come, we will try.' Then he led him
by dark passages to a smith's forge, took an axe, and with one blow
struck an anvil into the ground. 'I can do better than that,' said the
youth, and went to the other anvil. The old man placed himself near
and wanted to look on, and his white beard hung down. Then the youth
seized the axe, split the anvil with one blow, and in it caught the
old man's beard. 'Now I have you,' said the youth. 'Now it is your
turn to die.' Then he seized an iron bar and beat the old man till he
moaned and entreated him to stop, when he would give him great riches.
The youth drew out the axe and let him go. The old man led him back
into the castle, and in a cellar showed him three chests full of gold.
'Of these,' said he, 'one part is for the poor, the other for the
king, the third yours.' In the meantime it struck twelve, and the
spirit disappeared, so that the youth stood in darkness. 'I shall
still be able to find my way out,' said he, and felt about, found the
way into the room, and slept there by his fire. Next morning the king
came and said: 'Now you must have learnt what shuddering is?' 'No,' he
answered; 'what can it be? My dead cousin was here, and a bearded man
came and showed me a great deal of money down below, but no one told
me what it was to shudder.' 'Then,' said the king, 'you have saved the
castle, and shall marry my daughter.' 'That is all very well,' said
he, 'but still I do not know what it is to shudder!'

Then the gold was brought up and the wedding celebrated; but howsoever
much the young king loved his wife, and however happy he was, he still
said always: 'If I could but shudder--if I could but shudder.' And
this at last angered her. Her waiting-maid said: 'I will find a cure
for him; he shall soon learn what it is to shudder.' She went out to
the stream which flowed through the garden, and had a whole bucketful
of gudgeons brought to her. At night when the young king was sleeping,
his wife was to draw the clothes off him and empty the bucket full of
cold water with the gudgeons in it over him, so that the little fishes
would sprawl about him. Then he woke up and cried: 'Oh, what makes me
shudder so?-- what makes me shudder so, dear wife? Ah! now I know what
it is to shudder!'



\chapter{KING GRISLY-BEARD}

A great king of a land far away in the East had a daughter who was
very beautiful, but so proud, and haughty, and conceited, that none of
the princes who came to ask her in marriage was good enough for her,
and she only made sport of them.

Once upon a time the king held a great feast, and asked thither all
her suitors; and they all sat in a row, ranged according to their rank
--kings, and princes, and dukes, and earls, and counts, and barons,
and knights. Then the princess came in, and as she passed by them she
had something spiteful to say to every one. The first was too fat:
'He's as round as a tub,' said she. The next was too tall: 'What a
maypole!' said she. The next was too short: 'What a dumpling!' said
she. The fourth was too pale, and she called him 'Wallface.' The fifth
was too red, so she called him 'Coxcomb.' The sixth was not straight
enough; so she said he was like a green stick, that had been laid to
dry over a baker's oven. And thus she had some joke to crack upon
every one: but she laughed more than all at a good king who was there.
'Look at him,' said she; 'his beard is like an old mop; he shall be
called Grisly-beard.' So the king got the nickname of Grisly-beard.

But the old king was very angry when he saw how his daughter behaved,
and how she ill-treated all his guests; and he vowed that, willing or
unwilling, she should marry the first man, be he prince or beggar,
that came to the door.

Two days after there came by a travelling fiddler, who began to play
under the window and beg alms; and when the king heard him, he said,
'Let him come in.' So they brought in a dirty-looking fellow; and when
he had sung before the king and the princess, he begged a boon. Then
the king said, 'You have sung so well, that I will give you my
daughter for your wife.' The princess begged and prayed; but the king
said, 'I have sworn to give you to the first comer, and I will keep my
word.' So words and tears were of no avail; the parson was sent for,
and she was married to the fiddler. When this was over the king said,
'Now get ready to go--you must not stay here--you must travel on with
your husband.'

Then the fiddler went his way, and took her with him, and they soon
came to a great wood. 'Pray,' said she, 'whose is this wood?' 'It
belongs to King Grisly-beard,' answered he; 'hadst thou taken him, all
had been thine.' 'Ah! unlucky wretch that I am!' sighed she; 'would
that I had married King Grisly-beard!' Next they came to some fine
meadows. 'Whose are these beautiful green meadows?' said she. 'They
belong to King Grisly-beard, hadst thou taken him, they had all been
thine.' 'Ah! unlucky wretch that I am!' said she; 'would that I had
married King Grisly-beard!'

Then they came to a great city. 'Whose is this noble city?' said she.
'It belongs to King Grisly-beard; hadst thou taken him, it had all
been thine.' 'Ah! wretch that I am!' sighed she; 'why did I not marry
King Grisly-beard?' 'That is no business of mine,' said the fiddler:
'why should you wish for another husband? Am not I good enough for
you?'

At last they came to a small cottage. 'What a paltry place!' said she;
'to whom does that little dirty hole belong?' Then the fiddler said,
'That is your and my house, where we are to live.' 'Where are your
servants?' cried she. 'What do we want with servants?' said he; 'you
must do for yourself whatever is to be done. Now make the fire, and
put on water and cook my supper, for I am very tired.' But the
princess knew nothing of making fires and cooking, and the fiddler was
forced to help her. When they had eaten a very scanty meal they went
to bed; but the fiddler called her up very early in the morning to
clean the house. Thus they lived for two days: and when they had eaten
up all there was in the cottage, the man said, 'Wife, we can't go on
thus, spending money and earning nothing. You must learn to weave
baskets.' Then he went out and cut willows, and brought them home, and
she began to weave; but it made her fingers very sore. 'I see this
work won't do,' said he: 'try and spin; perhaps you will do that
better.' So she sat down and tried to spin; but the threads cut her
tender fingers till the blood ran. 'See now,' said the fiddler, 'you
are good for nothing; you can do no work: what a bargain I have got!
However, I'll try and set up a trade in pots and pans, and you shall
stand in the market and sell them.' 'Alas!' sighed she, 'if any of my
father's court should pass by and see me standing in the market, how
they will laugh at me!'

But her husband did not care for that, and said she must work, if she
did not wish to die of hunger. At first the trade went well; for many
people, seeing such a beautiful woman, went to buy her wares, and paid
their money without thinking of taking away the goods. They lived on
this as long as it lasted; and then her husband bought a fresh lot of
ware, and she sat herself down with it in the corner of the market;
but a drunken soldier soon came by, and rode his horse against her
stall, and broke all her goods into a thousand pieces. Then she began
to cry, and knew not what to do. 'Ah! what will become of me?' said
she; 'what will my husband say?' So she ran home and told him all.
'Who would have thought you would have been so silly,' said he, 'as to
put an earthenware stall in the corner of the market, where everybody
passes? but let us have no more crying; I see you are not fit for this
sort of work, so I have been to the king's palace, and asked if they
did not want a kitchen-maid; and they say they will take you, and
there you will have plenty to eat.'

Thus the princess became a kitchen-maid, and helped the cook to do all
the dirtiest work; but she was allowed to carry home some of the meat
that was left, and on this they lived.

She had not been there long before she heard that the king's eldest
son was passing by, going to be married; and she went to one of the
windows and looked out. Everything was ready, and all the pomp and
brightness of the court was there. Then she bitterly grieved for the
pride and folly which had brought her so low. And the servants gave
her some of the rich meats, which she put into her basket to take
home.

All on a sudden, as she was going out, in came the king's son in
golden clothes; and when he saw a beautiful woman at the door, he took
her by the hand, and said she should be his partner in the dance; but
she trembled for fear, for she saw that it was King Grisly-beard, who
was making sport of her. However, he kept fast hold, and led her in;
and the cover of the basket came off, so that the meats in it fell
about. Then everybody laughed and jeered at her; and she was so
abashed, that she wished herself a thousand feet deep in the earth.
She sprang to the door to run away; but on the steps King Grisly-beard
overtook her, and brought her back and said, 'Fear me not! I am the
fiddler who has lived with you in the hut. I brought you there because
I really loved you. I am also the soldier that overset your stall. I
have done all this only to cure you of your silly pride, and to show
you the folly of your ill-treatment of me. Now all is over: you have
learnt wisdom, and it is time to hold our marriage feast.'

Then the chamberlains came and brought her the most beautiful robes;
and her father and his whole court were there already, and welcomed
her home on her marriage. Joy was in every face and every heart. The
feast was grand; they danced and sang; all were merry; and I only wish
that you and I had been of the party.



\chapter{IRON HANS}

There was once upon a time a king who had a great forest near his
palace, full of all kinds of wild animals. One day he sent out a
huntsman to shoot him a roe, but he did not come back. 'Perhaps some
accident has befallen him,' said the king, and the next day he sent
out two more huntsmen who were to search for him, but they too stayed
away. Then on the third day, he sent for all his huntsmen, and said:
'Scour the whole forest through, and do not give up until you have
found all three.' But of these also, none came home again, none were
seen again. From that time forth, no one would any longer venture into
the forest, and it lay there in deep stillness and solitude, and
nothing was seen of it, but sometimes an eagle or a hawk flying over
it. This lasted for many years, when an unknown huntsman announced
himself to the king as seeking a situation, and offered to go into the
dangerous forest. The king, however, would not give his consent, and
said: 'It is not safe in there; I fear it would fare with you no
better than with the others, and you would never come out again.' The
huntsman replied: 'Lord, I will venture it at my own risk, of fear I
know nothing.'

The huntsman therefore betook himself with his dog to the forest. It
was not long before the dog fell in with some game on the way, and
wanted to pursue it; but hardly had the dog run two steps when it
stood before a deep pool, could go no farther, and a naked arm
stretched itself out of the water, seized it, and drew it under. When
the huntsman saw that, he went back and fetched three men to come with
buckets and bale out the water. When they could see to the bottom
there lay a wild man whose body was brown like rusty iron, and whose
hair hung over his face down to his knees. They bound him with cords,
and led him away to the castle. There was great astonishment over the
wild man; the king, however, had him put in an iron cage in his
courtyard, and forbade the door to be opened on pain of death, and the
queen herself was to take the key into her keeping. And from this time
forth everyone could again go into the forest with safety.

The king had a son of eight years, who was once playing in the
courtyard, and while he was playing, his golden ball fell into the
cage. The boy ran thither and said: 'Give me my ball out.' 'Not till
you have opened the door for me,' answered the man. 'No,' said the
boy, 'I will not do that; the king has forbidden it,' and ran away.
The next day he again went and asked for his ball; the wild man said:
'Open my door,' but the boy would not. On the third day the king had
ridden out hunting, and the boy went once more and said: 'I cannot
open the door even if I wished, for I have not the key.' Then the wild
man said: 'It lies under your mother's pillow, you can get it there.'
The boy, who wanted to have his ball back, cast all thought to the
winds, and brought the key. The door opened with difficulty, and the
boy pinched his fingers. When it was open the wild man stepped out,
gave him the golden ball, and hurried away. The boy had become afraid;
he called and cried after him: 'Oh, wild man, do not go away, or I
shall be beaten!' The wild man turned back, took him up, set him on
his shoulder, and went with hasty steps into the forest. When the king
came home, he observed the empty cage, and asked the queen how that
had happened. She knew nothing about it, and sought the key, but it
was gone. She called the boy, but no one answered. The king sent out
people to seek for him in the fields, but they did not find him. Then
he could easily guess what had happened, and much grief reigned in the
royal court.

When the wild man had once more reached the dark forest, he took the
boy down from his shoulder, and said to him: 'You will never see your
father and mother again, but I will keep you with me, for you have set
me free, and I have compassion on you. If you do all I bid you, you
shall fare well. Of treasure and gold have I enough, and more than
anyone in the world.' He made a bed of moss for the boy on which he
slept, and the next morning the man took him to a well, and said:
'Behold, the gold well is as bright and clear as crystal, you shall
sit beside it, and take care that nothing falls into it, or it will be
polluted. I will come every evening to see if you have obeyed my
order.' The boy placed himself by the brink of the well, and often saw
a golden fish or a golden snake show itself therein, and took care
that nothing fell in. As he was thus sitting, his finger hurt him so
violently that he involuntarily put it in the water. He drew it
quickly out again, but saw that it was quite gilded, and whatsoever
pains he took to wash the gold off again, all was to no purpose. In
the evening Iron Hans came back, looked at the boy, and said: 'What
has happened to the well?' 'Nothing nothing,' he answered, and held
his finger behind his back, that the man might not see it. But he
said: 'You have dipped your finger into the water, this time it may
pass, but take care you do not again let anything go in.' By daybreak
the boy was already sitting by the well and watching it. His finger
hurt him again and he passed it over his head, and then unhappily a
hair fell down into the well. He took it quickly out, but it was
already quite gilded. Iron Hans came, and already knew what had
happened. 'You have let a hair fall into the well,' said he. 'I will
allow you to watch by it once more, but if this happens for the third
time then the well is polluted and you can no longer remain with me.'

On the third day, the boy sat by the well, and did not stir his
finger, however much it hurt him. But the time was long to him, and he
looked at the reflection of his face on the surface of the water. And
as he still bent down more and more while he was doing so, and trying
to look straight into the eyes, his long hair fell down from his
shoulders into the water. He raised himself up quickly, but the whole
of the hair of his head was already golden and shone like the sun. You
can imagine how terrified the poor boy was! He took his pocket-
handkerchief and tied it round his head, in order that the man might
not see it. When he came he already knew everything, and said: 'Take
the handkerchief off.' Then the golden hair streamed forth, and let
the boy excuse himself as he might, it was of no use. 'You have not
stood the trial and can stay here no longer. Go forth into the world,
there you will learn what poverty is. But as you have not a bad heart,
and as I mean well by you, there is one thing I will grant you; if you
fall into any difficulty, come to the forest and cry: ``Iron Hans,'' and
then I will come and help you. My power is great, greater than you
think, and I have gold and silver in abundance.'

Then the king's son left the forest, and walked by beaten and unbeaten
paths ever onwards until at length he reached a great city. There he
looked for work, but could find none, and he learnt nothing by which
he could help himself. At length he went to the palace, and asked if
they would take him in. The people about court did not at all know
what use they could make of him, but they liked him, and told him to
stay. At length the cook took him into his service, and said he might
carry wood and water, and rake the cinders together. Once when it so
happened that no one else was at hand, the cook ordered him to carry
the food to the royal table, but as he did not like to let his golden
hair be seen, he kept his little cap on. Such a thing as that had
never yet come under the king's notice, and he said: 'When you come to
the royal table you must take your hat off.' He answered: 'Ah, Lord, I
cannot; I have a bad sore place on my head.' Then the king had the
cook called before him and scolded him, and asked how he could take
such a boy as that into his service; and that he was to send him away
at once. The cook, however, had pity on him, and exchanged him for the
gardener's boy.

And now the boy had to plant and water the garden, hoe and dig, and
bear the wind and bad weather. Once in summer when he was working
alone in the garden, the day was so warm he took his little cap off
that the air might cool him. As the sun shone on his hair it glittered
and flashed so that the rays fell into the bedroom of the king's
daughter, and up she sprang to see what that could be. Then she saw
the boy, and cried to him: 'Boy, bring me a wreath of flowers.' He put
his cap on with all haste, and gathered wild field-flowers and bound
them together. When he was ascending the stairs with them, the
gardener met him, and said: 'How can you take the king's daughter a
garland of such common flowers? Go quickly, and get another, and seek
out the prettiest and rarest.' 'Oh, no,' replied the boy, 'the wild
ones have more scent, and will please her better.' When he got into
the room, the king's daughter said: 'Take your cap off, it is not
seemly to keep it on in my presence.' He again said: 'I may not, I
have a sore head.' She, however, caught at his cap and pulled it off,
and then his golden hair rolled down on his shoulders, and it was
splendid to behold. He wanted to run out, but she held him by the arm,
and gave him a handful of ducats. With these he departed, but he cared
nothing for the gold pieces. He took them to the gardener, and said:
'I present them to your children, they can play with them.' The
following day the king's daughter again called to him that he was to
bring her a wreath of field-flowers, and then he went in with it, she
instantly snatched at his cap, and wanted to take it away from him,
but he held it fast with both hands. She again gave him a handful of
ducats, but he would not keep them, and gave them to the gardener for
playthings for his children. On the third day things went just the
same; she could not get his cap away from him, and he would not have
her money.

Not long afterwards, the country was overrun by war. The king gathered
together his people, and did not know whether or not he could offer
any opposition to the enemy, who was superior in strength and had a
mighty army. Then said the gardener's boy: 'I am grown up, and will go
to the wars also, only give me a horse.' The others laughed, and said:
'Seek one for yourself when we are gone, we will leave one behind us
in the stable for you.' When they had gone forth, he went into the
stable, and led the horse out; it was lame of one foot, and limped
hobblety jib, hobblety jib; nevertheless he mounted it, and rode away
to the dark forest. When he came to the outskirts, he called 'Iron
Hans' three times so loudly that it echoed through the trees.
Thereupon the wild man appeared immediately, and said: 'What do you
desire?' 'I want a strong steed, for I am going to the wars.' 'That
you shall have, and still more than you ask for.' Then the wild man
went back into the forest, and it was not long before a stable-boy
came out of it, who led a horse that snorted with its nostrils, and
could hardly be restrained, and behind them followed a great troop of
warriors entirely equipped in iron, and their swords flashed in the
sun. The youth made over his three-legged horse to the stable-boy,
mounted the other, and rode at the head of the soldiers. When he got
near the battlefield a great part of the king's men had already
fallen, and little was wanting to make the rest give way. Then the
youth galloped thither with his iron soldiers, broke like a hurricane
over the enemy, and beat down all who opposed him. They began to flee,
but the youth pursued, and never stopped, until there was not a single
man left. Instead of returning to the king, however, he conducted his
troop by byways back to the forest, and called forth Iron Hans. 'What
do you desire?' asked the wild man. 'Take back your horse and your
troops, and give me my three-legged horse again.' All that he asked
was done, and soon he was riding on his three-legged horse. When the
king returned to his palace, his daughter went to meet him, and wished
him joy of his victory. 'I am not the one who carried away the
victory,' said he, 'but a strange knight who came to my assistance
with his soldiers.' The daughter wanted to hear who the strange knight
was, but the king did not know, and said: 'He followed the enemy, and
I did not see him again.' She inquired of the gardener where his boy
was, but he smiled, and said: 'He has just come home on his three-
legged horse, and the others have been mocking him, and crying: ``Here
comes our hobblety jib back again!'' They asked, too: ``Under what hedge
have you been lying sleeping all the time?'' So he said: ``I did the
best of all, and it would have gone badly without me.'' And then he was
still more ridiculed.'

The king said to his daughter: 'I will proclaim a great feast that
shall last for three days, and you shall throw a golden apple. Perhaps
the unknown man will show himself.' When the feast was announced, the
youth went out to the forest, and called Iron Hans. 'What do you
desire?' asked he. 'That I may catch the king's daughter's golden
apple.' 'It is as safe as if you had it already,' said Iron Hans. 'You
shall likewise have a suit of red armour for the occasion, and ride on
a spirited chestnut-horse.' When the day came, the youth galloped to
the spot, took his place amongst the knights, and was recognized by no
one. The king's daughter came forward, and threw a golden apple to the
knights, but none of them caught it but he, only as soon as he had it
he galloped away.

On the second day Iron Hans equipped him as a white knight, and gave
him a white horse. Again he was the only one who caught the apple, and
he did not linger an instant, but galloped off with it. The king grew
angry, and said: 'That is not allowed; he must appear before me and
tell his name.' He gave the order that if the knight who caught the
apple, should go away again they should pursue him, and if he would
not come back willingly, they were to cut him down and stab him.

On the third day, he received from Iron Hans a suit of black armour
and a black horse, and again he caught the apple. But when he was
riding off with it, the king's attendants pursued him, and one of them
got so near him that he wounded the youth's leg with the point of his
sword. The youth nevertheless escaped from them, but his horse leapt
so violently that the helmet fell from the youth's head, and they
could see that he had golden hair. They rode back and announced this
to the king.

The following day the king's daughter asked the gardener about his
boy. 'He is at work in the garden; the queer creature has been at the
festival too, and only came home yesterday evening; he has likewise
shown my children three golden apples which he has won.'

The king had him summoned into his presence, and he came and again had
his little cap on his head. But the king's daughter went up to him and
took it off, and then his golden hair fell down over his shoulders,
and he was so handsome that all were amazed. 'Are you the knight who
came every day to the festival, always in different colours, and who
caught the three golden apples?' asked the king. 'Yes,' answered he,
'and here the apples are,' and he took them out of his pocket, and
returned them to the king. 'If you desire further proof, you may see
the wound which your people gave me when they followed me. But I am
likewise the knight who helped you to your victory over your enemies.'
'If you can perform such deeds as that, you are no gardener's boy;
tell me, who is your father?' 'My father is a mighty king, and gold
have I in plenty as great as I require.' 'I well see,' said the king,
'that I owe my thanks to you; can I do anything to please you?' 'Yes,'
answered he, 'that indeed you can. Give me your daughter to wife.' The
maiden laughed, and said: 'He does not stand much on ceremony, but I
have already seen by his golden hair that he was no gardener's boy,'
and then she went and kissed him. His father and mother came to the
wedding, and were in great delight, for they had given up all hope of
ever seeing their dear son again. And as they were sitting at the
marriage-feast, the music suddenly stopped, the doors opened, and a
stately king came in with a great retinue. He went up to the youth,
embraced him and said: 'I am Iron Hans, and was by enchantment a wild
man, but you have set me free; all the treasures which I possess,
shall be your property.'



\chapter{CAT-SKIN}

There was once a king, whose queen had hair of the purest gold, and
was so beautiful that her match was not to be met with on the whole
face of the earth. But this beautiful queen fell ill, and when she
felt that her end drew near she called the king to her and said,
'Promise me that you will never marry again, unless you meet with a
wife who is as beautiful as I am, and who has golden hair like mine.'
Then when the king in his grief promised all she asked, she shut her
eyes and died. But the king was not to be comforted, and for a long
time never thought of taking another wife. At last, however, his wise
men said, 'this will not do; the king must marry again, that we may
have a queen.' So messengers were sent far and wide, to seek for a
bride as beautiful as the late queen. But there was no princess in the
world so beautiful; and if there had been, still there was not one to
be found who had golden hair. So the messengers came home, and had had
all their trouble for nothing.

Now the king had a daughter, who was just as beautiful as her mother,
and had the same golden hair. And when she was grown up, the king
looked at her and saw that she was just like this late queen: then he
said to his courtiers, 'May I not marry my daughter? She is the very
image of my dead wife: unless I have her, I shall not find any bride
upon the whole earth, and you say there must be a queen.' When the
courtiers heard this they were shocked, and said, 'Heaven forbid that
a father should marry his daughter! Out of so great a sin no good can
come.' And his daughter was also shocked, but hoped the king would
soon give up such thoughts; so she said to him, 'Before I marry anyone
I must have three dresses: one must be of gold, like the sun; another
must be of shining silver, like the moon; and a third must be dazzling
as the stars: besides this, I want a mantle of a thousand different
kinds of fur put together, to which every beast in the kingdom must
give a part of his skin.' And thus she though he would think of the
matter no more. But the king made the most skilful workmen in his
kingdom weave the three dresses: one golden, like the sun; another
silvery, like the moon; and a third sparkling, like the stars: and his
hunters were told to hunt out all the beasts in his kingdom, and to
take the finest fur out of their skins: and thus a mantle of a
thousand furs was made.

When all were ready, the king sent them to her; but she got up in the
night when all were asleep, and took three of her trinkets, a golden
ring, a golden necklace, and a golden brooch, and packed the three
dresses--of the sun, the moon, and the stars--up in a nutshell, and
wrapped herself up in the mantle made of all sorts of fur, and
besmeared her face and hands with soot. Then she threw herself upon
Heaven for help in her need, and went away, and journeyed on the whole
night, till at last she came to a large wood. As she was very tired,
she sat herself down in the hollow of a tree and soon fell asleep: and
there she slept on till it was midday.

Now as the king to whom the wood belonged was hunting in it, his dogs
came to the tree, and began to snuff about, and run round and round,
and bark. 'Look sharp!' said the king to the huntsmen, 'and see what
sort of game lies there.' And the huntsmen went up to the tree, and
when they came back again said, 'In the hollow tree there lies a most
wonderful beast, such as we never saw before; its skin seems to be of
a thousand kinds of fur, but there it lies fast asleep.' 'See,' said
the king, 'if you can catch it alive, and we will take it with us.' So
the huntsmen took it up, and the maiden awoke and was greatly
frightened, and said, 'I am a poor child that has neither father nor
mother left; have pity on me and take me with you.' Then they said,
'Yes, Miss Cat-skin, you will do for the kitchen; you can sweep up the
ashes, and do things of that sort.' So they put her into the coach,
and took her home to the king's palace. Then they showed her a little
corner under the staircase, where no light of day ever peeped in, and
said, 'Cat-skin, you may lie and sleep there.' And she was sent into
the kitchen, and made to fetch wood and water, to blow the fire, pluck
the poultry, pick the herbs, sift the ashes, and do all the dirty
work.

Thus Cat-skin lived for a long time very sorrowfully. 'Ah! pretty
princess!' thought she, 'what will now become of thee?' But it
happened one day that a feast was to be held in the king's castle, so
she said to the cook, 'May I go up a little while and see what is
going on? I will take care and stand behind the door.' And the cook
said, 'Yes, you may go, but be back again in half an hour's time, to
rake out the ashes.' Then she took her little lamp, and went into her
cabin, and took off the fur skin, and washed the soot from off her
face and hands, so that her beauty shone forth like the sun from
behind the clouds. She next opened her nutshell, and brought out of it
the dress that shone like the sun, and so went to the feast. Everyone
made way for her, for nobody knew her, and they thought she could be
no less than a king's daughter. But the king came up to her, and held
out his hand and danced with her; and he thought in his heart, 'I
never saw any one half so beautiful.'

When the dance was at an end she curtsied; and when the king looked
round for her, she was gone, no one knew wither. The guards that stood
at the castle gate were called in: but they had seen no one. The truth
was, that she had run into her little cabin, pulled off her dress,
blackened her face and hands, put on the fur-skin cloak, and was Cat-
skin again. When she went into the kitchen to her work, and began to
rake the ashes, the cook said, 'Let that alone till the morning, and
heat the king's soup; I should like to run up now and give a peep: but
take care you don't let a hair fall into it, or you will run a chance
of never eating again.'

As soon as the cook went away, Cat-skin heated the king's soup, and
toasted a slice of bread first, as nicely as ever she could; and when
it was ready, she went and looked in the cabin for her little golden
ring, and put it into the dish in which the soup was. When the dance
was over, the king ordered his soup to be brought in; and it pleased
him so well, that he thought he had never tasted any so good before.
At the bottom he saw a gold ring lying; and as he could not make out
how it had got there, he ordered the cook to be sent for. The cook was
frightened when he heard the order, and said to Cat-skin, 'You must
have let a hair fall into the soup; if it be so, you will have a good
beating.' Then he went before the king, and he asked him who had
cooked the soup. 'I did,' answered the cook. But the king said, 'That
is not true; it was better done than you could do it.' Then he
answered, 'To tell the truth I did not cook it, but Cat-skin did.'
'Then let Cat-skin come up,' said the king: and when she came he said
to her, 'Who are you?' 'I am a poor child,' said she, 'that has lost
both father and mother.' 'How came you in my palace?' asked he. 'I am
good for nothing,' said she, 'but to be scullion-girl, and to have
boots and shoes thrown at my head.' 'But how did you get the ring that
was in the soup?' asked the king. Then she would not own that she knew
anything about the ring; so the king sent her away again about her
business.

After a time there was another feast, and Cat-skin asked the cook to
let her go up and see it as before. 'Yes,' said he, 'but come again in
half an hour, and cook the king the soup that he likes so much.' Then
she ran to her little cabin, washed herself quickly, and took her
dress out which was silvery as the moon, and put it on; and when she
went in, looking like a king's daughter, the king went up to her, and
rejoiced at seeing her again, and when the dance began he danced with
her. After the dance was at an end she managed to slip out, so slyly
that the king did not see where she was gone; but she sprang into her
little cabin, and made herself into Cat-skin again, and went into the
kitchen to cook the soup. Whilst the cook was above stairs, she got
the golden necklace and dropped it into the soup; then it was brought
to the king, who ate it, and it pleased him as well as before; so he
sent for the cook, who was again forced to tell him that Cat-skin had
cooked it. Cat-skin was brought again before the king, but she still
told him that she was only fit to have boots and shoes thrown at her
head.

But when the king had ordered a feast to be got ready for the third
time, it happened just the same as before. 'You must be a witch, Cat-
skin,' said the cook; 'for you always put something into your soup, so
that it pleases the king better than mine.' However, he let her go up
as before. Then she put on her dress which sparkled like the stars,
and went into the ball-room in it; and the king danced with her again,
and thought she had never looked so beautiful as she did then. So
whilst he was dancing with her, he put a gold ring on her finger
without her seeing it, and ordered that the dance should be kept up a
long time. When it was at an end, he would have held her fast by the
hand, but she slipped away, and sprang so quickly through the crowd
that he lost sight of her: and she ran as fast as she could into her
little cabin under the stairs. But this time she kept away too long,
and stayed beyond the half-hour; so she had not time to take off her
fine dress, and threw her fur mantle over it, and in her haste did not
blacken herself all over with soot, but left one of her fingers white.

Then she ran into the kitchen, and cooked the king's soup; and as soon
as the cook was gone, she put the golden brooch into the dish. When
the king got to the bottom, he ordered Cat-skin to be called once
more, and soon saw the white finger, and the ring that he had put on
it whilst they were dancing: so he seized her hand, and kept fast hold
of it, and when she wanted to loose herself and spring away, the fur
cloak fell off a little on one side, and the starry dress sparkled
underneath it.

Then he got hold of the fur and tore it off, and her golden hair and
beautiful form were seen, and she could no longer hide herself: so she
washed the soot and ashes from her face, and showed herself to be the
most beautiful princess upon the face of the earth. But the king said,
'You are my beloved bride, and we will never more be parted from each
other.' And the wedding feast was held, and a merry day it was, as
ever was heard of or seen in that country, or indeed in any other.



\chapter{SNOW-WHITE AND ROSE-RED}

There was once a poor widow who lived in a lonely cottage. In front of
the cottage was a garden wherein stood two rose-trees, one of which
bore white and the other red roses. She had two children who were like
the two rose-trees, and one was called Snow-white, and the other Rose-
red. They were as good and happy, as busy and cheerful as ever two
children in the world were, only Snow-white was more quiet and gentle
than Rose-red. Rose-red liked better to run about in the meadows and
fields seeking flowers and catching butterflies; but Snow-white sat at
home with her mother, and helped her with her housework, or read to
her when there was nothing to do.

The two children were so fond of one another that they always held
each other by the hand when they went out together, and when Snow-
white said: 'We will not leave each other,' Rose-red answered: 'Never
so long as we live,' and their mother would add: 'What one has she
must share with the other.'

They often ran about the forest alone and gathered red berries, and no
beasts did them any harm, but came close to them trustfully. The
little hare would eat a cabbage-leaf out of their hands, the roe
grazed by their side, the stag leapt merrily by them, and the birds
sat still upon the boughs, and sang whatever they knew.

No mishap overtook them; if they had stayed too late in the forest,
and night came on, they laid themselves down near one another upon the
moss, and slept until morning came, and their mother knew this and did
not worry on their account.

Once when they had spent the night in the wood and the dawn had roused
them, they saw a beautiful child in a shining white dress sitting near
their bed. He got up and looked quite kindly at them, but said nothing
and went into the forest. And when they looked round they found that
they had been sleeping quite close to a precipice, and would certainly
have fallen into it in the darkness if they had gone only a few paces
further. And their mother told them that it must have been the angel
who watches over good children.

Snow-white and Rose-red kept their mother's little cottage so neat
that it was a pleasure to look inside it. In the summer Rose-red took
care of the house, and every morning laid a wreath of flowers by her
mother's bed before she awoke, in which was a rose from each tree. In
the winter Snow-white lit the fire and hung the kettle on the hob. The
kettle was of brass and shone like gold, so brightly was it polished.
In the evening, when the snowflakes fell, the mother said: 'Go, Snow-
white, and bolt the door,' and then they sat round the hearth, and the
mother took her spectacles and read aloud out of a large book, and the
two girls listened as they sat and spun. And close by them lay a lamb
upon the floor, and behind them upon a perch sat a white dove with its
head hidden beneath its wings.

One evening, as they were thus sitting comfortably together, someone
knocked at the door as if he wished to be let in. The mother said:
'Quick, Rose-red, open the door, it must be a traveller who is seeking
shelter.' Rose-red went and pushed back the bolt, thinking that it was
a poor man, but it was not; it was a bear that stretched his broad,
black head within the door.

Rose-red screamed and sprang back, the lamb bleated, the dove
fluttered, and Snow-white hid herself behind her mother's bed. But the
bear began to speak and said: 'Do not be afraid, I will do you no
harm! I am half-frozen, and only want to warm myself a little beside
you.'

'Poor bear,' said the mother, 'lie down by the fire, only take care
that you do not burn your coat.' Then she cried: 'Snow-white, Rose-
red, come out, the bear will do you no harm, he means well.' So they
both came out, and by-and-by the lamb and dove came nearer, and were
not afraid of him. The bear said: 'Here, children, knock the snow out
of my coat a little'; so they brought the broom and swept the bear's
hide clean; and he stretched himself by the fire and growled
contentedly and comfortably. It was not long before they grew quite at
home, and played tricks with their clumsy guest. They tugged his hair
with their hands, put their feet upon his back and rolled him about,
or they took a hazel-switch and beat him, and when he growled they
laughed. But the bear took it all in good part, only when they were
too rough he called out: 'Leave me alive, children,

\begin{verse}
 'Snow-white, Rose-red,\\
  Will you beat your wooer dead?'
\end{verse}

When it was bed-time, and the others went to bed, the mother said to
the bear: 'You can lie there by the hearth, and then you will be safe
from the cold and the bad weather.' As soon as day dawned the two
children let him out, and he trotted across the snow into the forest.

Henceforth the bear came every evening at the same time, laid himself
down by the hearth, and let the children amuse themselves with him as
much as they liked; and they got so used to him that the doors were
never fastened until their black friend had arrived.

When spring had come and all outside was green, the bear said one
morning to Snow-white: 'Now I must go away, and cannot come back for
the whole summer.' 'Where are you going, then, dear bear?' asked Snow-
white. 'I must go into the forest and guard my treasures from the
wicked dwarfs. In the winter, when the earth is frozen hard, they are
obliged to stay below and cannot work their way through; but now, when
the sun has thawed and warmed the earth, they break through it, and
come out to pry and steal; and what once gets into their hands, and in
their caves, does not easily see daylight again.'

Snow-white was quite sorry at his departure, and as she unbolted the
door for him, and the bear was hurrying out, he caught against the
bolt and a piece of his hairy coat was torn off, and it seemed to
Snow-white as if she had seen gold shining through it, but she was not
sure about it. The bear ran away quickly, and was soon out of sight
behind the trees.

A short time afterwards the mother sent her children into the forest
to get firewood. There they found a big tree which lay felled on the
ground, and close by the trunk something was jumping backwards and
forwards in the grass, but they could not make out what it was. When
they came nearer they saw a dwarf with an old withered face and a
snow-white beard a yard long. The end of the beard was caught in a
crevice of the tree, and the little fellow was jumping about like a
dog tied to a rope, and did not know what to do.

He glared at the girls with his fiery red eyes and cried: 'Why do you
stand there? Can you not come here and help me?' 'What are you up to,
little man?' asked Rose-red. 'You stupid, prying goose!' answered the
dwarf: 'I was going to split the tree to get a little wood for
cooking. The little bit of food that we people get is immediately
burnt up with heavy logs; we do not swallow so much as you coarse,
greedy folk. I had just driven the wedge safely in, and everything was
going as I wished; but the cursed wedge was too smooth and suddenly
sprang out, and the tree closed so quickly that I could not pull out
my beautiful white beard; so now it is tight and I cannot get away,
and the silly, sleek, milk-faced things laugh! Ugh! how odious you
are!'

The children tried very hard, but they could not pull the beard out,
it was caught too fast. 'I will run and fetch someone,' said Rose-red.
'You senseless goose!' snarled the dwarf; 'why should you fetch
someone? You are already two too many for me; can you not think of
something better?' 'Don't be impatient,' said Snow-white, 'I will help
you,' and she pulled her scissors out of her pocket, and cut off the
end of the beard.

As soon as the dwarf felt himself free he laid hold of a bag which lay
amongst the roots of the tree, and which was full of gold, and lifted
it up, grumbling to himself: 'Uncouth people, to cut off a piece of my
fine beard. Bad luck to you!' and then he swung the bag upon his back,
and went off without even once looking at the children.

Some time afterwards Snow-white and Rose-red went to catch a dish of
fish. As they came near the brook they saw something like a large
grasshopper jumping towards the water, as if it were going to leap in.
They ran to it and found it was the dwarf. 'Where are you going?' said
Rose-red; 'you surely don't want to go into the water?' 'I am not such
a fool!' cried the dwarf; 'don't you see that the accursed fish wants
to pull me in?' The little man had been sitting there fishing, and
unluckily the wind had tangled up his beard with the fishing-line; a
moment later a big fish made a bite and the feeble creature had not
strength to pull it out; the fish kept the upper hand and pulled the
dwarf towards him. He held on to all the reeds and rushes, but it was
of little good, for he was forced to follow the movements of the fish,
and was in urgent danger of being dragged into the water.

The girls came just in time; they held him fast and tried to free his
beard from the line, but all in vain, beard and line were entangled
fast together. There was nothing to do but to bring out the scissors
and cut the beard, whereby a small part of it was lost. When the dwarf
saw that he screamed out: 'Is that civil, you toadstool, to disfigure
a man's face? Was it not enough to clip off the end of my beard? Now
you have cut off the best part of it. I cannot let myself be seen by
my people. I wish you had been made to run the soles off your shoes!'
Then he took out a sack of pearls which lay in the rushes, and without
another word he dragged it away and disappeared behind a stone.

It happened that soon afterwards the mother sent the two children to
the town to buy needles and thread, and laces and ribbons. The road
led them across a heath upon which huge pieces of rock lay strewn
about. There they noticed a large bird hovering in the air, flying
slowly round and round above them; it sank lower and lower, and at
last settled near a rock not far away. Immediately they heard a loud,
piteous cry. They ran up and saw with horror that the eagle had seized
their old acquaintance the dwarf, and was going to carry him off.

The children, full of pity, at once took tight hold of the little man,
and pulled against the eagle so long that at last he let his booty go.
As soon as the dwarf had recovered from his first fright he cried with
his shrill voice: 'Could you not have done it more carefully! You
dragged at my brown coat so that it is all torn and full of holes, you
clumsy creatures!' Then he took up a sack full of precious stones, and
slipped away again under the rock into his hole. The girls, who by
this time were used to his ingratitude, went on their way and did
their business in town.

As they crossed the heath again on their way home they surprised the
dwarf, who had emptied out his bag of precious stones in a clean spot,
and had not thought that anyone would come there so late. The evening
sun shone upon the brilliant stones; they glittered and sparkled with
all colours so beautifully that the children stood still and stared at
them. 'Why do you stand gaping there?' cried the dwarf, and his ashen-
grey face became copper-red with rage. He was still cursing when a
loud growling was heard, and a black bear came trotting towards them
out of the forest. The dwarf sprang up in a fright, but he could not
reach his cave, for the bear was already close. Then in the dread of
his heart he cried: 'Dear Mr Bear, spare me, I will give you all my
treasures; look, the beautiful jewels lying there! Grant me my life;
what do you want with such a slender little fellow as I? you would not
feel me between your teeth. Come, take these two wicked girls, they
are tender morsels for you, fat as young quails; for mercy's sake eat
them!' The bear took no heed of his words, but gave the wicked
creature a single blow with his paw, and he did not move again.

The girls had run away, but the bear called to them: 'Snow-white and
Rose-red, do not be afraid; wait, I will come with you.' Then they
recognized his voice and waited, and when he came up to them suddenly
his bearskin fell off, and he stood there a handsome man, clothed all
in gold. 'I am a king's son,' he said, 'and I was bewitched by that
wicked dwarf, who had stolen my treasures; I have had to run about the
forest as a savage bear until I was freed by his death. Now he has got
his well-deserved punishment.

Snow-white was married to him, and Rose-red to his brother, and they
divided between them the great treasure which the dwarf had gathered
together in his cave. The old mother lived peacefully and happily with
her children for many years. She took the two rose-trees with her, and
they stood before her window, and every year bore the most beautiful
roses, white and red.

\appendix

\chapter{ABOUT THE BROTHERS GRIMM}





The Brothers Grimm, Jacob (1785-1863) and Wilhelm (1786-1859), were
born in Hanau, near Frankfurt, in the German state of Hesse.
Throughout their lives they remained close friends, and both studied
law at Marburg University. Jacob was a pioneer in the study of German
philology, and although Wilhelm's work was hampered by poor health the
brothers collaborated in the creation of a German dictionary, not
completed until a century after their deaths. But they were best (and
universally) known for the collection of over two hundred folk tales
they made from oral sources and published in two volumes of 'Nursery
and Household Tales' in 1812 and 1814. Although their intention was to
preserve such material as part of German cultural and literary
history, and their collection was first published with scholarly notes
and no illustration, the tales soon came into the possession of young
readers. This was in part due to Edgar Taylor, who made the first
English translation in 1823, selecting about fifty stories 'with the
amusement of some young friends principally in view.' They have been
an essential ingredient of children's reading ever since.


\end{document}




























